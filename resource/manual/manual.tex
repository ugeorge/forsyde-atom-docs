\documentclass{book}
\usepackage{sty/atom-manual}
\usepackage{sty/haddock}
\usepackage{sty/urls}
\usepackage{sty/atom-vars}
\usepackage{ccicons}

\usepackage{draftwatermark}
\SetWatermarkText{\textsc{Draft}}
\SetWatermarkScale{5}

\title{\textsc{ForSyDe-Atom}\\User Manual}
\author{George Ungureanu}

\newcommand*{\RootPath}{./input}%
\addbibresource{refs.bib}
\defbibheading{subbibliography}{\section{References}}

\makeindex
\begin{document}
\dominitoc

\pagenumbering{Roman} 
\includepdf[pages={1}]{title.pdf}

%License
\begin{bottompar}\small
\begin{tabular}{p{.69\textwidth}r}
  \vspace{-.5cm} This work is licensed under a Creative Commons  Attribution-ShareAlike 4.0 Unported (CC BY-SA 4.0) License. & {\Huge\ccbysa}\\[3ex]
  \multicolumn{2}{p{\textwidth}}{The code listed throughout this document is generated from several projects mostly licensed under the BSD-3 Clause License, unless specified otherwise.} \\
\end{tabular}

\end{bottompar}

\tableofcontents
\clearpage
\listoffigures
\clearpage

\chapter*{List of Case Studies}
\label{cha:list-publications}
\markboth{List of Case Studies}{List of Case Studies}

\fullcite{Ungureanu17}\dotfill\cref{sec:getting-started:toy-example}

\vspace{1ex}
\noindent\fullcite{Ungureanu18a}\dotfill\cref{sec:hybrid:rc-oscillator}

\mainmatter

\pagenumbering{arabic}

\begin{refsection}
  \chapter{Introduction}
  \label{cha:introduction}

  \begin{summary}
    \input{\RootPath/abstract}
  \end{summary}
  \input{\RootPath/purpose}
  \input{\RootPath/getting}
  \input{\RootPath/usage}

  \printbibliography[heading=subbibliography]
\end{refsection}

\part{Examples \& Reports}
\label{part:introduction}

\begin{refsection}
  \chapter{Getting Started with {{\sc ForSyDe-Atom}}}
  \label{ch:getting-started}
  \renewcommand*{\RootPath}{%
    \AtomExamplesRoot/getting-started/docs/latex}%
  \graphicspath{{\AtomExamplesRoot/getting-started/docs/latex/figs/}}

  \begin{summary}
    \input{\RootPath/input/abstract}
  \end{summary}
  \minitoc
  \vspace{1ex}
  \input{\RootPath/input/info}
  \input{\RootPath/input/goals}
  \input{\RootPath/code/GettingStarted/Basics}
  \input{\RootPath/code/GettingStarted/Plot}
  \input{\RootPath/input/toy-example}
  \input{\RootPath/code/GettingStarted/TestSignals}
  \input{\RootPath/code/GettingStarted/SY}
  \input{\RootPath/code/GettingStarted/DE}
  \input{\RootPath/code/GettingStarted/CT}
  \input{\RootPath/code/GettingStarted/SDF}
  \input{\RootPath/code/GettingStarted/Polymorphic}
  \input{\RootPath/code/GettingStarted/CustomPattern}
  \input{\RootPath/input/conclusion}

  \printbibliography[heading=subbibliography]
\end{refsection}

\begin{refsection}
  \chapter{Hybrid CT/DT Models in {{\sc ForSyDe-Atom}}}
  \label{ch:hybrid}
  \renewcommand*{\RootPath}{\AtomExamplesRoot/hybrid/docs/latex}%
  \graphicspath{{\AtomExamplesRoot/hybrid/docs/latex/figs/}}

  \begin{summary}
    \input{\RootPath/input/abstract}
  \end{summary}
  \minitoc
  \vspace{1ex}
  \input{\RootPath/input/info}
  \input{\RootPath/input/goals}
  \input{\RootPath/code/Hybrid/RCOsc}
  \input{\RootPath/input/conclusion}

  \printbibliography[heading=subbibliography]
\end{refsection}

\part{Tools \& Libraries Documentation}
\label{part:api-documentation}

\chapter{The \textsc{ForSyDe-Atom} Standard Library}
\label{ch:forsyde-atom}
  \graphicspath{{./}}

\begin{refsection}
  \parskip=10pt plus2pt minus2pt
  \setlength{\parindent}{0cm}

  \begin{summary}
    This chapter is an extended API documentation of \textsc{ForSyDe-Atom} Standard Library version \AtomVersion\footnote{available at: \url{https://github.com/forsyde/forsyde-atom}} which was organized to also serve the purpose of technical report. It treats both theoretical and practical aspects, justifying the implementation and providing examples of usage for most of the library-exported functions. It has been generated from the inline documentation using a \href{https://www.haskell.org/haddock/}{\mbox{Haddock}}\footnote{custom build from: \url{https://github.com/ugeorge/haddock}} generator.

  \end{summary}
  \minitoc
  \section{Introduction}
\label{sec:forsyde-atom:introduction}

The ForSyDe (Formal System Design) methodology has been developed
with the objective to move system design to a higher level of
abstraction and to bridge the abstraction gap by transformational
design refinement. It targets the modelling and characterization of
cyber-physical systems inheriting the theory of
\href{http://ieeexplore.ieee.org/stamp/stamp.jsp?arnumber=736561}{Models of Computation (MoCs)},
providing both a correct-by-construction execution model and
analyzable entry point for further synthesis flows. For more information
about ForSyDe and its associated projects please consult the
\href{https://github.com/forsyde}{ForSyDe webpage}.\par
The \haddocktt{forsyde-atom} library is a shallow-embedded DSL implementing
the execution semantics of an \emph{atom-based approach} to ForSyDe. Its
purpose is to provide a modeling framework for cyber-physical
systems and to serve as a proof-of-concept for a future (non-strict
typed) DSL. Adhering to the formalism set by the
\href{http://www.diva-portal.org/smash/get/diva2:9340/FULLTEXT01.pdf}{PhD Thesis of Ingo Sander},
the current framework views systems as networks of processes
communicating through signals. The \emph{atom-based approach} to ForSyDe
adds some new important concepts:\par
\begin{itemize}
\item
the separation of concerns through semantically-independent
\textbf{\emph{layers}} of computation, behavior, synchronization and
structure. Each layer offers a different analyzable view of a
given system and is described using the concept of
\emph{higher-order functions}.\par

\item
the description of each layer as a network of primitive building
blocks called \textbf{\emph{atoms}}. Each atom embeds an undividable operation,
and wraps functions of lower layers with the semantics dictated by
a higher layer. They are described using the powerful concept of
\emph{applicative functors}.\par

\item
the complete \textbf{\emph{autonomy}} of atoms in relation to the patterns
they build. As such, process networks or constructors are nothing
but \emph{structural} ad-hoc compositions that aid in achieving complex
behaviors whereas the actual behavior is dictated by the atoms
alone. This is proved (for the time being) for the
\emph{Synchronization Layer} by implementing all MoCs as instances of
\emph{only one type class}.\par

\end{itemize}

While the host language limits the possibility of providing general
(e.g. proces) constructors due to type-strictness, the
\haddocktt{forsyde-atom} framework is by all means complete. In this sense the
user can create her own custom \emph{correct-by-design} constructors and
networks as compositions of the provided atoms and utilities. Also,
this haddock-generated page is organized both as an API
documentation and as a technical report to facilitate te use and
understanding of the formal principles behing the design process.\par

\subsection{Naming convention}
\label{sec:forsyde-atom:naming-convention}

All multi-parameter patterns and
utilities provided by the library API as higher-order functions are
named along the lines of \haddocktt{functionMN} where \haddocktt{M} represents the
number of \textbf{\emph{curried}} inputs (i.e. \haddocktt{a1\ ->\ a2\ ->\ ...\ ->\ aM}),
while \haddocktt{N} represents the number of \textbf{\emph{tupled}} outputs
(i.e. \haddocktt{(b1,b2,...,bN)}). To avoid repetition we only provide
documentation for functions with 2 inputs and 2 outputs
(i.e. \haddocktt{function22}), while the available ones are mentioned as a 
regex (i.e. \haddocktt{function[1-4][1-4]}). In case the provided functions do not suffice,
feel free to implement your own patterns following the example in \cref{sec:getting-started:making-your-own}.\par

  \input{input/ForSyDe-Atom}
  \vspace{3ex}
\haddockmoduleheading{ForSyDe.Atom.ExB}
\label{module:ForSyDe.Atom.ExB}
\haddockbeginheader
{\haddockverb\begin{verbatim}
module ForSyDe.Atom.ExB (
    ExB(extend, (/.\), (/*\), (/&\), (/!\)),  res22,  filter,  filter', 
    degrade,  ignore22
  ) where\end{verbatim}}
\haddockendheader

This module exports the core entities of the extended behavior
 layer: interfaces for atoms and common patterns of atoms. It does
 \emph{NOT} export any implementation or instantiation of any specific
 behavior extension type.\par
\begin{mdframed}[style=reminder,frametitle=Reminder]Make sure to consult naming conventions in  \cref{sec:forsyde-atom:naming-convention} in order to interpret the names and type signatures correctly.\end{mdframed}\par

\subsection{Atoms}
\begin{haddockdesc}
\item[\begin{tabular}{@{}l}
class\ Functor\ b\ =>\ ExB\ b\ where
\end{tabular}]\haddockbegindoc
Class which defines the atoms for the extended behavior layer.\par
As its name suggests, this layer is extending the behavior of
 processes (or merely of functions if we ignore timing semantics
 completely), by expanding the domains of the wrapped layer
 (e.g. the set of values) with symbols having clearly-defined
 semantics (e.g. special events with known responses).\par
The types associated with this layer can simply be describes as:\par
\haddockeq{fig/eqs-exb-types.pdf}\par
where  \emph{α} is a base type and \emph{b} is the type extension,
 i.e. a set of symbols with clearly-defined semantics.\par
Extended behavior atoms are functions of these types, defined as
 interfaces in the \haddockid{ExB} type class.\par

\haddockpremethods{}\textbf{Methods}
\begin{haddockdesc}
\item[\begin{tabular}{@{}l}\haddockid{extend}\ ::\ a\ ->\ b\ a\ Source\ \end{tabular}]
\haddockbegindoc
Extends a value (from a layer below) with a set of symbols with known semantics, as described by a type instantiating this class.\par

\item[\begin{tabular}{@{}l}\haddockid{(/.\textbackslash)}\ ::\ (a\ ->\ a)\ ->\ b\ a\ ->\ b\ a\end{tabular}]
\haddockbegindoc
Basic functor operator. Lifts a function (from a layer below) into the domain of the extended behavior layer.\haddockeq{fig/eqs-exb-atom-func.pdf}\par

\item[\begin{tabular}{@{}l}\haddockid{(/*\textbackslash)}\ ::\ b\ (a\ ->\ a)\ ->\ b\ a\ ->\ b\ a\end{tabular}]
\haddockbegindoc
Applicative operator. Defines a res between two extended behavior symbols.\haddockeq{fig/eqs-exb-atom-app.pdf}\par

\item[\begin{tabular}{@{}l}\haddockid{(/\&\textbackslash)}\ ::\ b\ Bool\ ->\ b\ a\ ->\ b\ a\ Source\ \end{tabular}]
\haddockbegindoc
Predicate operator. Generates a defined behavior based on an extended Boolean predicate.\haddockeq{fig/eqs-exb-atom-phi.pdf}\par

\item[\begin{tabular}{@{}l}\haddockid{(/"!\textbackslash)}\ ::\ a\ ->\ b\ a\ ->\ a\ Source\ \end{tabular}]
\haddockbegindoc
Degrade operator. Degrades a behavior-extended value into a non-extended one (from a layer below), based on a kernel value. Used also to throw exceptions.\haddockeq{fig/eqs-exb-atom-deg.pdf}\par

\end{haddockdesc}


\item[\begin{tabular}{@{}l}
instance\ ExB\ AbstExt
\end{tabular}]\haddockbegindoc
Implements the absent semantics of the extended behavior atoms.\par

\end{haddockdesc}
\subsection{Patterns}
\begin{haddockdesc}
\item[\begin{tabular}{@{}l}
res22
\end{tabular}]\haddockbegindoc
\haddockbeginargs
\haddockdecltt{::} & ExB b \\
                     \haddockdecltt{=>} & \haddockdecltt{(a1
                                                          -> a2
                                                             -> (a1', a2'))} & function on values \\
                                                                               \haddockdecltt{->} & \haddockdecltt{b a1} & first input \\
                                                                                                                           \haddockdecltt{->} & \haddockdecltt{b a2} & second input \\
                                                                                                                                                                       \haddockdecltt{->} & \haddockdecltt{(b a1', b a2')} & tupled output \\
\end{tabulary}\par
\haddockeq{fig/eqs-exb-pattern-resolution.pdf}\par
The \haddocktt{res} behavior pattern lifts a function on values to the
 extended behavior domain, and applies a resolution between two
 extended behavior symbols.\par
Constructors: \haddocktt{res{\char 91}1-8{\char 93}{\char 91}1-4{\char 93}}.\par


\item[\begin{tabular}{@{}l}
filter\ ::\ ExB\ b\ =>\ b\ Bool\ ->\ b\ a\ ->\ b\ a
\end{tabular}]\haddockbegindoc
Prefix name for the prefix operator \haddockid{/{\char '46}{\char '134}}.\par


\item[\begin{tabular}{@{}l}
filter'\ ::\ ExB\ b\ =>\ Bool\ ->\ a\ ->\ b\ a
\end{tabular}]\haddockbegindoc
Same as \haddockid{filter} but takes base (non-extended) values as
 input arguments.\par


\item[\begin{tabular}{@{}l}
degrade\ ::\ ExB\ b\ =>\ a\ ->\ b\ a\ ->\ a
\end{tabular}]\haddockbegindoc
Prefix name for the degrade operator \haddockid{/"!{\char '134}}.\par


\item[\begin{tabular}{@{}l}
ignore22
\end{tabular}]\haddockbegindoc
\haddockbeginargs
\haddockdecltt{::} & ExB b \\
                     \haddockdecltt{=>} & \haddockdecltt{(a1
                                                          -> a2
                                                             -> a1'
                                                                -> a2'
                                                                   -> (a1, a2))} & function of \haddocktt{Y\ +\ X} arguments \\
                                                                                   \haddockdecltt{->} & \haddockdecltt{a1} & \\
                                                                                                                             \haddockdecltt{->} & \haddockdecltt{a2} & \\
                                                                                                                                                                       \haddockdecltt{->} & \haddockdecltt{b a1'} & \\
                                                                                                                                                                                                                    \haddockdecltt{->} & \haddockdecltt{b a2'} & \\
                                                                                                                                                                                                                                                                 \haddockdecltt{->} & \haddockdecltt{(a1, a2)} & \\
\end{tabulary}\par
\haddockeq{fig/eqs-exb-pattern-ignore.pdf}\par
The \haddocktt{ignoreXY} pattern takes a function of \haddocktt{Y\ +\ X} arguments, \haddocktt{Y}
 basic inputs followed by \haddocktt{X} behavior-extended inputs. The \haddocktt{X}
 behavior-extended arguments are subjugated to a res, and the
 result is then degraded using the first \haddocktt{Y} arguments as
 fallback. The effect is similar to "ignoring" a the result of a
 res function if ∈ \emph{b}.\par
The main application of this pattern is as extended behavior
 wrapper for state machine functions which do not "understand"
 extended behavior semantics, i.e. it simply propagates the current
 state (∈ \emph{α}) if the inputs (their res) belongs
 to the set of extended values (∈ \emph{b}).\par
Constructors: \haddocktt{ignore{\char 91}1-4{\char 93}{\char 91}1-4{\char 93}}.\par

\end{haddockdesc}
  \input{input/ForSyDe-Atom-ExB-Absent}
  \haddockmoduleheading{ForSyDe.Atom.MoC}
\label{module:ForSyDe.Atom.MoC}
\haddockbeginheader
{\haddockverb\begin{verbatim}
module ForSyDe.Atom.MoC (
    MoC(Fun, Ret, (-.-), (-*-), (-*), (-<-), (-&-)),  delay,  comb22, 
    reconfig22,  state22,  stated22,  moore22,  mealy22,  ctxt22,  warg,  wres, 
    (-*<)
  ) where\end{verbatim}}
\haddockendheader

This module exports the core entities of the MoC layer: interfaces
 for atoms and process constructors as patterns of atoms. It does
 \emph{NOT} export any implementation or instantiation of any specific
 MoC.\par
Current MoC implementations can be used by importing their
 respective modules:\par
\begin{itemize}
\item
\haddocktt{ForSyDe.Atom.MoC.CT}\par

\item
\haddocktt{ForSyDe.Atom.MoC.DE}\par

\item
\haddocktt{ForSyDe.Atom.MoC.SY}\par

\item
\haddocktt{ForSyDe.Atom.MoC.SDF}\par

\end{itemize}
\begin{mdframed}[style=reminder,frametitle=Reminder]Make sure to consult naming conventions in  \cref{sec:forsyde-atom:naming-convention} in order to interpret the names and type signatures correctly.\end{mdframed}\par

\subsection{Atoms}
\begin{haddockdesc}
\item[\begin{tabular}{@{}l}
class\ Applicative\ e\ =>\ MoC\ e\ where
\end{tabular}]\haddockbegindoc
This is a type class defining interfaces for the MoC layer
 atoms. Each model of computation exposes its tag system through a
 unique event constructor as an instance of this class, which
 defines \emph{T} × \emph{V}.\par
 To express all possible MoCs which can be described using
 a \emph{tagged} \emph{signal} \emph{model} we need to capture the most general
 form of their atoms. Recall that all atoms in the layered framework
 are represented as higher-order functions on structured types
 (instances of this class), taking functions of other (lower) layers
 as arguments. While this principle stands also for this layer, the
 functions taken as arguments need to be formatted for each MoC in
 particular in order to capture additional information, which we can
 call in general terms as the \emph{execution context}.\par
One typical example of additional information is the consumption
 and production rates of for data flow MoCs (e.g. SDF). In this case
 the passed functions are defined over "partitions" of events,
 i.e. groupings of events with the same partial order in relation
 to, for example, a process firing. The formal description of such a
 "formatted function" taken as argument by a MoC entity is:\par
\haddockeq{fig/eqs-moc-atom-formatted-func.pdf}\par
where \emph{a} and \emph{b} might be Cartesian products of different types,
 corresponding to how many signals the constructor is applied to or
 how many signals it yields, and each type is expressed as:\par
\haddockeq{fig/eqs-moc-atom-formatted-arg.pdf}\par
While, as you can see above, the execution context can be extracted
 from the type information, working with type-level parameters is
 not a trivial task in Haskell, especially if we want to describe a
 general and extensible type class. This is why we have chosen a
 pragmatic approach in implementing the \haddockid{MoC} class:\par
\begin{itemize}
\item
any (possible) Cartesian product of \emph{α} is represented using
 a recursive type, namely a list {\char 91}\emph{α}{\char 93}.\par

\item
as the execution context cannot (or can hardly) be extracted from
 the recursive type, in the most general case we pass both context
 \emph{and} argument as a pair (see each instance in particular). To aid
 in pairing contexts with each argument in a function, the \haddocktt{ctxt}
 utilities are provided (see \haddockid{ctxt22}).\par

\item
this artifice was masked using the generic type families \haddockid{Fun}
 and \haddocktt{Res}. \par

\end{itemize}

\haddockpremethods{}\textbf{Methods}
\begin{haddockdesc}
\item[\begin{tabular}{@{}l}\ \haddockid{Fun}\ e\ a\ b\ Source\ \end{tabular}]
\haddockbegindoc
This is a type family alias for a context-bound function passed as an argument to a MoC atom. In the most simple case it can be regarded as an enhanced -> type operator. While hiding the explicit definition of arguments, this implementation choice certainly has its advantages in avoiding unnecessary or redundant type constructors (see version 0.1.1 and prior). Aliases are replaced at compile time, thus not affecting run-time performance.\haddockeq{fig/eqs-moc-atom-function.pdf}\par

\item[\begin{tabular}{@{}l}\ \haddockid{Ret}\ e\ b\ Source\ \end{tabular}]
\haddockbegindoc
As with \haddockid{Fun}, this alias hides a context-bound value (e.g. function return). Although the definition seems to be redundant with \haddockid{Fun}, this alias is needed for utilities to recreate clean types again (see \haddockid{-*}).\haddockeq{fig/eqs-moc-atom-result.pdf}\par

\item[\begin{tabular}{@{}l}\haddockid{(-.-)}\ ::\ Fun\ e\ a\ b\ ->\ Stream\ (e\ a)\ ->\ Stream\ (e\ b)\ Source\ \end{tabular}]
\haddockbegindoc
This atom is mapping a function on values (in the presence of a context) to a signal, i.e. stream of tagged events. As ForSyDe deals with \emph{determinate}, \emph{functional} processes, this atom defines the (only) \emph{behavior} of a process in rapport to one input signal  \cite{Lee98}.\haddockeq{fig/eqs-moc-atom-dot.pdf}\par

\item[\begin{tabular}{@{}l}\haddockid{(-*-)}\ ::\ Stream\ (e\ (Fun\ e\ a\ b))\ ->\ Stream\ (e\ a)\ ->\ Stream\ (e\ b)\ Source\ \end{tabular}]
\haddockbegindoc
This atom synchronizes two signals, one carrying functions on values (in the presence of a context), and the other containing values, during which it applies the former on the latter. As concerning the process created, this atom defines a \emph{relation} between two signals  \cite{Lee98}.\haddockeq{fig/eqs-moc-atom-star.pdf}\par

\item[\begin{tabular}{@{}l}\haddockid{(-*)}\ ::\ Stream\ (e\ (Ret\ e\ b))\ ->\ Stream\ (e\ b)\ Source\ \end{tabular}]
\haddockbegindoc
Artificial \emph{utility} which drops the context and/or partitioning yielding a clean signal type.\haddockeq{fig/eqs-moc-atom-post.pdf}\par

\item[\begin{tabular}{@{}l}\haddockid{(-<-)}\ ::\ Stream\ (e\ a)\ ->\ Stream\ (e\ a)\ ->\ Stream\ (e\ a)\ Source\ \end{tabular}]
\haddockbegindoc
This atom appends a (partition of) events at the beginning of a signal. This atom is necessary to ensure \emph{complete partial order} of a signal and assures the \emph{least upper bound} necessary for example in the evaluation of feedback loops  \cite{Lee98}.\haddockeq{fig/eqs-moc-atom-pre.pdf}Notice the difference between the formal and the implemented type signatures. In the implementation the value/partition is wrapped inside an event type to enable smooth composition. You might also notice the type for the initial event(s) as being wrapped inside a signal constructor. This allows defining an DSL for this layer which is centered around signals exclusively, while also enabling to define atoms as homomorphisms to certain extent  \cite{Bird97}. Certain MoCs might have additional constraints on the first operand to be finite.\par

\item[\begin{tabular}{@{}l}\haddockid{(-\&-)}\ ::\ Stream\ (e\ a)\ ->\ Stream\ (e\ a)\ ->\ Stream\ (e\ a)\ Source\ \end{tabular}]
\haddockbegindoc
This atom allows the manipulation of tags in a signal in a restrictive way which preserves \emph{monotonicity} and \emph{continuity} in a process  \cite{Lee98}, namely by phase-shifting all tags in a signal with the appropriate metric corresponding to each MoC. Thus it preserves the characteristic function intact  \cite{Sander04}.\haddockeq{fig/eqs-moc-atom-phi.pdf}As with the \haddockid{-<-} atom, we can justify the type signature for smooth composition and the definition of atoms as homomorphisms to certain extent. This in turn allows the interpretation of the \haddockid{-\haddockpremethods{}\textbf{Methods}-} operator as aligning the phases of two signals: the second operand is aligned based on the first.\par

\end{haddockdesc}


\item[\begin{tabular}{@{}l}
instance\ MoC\ SY
\end{tabular}]\haddockbegindoc
Implenents the execution and synchronization semantics for the SY
 MoC through its atoms.\par


\item[\begin{tabular}{@{}l}
instance\ MoC\ SDF
\end{tabular}]\haddockbegindoc
Implenents the SDF semantics for the MoC atoms\par


\item[\begin{tabular}{@{}l}
instance\ MoC\ DE
\end{tabular}]\haddockbegindoc
Implenents the execution and synchronization semantics for the DE
 MoC through its atoms.\par


\item[\begin{tabular}{@{}l}
instance\ MoC\ CT
\end{tabular}]\haddockbegindoc
Implenents the execution and synchronization semantics for the CT
 MoC through its atoms.\par

\end{haddockdesc}
\subsection{Process constructors}
Process constructors are defined as patterns of MoC
 atoms. Check the \hyperref[naming_conv]{naming convention} of the API in
 the page description.\par

\begin{haddockdesc}
\item[\begin{tabular}{@{}l}
delay\ ::\ MoC\ e\ =>\ Stream\ (e\ a)\ ->\ Stream\ (e\ a)\ ->\ Stream\ (e\ a)
\end{tabular}]\haddockbegindoc
\haddockeq{fig/eqs-moc-pattern-delay.pdf}
   \haddockfig{fig/moc-pattern-delay.pdf}\par
The \haddockid{delay} process provides both initial token(s) and shifts the
 phase of the signal. In other words, it "delays" a signal with
 one or several events.\par
There is also an infix variant \haddockid{-{\char '46}>-} (\haddocktt{infixl\ 3}). To justify the
 first argument, see the documentation of the \haddockid{-<-} atom.\par


\item[\begin{tabular}{@{}l}
comb22
\end{tabular}]\haddockbegindoc
\haddockbeginargs
\haddockdecltt{::} & MoC e \\
                     \haddockdecltt{=>} & \haddockdecltt{Fun e a1 (Fun e a2 (Ret e b1, Ret e b2))} & combinational function (\hyperref[comb22f]{*}) \\
                                                                                                     \haddockdecltt{->} & \haddockdecltt{Stream (e a1)} & first input signal \\
                                                                                                                                                          \haddockdecltt{->} & \haddockdecltt{Stream (e a2)} & second input signal \\
                                                                                                                                                                                                               \haddockdecltt{->} & \haddockdecltt{(Stream (e b1), Stream (e b2))} & two output signals \\
\end{tabulary}\par
 \emph{(*) to be read } \haddocktt{a1\ ->\ a2\ ->\ (b1,\ b2)} \emph{where each}
 \emph{argument and result might be individually wrapped with a context}
 \emph{and might also express a partition.}\par
\haddockeq{fig/eqs-moc-pattern-comb.pdf}
 \haddockfig{fig/moc-pattern-comb.pdf}\par
The \haddocktt{comb} processes takes care of synchronization between signals
 and maps combinatorial functions on their event values. \par
This library exports constructors of type \haddocktt{comb{\char 91}1-8{\char 93}{\char 91}1-4{\char 93}}.\par


\item[\begin{tabular}{@{}l}
reconfig22
\end{tabular}]\haddockbegindoc
\haddockbeginargs
\haddockdecltt{::} & MoC e \\
                     \haddockdecltt{=>} & \haddockdecltt{Stream (e (Fun e a1 (Fun e a2 (Ret e b1, Ret e b2))))} & signal carrying functions (\hyperref[reconfig22f]{*}) \\
                                                                                                                  \haddockdecltt{->} & \haddockdecltt{Stream (e a1)} & first input signal \\
                                                                                                                                                                       \haddockdecltt{->} & \haddockdecltt{Stream (e a2)} & second input signal \\
                                                                                                                                                                                                                            \haddockdecltt{->} & \haddockdecltt{(Stream (e b1), Stream (e b2))} & two output signals \\
\end{tabulary}\par
 \emph{(*) to be read } \haddocktt{a1\ ->\ a2\ ->\ (b1,\ b2)} \emph{where each}
 \emph{argument and result might be individually wrapped with a context}
 \emph{and might also express a partition.}\par
\haddockeq{fig/eqs-moc-pattern-reconfig.pdf}
 \haddockfig{fig/moc-pattern-reconfig.pdf}\par
The \haddocktt{reconfig} processes constructs adaptive processes, where the
 first signal carries functions, and it is synchronized with all the
 other signals. \par
This library exports constructors of type \haddocktt{reconfig{\char 91}1-8{\char 93}{\char 91}1-4{\char 93}}.\par


\item[\begin{tabular}{@{}l}
state22
\end{tabular}]\haddockbegindoc
\haddockbeginargs
\haddockdecltt{::} & MoC e \\
                     \haddockdecltt{=>} & \haddockdecltt{Fun e st1 (Fun e st2 (Fun e a1 (Fun e a2 (Ret e st1, Ret e st2))))} & next state function (\hyperref[state22ns]{*}) \\
                                                                                                                               \haddockdecltt{->} & \haddockdecltt{(Stream (e st1), Stream (e st2))} & initial state(s) (\hyperref[state22i]{**}) \\
                                                                                                                                                                                                       \haddockdecltt{->} & \haddockdecltt{Stream (e a1)} & first input signal \\
                                                                                                                                                                                                                                                            \haddockdecltt{->} & \haddockdecltt{Stream (e a2)} & second input signal \\
                                                                                                                                                                                                                                                                                                                 \haddockdecltt{->} & \haddockdecltt{(Stream (e st1), Stream (e st2))} & output signals mirroring the next state(s). \\
\end{tabulary}\par
 \emph{(*) meaning } \haddocktt{st1\ ->\ st2\ ->\ a1\ ->\ a2\ ->\ (st1,st2)}
 \emph{where each argument and result might be individually wrapped}
 \emph{with a context and might also express a partition.}\par
 \emph{(**) see the documentation for \haddockid{-<-} for justification}
 \emph{of the type}\par
\haddockeq{fig/eqs-moc-pattern-state.pdf}
 \haddockfig{fig/moc-pattern-state.pdf}\par
The \haddocktt{state} processes generate process networks corresponding to a
 simple state machine like in the graph above. \par
This library exports constructors of type \haddocktt{state{\char 91}1-4{\char 93}{\char 91}1-4{\char 93}}.\par


\item[\begin{tabular}{@{}l}
stated22
\end{tabular}]\haddockbegindoc
\haddockbeginargs
\haddockdecltt{::} & MoC e \\
                     \haddockdecltt{=>} & \haddockdecltt{Fun e st1 (Fun e st2 (Fun e a1 (Fun e a2 (Ret e st1, Ret e st2))))} & next state function (\hyperref[stated22ns]{*}) \\
                                                                                                                               \haddockdecltt{->} & \haddockdecltt{(Stream (e st1), Stream (e st2))} & initial state(s) (\hyperref[stated22i]{**}) \\
                                                                                                                                                                                                       \haddockdecltt{->} & \haddockdecltt{Stream (e a1)} & first input signal \\
                                                                                                                                                                                                                                                            \haddockdecltt{->} & \haddockdecltt{Stream (e a2)} & second input signal \\
                                                                                                                                                                                                                                                                                                                 \haddockdecltt{->} & \haddockdecltt{(Stream (e st1), Stream (e st2))} & output signals mirroring the next state(s). \\
\end{tabulary}\par
 \emph{(*) meaning } \haddocktt{st1\ ->\ st2\ ->\ a1\ ->\ a2\ ->\ (st1,st2)}
 \emph{where each argument and result might be individually wrapped}
 \emph{with a context and might also express a partition.}\par
 \emph{(**) see the documentation for \haddockid{-<-} for justification}
 \emph{of the type}\par
\haddockeq{fig/eqs-moc-pattern-stated.pdf}
 \haddockfig{fig/moc-pattern-stated.pdf}\par
The \haddocktt{state} processes generate process networks corresponding to a
 simple state machine like in the graph above. The difference
 between \haddockid{state22} and \haddockid{stated22} is that the latter outputs the
 current state rather than the next one. There exists a variant with
 0 input signals, in which case the process is a signal
 generator.\par
This library exports constructors of type \haddocktt{stated{\char 91}0-4{\char 93}{\char 91}1-4{\char 93}}.\par


\item[\begin{tabular}{@{}l}
moore22
\end{tabular}]\haddockbegindoc
\haddockbeginargs
\haddockdecltt{::} & MoC e \\
                     \haddockdecltt{=>} & \haddockdecltt{Fun e st (Fun e a1 (Fun e a2 (Ret e st)))} & next state function (\hyperref[moore22ns]{*}) \\
                                                                                                      \haddockdecltt{->} & \haddockdecltt{Fun e st (Ret e b1, Ret e b2)} & output decoder (\hyperref[moore22od]{**}) \\
                                                                                                                                                                           \haddockdecltt{->} & \haddockdecltt{Stream (e st)} & initial state (\hyperref[moore22i]{***}) \\
                                                                                                                                                                                                                                \haddockdecltt{->} & \haddockdecltt{Stream (e a1)} & first input signal \\
                                                                                                                                                                                                                                                                                     \haddockdecltt{->} & \haddockdecltt{Stream (e a2)} & second input signal \\
                                                                                                                                                                                                                                                                                                                                          \haddockdecltt{->} & \haddockdecltt{(Stream (e b1), Stream (e b2))} & output signals \\
\end{tabulary}\par
 \emph{(*) meaning } \haddocktt{st\ ->\ a1\ ->\ a2\ ->\ st\ } \emph{where each}
 \emph{argument and result might be individually wrapped with a context}
 \emph{and might also express a partition.}\par
 \emph{(**) meaning } \haddocktt{st\ ->\ (b1,\ b2)\ } \emph{where each argument}
 \emph{and result might be individually wrapped with a context and might}
 \emph{also express a partition.}\par
 \emph{(***) see the documentation for \haddockid{-<-} for justification}
 \emph{of the type}\par
\haddockeq{fig/eqs-moc-pattern-moore.pdf}
 \haddockfig{fig/moc-pattern-moore.pdf}\par
The \haddocktt{moore} processes model Moore state machines.\par
This library exports constructors of type \haddocktt{moore{\char 91}1-4{\char 93}{\char 91}1-4{\char 93}}.\par


\item[\begin{tabular}{@{}l}
mealy22
\end{tabular}]\haddockbegindoc
\haddockbeginargs
\haddockdecltt{::} & MoC e \\
                     \haddockdecltt{=>} & \haddockdecltt{Fun e st (Fun e a1 (Fun e a2 (Ret e st)))} & next state function (\hyperref[mealy22ns]{*}) \\
                                                                                                      \haddockdecltt{->} & \haddockdecltt{Fun e st (Fun e a1 (Fun e a2 (Ret e b1, Ret e b2)))} & output decoder (\hyperref[mealy22od]{**}) \\
                                                                                                                                                                                                 \haddockdecltt{->} & \haddockdecltt{Stream (e st)} & initial state (\hyperref[mealy22i]{***}) \\
                                                                                                                                                                                                                                                      \haddockdecltt{->} & \haddockdecltt{Stream (e a1)} & first input signal \\
                                                                                                                                                                                                                                                                                                           \haddockdecltt{->} & \haddockdecltt{Stream (e a2)} & second input signal \\
                                                                                                                                                                                                                                                                                                                                                                \haddockdecltt{->} & \haddockdecltt{(Stream (e b1), Stream (e b2))} & output signals \\
\end{tabulary}\par
 \emph{(*) meaning } \haddocktt{st\ ->\ a1\ ->\ a2\ ->\ st\ } \emph{where each}
 \emph{argument and result might be individually wrapped with a context}
 \emph{and might also express a partition.}\par
 \emph{(**) meaning } \haddocktt{st\ ->\ a1\ ->\ a2\ ->\ (b1,\ b2)\ } \emph{where}
 \emph{each argument and result might be individually wrapped with a}
 \emph{context and might also express a partition.}\par
 \emph{(***) see the documentation for \haddockid{-<-} for justification}
 \emph{of the type}\par
\haddockeq{fig/eqs-moc-pattern-mealy.pdf}
 \haddockfig{fig/moc-pattern-mealy.pdf}\par
The \haddocktt{mealy} processes model Mealy state machines.\par
This library exports constructors of type \haddocktt{mealy{\char 91}1-4{\char 93}{\char 91}1-4{\char 93}}.\par

\end{haddockdesc}
\subsection{Utilities}
\begin{haddockdesc}
\item[\begin{tabular}{@{}l}
ctxt22
\end{tabular}]\haddockbegindoc
\haddockbeginargs
\haddockdecltt{::} & \haddockdecltt{(ctx, ctx)} & argument contexts (e.g. consumption rates in SDF) \\
                                                  \haddockdecltt{->} & \haddockdecltt{(ctx, ctx)} & result contexts (e.g. production rates in SDF) \\
                                                                                                    \haddockdecltt{->} & \haddockdecltt{(a1
                                                                                                                                         -> a2
                                                                                                                                            -> (b1, b2))} & function on values/partitions of values \\
                                                                                                                                                            \haddockdecltt{->} & \haddockdecltt{(ctx, a1
                                                                                                                                                                                                      -> (ctx, a2
                                                                                                                                                                                                               -> ((ctx, b1), (ctx, b2))))} & context-wrapped form of the previous function \\
\end{tabulary}\par
\haddockeq{fig/eqs-moc-atom-context.pdf}\par
Wraps a function with the context needed by some MoCs for their
 constructors (e.g. rates in SDF).\par
This library exports wrappers of type \haddocktt{ctxt{\char 91}1-8{\char 93}{\char 91}1-4{\char 93}}.\par


\item[\begin{tabular}{@{}l}
warg\ ::\ c\ ->\ (a\ ->\ b)\ ->\ (c,\ a\ ->\ b)
\end{tabular}]\haddockbegindoc
Attaches a context parameter to a function argument (e.g
 consumption rates in SDF). Used as kernel function in defining
 e.g. \haddockid{ctxt22}.\par


\item[\begin{tabular}{@{}l}
wres\ ::\ p\ ->\ b\ ->\ (p,\ b)
\end{tabular}]\haddockbegindoc
Attaches a context parameter to a function's result (e.g
 production rates in SDF). Used as kernel function in defining
 e.g. \haddockid{ctxt22}.\par


\item[\begin{tabular}{@{}l}
(-*<)\ ::\ MoC\ e\ =>\\\ \ \ \ \ \ \ \ \ Stream\ (e\ (Ret\ e\ b1,\ Ret\ e\ b2))\ ->\ (Stream\ (e\ b1),\ Stream\ (e\ b2))
\end{tabular}]\haddockbegindoc
Utilities for extending the \haddockid{-*} atom for dealing with tupled
 outputs. This library exports operators of form \haddocktt{-*<{\char '173}1,8{\char '175}}.\par

\end{haddockdesc}
  \input{input/ForSyDe-Atom-MoC-Stream}
  \input{input/ForSyDe-Atom-MoC-SY}
  \input{input/ForSyDe-Atom-MoC-DE}
  \input{input/ForSyDe-Atom-MoC-CT}
  \haddockmoduleheading{ForSyDe.Atom.MoC.SDF}
\label{module:ForSyDe.Atom.MoC.SDF}
\haddockbeginheader
{\haddockverb\begin{verbatim}
module ForSyDe.Atom.MoC.SDF (
    SDF(SDF, val),  Signal,  Prod,  Cons,  signal,  readSignal,  delay, 
    delay',  comb22,  reconfig22,  constant2,  generate2,  stated22,  state22, 
    moore22,  mealy22,  toSY2,  zipx,  unzipx
  ) where\end{verbatim}}
\haddockendheader

The \haddocktt{SDF} library implements the atoms holding the sematics for the
 synchronous data flow computation model. It also provides a set of
 helpers for properly instantiating process network patterns as
 process constructors.\par
\begin{mdframed}[style=reminder,frametitle=Reminder]Make sure to consult naming conventions in  \cref{sec:forsyde-atom:naming-convention} in order to interpret the names and type signatures correctly.\end{mdframed}\par

\subsection{Synchronous data flow (\haddocktt{SDF}) event}
The synchronous data flow (\haddocktt{SDF}) MoC is the first untimed MoC
 implemented by the \haddocktt{forsyde-atom} framework. On untimed MoCs,
 \cite{Lee98} states that: "when tags are
 partially ordered rather than totally ordered, we say that the
 system is untimed. Untimed systems cannot have the same notion of
 causality as timed systems {\char 91}see \haddockid{SY}{\char 93}. (...)
 Processes defined in terms of constraints on the tags in the
 signals (...) have a \emph{consistent cut} rather than
 \emph{simultaneity}."  Regarding SDF, it states that "is a special
 case of Kahn process networks
 \cite{Kahn76}. A dataflow process is a Kahn
 process that is also sequential, where the events on the
 self-loop signal denote the firings of the dataflow actor. The
 firing rules of a dataflow actor are partial ordering constraints
 between these events and events on the inputs. (...)
 Produced/consumed events are defined in terms of relations with
 the events in the firing signal. It results that for the same
 firing \emph{i}, eᵢ < eₒ, as an intuitive sort of
 causality constraint."\par
Based on the above insights, we can formulate a simplified
 definition of the \haddocktt{forsyde-atom} interpretation of SDF:\par
\begin{description}
\item[The SDF MoC] is abstracting the execution semantics of a system
 where computation is performed according to firing rules where
 the production and the consumption rates are fixed.
\end{description}Below is a \emph{possible} behavior in time of the input and the
 output signals of a SDF process. Events sharing the same partial
 ordering in relation to one firing are overlined:\par
                 \haddockfig{fig/moc-sdf-example.pdf}\par
                 Implementing the SDF tag system implied a series of engineering
 decisions which lead to the following particularities:\par
                 \begin{enumerate}
                 \item 
                 signals represent FIFO channels, and tags are implicit from
 their position in the \haddockid{Stream}
 structure. Internally, \haddockid{SDF} signals have
 exactly the same structure as \haddockid{SY} signals,
 whereas the partial ordering is imposed by the processes alone.\par
                 
                 \item 
                 the \haddockid{SDF} event constructor wraps only a
 value.\par
                 
                 \item 
                 being an \emph{untimed MoC}, the order between events is partial to
 the firings of processes. An SDF atom will fire only when there
 are enough events to trigger its inputs. Once a firing occurs, it
 will take care of partitioning the input or output signals.\par
                 
                 \item 
                 SDF atoms \emph{do} require a context: the consumption \emph{c} and
 production \emph{p} rates. Also, the functions passed as arguments
 reflect the fact that multiple events are handled during a
 firing.\par
                 
                 \item 
                 the previous statement can be synthesized into the following
 execution context (see \cref{module:ForSyDe.Atom.MoC}), which also
 justifies the SDF implementation of \haddockid{Fun} and
 for \haddockid{Ret}:\par
                 
                 \end{enumerate}
                 \haddockeq{fig/eqs-moc-sdf-context.pdf}\par
                 
\begin{haddockdesc}
\item[\begin{tabular}{@{}l}
newtype\ SDF\ a
\end{tabular}]\haddockbegindoc
\haddockbeginconstrs
\haddockdecltt{=} & \haddockdecltt{SDF} & \\
                    \haddockdecltt{>}&\haddockdecltt{val :: a} &\\
\end{tabulary}\par

The CT type, identifying a discrete time event and implementing an
 instance of the \haddockid{MoC} class. A discrete event explicitates its tag
 which is represented as an integer.\par


\item[\begin{tabular}{@{}l}
instance\ Functor\ SDF
\end{tabular}]\haddockbegindoc
Allows for mapping of functions on a SDF event.\par


\item[\begin{tabular}{@{}l}
instance\ Applicative\ SDF
\end{tabular}]\haddockbegindoc
Allows for lifting functions on a pair of SDF events.\par


\item[\begin{tabular}{@{}l}
instance\ Foldable\ SDF\\instance\ Traversable\ SDF
\end{tabular}]

\item[\begin{tabular}{@{}l}
instance\ MoC\ SDF
\end{tabular}]\haddockbegindoc
Implenents the SDF semantics for the MoC atoms\par


\item[\begin{tabular}{@{}l}
instance\ Read\ a\ =>\ Read\ (SDF\ a)
\end{tabular}]\haddockbegindoc
Reads the value wrapped\par


\item[\begin{tabular}{@{}l}
instance\ Show\ a\ =>\ Show\ (SDF\ a)
\end{tabular}]\haddockbegindoc
Shows the value wrapped\par


\item[\begin{tabular}{@{}l}
instance\ Plottable\ a\ =>\ Plot\ (Signal\ a)
\end{tabular}]\haddockbegindoc
\haddockid{SDF} signals.\par


\item[\begin{tabular}{@{}l}
instance\ type\ Ret\ SDF\ a\ =\ (Prod,\ {\char 91}a{\char 93})\\instance\ type\ Fun\ SDF\ a\ b\ =\ (Cons,\ {\char 91}a{\char 93}\ ->\ b)
\end{tabular}]
\end{haddockdesc}
\subsection{Aliases {\char '46} utilities}
A set of type synonyms and utilities are provided for
 convenience. The API type signatures will feature these aliases
 to hide the cumbersome construction of atoms and patters as seen
 in \haddocktt{ForSyDe.Atom.MoC}.\par

\begin{haddockdesc}
\item[\begin{tabular}{@{}l}
type\ Signal\ a\ =\ Stream\ (SDF\ a)
\end{tabular}]\haddockbegindoc
Type synonym for a SY signal, i.e. "a signal of SY events"\par


\item[\begin{tabular}{@{}l}
type\ Prod\ =\ Int
\end{tabular}]\haddockbegindoc
Type synonym for consumption rate\par


\item[\begin{tabular}{@{}l}
type\ Cons\ =\ Int
\end{tabular}]\haddockbegindoc
Type synonym for production rate\par


\item[\begin{tabular}{@{}l}
signal\ ::\ {\char 91}a{\char 93}\ ->\ Signal\ a
\end{tabular}]\haddockbegindoc
Transforms a list of values into a SDF signal with only one
 partition, i.e. all events share the same (initial) tag.\par


\item[\begin{tabular}{@{}l}
readSignal\ ::\ Read\ a\ =>\ String\ ->\ Signal\ a
\end{tabular}]\haddockbegindoc
Reads a signal from a string. Like with the \haddocktt{read} function from
 \haddocktt{Prelude}, you must specify the tipe of the signal.\par
\begin{interactive}
λ> readSignal "{1,2,3,4,5}" :: Signal Int
{1,2,3,4,5}

\end{interactive}
\end{haddockdesc}
These SY process constructors are basically specific
 instantiations of the patterns of atoms defined in
 \haddocktt{ForSyDe.Atom.MoC}. Some are also wrapping functions in an
 extended behavioural model.\par

\subsubsection{Simple}
\begin{haddockdesc}
\item[\begin{tabular}{@{}l}
delay
\end{tabular}]\haddockbegindoc
\haddockbeginargs
\haddockdecltt{::} & \haddockdecltt{[a]} & list of initial values \\
                                           \haddockdecltt{->} & \haddockdecltt{Signal a} & input signal \\
                                                                                           \haddockdecltt{->} & \haddockdecltt{Signal a} & output signal \\
\end{tabulary}\par
The \haddocktt{delay} process "delays" a signal with initial events built
 from a list. It is an instantiation of the \haddockid{delay}
 constructor.\par
\begin{interactive}
λ> let s = signal [1,2,3,4,5]
λ> delay [0,0,0] s
{0,0,0,1,2,3,4,5}

\end{interactive}\haddockfig{fig/moc-sdf-pattern-delay.pdf}\par
           

\item[\begin{tabular}{@{}l}
delay'
\end{tabular}]\haddockbegindoc
\haddockbeginargs
\haddockdecltt{::} & \haddockdecltt{Signal a} & signal containing the initial tokens \\
                                                \haddockdecltt{->} & \haddockdecltt{Signal a} & input signal \\
                                                                                                \haddockdecltt{->} & \haddockdecltt{Signal a} & output signal \\
\end{tabulary}\par
Similar to the previous, but this is the raw instantiation of the
 \haddockid{delay} pattern. It appends the contents of one
 signal at the head of another signal.\par
\begin{interactive}
λ> let s1 = signal [0,0,0]
λ> let s2 = signal [1,2,3,4,5]
λ> delay' s1 s2
{0,0,0,1,2,3,4,5}

\end{interactive}\haddockfig{fig/moc-sdf-pattern-delayp.pdf}\par
           

\item[\begin{tabular}{@{}l}
comb22
\end{tabular}]\haddockbegindoc
\haddockbeginargs
\haddockdecltt{::} & \haddockdecltt{((Cons, Cons), (Prod, Prod), [a1]
                                                                 -> [a2]
                                                                    -> ([b1], [b2]))} & function on lists of values, tupled with consumption /
 production rates \\
                                                                                        \haddockdecltt{->} & \haddockdecltt{Signal a1} & first input signal \\
                                                                                                                                         \haddockdecltt{->} & \haddockdecltt{Signal a2} & second input signal \\
                                                                                                                                                                                          \haddockdecltt{->} & \haddockdecltt{(Signal b1, Signal b2)} & two output signals \\
\end{tabulary}\par
\haddocktt{comb} processes map combinatorial functions on signals and take
 care of synchronization between input signals. It instantiates the
 \haddocktt{comb} atom pattern (see \haddockid{comb22}).\par
Constructors: \haddocktt{comb{\char 91}1-4{\char 93}{\char 91}1-4{\char 93}}.\par
\begin{interactive}
λ> let s1 = signal [1..]
λ> let s2 = signal [1,1,1,1,1,1,1]
λ> let f [a,b,c] [d,e] = [a+d, c+e]
λ> comb21 ((3,2),2,f) s1 s2
{2,4,5,7,8,10}

\end{interactive}Incorrect usage (not covered by \haddocktt{doctest}):\par
           \begin{quote}
           {\haddockverb\begin{verbatim}
           λ> comb21 ((3,2),3,f) s1 s2
*** Exception: [MoC.SDF] Wrong production\end{verbatim}}
           \end{quote}
           \haddockfig{fig/moc-sdf-pattern-comb.pdf}\par
           

\item[\begin{tabular}{@{}l}
reconfig22
\end{tabular}]\haddockbegindoc
\haddockbeginargs
\haddockdecltt{::} & \haddockdecltt{((Cons, Cons), (Prod, Prod))} & \\
                                                                    \haddockdecltt{->} & \haddockdecltt{Signal ([a1]
                                                                                                                -> [a2]
                                                                                                                   -> ([b1], [b2]))} & function on lists of values, tupled with consumption /
 production rates \\
                                                                                                                                       \haddockdecltt{->} & \haddockdecltt{Signal a1} & first input signal \\
                                                                                                                                                                                        \haddockdecltt{->} & \haddockdecltt{Signal a2} & second input signal \\
                                                                                                                                                                                                                                         \haddockdecltt{->} & \haddockdecltt{(Signal b1, Signal b2)} & two output signals \\
\end{tabulary}\par
\haddocktt{reconfig} creates an SDF adaptive process where the first signal
 carries functions and the other carry the arguments. It
 instantiates the \haddocktt{reconfig} atom pattern (see
 \haddockid{reconfig22}). According to our SDF definition,
 the production and consumption rates need to be fixed, so they are
 passed as parameters to the constructor, whereas the first signal
 carries adaptive functions only. For the adaptive signal it only
 makes sense that the consumption rate is always 1.\par
Constructors: \haddocktt{reconfig{\char 91}1-4{\char 93}{\char 91}1-4{\char 93}}.\par
\begin{interactive}
λ> let f1 a = [sum a]
λ> let f2 a = [maximum a]
λ> let sf = signal [f1,f2,f1,f2,f1,f2,f1]
λ> let s1 = signal [1..]
λ> reconfig11 (4,1) sf s1
{10,8,42,16,74,24,106}

\end{interactive}\haddockfig{fig/moc-sdf-pattern-reconfig.pdf}\par
           

\item[\begin{tabular}{@{}l}
constant2
\end{tabular}]\haddockbegindoc
\haddockbeginargs
\haddockdecltt{::} & \haddockdecltt{([b1], [b2])} & values to be repeated \\
                                                    \haddockdecltt{->} & \haddockdecltt{(Signal b1, Signal b2)} & generated signals \\
\end{tabulary}\par
A signal generator which repeats the initial tokens
 indefinitely. It is actually an instantiation of the \haddocktt{stated0X}
 constructor (check \haddockid{stated22}).\par
Constructors: \haddocktt{constant{\char 91}1-4{\char 93}}.\par
\begin{interactive}
λ> let (s1, s2) = constant2 ([1,2,3],[2,1])
λ> takeS 7 s1
{1,2,3,1,2,3,1}
λ> takeS 5 s2
{2,1,2,1,2}

\end{interactive}\haddockfig{fig/moc-sdf-pattern-constant.pdf}\par
           

\item[\begin{tabular}{@{}l}
generate2
\end{tabular}]\haddockbegindoc
\haddockbeginargs
\haddockdecltt{::} & \haddockdecltt{((Cons, Cons), (Prod, Prod), [b1]
                                                                 -> [b2]
                                                                    -> ([b1], [b2]))} & function to generate next value, tupled with
 consumption / production rates \\
                                                                                        \haddockdecltt{->} & \haddockdecltt{([b1], [b2])} & values of initial tokens \\
                                                                                                                                            \haddockdecltt{->} & \haddockdecltt{(Signal b1, Signal b2)} & generated signals \\
\end{tabulary}\par
A signal generator based on a function and a kernel value. It
 is actually an instantiation of the \haddocktt{stated0X} constructor
 (check \haddockid{stated22}).\par
Constructors: \haddocktt{generate{\char 91}1-4{\char 93}}.\par
\begin{interactive}
λ> let f a b = ([sum a, sum a],[sum b, sum b, sum b])
λ> let (s1,s2) = generate2 ((2,3),(2,3),f) ([1,1],[2,2,2])
λ> takeS 7 s1
{1,1,2,2,4,4,8}
λ> takeS 8 s2
{2,2,2,6,6,6,18,18}

\end{interactive}\haddockfig{fig/moc-sdf-pattern-generate.pdf}\par
           

\item[\begin{tabular}{@{}l}
stated22
\end{tabular}]\haddockbegindoc
\haddockbeginargs
\haddockdecltt{::} & \haddockdecltt{((Cons, Cons, Cons, Cons), (Prod, Prod), [b1]
                                                                             -> [b2]
                                                                                -> [a1]
                                                                                   -> [a2]
                                                                                      -> ([b1], [b2]))} & next state function, tupled with
 consumption / production rates \\
                                                                                                          \haddockdecltt{->} & \haddockdecltt{([b1], [b2])} & initial state partitions of values \\
                                                                                                                                                              \haddockdecltt{->} & \haddockdecltt{Signal a1} & first input signal \\
                                                                                                                                                                                                               \haddockdecltt{->} & \haddockdecltt{Signal a2} & second input signal \\
                                                                                                                                                                                                                                                                \haddockdecltt{->} & \haddockdecltt{(Signal b1, Signal b2)} & output signals \\
\end{tabulary}\par
\haddocktt{stated} is a state machine without an output decoder. It is an
 instantiation of the \haddocktt{state} MoC constructor (see
 \haddockid{stated22}).\par
Constructors: \haddocktt{stated{\char 91}1-4{\char 93}{\char 91}1-4{\char 93}}.\par
\begin{interactive}
λ> let f [a] [b,c] = [a+b+c]
λ> let s = signal [1,2,3,4,5,6,7]
λ> stated11 ((1,2),1,f) [1] s
{1,4,11,22}

\end{interactive}\haddockfig{fig/moc-sdf-pattern-stated.pdf}\par
           

\item[\begin{tabular}{@{}l}
state22
\end{tabular}]\haddockbegindoc
\haddockbeginargs
\haddockdecltt{::} & \haddockdecltt{((Cons, Cons, Cons, Cons), (Prod, Prod), [b1]
                                                                             -> [b2]
                                                                                -> [a1]
                                                                                   -> [a2]
                                                                                      -> ([b1], [b2]))} & next state function, tupled with consumption /
 production rates \\
                                                                                                          \haddockdecltt{->} & \haddockdecltt{([b1], [b2])} & initial partitions of values \\
                                                                                                                                                              \haddockdecltt{->} & \haddockdecltt{Signal a1} & first input signal \\
                                                                                                                                                                                                               \haddockdecltt{->} & \haddockdecltt{Signal a2} & second input signal \\
                                                                                                                                                                                                                                                                \haddockdecltt{->} & \haddockdecltt{(Signal b1, Signal b2)} & output signals \\
\end{tabulary}\par
\haddocktt{state} is a state machine without an output decoder. It is an
 instantiation of the \haddocktt{stated} MoC constructor (see
 \haddockid{state22}).\par
Constructors: \haddocktt{state{\char 91}1-4{\char 93}{\char 91}1-4{\char 93}}.\par
\begin{interactive}
λ> let f [a] [b,c] = [a+b+c]
λ> let s = signal [1,2,3,4,5,6,7]
λ> state11 ((1,2),1,f) [1] s
{4,11,22}

\end{interactive}\haddockfig{fig/moc-sdf-pattern-state.pdf}\par
           

\item[\begin{tabular}{@{}l}
moore22
\end{tabular}]\haddockbegindoc
\haddockbeginargs
\haddockdecltt{::} & \haddockdecltt{((Cons, Cons, Cons), Prod, [st]
                                                               -> [a1]
                                                                  -> [a2]
                                                                     -> [st])} & next state function, tupled with consumption / production
 rates \\
                                                                                 \haddockdecltt{->} & \haddockdecltt{(Cons, (Prod, Prod), [st]
                                                                                                                                          -> ([b1], [b2]))} & output decoder, tupled with consumption / production
 rates \\
                                                                                                                                                              \haddockdecltt{->} & \haddockdecltt{[st]} & initial state values \\
                                                                                                                                                                                                          \haddockdecltt{->} & \haddockdecltt{Signal a1} & \\
                                                                                                                                                                                                                                                           \haddockdecltt{->} & \haddockdecltt{Signal a2} & \\
                                                                                                                                                                                                                                                                                                            \haddockdecltt{->} & \haddockdecltt{(Signal b1, Signal b2)} & \\
\end{tabulary}\par
\haddocktt{moore} processes model Moore state machines. It is an
 instantiation of the \haddocktt{moore} MoC constructor (see
 \haddockid{moore22}).\par
Constructors: \haddocktt{moore{\char 91}1-4{\char 93}{\char 91}1-4{\char 93}}.\par
\begin{interactive}
λ> let ns [a] [b,c] = [a+b+c]
λ> let od [a]       = [a+1,a*2]
λ> let s = signal [1,2,3,4,5,6,7]
λ> moore11 ((1,2),1,ns) (1,2,od) [1] s
{2,2,5,8,12,22,23,44}

\end{interactive}\haddockfig{fig/moc-sdf-pattern-moore.pdf}\par
           

\item[\begin{tabular}{@{}l}
mealy22
\end{tabular}]\haddockbegindoc
\haddockbeginargs
\haddockdecltt{::} & \haddockdecltt{((Cons, Cons, Cons), Prod, [st]
                                                               -> [a1]
                                                                  -> [a2]
                                                                     -> [st])} & next state function, tupled with consumption / production
 rates \\
                                                                                 \haddockdecltt{->} & \haddockdecltt{((Cons, Cons, Cons), (Prod, Prod), [st]
                                                                                                                                                        -> [a1]
                                                                                                                                                           -> [a2]
                                                                                                                                                              -> ([b1], [b2]))} & outpt decoder, tupled with consumption / production rates \\
                                                                                                                                                                                  \haddockdecltt{->} & \haddockdecltt{[st]} & initial state values \\
                                                                                                                                                                                                                              \haddockdecltt{->} & \haddockdecltt{Signal a1} & \\
                                                                                                                                                                                                                                                                               \haddockdecltt{->} & \haddockdecltt{Signal a2} & \\
                                                                                                                                                                                                                                                                                                                                \haddockdecltt{->} & \haddockdecltt{(Signal b1, Signal b2)} & \\
\end{tabulary}\par
\haddocktt{mealy} processes model Mealy state machines. It is an
 instantiation of the \haddocktt{mealy} MoC constructor
 (see \haddockid{mealy22}).\par
Constructors: \haddocktt{mealy{\char 91}1-4{\char 93}{\char 91}1-4{\char 93}}.\par
\begin{interactive}
λ> let ns [a] [b,c] = [a+b+c]
λ> let od [a] [b]   = [a+b,a*b]
λ> let s = signal [1,2,3,4,5,6,7]
λ> mealy11 ((1,2),1,ns) ((1,1),2,od) [1] s
{2,1,6,8,14,33,26,88}

\end{interactive}\haddockfig{fig/moc-sdf-pattern-mealy.pdf}\par
           
\end{haddockdesc}
\subsubsection{Interfaces}
\begin{haddockdesc}
\item[\begin{tabular}{@{}l}
toSY2\ ::\ Signal\ a\ ->\ Signal\ b\ ->\ (Signal\ a,\ Signal\ b)
\end{tabular}]\haddockbegindoc
Transforms a (set of) \haddockid{SDF} signal(s) into
 the equivalent \haddockid{SY} signal(s). The only change
 is the event consructor. The partial order of DE is interpreted as
 SY's total order, based on the positioning of events in the signals
 (e.g. FIFO buffers) at that moment.\par
Constructors: \haddocktt{toSY{\char 91}1-4{\char 93}}.\par
\begin{interactive}
λ> let s = SDF.signal [1,2,3,4,5]
λ> toSY s
{1,2,3,4,5}

\end{interactive}\haddockfig{fig/moc-sdf-tosy.pdf}\par
           

\item[\begin{tabular}{@{}l}
zipx
\end{tabular}]\haddockbegindoc
\haddockbeginargs
\haddockdecltt{::} & \haddockdecltt{Vector Cons} & consumption rates \\
                                                   \haddockdecltt{->} & \haddockdecltt{Vector (Signal a)} & vector of signals \\
                                                                                                            \haddockdecltt{->} & \haddockdecltt{Signal (Vector a)} & signal of vectors \\
\end{tabulary}\par
Consumes tokens from a vector of signals and merges them into a
 signal of vectors, with a production rate of 1. It instantiates the
 \haddockid{zipx} skeleton.\par
\begin{interactive}
λ> let s1 = SDF.signal [1,2,3,4,5]
λ> let s2 = SDF.signal [11,12,13,14,15]
λ> let v1 = V.vector [s1,s1,s2,s2]
λ> let r  = V.vector [2,1,2,1]
λ> v1
<{1,2,3,4,5},{1,2,3,4,5},{11,12,13,14,15},{11,12,13,14,15}>
λ> zipx r v1
{<1,2,1,11,12,11>,<3,4,2,13,14,12>}

\end{interactive}\haddockfig{fig/moc-sdf-zipx.pdf}\par
           

\item[\begin{tabular}{@{}l}
unzipx
\end{tabular}]\haddockbegindoc
\haddockbeginargs
\haddockdecltt{::} & \haddockdecltt{Vector Prod} & production rates (in reverse order) \\
                                                   \haddockdecltt{->} & \haddockdecltt{Signal (Vector a)} & signal of vectors \\
                                                                                                            \haddockdecltt{->} & \haddockdecltt{Vector (Signal a)} & vector of signals \\
\end{tabulary}\par
Consumes the vectors carried by a signal with a rate of 1, and
 unzips them into a vector of signals based on the user provided
 rates. It instantiates the \haddockid{unzipx}
 skeleton.\par
\textbf{OBS:} due to the \haddockid{recur} pattern
 contained by \haddockid{unzipx}, the vector of
 production rates needs to be provided in reverse order (see
 \haddocktt{ForSyDe.Atom.Skeleton.Vector}).\par
\begin{interactive}
λ> let s1 = SDF.signal [1,2,3,4,5]
λ> let s2 = SDF.signal [11,12,13,14,15]
λ> let v1 = V.vector [s1,s1,s2,s2]
λ> let r  = V.vector [2,1,2,1]
λ> let sz = zipx r v1
λ> v1
<{1,2,3,4,5},{1,2,3,4,5},{11,12,13,14,15},{11,12,13,14,15}>
λ> sz
{<1,2,1,11,12,11>,<3,4,2,13,14,12>}
λ> unzipx (V.reverse r) sz
<{1,2,3,4},{1,2},{11,12,13,14},{11,12}>

\end{interactive}\haddockfig{fig/moc-sdf-unzipx.pdf}\par
           
\end{haddockdesc}
  \input{input/ForSyDe-Atom-MoC-Time}
  \input{input/ForSyDe-Atom-MoC-TimeStamp}
  \haddockmoduleheading{ForSyDe.Atom.Skeleton}
\label{module:ForSyDe.Atom.Skeleton}
\haddockbeginheader
{\haddockverb\begin{verbatim}
module ForSyDe.Atom.Skeleton (
    Skeleton((=.=), (=*=), (=\=), (=<<=), first, last),  farm22,  reduce, 
    reducei,  pipe,  pipe2
  ) where\end{verbatim}}
\haddockendheader

This module exports a type class with the interfaces for the
 Skeleton layer atoms. It does \emph{NOT} export any implementation of
 atoms not any constructor as composition of atoms.\par
\begin{mdframed}[style=reminder,frametitle=Reminder]Make sure to consult naming conventions in  \cref{sec:forsyde-atom:naming-convention} in order to interpret the names and type signatures correctly.\end{mdframed}\par

\subsection{Atoms}
\begin{haddockdesc}
\item[\begin{tabular}{@{}l}
class\ Functor\ c\ =>\ Skeleton\ c\ where
\end{tabular}]\haddockbegindoc
Class containing all the Skeleton layer atoms.\par
This class is instantiated by a set of categorical types,
 i.e. types which describe an inherent potential for being evaluated
 in parallel. Skeletons are patterns from this layer. When skeletons
 take as arguments entities from the MoC layer (i.e. processes), the
 results themselves are parallel process networks which describe
 systems with an inherent potential to be implemented on parallel
 platforms. All skeletons can be described as composition of the
 three atoms below (\haddockid{=<<=} being just a specific instantiation of
 \haddockid{={\char '134}=}). This possible due to an existing theorem in the categorical
 type theory, also called the Bird-Merteens formalism
 \cite{Bird97}:\par
\par
\begin{description}
\item[factorization] A function on a categorical type is an algorithmic
 skeleton (i.e. catamorphism) \emph{iff} it can be represented in a
 factorized form, i.e. as a \emph{map} composed with a \emph{reduce}.
\end{description}Consequently, most of the skeletons for the implemented categorical
 types are described in their factorized form, taking as arguments
 either:\par
                 \begin{itemize}
                 \item
                 type constructors or functions derived from type constructors\par
                 
                 \item
                 processes, i.e. MoC layer entities\par
                 
                 \end{itemize}
                 Most of the ground-work on algorithmic skeletons on which this
 module is founded has been laid in \cite{Bird97},
 \cite{Skillicorn05} and it founds many
 of the frameworks collected in \cite{Gorlatch03}.\par
                 
\haddockpremethods{}\textbf{Methods}
\begin{haddockdesc}
\item[\begin{tabular}{@{}l}\haddockid{(=.=)}\ ::\ (a\ ->\ b)\ ->\ c\ a\ ->\ c\ b\ \end{tabular}]
\haddockbegindoc
Atom which maps a function on each element of a structure (i.e. categorical type), defined as:\haddockeq{fig/eqs-skel-atom-dot.pdf}\haddockid{=.=} together with \haddockid{=*=} form the map pattern.\par

\item[\begin{tabular}{@{}l}\haddockid{(=*=)}\ ::\ c\ (a\ ->\ b)\ ->\ c\ a\ ->\ c\ b\ \end{tabular}]
\haddockbegindoc
Atom which applies the functions contained by as structure (i.e. categorical type), on the elements of another structure, defined as:\haddockeq{fig/eqs-skel-atom-star.pdf}\haddockid{=.=} together with \haddockid{=*=} form the map pattern.\par

\item[\begin{tabular}{@{}l}\haddockid{(=\textbackslash=)}\ ::\ (a\ ->\ a\ ->\ a)\ ->\ c\ a\ ->\ a\ \end{tabular}]
\haddockbegindoc
Atom which reduces a structure to an element based on an \emph{associative} function, defined as:\haddockeq{fig/eqs-skel-atom-red.pdf}\par

\item[\begin{tabular}{@{}l}\haddockid{(=<<=)}\ \end{tabular}]
\haddockbegindoc
Skeleton which \emph{pipes} an element through all the functions contained by a structure. N.B.: this is not an atom. It has an implicit definition which might be augmented by instances of this class to include edge cases.\haddockeq{fig/eqs-skel-pattern-pipe.pdf}As the composition operation is not associative, we cannot treat pipe as a true reduction. Alas, it can still be exploited in parallel since it exposes another type of parallelism: time parallelism.\par

\item[\begin{tabular}{@{}l}\haddockid{first}\ ::\ c\ a\ ->\ a\ \end{tabular}]
\haddockbegindoc
Returns the first element in a structure.N.B.: this is not an atom. It has an implicit definition which might be replaced by instances of this class with a more efficient implementation.\haddockeq{fig/eqs-skel-pattern-first.pdf}\par

\item[\begin{tabular}{@{}l}\haddockid{last}\ ::\ c\ a\ ->\ a\ \end{tabular}]
\haddockbegindoc
Returns the last element in a structure.N.B.: this is not an atom. It has an implicit definition which might be replaced by instances of this class with a more efficient implementation.\haddockeq{fig/eqs-skel-pattern-last.pdf}\par

\end{haddockdesc}


\item[\begin{tabular}{@{}l}
instance\ Skeleton\ Vector
\end{tabular}]\haddockbegindoc
Ensures that \haddockid{Vector} is a structure associated with the Skeleton Layer.\par

\end{haddockdesc}

\subsection{Skeleton constructors}
Patterns of in the skeleton layer are provided, like all other
 patterns in ForSyDe-Atom, as constructors. If the layer below
 this one is the \haddockid{MoC} layer, i.e. the functions
 taken as arguments are processes, then these skeletons can be
 regarded as process network constructors, as the structures
 created are process networks with inherent potential for parallel
 implementation.\par

\begin{haddockdesc}
\item[\begin{tabular}{@{}l}
farm22\ ::\ Skeleton\ c\ =>\\\ \ \ \ \ \ \ \ \ \ (a1\ ->\ a2\ ->\ (b1,\ b2))\ ->\ c\ a1\ ->\ c\ a2\ ->\ (c\ b1,\ c\ b2)
\end{tabular}]\haddockbegindoc
\haddocktt{farm} maps a function on a vector. It is the embodiment of the
 \haddocktt{map} homomorphism, and its naming is inspired from the pattern
 predominant in HPC. Indeed, if we consider the layer below as being
 the \haddockid{MoC} layer (i.e. the passed functions are
 processes), the resulting structure could be regarded as a "farm of
 data-parallel processes".\par
Constructors: \haddocktt{farm{\char 91}1-8{\char 93}{\char 91}1-4{\char 93}}.\par
\haddockeq{fig/eqs-skel-pattern-farm.pdf}
 \haddockfig{fig/skel-pattern-farm.pdf}\par


\item[\begin{tabular}{@{}l}
reduce
\end{tabular}]\haddockbegindoc
\haddockbeginargs
\haddockdecltt{::} & Skeleton c \\
                     \haddockdecltt{=>} & \haddockdecltt{(a
                                                          -> a
                                                             -> a)} & associative function (*) \\
                                                                      \haddockdecltt{->} & \haddockdecltt{c a} & structure \\
                                                                                                                 \haddockdecltt{->} & \haddockdecltt{a} & reduced element \\
\end{tabulary}\par
Infix name for the \haddockid{={\char '134}=} atom operator.\par
(*) if the operation is not associative then the network can be
 treated like a pipeline.\par


\item[\begin{tabular}{@{}l}
reducei
\end{tabular}]\haddockbegindoc
\haddockbeginargs
\haddockdecltt{::} & Skeleton c \\
                     \haddockdecltt{=>} & \haddockdecltt{(a
                                                          -> a
                                                             -> a)} & associative function (*) \\
                                                                      \haddockdecltt{->} & \haddockdecltt{a} & initial element of structure \\
                                                                                                               \haddockdecltt{->} & \haddockdecltt{c a} & structure \\
                                                                                                                                                          \haddockdecltt{->} & \haddockdecltt{a} & reduced element \\
\end{tabulary}\par
\haddockid{reducei} is special case of \haddockid{reduce} where an initial element is
 specified outside the reduced vector. It is implemented as a
 \haddockid{pipe} with switched arguments, and the reduction function is
 constrained to be associative. It is semantically equivalent to the
 pattern depicted below.\par
(*) if the operation is not associative then the network is
 semantically equivalent to \haddocktt{pipe1} (see \haddockid{pipe2}).\par
\haddockeq{fig/eqs-skel-pattern-reducei.pdf}
 \haddockfig{fig/skel-pattern-reducei.pdf}\par


\item[\begin{tabular}{@{}l}
pipe
\end{tabular}]\haddockbegindoc
\haddockbeginargs
\haddockdecltt{::} & Skeleton c \\
                     \haddockdecltt{=>} & \haddockdecltt{c (a
                                                            -> a)} & vector of functions \\
                                                                     \haddockdecltt{->} & \haddockdecltt{a} & kernel element \\
                                                                                                              \haddockdecltt{->} & \haddockdecltt{a} & result  \\
\end{tabulary}\par
Infix name for the \haddockid{=<<=} skeleton operator.\par


\item[\begin{tabular}{@{}l}
pipe2\ ::\ Skeleton\ c\ =>\\\ \ \ \ \ \ \ \ \ (a1\ ->\ a2\ ->\ a\ ->\ a)\ ->\ c\ a1\ ->\ c\ a2\ ->\ a\ ->\ a
\end{tabular}]\haddockbegindoc
The \haddocktt{pipe} constructors are a more generic form of the \haddockid{=<<=}
 (\haddockid{pipe}) skeleton apt for successive partial application and create
 more robust parameterizable pipeline networks.\par
Constructors: \haddocktt{comb{\char 91}1-8{\char 93}}.\par
\haddockeq{fig/eqs-skel-pattern-pipe1.pdf}
 \haddockfig{fig/skel-pattern-pipe1.pdf}\par

\end{haddockdesc}
  \haddockmoduleheading{ForSyDe.Atom.Skeleton.Vector}
\label{module:ForSyDe.Atom.Skeleton.Vector}
\haddockbeginheader
{\haddockverb\begin{verbatim}
module ForSyDe.Atom.Skeleton.Vector (
    Vector(Null, (:>)),  null,  unit,  (<++>),  vector,  fromVector,  indexes, 
    isNull,  (<:),  farm22,  reduce,  prefix,  suffix,  pipe,  (=/=),  recur, 
    cascade2,  mesh2,  length,  index,  fanout,  fanoutn,  generate,  iterate, 
    first,  last,  inits,  tails,  init,  tail,  concat,  reverse,  group, 
    shiftr,  shiftl,  rotr,  rotl,  take,  drop,  takeWhile,  filterIdx,  odds, 
    evens,  stride,  get,  (<@),  (<@!),  gather1,  (<@>),  replace,  scatter, 
    bitrev,  duals,  unduals,  zipx,  unzipx
  ) where\end{verbatim}}
\haddockendheader

This module defines the data type \haddockid{Vector} as a categorical type,
 and implements the atoms for the \haddockid{Skeleton}
 class. Algorithmic skeletons for \haddockid{Vector} are mostly described in
 their factorized form, which ensures that they are catamorphisms
 (see the \href{ForSyDe-Atom-Skeleton.html#factorization}{factorization}
 theorem). Where efficiency or practicality is a concern, some
 skeletons are implemented as recurrences. One can still prove that
 they are catamorphisms through alternative theorems (see
 \cite{Skillicorn05}).\par
\begin{mdframed}[style=reminder,frametitle=Reminder]Make sure to consult naming conventions in  \cref{sec:forsyde-atom:naming-convention} in order to interpret the names and type signatures correctly.\end{mdframed}\par

\subsection{Vector data type}
\begin{haddockdesc}
\item[\begin{tabular}{@{}l}
data\ Vector\ a
\end{tabular}]\haddockbegindoc
\haddockbeginconstrs
\haddockdecltt{=} & \haddockdecltt{Null} & Null element. Terminates a vector. \\
\haddockdecltt{|} & \haddockdecltt{a (:>) (Vector a)} & appends an element at the head of a vector. \\
\end{tabulary}\par

The  \haddockid{Vector}, or at least its interpretation, is the
 exact equivalent of an infinite list, as defined in
 \cite{Bird97}. Its name though is borrowed
 from \cite{Reekie95}, since it is more
 suggestive in the context of process networks.\par
According to \cite{Bird97}, \haddockid{Vector}
 should be implemented as following:\par
\begin{code}
data Vector a = Null                   -- null element
              | Unit a                 -- singleton vector
              | Vector a <++> Vector a -- concatenate two vectors\end{code}
This construction suggests the possibility of splitting a \haddockid{Vector}
 into multiple parts and evaluating it in parallel. Due to reasons
 of efficiency, and to ensure that the structure is flat and
 homogeneous, \haddockid{Vector} is implemented using the same constructors as
 an infinite list like in \cite{Bird87} (see
 below). When defining skeletons of vectors we will not use the real
 constructors though, but the theoretical ones defined above and
 provided as \hyperref[g:2]{functions} . This way we align ForSyDe-Atom's
 \haddockid{Vector} type with the categorical type theory and its theorems.\par
Another particularity of \haddockid{Vector} is that it instantiates the
 reduction atom \haddockid{={\char '134}=} as a \emph{right fold}, as it is the most efficient
 implementation in the context of lazy evaluation. As a consequence
 reduction is performed \textbf{\emph{from right to left}}. This is noticeable
 especially in the case of pipeline-based skeletons (where \haddockid{pipe}
 itself is a reduction with the right-associative composition
 operator \haddockid{.}) is performed from right to left, which
 comes in natural when considering the order of function
 composition. Thus for \haddockid{reduce}-based skeletons (e.g. \haddocktt{prefix},
 \haddocktt{suffix}, \haddocktt{recur}, \haddocktt{cascade}, \haddocktt{mesh}) the result vectors shall be
 read from end to beginning.\par


\item[\begin{tabular}{@{}l}
instance\ Functor\ Vector
\end{tabular}]\haddockbegindoc
Provides an implementation for \haddockid{=.=}.\par


\item[\begin{tabular}{@{}l}
instance\ Applicative\ Vector
\end{tabular}]\haddockbegindoc
Provides an implementation for \haddockid{=*=}.\par


\item[\begin{tabular}{@{}l}
instance\ Foldable\ Vector
\end{tabular}]\haddockbegindoc
Provides an implementation for \haddockid{={\char '134}=}.\par


\item[\begin{tabular}{@{}l}
instance\ Skeleton\ Vector
\end{tabular}]\haddockbegindoc
Ensures that \haddockid{Vector} is a structure associated with the Skeleton Layer.\par


\item[\begin{tabular}{@{}l}
instance\ Eq\ a\ =>\ Eq\ (Vector\ a)
\end{tabular}]

\item[\begin{tabular}{@{}l}
instance\ Read\ a\ =>\ Read\ (Vector\ a)
\end{tabular}]\haddockbegindoc
The vector 1 :> 2 :> Null is read using the string "<1,2>".\par


\item[\begin{tabular}{@{}l}
instance\ Show\ a\ =>\ Show\ (Vector\ a)
\end{tabular}]\haddockbegindoc
The vector 1 :> 2 :> Null is represented as <1,2>.\par


\item[\begin{tabular}{@{}l}
instance\ Plottable\ a\ =>\ Plottable\ (Vector\ a)
\end{tabular}]\haddockbegindoc
Vectors of plottable types\par


\item[\begin{tabular}{@{}l}
instance\ Plottable\ a\ =>\ Plot\ (Vector\ a)
\end{tabular}]\haddockbegindoc
vectors of coordinates\par

\end{haddockdesc}
\subsection{"Constructors"}
Theoretical constructors for the \haddockid{Vector} type, used in the
 definition of skeletons as catamorphisms.\par

\begin{haddockdesc}
\item[\begin{tabular}{@{}l}
null\ ::\ Vector\ a
\end{tabular}]\haddockbegindoc
Constructs a null vector.\par
\begin{interactive}
λ> null
<>

\end{interactive}

\item[\begin{tabular}{@{}l}
unit\ ::\ a\ ->\ Vector\ a
\end{tabular}]\haddockbegindoc
Constructs a singleton vector.\par
\begin{interactive}
λ> unit 1
<1>

\end{interactive}

\item[\begin{tabular}{@{}l}
(<++>)\ ::\ Vector\ a\ ->\ Vector\ a\ ->\ Vector\ a
\end{tabular}]\haddockbegindoc
Constructs a vector by appending two existing vectors.\par
\begin{interactive}
λ> unit 1 <++> unit 2
<1,2>

\end{interactive}
\end{haddockdesc}
\subsection{Utilities}
\begin{haddockdesc}
\item[\begin{tabular}{@{}l}
vector\ ::\ {\char 91}a{\char 93}\ ->\ Vector\ a
\end{tabular}]\haddockbegindoc
Converts a list to a vector.\par


\item[\begin{tabular}{@{}l}
fromVector\ ::\ Vector\ a\ ->\ {\char 91}a{\char 93}
\end{tabular}]\haddockbegindoc
Converts a vector to a list.\par


\item[\begin{tabular}{@{}l}
indexes\ ::\ Vector\ Integer
\end{tabular}]\haddockbegindoc
Creates the infinite vector:\par
\begin{interactive}
<1,2,3,4,...>\end{interactive}
Used mainly for operation on indexes.\par


\item[\begin{tabular}{@{}l}
isNull\ ::\ Vector\ a\ ->\ Bool
\end{tabular}]\haddockbegindoc
Returns \haddocktt{True} if the argument is a null vector.\par


\item[\begin{tabular}{@{}l}
(<:)\ ::\ Vector\ a\ ->\ a\ ->\ Vector\ a
\end{tabular}]\haddockbegindoc
Appends an element at the end of a vector.\par

\end{haddockdesc}
\subsection{Skeletons}
Algorithmic skeletons on vectors are mainly presented in terms
 of compositions of the atoms associated with the
 \haddocktt{Skeleton} Layer. When defining them,
 we use the following operators:\par
\haddockeq{fig/eqs-skel-vector-operators.pdf}\par
where:\par
\begin{itemize}
\item
(1) is the \haddockid{unit} constructor, constructing a singleton vector.\par

\item
(2) is the \haddockid{<++>} constructor, concatenating two vectors.\par

\item
(3) is the \haddocktt{<@!>} selector. The subscript notation is used to
 denote element at position \emph{n} in a vector.\par

\item
(4) suggests an arbitrary selector which returns a vector with
 another one's elements, based on some indices. The shown example
 is an alternative notation for the \haddockid{tail} skeleton.\par

\end{itemize}

\subsubsection{Functional networks}
This sub-category denotes skeletons (patterns) which are take
 functions as arguments. If the functions are
 \haddockid{MoC} layer entities, i.e. processes, then these
 patterns are capable of constructing parallel process
 networks. Using the applicative mechanism, the designer has a
 high degree of freedom when customizing process networks through
 systematic partial application, rendering numerous possible
 usages for the same pattern. To avoid over-encumbering the
 figures, they depict small test cases, which might not expose the
 full potential of the constructors.\par
see the \href{ForSyDe-Atom.html#naming_conv}{naming convention} rules
 on how to interpret, use and develop your own constructors.\par

\begin{haddockdesc}
\item[\begin{tabular}{@{}l}
farm22
\end{tabular}]\haddockbegindoc
\haddockbeginargs
\haddockdecltt{::} & \haddockdecltt{(a1
                                     -> a2
                                        -> (b1, b2))} & function (e.g. process) \\
                                                        \haddockdecltt{->} & \haddockdecltt{Vector a1} & first input vector \\
                                                                                                         \haddockdecltt{->} & \haddockdecltt{Vector a2} & second input vector \\
                                                                                                                                                          \haddockdecltt{->} & \haddockdecltt{(Vector b1, Vector b2)} & two output vectors \\
\end{tabulary}\par
\haddocktt{farm} is simply the \haddockid{Vector} instance of the skeletom \haddocktt{farm}
 pattern (see \haddockid{farm22}). If the function taken
 as argument is a process, then it creates a farm network of data
 parallel processes.\par
Constructors: \haddocktt{farm{\char 91}1-4{\char 93}{\char 91}1-4{\char 93}}.\par
\begin{interactive}
λ> let v1 = vector [1,2,3,4,5]
λ> S.farm21 (+) v1 v1
<2,4,6,8,10>
λ> let s1 = SY.signal [1,2,3,4,5]
λ> let v2 = vector [s1,s1,s1]
λ> S.farm11 (comb11 (+1)) v2
<{2,3,4,5,6},{2,3,4,5,6},{2,3,4,5,6}>
λ> S.farm21 (\x -> comb11 (+x)) v1 v2
<{2,3,4,5,6},{3,4,5,6,7},{4,5,6,7,8}>

\end{interactive}\haddockeq{fig/eqs-skel-vector-farm.pdf}\par
           \haddockdoublefig{fig/skel-vector-func-farm.pdf}{fig/skel-vector-func-farm-net.pdf}\par
           

\item[\begin{tabular}{@{}l}
reduce\ ::\ (a\ ->\ a\ ->\ a)\ ->\ Vector\ a\ ->\ a
\end{tabular}]\haddockbegindoc
As the name suggests, it reduces a vector to an element based on
 an associative function. If the function is not associative, it can be treated like a pipeline.\par
\haddockid{Vector} instantiates the skeletons for both
 \haddockid{reduce} and \haddockid{reducei}.\par
\begin{interactive}
λ> let v1 = vector [1,2,3,4,5]
λ> S.reduce (+) v1
15
λ> let s1 = SY.signal [1,2,3,4,5]
λ> let s2 = SY.signal [10,10,10,10,10]
λ> let v2 = vector [s1,s1,s1]
λ> S.reduce (comb21 (+)) v2
{3,6,9,12,15}
λ> S.reducei (comb21 (+)) s2 v2
{13,16,19,22,25}

\end{interactive}\haddockdoublefig{fig/skel-vector-func-reducei.pdf}{fig/skel-vector-func-reducei-net.pdf}\par
           

\item[\begin{tabular}{@{}l}
prefix\ ::\ (b\ ->\ b\ ->\ b)\ ->\ Vector\ b\ ->\ Vector\ b
\end{tabular}]\haddockbegindoc
\haddocktt{prefix} peforms the \emph{parallel prefix} operation on a vector.
 Equivalent process networks are constructed if processes are passed
 as arguments.\par
Similar to \haddockid{reduce} and \haddockid{reducei}, two versions \haddockid{prefix} and
 \haddocktt{prefixi} are provided.\par
\begin{interactive}
λ> let v1 = vector [1,2,3,4,5]
λ> prefix (+) v1
<15,14,12,9,5>
λ> let s1 = SY.signal [1,2,3,4,5]
λ> let s2 = SY.signal [10,10,10,10,10]
λ> let v2 = vector [s1,s1,s1]
λ> prefix (comb21 (+)) v2
<{3,6,9,12,15},{2,4,6,8,10},{1,2,3,4,5}>
λ> prefixi (comb21 (+)) s2 v2
<{13,16,19,22,25},{12,14,16,18,20},{11,12,13,14,15}>

\end{interactive}\haddockeq{fig/eqs-skel-vector-prefix.pdf}\par
           \haddockdoublefig{fig/skel-vector-func-prefix.pdf}{fig/skel-vector-func-prefix-net.pdf}\par
           \haddockdoublefig{fig/skel-vector-func-prefixi.pdf}{fig/skel-vector-func-prefixi-net.pdf}\par
           

\item[\begin{tabular}{@{}l}
suffix\ ::\ (b\ ->\ b\ ->\ b)\ ->\ Vector\ b\ ->\ Vector\ b
\end{tabular}]\haddockbegindoc
\haddocktt{suffix} peforms the \emph{parallel suffix} operation on a vector.
 Equivalent process networks are constructed if processes are passed
 as arguments.\par
Similar to \haddockid{reduce} and \haddockid{reducei}, two versions \haddockid{suffix} and
 \haddocktt{suffixi} are provided.\par
\begin{interactive}
λ> let v1 = vector [1,2,3,4,5]
λ> suffix (+) v1
<1,3,6,10,15>
λ> let s1 = SY.signal [1,2,3,4,5]
λ> let s2 = SY.signal [10,10,10,10,10]
λ> let v2 = vector [s1,s1,s1]
λ> suffix (comb21 (+)) v2
<{1,2,3,4,5},{2,4,6,8,10},{3,6,9,12,15}>
λ> suffixi (comb21 (+)) s2 v2
<{11,12,13,14,15},{12,14,16,18,20},{13,16,19,22,25}>

\end{interactive}\haddockeq{fig/eqs-skel-vector-suffix.pdf}\par
           \haddockdoublefig{fig/skel-vector-func-suffix.pdf}{fig/skel-vector-func-suffix-net.pdf}\par
           \haddockdoublefig{fig/skel-vector-func-suffixi.pdf}{fig/skel-vector-func-suffixi-net.pdf}\par
           

\item[\begin{tabular}{@{}l}
pipe
\end{tabular}]\haddockbegindoc
\haddockbeginargs
\haddockdecltt{::} & \haddockdecltt{Vector (a -> a)} & vector of functions \\
                                                       \haddockdecltt{->} & \haddockdecltt{a} & input \\
                                                                                                \haddockdecltt{->} & \haddockdecltt{a} & output \\
\end{tabulary}\par
\haddocktt{pipe} creates a pipeline of functions from a vector. \haddockid{pipe}
  simply instantiates the \haddockid{=<<=} atom whereas \haddocktt{pipeX} instantiate
  their omologi from the \haddocktt{ForSyDe.Atom.Skeleton} module (see
  \haddockid{pipe2}).\par
\textbf{OBS:} the pipelining is done in the order dictated by the
 function composition operator: from right to left.\par
Constructors: \haddocktt{pipe{\char 91}1-4{\char 93}}.\par
\begin{interactive}
λ> let v1 = vector [(+1),(+1),(+1)]
λ> S.pipe v1 1
4
λ> let s1 = SY.signal [1,2,3,4]
λ> let v2 = vector [1,2,3,4]
λ> S.pipe1 (\x -> comb11 (+x)) v2 s1
{11,12,13,14}

\end{interactive}\haddockdoublefig{fig/skel-vector-func-pipe.pdf}{fig/skel-vector-func-pipe-net.pdf}\par
           

\item[\begin{tabular}{@{}l}
(=/=)\ ::\ Vector\ (a\ ->\ a)\ ->\ a\ ->\ Vector\ a
\end{tabular}]\haddockbegindoc
Infix operator for \haddockid{recur}.\par


\item[\begin{tabular}{@{}l}
recur
\end{tabular}]\haddockbegindoc
\haddockbeginargs
\haddockdecltt{::} & \haddockdecltt{Vector (a -> a)} & vector of functions \\
                                                       \haddockdecltt{->} & \haddockdecltt{a} & input \\
                                                                                                \haddockdecltt{->} & \haddockdecltt{Vector a} & output  \\
\end{tabulary}\par
\haddocktt{recur} creates a systolic array from a vector of
 functions. Just like \haddockid{pipe} and \haddocktt{pipeX}, there exists a raw
 \haddockid{recur} version with an infix operator \haddockid{=/=}, and the enhanced
 \haddocktt{recurX} which is meant for systematic partial application of a
 function on an arbitrary number of vectors until the desired vector
 of functions is obtained.\par
Constructors: \haddocktt{(=/=)}, \haddocktt{recur}, \haddocktt{recuri}, \haddocktt{recur{\char 91}1-4{\char 93}{\char 91}1-4{\char 93}}.\par
\begin{interactive}
λ> let v1 = vector [(+1),(+1),(+1)]
λ> recur v1 1
<4,3,2>
λ> recuri v1 1
<4,3,2,1>
λ> let s1 = SY.signal [1,2,3,4]
λ> let v2 = vector [1,2,3,4]
λ> recur1 (\x -> comb11 (+x)) v2 s1
<{11,12,13,14},{10,11,12,13},{8,9,10,11},{5,6,7,8}>

\end{interactive}\haddockeq{fig/eqs-skel-vector-recur.pdf}\par
           \haddockdoublefig{fig/skel-vector-func-recur.pdf}{fig/skel-vector-func-recur-net.pdf}\par
           

\item[\begin{tabular}{@{}l}
cascade2
\end{tabular}]\haddockbegindoc
\haddockbeginargs
\haddockdecltt{::} & \haddockdecltt{(a2
                                     -> a1
                                        -> a
                                           -> a
                                              -> a)} & \haddocktt{function41} which needs to be applied to \haddocktt{function21} \\
                                                       \haddockdecltt{->} & \haddockdecltt{Vector (Vector a2)} & fills in the first argument in the function above \\
                                                                                                                 \haddockdecltt{->} & \haddockdecltt{Vector (Vector a1)} & fills in the second argument in the function above \\
                                                                                                                                                                           \haddockdecltt{->} & \haddockdecltt{Vector a} & first input vector (e.g. of signals) \\
                                                                                                                                                                                                                           \haddockdecltt{->} & \haddockdecltt{Vector a} & second input vector (e.g. of signals) \\
                                                                                                                                                                                                                                                                           \haddockdecltt{->} & \haddockdecltt{Vector a} & output \\
\end{tabulary}\par
\haddocktt{cascade} creates a "cascading mesh" as a result of piping a
 vector into a vector of recur arrays. \par
Constructors: \haddocktt{cascade}, \haddocktt{cascade{\char 91}1-4{\char 93}}.\par
\begin{interactive}
λ> let v1 = vector [1,2,3,4]
λ> cascade (+) v1 v1
<238,119,49,14>
λ> let s1 = SY.signal [1,2,3,4]
λ> let vs = vector [s1, s1, s1]
λ> cascade (comb21 (+)) vs vs
<{20,40,60,80},{10,20,30,40},{4,8,12,16}>
λ> let vv = vector [vector [1,-1,1], vector [-1,1,-1], vector [1,-1,1] ]
λ> cascade1 (\x -> comb21 (\y z-> x*(y+z))) vv vs vs
<{16,32,48,64},{8,16,24,32},{-2,-4,-6,-8}>

\end{interactive}\haddockeq{fig/eqs-skel-vector-cascade.pdf}\par
           \haddockdoublefig{fig/skel-vector-func-cascade.pdf}{fig/skel-vector-func-cascade-net.pdf}\par
           

\item[\begin{tabular}{@{}l}
mesh2
\end{tabular}]\haddockbegindoc
\haddockbeginargs
\haddockdecltt{::} & \haddockdecltt{(a2
                                     -> a1
                                        -> a
                                           -> a
                                              -> a)} & \haddocktt{function41} which needs to be applied to \haddocktt{function21} \\
                                                       \haddockdecltt{->} & \haddockdecltt{Vector (Vector a2)} & fills in the first argument in the function above \\
                                                                                                                 \haddockdecltt{->} & \haddockdecltt{Vector (Vector a1)} & fills in the second argument in the function above \\
                                                                                                                                                                           \haddockdecltt{->} & \haddockdecltt{Vector a} & first input vector (e.g. of signals) \\
                                                                                                                                                                                                                           \haddockdecltt{->} & \haddockdecltt{Vector a} & second input vector (e.g. of signals) \\
                                                                                                                                                                                                                                                                           \haddockdecltt{->} & \haddockdecltt{Vector (Vector a)} & output, a 2D vector \\
\end{tabulary}\par
\haddocktt{mesh} creates a 2D systolic array as a result of piping a vector
 into a vector of 1D systolic arrays.\par
Constructors: \haddocktt{mesh}, \haddocktt{mesh{\char 91}1-4{\char 93}}.\par
\begin{interactive}
λ> let v1 = vector [1,2,3,4]
λ> mesh (+) v1 v1
<<238,119,49,14>,<119,70,35,13>,<49,35,22,11>,<14,13,11,8>>
λ> let s1 = SY.signal [1,2,3,4]
λ> let vs = vector [s1, s1, s1]
λ> mesh (comb21 (+)) vs vs
<<{20,40,60,80},{10,20,30,40},{4,8,12,16}>,<{10,20,30,40},{6,12,18,24},{3,6,9,12}>,<{4,8,12,16}, {3,6,9,12},{2,4,6,8}>>
λ> let vv = vector [vector [1,-1,1], vector [-1,1,-1], vector [1,-1,1]]
λ> mesh1 (\x -> comb21 (\y z-> x*(y+z))) vv vs vs
<<{16,32,48,64},{8,16,24,32},{-2,-4,-6,-8}>,<{8,16,24,32},{-6,-12,-18,-24},{-3,-6,-9,-12}>, <{-2,-4,-6,-8},{-3,-6,-9,-12},{2,4,6,8}>>

\end{interactive}\haddockeq{fig/eqs-skel-vector-mesh.pdf}\par
           \haddockdoublefig{fig/skel-vector-func-mesh.pdf}{fig/skel-vector-func-mesh-net.pdf}\par
           
\end{haddockdesc}
\subsubsection{Queries}
Queries return various information about a vector. They are
 also built as skeletons.\par

\begin{haddockdesc}
\item[\begin{tabular}{@{}l}
length\ ::\ Num\ p\ =>\ Vector\ a\ ->\ p
\end{tabular}]\haddockbegindoc
returns the number of elements in a value.\par
\begin{interactive}
λ> length $ vector [1,2,3,4,5]
5

\end{interactive}\haddockeq{fig/eqs-skel-vector-length.pdf}\par
           

\item[\begin{tabular}{@{}l}
index\ ::\ Vector\ a2\ ->\ Vector\ Integer
\end{tabular}]\haddockbegindoc
returns a vector with the indexes from another vector.\par
\begin{interactive}
λ> index $ vector [1,1,1,1,1,1,1]
<1,2,3,4,5,6,7>

\end{interactive}
\end{haddockdesc}
\subsubsection{Generators}
Generators are specific applications of the \haddocktt{prefix} or
 \haddocktt{suffix} skeletons.\par

\begin{haddockdesc}
\item[\begin{tabular}{@{}l}
fanout\ ::\ t\ ->\ Vector\ t
\end{tabular}]\haddockbegindoc
\haddockid{fanout} repeats an element. As a process network it distributes
 the same value or signal to all the connected processes down the
 line. Depending on the target platform and the refinement decisions
 involved, it may be interpreted in the following implementations:\par
\begin{itemize}
\item
global or shared memory in case of a massively parallel platform
  (e.g. GPU)\par

\item
a (static) memory or cache location in memory-driven
  architectures (e.g. CPU)\par

\item
a fanout in case of a HDL system\par

\item
a broadcast in case of a distributed system\par

\end{itemize}


\item[\begin{tabular}{@{}l}
fanoutn\ ::\ (Num\ t,\ Ord\ t)\ =>\ t\ ->\ a\ ->\ Vector\ a
\end{tabular}]\haddockbegindoc
\haddockid{fanoutn} is the same as \haddockid{fanout}, but the length of the result
 is also provided.\par
\begin{interactive}
λ> fanoutn 5 1
<1,1,1,1,1>

\end{interactive}

\item[\begin{tabular}{@{}l}
generate\ ::\ (Num\ t,\ Ord\ t)\ =>\ t\ ->\ (a\ ->\ a)\ ->\ a\ ->\ Vector\ a
\end{tabular}]\haddockbegindoc
\haddockid{generate} creates a vector based on a kernel function. It is
 just a restricted version of \haddockid{recur}.\par
\begin{interactive}
λ> generate 5 (+1) 1
<6,5,4,3,2>

\end{interactive}\haddockeq{fig/eqs-skel-vector-generate.pdf}\par
           

\item[\begin{tabular}{@{}l}
iterate\ ::\ (Num\ t,\ Ord\ t)\ =>\ t\ ->\ (a\ ->\ a)\ ->\ a\ ->\ Vector\ a
\end{tabular}]\haddockbegindoc
\haddockid{iterate} is a version of \haddockid{generate} which keeps the initial
 element as well. It is a restricted version of \haddocktt{recuri}.\par
\begin{interactive}
λ> iterate 5 (+1) 1
<5,4,3,2,1>

\end{interactive}
\end{haddockdesc}
\subsubsection{Permutators}
Permutators perform operations on the very structure of
 vectors, and make heavy use of the vector constructors.\par

\begin{haddockdesc}
\item[\begin{tabular}{@{}l}
first\ ::\ Vector\ a\ ->\ a
\end{tabular}]\haddockbegindoc
Instance of \haddockid{first}\par
\begin{interactive}
λ> S.first $ vector [1,2,3,4,5]
1

\end{interactive}

\item[\begin{tabular}{@{}l}
last\ ::\ Vector\ a\ ->\ a
\end{tabular}]\haddockbegindoc
Instance of \haddockid{last}\par
\begin{interactive}
λ> S.last $ vector [1,2,3,4,5]
5

\end{interactive}

\item[\begin{tabular}{@{}l}
inits\ ::\ Vector\ a\ ->\ Vector\ (Vector\ a)
\end{tabular}]\haddockbegindoc
creates a vector of all the initial segments in a vector.\par
\begin{interactive}
λ> inits $ vector [1,2,3,4,5]
<<1>,<1,2>,<1,2,3>,<1,2,3,4>,<1,2,3,4,5>>

\end{interactive}\haddockeq{fig/eqs-skel-vector-inits.pdf}\par
           \haddockdoublefig{fig/skel-vector-comm-inits.pdf}{fig/skel-vector-comm-inits-net.pdf}\par
           

\item[\begin{tabular}{@{}l}
tails\ ::\ Vector\ a\ ->\ Vector\ (Vector\ a)
\end{tabular}]\haddockbegindoc
creates a vector of all the final segments in a vector.\par
\begin{interactive}
λ> tails $ vector [1,2,3,4,5]
<<1,2,3,4,5>,<2,3,4,5>,<3,4,5>,<4,5>,<5>>

\end{interactive}\haddockeq{fig/eqs-skel-vector-tails.pdf}\par
           \haddockdoublefig{fig/skel-vector-comm-tails.pdf}{fig/skel-vector-comm-tails-net.pdf}\par
           

\item[\begin{tabular}{@{}l}
init\ ::\ Vector\ a\ ->\ Vector\ a
\end{tabular}]\haddockbegindoc
Returns the initial segment of a vector.\par
\begin{interactive}
λ> init $ vector [1,2,3,4,5]
<1,2,3,4>

\end{interactive}\haddockeq{fig/eqs-skel-vector-init.pdf}\par
           

\item[\begin{tabular}{@{}l}
tail\ ::\ Vector\ a\ ->\ Vector\ a
\end{tabular}]\haddockbegindoc
Returns the tail of a vector.\par
\begin{interactive}
λ> tail $ vector [1,2,3,4,5]
<2,3,4,5>

\end{interactive}\haddockeq{fig/eqs-skel-vector-tail.pdf}\par
           

\item[\begin{tabular}{@{}l}
concat\ ::\ Vector\ (Vector\ a)\ ->\ Vector\ a
\end{tabular}]\haddockbegindoc
concatenates a vector of vectors.\par
\begin{interactive}
λ> concat $ vector [vector[1,2,3,4], vector[5,6,7]]
<1,2,3,4,5,6,7>

\end{interactive}\haddockeq{fig/eqs-skel-vector-concat.pdf}\par
           

\item[\begin{tabular}{@{}l}
reverse\ ::\ Vector\ a\ ->\ Vector\ a
\end{tabular}]\haddockbegindoc
reverses the elements in a vector.\par
\begin{interactive}
λ> reverse $ vector [1,2,3,4,5]
<5,4,3,2,1>

\end{interactive}\haddockeq{fig/eqs-skel-vector-reverse.pdf}\par
           \haddockdoublefig{fig/skel-vector-comm-reverse.pdf}{fig/skel-vector-comm-reverse-net.pdf}\par
           

\item[\begin{tabular}{@{}l}
group\ ::\ Integer\ ->\ Vector\ a\ ->\ Vector\ (Vector\ a)
\end{tabular}]\haddockbegindoc
groups a vector into sub-vectors of \emph{n} elements.\par
\begin{interactive}
λ> group 3 $ vector [1,2,3,4,5,6,7,8]
<<1,2,3>,<4,5,6>,<7,8>>

\end{interactive}\haddockeq{fig/eqs-skel-vector-group.pdf}\par
           \haddockdoublefig{fig/skel-vector-comm-group.pdf}{fig/skel-vector-comm-group-net.pdf}\par
           

\item[\begin{tabular}{@{}l}
shiftr\ ::\ Vector\ a\ ->\ a\ ->\ Vector\ a
\end{tabular}]\haddockbegindoc
right-shifts a vector with an element.\par
\begin{interactive}
λ> vector [1,2,3,4] `shiftr` 8
<8,1,2,3>

\end{interactive}\haddockdoublefig{fig/skel-vector-comm-shiftr.pdf}{fig/skel-vector-comm-shiftr-net.pdf}\par
           

\item[\begin{tabular}{@{}l}
shiftl\ ::\ Vector\ a\ ->\ a\ ->\ Vector\ a
\end{tabular}]\haddockbegindoc
left-shifts a vector with an element.\par
\begin{interactive}
λ> vector [1,2,3,4] `shiftl` 8
<2,3,4,8>

\end{interactive}\haddockdoublefig{fig/skel-vector-comm-shiftl.pdf}{fig/skel-vector-comm-shiftl-net.pdf}\par
           

\item[\begin{tabular}{@{}l}
rotr\ ::\ Vector\ a\ ->\ Vector\ a
\end{tabular}]\haddockbegindoc
rotates a vector to the right.\par
\begin{interactive}
λ> rotr $ vector [1,2,3,4]
<4,1,2,3>

\end{interactive}\haddockdoublefig{fig/skel-vector-comm-rotr.pdf}{fig/skel-vector-comm-rotr-net.pdf}\par
           

\item[\begin{tabular}{@{}l}
rotl\ ::\ Vector\ a\ ->\ Vector\ a
\end{tabular}]\haddockbegindoc
rotates a vector to the left.\par
\begin{interactive}
λ> rotl $ vector [1,2,3,4]
<2,3,4,1>

\end{interactive}\haddockdoublefig{fig/skel-vector-comm-rotl.pdf}{fig/skel-vector-comm-rotl-net.pdf}\par
           

\item[\begin{tabular}{@{}l}
take\ ::\ Integer\ ->\ Vector\ a\ ->\ Vector\ a
\end{tabular}]\haddockbegindoc
takes the first \emph{n} elements of a vector.\par
\begin{interactive}
λ> take 5 $ vector [1,2,3,4,5,6,7,8,9]
<1,2,3,4,5>

\end{interactive}\haddockeq{fig/eqs-skel-vector-take.pdf}\par
           

\item[\begin{tabular}{@{}l}
drop\ ::\ Integer\ ->\ Vector\ a\ ->\ Vector\ a
\end{tabular}]\haddockbegindoc
drops the first \emph{n} elements of a vector.\par
\begin{interactive}
λ> drop 5 $ vector [1,2,3,4,5,6,7,8,9]
<6,7,8,9>

\end{interactive}\haddockeq{fig/eqs-skel-vector-drop.pdf}\par
           

\item[\begin{tabular}{@{}l}
takeWhile\ ::\ (a\ ->\ Bool)\ ->\ Vector\ a\ ->\ Vector\ a
\end{tabular}]\haddockbegindoc
takes the first elements in a vector until the first element that
 does not fulfill a predicate.\par
\begin{interactive}
λ> takeWhile (<5) $ vector [1,2,3,4,5,6,7,8,9]
<1,2,3,4>

\end{interactive}\haddockeq{fig/eqs-skel-vector-takewhile.pdf}\par
           

\item[\begin{tabular}{@{}l}
filterIdx\ ::\ (Integer\ ->\ Bool)\ ->\ Vector\ a\ ->\ Vector\ a
\end{tabular}]\haddockbegindoc
returns a vector containing only the elements of another vector
 whose index satisfies a predicate.\par
\begin{interactive}
λ> filterIdx (\x -> x `mod` 3 == 0) $ vector [0,1,2,3,4,5,6,7,8,9]
<2,5,8>

\end{interactive}\haddockeq{fig/eqs-skel-vector-filteridx.pdf}\par
           \haddockdoublefig{fig/skel-vector-comm-filteridx.pdf}{fig/skel-vector-comm-filteridx-net.pdf}\par
           

\item[\begin{tabular}{@{}l}
odds\ ::\ Vector\ a\ ->\ Vector\ a
\end{tabular}]\haddockbegindoc
\haddockeq{fig/eqs-skel-vector-odds.pdf}\par


\item[\begin{tabular}{@{}l}
evens\ ::\ Vector\ a\ ->\ Vector\ a
\end{tabular}]\haddockbegindoc
\haddockeq{fig/eqs-skel-vector-evens.pdf}\par


\item[\begin{tabular}{@{}l}
stride
\end{tabular}]\haddockbegindoc
\haddockbeginargs
\haddockdecltt{::} & \haddockdecltt{Integer} & first index \\
                                               \haddockdecltt{->} & \haddockdecltt{Integer} & stride length \\
                                                                                              \haddockdecltt{->} & \haddockdecltt{Vector a} & \\
                                                                                                                                              \haddockdecltt{->} & \haddockdecltt{Vector a} & \\
\end{tabulary}\par
does a stride-selection on a vector.\par
\begin{interactive}
λ> stride 1 3 $ vector [1,2,3,4,5,6,7,8,9]
<1,4,7>

\end{interactive}\haddockeq{fig/eqs-skel-vector-stride.pdf}\par
           \haddockdoublefig{fig/skel-vector-comm-inits.pdf}{fig/skel-vector-comm-inits-net.pdf}\par
           

\item[\begin{tabular}{@{}l}
get\ ::\ Integer\ ->\ Vector\ a\ ->\ Maybe\ a
\end{tabular}]\haddockbegindoc
returns the \emph{n}-th element in a vector, or \haddocktt{Nothing} if \emph{n > l}.\par
\begin{interactive}
λ> get 3 $ vector [1,2,3,4,5]
Just 3

\end{interactive}\haddockeq{fig/eqs-skel-vector-get.pdf}\par
           

\item[\begin{tabular}{@{}l}
(<@)\ ::\ Vector\ a\ ->\ Integer\ ->\ Maybe\ a
\end{tabular}]\haddockbegindoc
the same as \haddockid{get} but with flipped arguments.\par


\item[\begin{tabular}{@{}l}
(<@!)\ ::\ Vector\ p\ ->\ Integer\ ->\ p
\end{tabular}]\haddockbegindoc
unsafe version of \haddockid{<@>}. Throws an exception if \emph{n > l}.\par


\item[\begin{tabular}{@{}l}
gather1
\end{tabular}]\haddockbegindoc
\haddockbeginargs
\haddockdecltt{::} & \haddockdecltt{Vector Integer} & vector of indexes \\
                                                      \haddockdecltt{->} & \haddockdecltt{Vector a} & input vector \\
                                                                                                      \haddockdecltt{->} & \haddockdecltt{Vector (Maybe a)} & \\
\end{tabulary}\par
selects the elements in a vector at the incexes contained by another vector.\par
The following versions of this skeleton are available, the number
 suggesting how many nested vectors it is operating upon: \haddocktt{gather{\char 91}1-5{\char 93}}\par
\begin{interactive}
λ> let ix = vector [vector [1,3,4], vector [3,5,1], vector [5,8,9]]
λ> let v = vector [11,12,13,14,15]
λ> gather2 ix v
<<Just 11,Just 13,Just 14>,<Just 13,Just 15,Just 11>,<Just 15,Nothing,Nothing>>

\end{interactive}\haddockeq{fig/eqs-skel-vector-gather.pdf}\par
           \haddockdoublefig{fig/skel-vector-comm-gather.pdf}{fig/skel-vector-comm-gather-net.pdf}\par
           

\item[\begin{tabular}{@{}l}
(<@>)
\end{tabular}]\haddockbegindoc
\haddockbeginargs
\haddockdecltt{::} & \haddockdecltt{Vector a} & input vector \\
                                                \haddockdecltt{->} & \haddockdecltt{Vector Integer} & vector of indexes \\
                                                                                                      \haddockdecltt{->} & \haddockdecltt{Vector (Maybe a)} & \\
\end{tabulary}\par
the same as \haddockid{gather1} but with flipped arguments\par
The following versions of this skeleton are available, the number
 suggesting how many nested vectors it is operating upon.\par
\begin{code}
(<@>), (<<@>>), (<<<@>>>), (<<<<@>>>>), (<<<<<@>>>>>),\end{code}


\item[\begin{tabular}{@{}l}
replace\ ::\ Integer\ ->\ p\ ->\ Vector\ p\ ->\ Vector\ p
\end{tabular}]\haddockbegindoc
replaces the \emph{n}-th element in a vector with another.\par
\begin{interactive}
λ> replace 5 15 $ vector [1,2,3,4,5,6,7,8,9]
<1,2,3,4,15,6,7,8,9>

\end{interactive}\haddockeq{fig/eqs-skel-vector-replace.pdf}\par
           \haddockdoublefig{fig/skel-vector-comm-replace.pdf}{fig/skel-vector-comm-replace-net.pdf}\par
           

\item[\begin{tabular}{@{}l}
scatter\ ::\ Vector\ Integer\ ->\ Vector\ p\ ->\ Vector\ p\ ->\ Vector\ p
\end{tabular}]\haddockbegindoc
scatters the elements in a vector based on the indexes contained by another vector.\par
\begin{interactive}
λ> scatter (vector [2,4,5]) (vector [0,0,0,0,0,0,0,0]) (vector [1,1,1])
<0,1,0,1,1,0,0,0>

\end{interactive}\haddockeq{fig/eqs-skel-vector-scatter.pdf}\par
           \haddockdoublefig{fig/skel-vector-comm-scatter.pdf}{fig/skel-vector-comm-scatter-net.pdf}\par
           

\item[\begin{tabular}{@{}l}
bitrev\ ::\ Vector\ a\ ->\ Vector\ a
\end{tabular}]\haddockbegindoc
performs a bit-reverse permutation.\par
\haddockdoublefig{fig/skel-vector-comm-bitrev.pdf}{fig/skel-vector-comm-bitrev-net.pdf}\par
\begin{interactive}
λ> bitrev $ vector ["000","001","010","011","100","101","110","111"]
<"111","011","101","001","110","010","100","000">

\end{interactive}

\item[\begin{tabular}{@{}l}
duals\ ::\ Vector\ b2\ ->\ (Vector\ b2,\ Vector\ b2)
\end{tabular}]\haddockbegindoc
splits a vector in two equal parts.\par
\begin{interactive}
λ> duals $ vector [1,2,3,4,5,6,7]
(<1,2,3>,<4,5,6>)

\end{interactive}

\item[\begin{tabular}{@{}l}
unduals\ ::\ Vector\ a\ ->\ Vector\ a\ ->\ Vector\ a
\end{tabular}]\haddockbegindoc
concatenates a previously split vector. See also \haddockid{duals}\par

\end{haddockdesc}
\subsubsection{Interfaces}
\begin{haddockdesc}
\item[\begin{tabular}{@{}l}
zipx
\end{tabular}]\haddockbegindoc
\haddockbeginargs
\haddockdecltt{::} & MoC e \\
                     \haddockdecltt{=>} & \haddockdecltt{Vector ((Vector a
                                                                  -> Vector a
                                                                     -> Vector a)
                                                                 -> Fun e (Vector a) (Fun e (Vector a) (Ret e (Vector a))))} & vector of MoC-specific context wrappers for the function
 \haddockid{<++>} \\
                                                                                                                               \haddockdecltt{->} & \haddockdecltt{Vector (Stream (e a))} & input vector of signals \\
                                                                                                                                                                                            \haddockdecltt{->} & \haddockdecltt{Stream (e (Vector a))} & output signal of vectors \\
\end{tabulary}\par
\haddockid{zipx} is a template skeleton for "zipping" a vector of
 signals. It synchronizes all signals (of the same MoC) in a vector
 and outputs one signal with vectors of the synced values. For each
 signal in the input vector it requires a function which
 \emph{translates} a partition of events (see \haddocktt{ForSyDe.Atom.MoC}) into
 sub-vectors.\par
There exist helper intances of the \haddockid{zipx} skeleton interface for
 all supported MoCs.\par
\haddockeq{fig/eqs-skel-vector-zipx.pdf}
 \haddockfig{fig/skel-vector-comm-zipx.pdf}\par


\item[\begin{tabular}{@{}l}
unzipx\ ::\ MoC\ e\ =>\\\ \ \ \ \ \ \ \ \ \ (Vector\ a\ ->\ Vector\ (Ret\ e\ a))\\\ \ \ \ \ \ \ \ \ \ ->\ Integer\ ->\ Stream\ (e\ (Vector\ a))\ ->\ Vector\ (Stream\ (e\ a))
\end{tabular}]\haddockbegindoc
\haddockid{unzipx} is a template skeleton to unzip a signal carrying
 vectors into a vector of multiple signals. It required a function
 that \emph{splits} a vector of values into a vector of event partitions
 belonging to output signals. Unlike \haddockid{zipx}, it also requires the
 number of output signals. The reason for this is that it is
 impossible to determine the length of the output vector without
 "sniffing" the content of the input events, which is out of the
 scope of skeletons and may lead to unsafe behavior. The length of
 the output vector is needed in order to avoid infinite recurrence.\par
There exist helper intances of the \haddockid{unzipx} skeleton interface for
 all supported MoCs.\par
\haddockeq{fig/eqs-skel-vector-unzipx.pdf}{fig/skel-vector-comm-unzipx.pdf}\par

\end{haddockdesc}
  % \input{input/ForSyDe-Atom-Utility}
  \input{input/ForSyDe-Atom-Utility-Plot}
  \input{input/ForSyDe-Atom-Utility-Tuple}
  
  \printbibliography[heading=subbibliography]
  \printindex
\end{refsection}

\end{document}

%%% Local Variables:
%%% TeX-command-default: "Make"
%%% mode: latex
%%% TeX-master: t
%%% End:
