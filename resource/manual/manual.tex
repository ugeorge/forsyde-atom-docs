\documentclass{book}
\usepackage{sty/atom-manual}
\usepackage{sty/haddock}
\usepackage{sty/urls}
\usepackage{sty/atom-vars}
\usepackage{ccicons}

\title{\textsc{ForSyDe-Atom}\\User Manual}
\author{George Ungureanu}

\newcommand*{\RootPath}{./input}%
\addbibresource{refs.bib}

\makeindex
\begin{document}
\dominitoc

\pagenumbering{Roman} 
\includepdf[pages={1}]{title.pdf}

%License
\begin{bottompar}\small
\begin{tabular}{p{.69\textwidth}r}
  \vspace{-.5cm} This work is licensed under a Creative Commons  Attribution-ShareAlike 4.0 Unported (CC BY-SA 4.0) License. & {\Huge\ccbysa}\\[3ex]
  \multicolumn{2}{p{\textwidth}}{The code listed throughout this document is generated from several projects mostly licensed under the BSD-3 Clause License, unless specified otherwise.} \\
\end{tabular}

\end{bottompar}

\tableofcontents
\clearpage
\listoffigures
\clearpage

\chapter*{List of Case Studies}
\label{cha:list-publications}
\markboth{List of Case Studies}{List of Case Studies}

\fullcite{Ungureanu17}\dotfill\cref{sec:getting-started:toy-example}

\vspace{1ex}
\noindent\fullcite{Ungureanu18a}\dotfill\cref{sec:hybrid:rc-oscillator}

\mainmatter

\pagenumbering{arabic}

\begin{refsection}
  \chapter{Introduction}
  \label{cha:introduction}

  \begin{summary}
    In this chapter we introduce the purpose, organization and usage of this document, as well as brief instructions and references for helping to set up the \textsc{ForSyDe-Atom} libraries. The scope is to facilitate the reader's progression through this document.

%%% Local Variables:
%%% mode: latex
%%% TeX-master: "manual"
%%% End:

  \end{summary}
  \section{Purpose \& organization}
\label{sec:purp-organ}

This book is a living document which gathers material related to \textsc{ForSyDe-Atom} and binds it in form of a user manual. Most of the text contained by this book originates from actual inline or literate source code documentation, in form of examples, tutorials, reports and even library API documentation. This means that this document evolves with the \textsc{ForSyDe-Atom} project itself and is periodically updated.

\textsc{ForSyDe-Atom} is a shallow-embedded DSL in the functional programming language Haskell for modeling cyber-physical and parallel systems. It enforces a disciplined way of modeling by separating the manifold concerns of systems into orthogonal \emph{layers}. The \textsc{ForSyDe-Atom} formal framework aims to providing (where possible) a minimum set of primary common building blocks for each layer called \emph{atoms}, capturing elementary semantics. Even so, the modeling framework provides library blocks and modules commonly used in CPS defined in terms of \emph{patterns} of atoms. For more information on \textsc{ForSyDe-Atom} itself, please consult the associated scientific publications or the API extended documentation\footnote{currently available online at \url{https://forsyde.github.io/forsyde-atom/}}.

This document is structured in two parts. The first part gathers examples, tutorials and experiments in a learning progression. Each chapter associated with (and actually generated from) a \href{https://www.haskell.org/cabal/}{Haskell Cabal project} included in the \texttt{forsyde-atom-examples}\footnote{available online at \url{https://github.com/forsyde/forsyde-atom-examples}} repository. This means that the first part of the book is meant to be read in parallel with running the associated example project which conveniently exports functions to test the listed code ``on-the-fly''. The second part of the book is the actual inline API documentation of \textsc{ForSyDe-Atom} generated with \href{https://www.haskell.org/haddock/}{Haddock}, which serves also as an extended library report and provides information both on theoretical and implementation issues.

\begin{summary}
  If for any reason you have difficulties following the document or you encounter bugs or discrepancies in the code and text do not hesitate to contact the author(s) or maintainer(s) on GitHub or by email. 
\end{summary}

%%% Local Variables:
%%% mode: latex
%%% TeX-master: "manual"
%%% End:

  
\section{Getting {{\sc ForSyDe-Atom}}}
\label{sec:getting-forsyde-atom}

The \textsc{ForSyDe-Atom} EDSL can be downloaded from \url{https://github.com/forsyde/forsyde-atom}. The main page contains enough information for acquiring the dependencies and installing the library on your own. However, each example project from the \href{https://github.com/forsyde/forsyde-atom-examples}{\texttt{forsyde-atom-examples}} repository comes with a set of installation scripts written for user convenience, which should be enough for traversing this manual.

\begin{mdframed}[style=attention,frametitle=Attention!]
Be advised that different chapters of the book have been written during different development stages of \textsc{ForSyDe-Atom} and are compatible with different library releases. The code listed throughout the book \emph{might} be incompatible with the latest release. We strongly recommend using the installation scripts included in each project associated with a chapter, which acquire and install the right dependencies in a local sandbox.
\end{mdframed}

Provided you have an OS installation where the minimum dependencies (\href{https://www.gnu.org/software/make/}{GNU make}, a \href{https://git-scm.com/downloads}{Git CLI client}, \href{https://www.haskell.org/platform/}{Haskell Platform} and \href{https://www.haskell.org/cabal/download.html}{cabal-install}) are working and accessible by your user profile, to run the \texttt{getting-started} example associated with \cref{ch:getting-started}, you simply need to type in the terminal:

\begin{verbatim}
# download the examples
git clone https://github.com/forsyde/forsyde-atom-examples.git

# change directory to the desired project folder
cd forsyde-atom-examples/getting-started

# install the project (and dependencies) in a sandbox
make install

# open an interpreter session with the examples loaded
cabal repl
\end{verbatim}

%%% Local Variables:
%%% mode: latex
%%% TeX-master: "manual"
%%% End:

  \section{Using this document}
\label{sec:using-this-document}


\begin{mdframed}[style=attention,frametitle=Disclaimer]
The document assumes that the reader is familiar with the syntax of Haskell and the usage of a Haskell interpreter (e.g. \texttt{ghci}). Otherwise, we recommend consulting at least the introductory chapters of one of the following books by \cite{Lipovaca11} and \cite{Hutton16} or other recent books in Haskell.
\end{mdframed}

Most of this document has been created using literate programming. This means that all code shown in the listings is compatible with the \textsc{ForSyDe-Atom} library version mentioned at the beginning of \DivergeDoc{the}{each} \SelfRef. Throughout the document you will find two listing styles. This style
\begin{code}
-- | API documentation comment 
myIdFunc :: a -> a
myIdFunc = id
\end{code}
shows \emph{source code} as it is found in the implementation files. We have taken the liberty to display some code characters as their literate equivalent (e.g. \texttt{->} is shown as $\rightarrow$, \texttt{\textbackslash} is shown as $\lambda$, and so on). \DivergeDoc{}{
 
} This style
\begin{interactive}
Prelude> 1 + 1
2
\end{interactive}
suggests \emph{interactive commands} given by the user in a terminal or an interpreter session. The listing above shows a typical \texttt{ghci} session, where the string after the prompter symbol \texttt{>} suggests the user input (e.g. \texttt{1 + 1}). Whenever relevant, the expected output is printed one row below that (e.g. \texttt{2}).

The code examples are bundled as separate \href{https://www.haskell.org/cabal/}{Cabal} packages and is provided as libraries meant to be loaded in an interpreter session in parallel with reading this document. Detailed instructions on how to install the packages can be found in the \texttt{README.md} file in each project. The best way to install the packages is within sandboxed environments with all dependencies taken care of, usually scripted within the \texttt{make} commands. After a successful installation, to open an interpreter session pre-loaded with the main sandboxed library, you just need to type in the following command in a terminal from the package root path (the one containing the \texttt{.cabal} file):
\begin{interactive}
# cabal repl
\end{interactive}

Each section of this document contains a small example written within a library \emph{module}, like:
\begin{code}
module X where
\end{code}
One can access all functions in module \texttt{X} by importing it in the interpreter session, unless otherwise noted (e.g. library \texttt{X} is re-exported by \texttt{Y}).
\begin{interactive}
*Y> import X
\end{interactive}
Now suppose that function \texttt{myIdFunc} above was defined in module \texttt{X}, then one would have direct access to it, e.g.:
\begin{interactive}
*Y X> :t myIdFunc
myIdFunc :: a -> a
*Y X> myIdFunc 3
3  
\end{interactive}
By all means, the code for \texttt{myIdFunc} or any source code for that matter can be copied/pasted in a custom \texttt{.hs} file and compiled or used in any relevant means. The current format was chosen because it is convenient to ``get your hands dirty'' quickly without thinking of issues associated with compiler suites.

A final tip: if you think that the full name of \texttt{X} is polluting your prompter or is hard to use, then you can import it using an alias:
\begin{interactive}
*Y> import Extremely.Long.Full.Name.For.X as ShortAlias
*Y ShortAlias> 
\end{interactive}

  \printbibliography[heading=subbibliography]
\end{refsection}


\part{Examples \& Reports}
\label{part:introduction}

\begin{refsection}
  \chapter{Getting Started with {{\sc ForSyDe-Atom}}}
  \label{ch:getting-started}
  \renewcommand*{\RootPath}{%
    \AtomExamplesRoot/getting-started/docs/latex}%
  \graphicspath{{\AtomExamplesRoot/getting-started/docs/latex/figs/}}

  \begin{summary}
    In this chapter we introduce the purpose, organization and usage of this document, as well as brief instructions and references for helping to set up the \textsc{ForSyDe-Atom} libraries. The scope is to facilitate the reader's progression through this document.

%%% Local Variables:
%%% mode: latex
%%% TeX-master: "manual"
%%% End:

  \end{summary}
  \minitoc
  \vspace{1ex}
  \input{\RootPath/input/info}
  \input{\RootPath/input/goals}
  \input{\RootPath/code/GettingStarted/Basics}
  \input{\RootPath/code/GettingStarted/Plot}
  \input{\RootPath/input/toy-example}
  \input{\RootPath/code/GettingStarted/TestSignals}
  \input{\RootPath/code/GettingStarted/SY}
  \input{\RootPath/code/GettingStarted/DE}
  \input{\RootPath/code/GettingStarted/CT}
  \input{\RootPath/code/GettingStarted/SDF}
  \input{\RootPath/code/GettingStarted/Polymorphic}
  \input{\RootPath/code/GettingStarted/CustomPattern}
  \input{\RootPath/input/conclusion}

  \printbibliography[heading=subbibliography]
\end{refsection}

\begin{refsection}
  \chapter{Hybrid CT/DT Models in {{\sc ForSyDe-Atom}}}
  \label{ch:hybrid}
  \renewcommand*{\RootPath}{\AtomExamplesRoot/hybrid/docs/latex}%
  \graphicspath{{\AtomExamplesRoot/hybrid/docs/latex/figs/}}

  \begin{summary}
    In this chapter we introduce the purpose, organization and usage of this document, as well as brief instructions and references for helping to set up the \textsc{ForSyDe-Atom} libraries. The scope is to facilitate the reader's progression through this document.

%%% Local Variables:
%%% mode: latex
%%% TeX-master: "manual"
%%% End:

  \end{summary}
  \minitoc
  \vspace{1ex}
  \input{\RootPath/input/info}
  \input{\RootPath/input/goals}
  \input{\RootPath/code/Hybrid/RCOsc}
  \input{\RootPath/input/conclusion}

  \printbibliography[heading=subbibliography]
\end{refsection}

\part{Tools \& Libraries Documentation}
\label{part:api-documentation}

\chapter{The \textsc{ForSyDe-Atom} Standard Library}
\label{ch:forsyde-atom}
  \graphicspath{{./}}

\begin{refsection}
  \parskip=10pt plus2pt minus2pt
  \setlength{\parindent}{0cm}

  \begin{summary}
    This chapter is an extended API documentation of \textsc{ForSyDe-Atom} Standard Library version \AtomVersion\footnote{available at: \url{https://github.com/forsyde/forsyde-atom}} which was organized to also serve the purpose of technical report. It treats both theoretical and practical aspects, justifying the implementation and providing examples of usage for most of the library-exported functions. It has been generated from the inline documentation using a \href{https://www.haskell.org/haddock/}{\mbox{Haddock}}\footnote{custom build from: \url{https://github.com/ugeorge/haddock}} generator.

  \end{summary}
  \minitoc
  \section{Introduction}
\label{sec:forsyde-atom:introduction}

The ForSyDe (Formal System Design) methodology has been developed
with the objective to move system design to a higher level of
abstraction and to bridge the abstraction gap by transformational
design refinement. It targets the modelling and characterization of
cyber-physical systems inheriting the theory of
\href{http://ieeexplore.ieee.org/stamp/stamp.jsp?arnumber=736561}{Models of Computation (MoCs)},
providing both a correct-by-construction execution model and
analyzable entry point for further synthesis flows. For more information
about ForSyDe and its associated projects please consult the
\href{https://github.com/forsyde}{ForSyDe webpage}.\par
The \haddocktt{forsyde-atom} library is a shallow-embedded DSL implementing
the execution semantics of an \emph{atom-based approach} to ForSyDe. Its
purpose is to provide a modeling framework for cyber-physical
systems and to serve as a proof-of-concept for a future (non-strict
typed) DSL. Adhering to the formalism set by the
\href{http://www.diva-portal.org/smash/get/diva2:9340/FULLTEXT01.pdf}{PhD Thesis of Ingo Sander},
the current framework views systems as networks of processes
communicating through signals. The \emph{atom-based approach} to ForSyDe
adds some new important concepts:\par
\begin{itemize}
\item
the separation of concerns through semantically-independent
\textbf{\emph{layers}} of computation, behavior, synchronization and
structure. Each layer offers a different analyzable view of a
given system and is described using the concept of
\emph{higher-order functions}.\par

\item
the description of each layer as a network of primitive building
blocks called \textbf{\emph{atoms}}. Each atom embeds an undividable operation,
and wraps functions of lower layers with the semantics dictated by
a higher layer. They are described using the powerful concept of
\emph{applicative functors}.\par

\item
the complete \textbf{\emph{autonomy}} of atoms in relation to the patterns
they build. As such, process networks or constructors are nothing
but \emph{structural} ad-hoc compositions that aid in achieving complex
behaviors whereas the actual behavior is dictated by the atoms
alone. This is proved (for the time being) for the
\emph{Synchronization Layer} by implementing all MoCs as instances of
\emph{only one type class}.\par

\end{itemize}

While the host language limits the possibility of providing general
(e.g. proces) constructors due to type-strictness, the
\haddocktt{forsyde-atom} framework is by all means complete. In this sense the
user can create her own custom \emph{correct-by-design} constructors and
networks as compositions of the provided atoms and utilities. Also,
this haddock-generated page is organized both as an API
documentation and as a technical report to facilitate te use and
understanding of the formal principles behing the design process.\par

\subsection{Naming convention}
\label{sec:forsyde-atom:naming-convention}

All multi-parameter patterns and
utilities provided by the library API as higher-order functions are
named along the lines of \haddocktt{functionMN} where \haddocktt{M} represents the
number of \textbf{\emph{curried}} inputs (i.e. \haddocktt{a1\ ->\ a2\ ->\ ...\ ->\ aM}),
while \haddocktt{N} represents the number of \textbf{\emph{tupled}} outputs
(i.e. \haddocktt{(b1,b2,...,bN)}). To avoid repetition we only provide
documentation for functions with 2 inputs and 2 outputs
(i.e. \haddocktt{function22}). In case the provided functions do not suffice,
feel free to implement your own patterns following the examples in
the source code.\par

  \haddockmoduleheading{ForSyDe.Atom}
\label{module:ForSyDe.Atom}
\haddockbeginheader
{\haddockverb\begin{verbatim}
module ForSyDe.Atom (
    ExB(extend, (/.\), (/*\), (/&\), (/!\)),  Stream(NullS, (:-)), 
    module ForSyDe.Atom.MoC.Stream,  Time,  TimeStamp, 
    MoC(Fun, Ret, (-.-), (-*-), (-*), (-<-), (-&-)), 
    Skeleton((=.=), (=*=), (=\=), (=<<=), first, last), 
    module ForSyDe.Atom.Utility,  (||<)
  ) where\end{verbatim}}
\haddockendheader

The formal foundation upon which ForSyDe \cite{Sander04}
 defines its semantics is the \emph{tagged signal model}
 \cite{Lee98}.  This is a denotational
 framework introduced by Lee and Sangiovanni-Vincentelli as a common
 meta model for describing properties of concurrent systems in
 general terms as sets of possible behaviors. Systems are regarded
 as \emph{compositions} of \emph{processes} acting on \emph{signals} which are sets
 of \emph{tagged events}. Signals are characterized by a \emph{tag system}
 which determines causality between events, and could model time,
 precedence relationships, synchronization points, and other key
 properties. Based on how tag systems are defined, one can identify
 several \emph{Models of Computations (MoCs)} as classes of behaviors
 dictating the semantics of execution and concurrency in a network
 of processes.\par
These concepts are the supporting pillars of ForSyDe's philosophy,
 and state the purpose of the \haddocktt{forsyde-atom} library: it is supposed
 to be a modelling framework used as a proof-of-concept for the
 atom-based approach to cyber-physical systems
 \cite{Ungureanu17}. This approach
 extends the tagged signal model by systematically deconstructing
 processes to their basic semantics and recreating them using a
 minimal language of primitive building blocks called \emph{atoms}. It
 also tries to expand the scope of this model by exploiting more
 aspects than just timing, by adding primitives for parallelism,
 behavior, etc.\par
The API documentation is structured as follows: this page provides
 an overview of the general notions and concepts, gently introducing
 the separate modules and the motivation behind them. Each major
 module corresponds to a separate \emph{layer}
 \cite{Ungureanu17} implemented as a type
 class. The documentation pages for each layer and for each of their
 instances contains in-depth knowledge and examples, and can be
 accessed from the contents page or by following the links
 suggested. For more complex examples and tutorials follow the links
 in the \href{https://github.com/forsyde/forsyde-atom}{project web page}.\par
 \begin{mdframed}[style=reminder,frametitle=Reminder]Make sure to consult naming conventions in  \cref{sec:forsyde-atom:naming-convention} in order to interpret the names and type signatures correctly.\end{mdframed}\par

\subsection{The layered process model}
The \haddocktt{forsyde-atom} project is led by three main policies:\par
\begin{enumerate}
\item 
in order to cope with the complexity of cyber-physical systems
 (CPS) it tries to separate the concerns such as computation,
 timing, synchronization, parallelism, structure, behavior, etc.\par

\item 
in order to have a small, ideally minimal grammar to reason
 about systems correctness, it aims to provide primitive
 (indivisible) operators called \emph{atoms} as building blocks for
 independently developing complex aspects of a system's execution
 through means of composition or generalization.\par

\item 
in order to express complex behaviors with a minimal grammar,
 it decouples structure (composition) from meaning (semantics),
 the only semantics carriers being atoms. Thus complex behaviors
 can be described in terms of \emph{patterns of atoms}. Using ad-hoc
 polymorphism, atoms can be overloaded with different semantics
 triggered by the data type they input, whereas their composition
 is always the same.\par

\end{enumerate}
\begin{description}
\item[\haddocktt{atom}] the elementary (primitive, indivisible) constructor
 which embeds a set of semantics relevant for their respective
 layer (e.g. timing, behavioural, structural, etc.)
\item[\haddocktt{atom\ patterns}] meaningful compositions of atoms. They are
 provided as constructors which need to be properly instantiated
 in order to be used. We also use the term "pattern" to
 differentiate atom compositions as constructors from atoms as
 constructors.
\end{description}The first policy, i.e. the separation of concerns led to the
 so-called \emph{layered process model} which is reflected in the
 library by providing separate major modules associated with each
 layer. Layers as such are independent collections of entities for
 modeling different aspects of CPS. These aspects interact through
 means of higher-order functions, wrapping each other in as
 structured fashion in a way which can be visualized as below.\par
                 
 \haddockfig{fig/misc-layered-model.pdf}\par
                 Layers are implemented as type classes which imply:\par
                 \begin{itemize}
                 \item
                 \textbf{atoms} as function signatures belonging to the type class;\par
                 
                 \item
                 \textbf{patterns} which are compositions atoms, provided as constructors;\par
                 
                 \item
                 \textbf{data types} for all the classes of behaviors concerning the
 aspect described by the layer in question. These types
 instantiate the above type class and overload the atoms with
 semantics in accordance to the behavior described. For example,
 the \haddockid{MoC} layer is currently instantiated by
 types describing the \haddockid{CT},
 \haddockid{DE}, \haddockid{SY} and
 \haddockid{SDF} MoCs.\par
                 
                 \end{itemize}
                 In order to model interleaving aspects of CPS, layers interact
 with each other through means of higher order functions. As such,
 each layer describes some atoms as higher-order functions which
 take entities belonging to another layer as arguments.
 Intrinsically, the data types belonging to a layer may be wrapping
 types of other layers, as depicted in the \hyperref[layered-model]{figure}
 above. For a short comprehensive overview on layers, please refer
 to \cite{Ungureanu17}.\par
                 By convention, the first (innermost) layer is always the
 \emph{function layer} which describes arbitrary functions on data and
 expresses the system's functional aspects. In the following
 paragraphs we will give an overview of the "outer" layers
 currently implemented in \haddocktt{forsyde-atom}, which in comparison,
 express the extra-functional aspects of a system (timing,
 behavior, synchronization, and so on).\par
                 
\subsection{The Extended Behavior (ExB) Layer}
As seen in \hyperref[layered-model]{layered process model}, the extended
 behavior layer expands the set of possible behaviors implied by a
 layer (typically the function layer), by defining a set of
 symbols with \emph{known} semantics, and adding it to (i.e. wrapping)
 the pool of possible values or states.\par
While semantically the \haddockid{ExB} layer extends the value pool in
 order to express special events (e.g. error messages or even the
 complete absence of events), it practically provides an
 independent environment to model events with a default/known
 response, independently of the data path. These responses are
 particularly captured by atoms, thus enforcing the high-level
 separation of concerns between e.g. control and data paths.\par
This layer provides:\par
\begin{itemize}
\item
a set of extended behavior atoms defining the interfaces for
 the resolution and response functions, as part of the \haddockid{ExB} type
 class \emph{(see below)}.\par

\item
a library of function wrappers as specific atom patterns
 (\emph{Check the \haddocktt{ForSyDe.Atom.ExB} module for extensive}
 \emph{documentation}).\par

\item
a set of data types defining classes of behaviors and
 instantiating the \haddockid{ExB} type class (\emph{check the links in the}
 \emph{\hyperref[section.i:ExB]{instances} section for extensive documentation}).\par

\end{itemize}

\begin{haddockdesc}
\item[\begin{tabular}{@{}l}
class\ Functor\ b\ =>\ ExB\ b\ where
\end{tabular}]\haddockbegindoc
Class which defines the atoms for the extended behavior layer.\par
As its name suggests, this layer is extending the behavior of
 processes (or merely of functions if we ignore timing semantics
 completely), by expanding the domains of the wrapped layer
 (e.g. the set of values) with symbols having clearly-defined
 semantics (e.g. special events with known responses).\par
The types associated with this layer can simply be describes as:\par
\haddockeq{fig/eqs-exb-types.pdf}\par
where  \emph{α} is a base type and \emph{b} is the type extension,
 i.e. a set of symbols with clearly-defined semantics.\par
Extended behavior atoms are functions of these types, defined as
 interfaces in the \haddockid{ExB} type class.\par

\haddockpremethods{}\textbf{Methods}
\begin{haddockdesc}
\item[\begin{tabular}{@{}l}\haddockid{extend}\ ::\ a\ ->\ b\ a\ \end{tabular}]
\haddockbegindoc
Extends a value (from a layer below) with a set of symbols with known semantics, as described by a type instantiating this class.\par

\item[\begin{tabular}{@{}l}\haddockid{(/.\textbackslash)}\ ::\ (a\ ->\ a)\ ->\ b\ a\ ->\ b\ a\end{tabular}]
\haddockbegindoc
Basic functor operator. Lifts a function (from a layer below) into the domain of the extended behavior layer.\haddockeq{fig/eqs-exb-atom-func.pdf}\par

\item[\begin{tabular}{@{}l}\haddockid{(/*\textbackslash)}\ ::\ b\ (a\ ->\ a)\ ->\ b\ a\ ->\ b\ a\end{tabular}]
\haddockbegindoc
Applicative operator. Defines a res between two extended behavior symbols.\haddockeq{fig/eqs-exb-atom-app.pdf}\par

\item[\begin{tabular}{@{}l}\haddockid{(/\&\textbackslash)}\ ::\ b\ Bool\ ->\ b\ a\ ->\ b\ a\ \end{tabular}]
\haddockbegindoc
Predicate operator. Generates a defined behavior based on an extended Boolean predicate.\haddockeq{fig/eqs-exb-atom-phi.pdf}\par

\item[\begin{tabular}{@{}l}\haddockid{(/"!\textbackslash)}\ ::\ a\ ->\ b\ a\ ->\ a\ \end{tabular}]
\haddockbegindoc
Degrade operator. Degrades a behavior-extended value into a non-extended one (from a layer below), based on a kernel value. Used also to throw exceptions.\haddockeq{fig/eqs-exb-atom-deg.pdf}\par

\end{haddockdesc}


\item[\begin{tabular}{@{}l}
instance\ ExB\ AbstExt
\end{tabular}]\haddockbegindoc
Implements the absent semantics of the extended behavior atoms.\par

\end{haddockdesc}
\subsection{The Model of Computation (MoC) Layer}
This layer represents a major part of the \haddocktt{forsyde-atom}
 library and is concerned in modeling the timing aspects of
 CPS. While its foundations have been layered in the classical
 ForSyDe \cite{Sander04}, it is mainly inspired from
 \cite{Lee98} an it tries to follow the tagged signal model as
 closely as it is permitted by the host language, and with the
 adaptations require by the atom approach.\par
Although a short introduction of the tagged signal model has been
 written in the introduction of this documentation, we feel
 obliged to provide a primer in the classical ForSyDe theory in
 order to understand how everything fits together.\par

\subsubsection{Signals}
\cite{Lee98} defines signals as (ordered) sets of events
 where each event is composed of a tag \emph{T} and a value
 \emph{V}. Similarly, in ForSyDe a signal is defined as a (partially or
 totally) \emph{ordered sequence} of events that enables processes to
 communicate and synchronize. Sequencing might infer an implicit
 order of events, but more importantly it determines an order of
 evaluation, which is a key piece of a simulation engine.\par
\haddockeq{fig/misc-tagged-signal.pdf}\par
In ForSyDe-Atom, sequencing is achieved using the \haddockid{Stream} data
 type, inspired from \cite{Reekie95}. In ForSyDe-Atom,
 signals are streams that carry \emph{events}, where each type of event
 is identified by a type constructor which defines its tag
 system. In other words, we can state that through its tag system,
 a signal is \emph{bound} to a MoC.\par

\begin{haddockdesc}
\item[\begin{tabular}{@{}l}
data\ Stream\ e
\end{tabular}]\haddockbegindoc
\haddockbeginconstrs
\haddockdecltt{=} & \haddockdecltt{NullS} & terminates a signal \\
\haddockdecltt{|} & \haddockdecltt{e (:-) (Stream e)} & the default constructor appends an\\
 &&event to the head of the stream \\
\end{tabulary}\par

Defines a stream of events, encapsulating them in a structure
 isomorphic to an infinite list \cite{Bird87},
 thus all properties of lists may also be applied to
 \haddockid{Stream}s. While, in combination with lazy evaluation, it is
 possible to create and simulate infinite signals, we need to ensure
 that the first/previous event is always fully evaluated. This can
 be translated into the following rule:\par
\begin{description}
\item[zero-delay feedbacks] are forbidden, due to un-evaluated
 self-referential calls. In a feedback loop, there always has to be
 enough events to ensure the data flow.
\end{description}This rule imposes that the stream of data is uninterrupted in order
 to have an evaluatable kernel every time a new event is produced
 (i.e. to avoid deadlocks). Thus we can add the rule:\par
                 \begin{description}
                 \item[cleaning of output events] is also forbidden.  In other words, for
 each new input at any instant in time, a process must react with
 \emph{at least} one output event.
                 \end{description}

\item[\begin{tabular}{@{}l}
instance\ Functor\ Stream
\end{tabular}]\haddockbegindoc
allows for the mapping of an arbitrary function \haddocktt{(a\ ->\ b)} upon
 all the events of a \haddocktt{(Stream\ a)}.\par


\item[\begin{tabular}{@{}l}
instance\ Applicative\ Stream
\end{tabular}]\haddockbegindoc
enables the \haddockid{Stream} to behave like a \haddockid{ZipList}\par


\item[\begin{tabular}{@{}l}
instance\ Foldable\ Stream
\end{tabular}]\haddockbegindoc
provides folding functions useful for implementing utilities, such as \haddockid{length}.\par


\item[\begin{tabular}{@{}l}
instance\ Read\ a\ =>\ Read\ (Stream\ a)
\end{tabular}]\haddockbegindoc
signal \haddocktt{(1\ :-\ 2\ :-\ NullS)} is read using the string \haddocktt{"{\char '173}1,2{\char '175}"}.\par


\item[\begin{tabular}{@{}l}
instance\ Show\ a\ =>\ Show\ (Stream\ a)
\end{tabular}]\haddockbegindoc
signal \haddocktt{(1\ :-\ 2\ :-\ NullS)} is represented as \haddocktt{{\char '173}1,2{\char '175}}.\par


\item[\begin{tabular}{@{}l}
instance\ Plottable\ a\ =>\ Plot\ (Signal\ a)
\end{tabular}]\haddockbegindoc
\haddockid{SY} signals.\par


\item[\begin{tabular}{@{}l}
instance\ Plottable\ a\ =>\ Plot\ (Signal\ a)
\end{tabular}]\haddockbegindoc
\haddockid{SDF} signals.\par


\item[\begin{tabular}{@{}l}
instance\ Plottable\ a\ =>\ Plot\ (Signal\ a)
\end{tabular}]\haddockbegindoc
\haddockid{DE} signals.\par


\item[\begin{tabular}{@{}l}
instance\ Plottable\ a\ =>\ Plot\ (Signal\ a)
\end{tabular}]\haddockbegindoc
\haddockid{CT} signals.\par

\end{haddockdesc}
For extended documentation and a list of all utilities associated
 with the \haddockid{Stream} type you can consult:\par

\begin{haddockdesc}
\item[\begin{tabular}{@{}l}
module\ ForSyDe.Atom.MoC.Stream
\end{tabular}]
\end{haddockdesc}
\subsubsection{Processes}
As described in \cite{Lee98}, processes are either "set of
 possible behaviors" of signals or "relations" between multiple
 signals. One can describe complex systems by composing processes,
 which in this case is interpreted as the "intersection of the
 behaviors of each of the processes being involved".\par
\begin{description}
\item[monotonicity] In order to ensure causal order and determinancy,
 processes need to be \emph{monotonic} \cite{Lee98}. A signal's
 tags (if explicit) \emph{must be} a partial or total order and all tag
 alterations must be monotonic.
\end{description}ForSyDe inherits this definition with respect to a functional
 view, thus a \textbf{process} \emph{p} is a functional mapping over (the
 history of) signals. A process can \emph{only} be instantiated using a
 \textbf{process constructor} \emph{pc}, which is a higher order function
 embedding MoC semantics and/or a specific composition, but
 lacking functionality.\par
                 
 \haddockeq{fig/misc-process.pdf}\par
                 Since processes are functions, process composition is equivalent
 to function composition. This means that composing two processes
 \haddocktt{p1} and \haddocktt{p2} grants the process \haddocktt{p2\ .\ p1}\par
\begin{quote}
  {\haddockverb\begin{verbatim}
p1      :: Signal α -> Signal β
p2      :: Signal β -> Signal γ
p2 . p1 :: Signal α -> Signal γ\end{verbatim}}
                 \end{quote}
                 This implies that there is a signal \haddocktt{Signal\ $\beta$ } that
  "streams" the result from \haddocktt{p1} to \haddocktt{p2}, as suggested in the
  drawing:\par
                 \haddockeq{fig/misc-ser-composition.pdf}\par
                 \textbf{Process networks} describe ForSyDe systems in terms of
 compositions of processes and originate from Reekie's process
 nets \cite{Reekie95}. A process network is a process itself,
 i.e. function from signal(s) to signal(s). The composition above
 \haddocktt{p2\ .\ p1\ } can also be regarded as a process network.\par
                 In ForSyDe-Atom atoms can be regarded as process constructors as
 their instantiations are functions on signals of events.
 Instantiations of atom patterns are the exact equivalent of
 process networks, which themselves are also processes, depending
 on the level of abstraction you are working with (hierarchical
 blocks vs. flat structures).\par
                 To understand the versatility of composition and partial
 application in building process constructors, consider the
 example above where composition of two processes infers a signal
 between them. This mechanism also works when composing
 constructors (un-instantiated atoms), which yields another
 constructor. By instantiating (fully applying) the new
 constructor we obtain a process network equivalent to the
 composition of the respective primitive processes obtained by
 instantiating (fully applying) the component atoms, like in the
 example below:\par
                 \haddockeq{fig/misc-process-constructor.pdf}\par
                 Now if we visualize process networks as graphs, where processes
 are nodes and signals are edges, a meaningful process composition
 could be regarded as graph patterns. Therefore it is safe to
 associate process constructors as patterns in process networks.\par
                 
\subsubsection{Models of Computation}
As mentioned in the introduction, \emph{MoCs} are classes of behaviors
 dictating the semantics of execution and concurrency in a network
 of processes. Based on the definitions of their tag systems
 ForSyDe identifies MoCs as:\par
\begin{enumerate}
\item 
\emph{timed} where \emph{T} is a totally ordered set and \emph{t} express the
 notion of physical time (e.g. continuous time
 \haddockid{CT}, discrete event
 \haddockid{DE}) or precedence (e.g. synchronous
 \haddockid{SY});\par

\item 
\emph{untimed}, where \emph{T} is a partially ordered set and \emph{t} is
 expressed in terms of constraints on the tags in signals
 (e.g. dataflow, synchronous data flow
 \haddockid{SDF}).\par

\end{enumerate}
As concerning MoCs, ForSyDe implements the execution semantics
 \emph{through process constructors}, abstracting the timing model and
 inferring a schedule of the process network. In ForSyDe-Atom all
 atoms embed operating semantics dictated by a certain MoC and are
 side-effect-free. This ensures the functional correctness of a
 system even from early design stages.\par

\subsubsection{Representing Time}
For explicit time representation, ForSyDe-atom provides two
 distinct data types.\par

\begin{haddockdesc}
\item[\begin{tabular}{@{}l}
type\ Time\ =\ Rational
\end{tabular}]\haddockbegindoc
Type alias for the type to represent metric (continuous)
 time. Underneath we use \haddockid{Rational} that is able to represent any
 \emph{t} between \emph{t₁} < \emph{t₂} ∈ \emph{T}.\par


\item[\begin{tabular}{@{}l}
type\ TimeStamp\ =\ DiffTime
\end{tabular}]\haddockbegindoc
Alias for the type representing discrete time. It is inherently
 quantizable, the quantum being a picosecond (10⁻¹²
 seconds), thus it can be considered order-isomorphic with a set of
 integers, i.e. between any two timestamps there is a finite number
 of timestamps. Moreover, a timestamp can be easily translated into
 a rational number representing fractions of a second, so the
 conversion between timestamps (discrete time) and rationals
 (analog/continuous time) is straightforward.\par
This type is used in the explicit tags of the
 \haddockid{DE} MoC (and subsequently the discrete event
 evaluation engine for simulating the \haddockid{CT} MoC).\par

\end{haddockdesc}
\subsubsection{MoC Layer Overview}
This layer consists of:\par
\begin{itemize}
\item
4 atoms as infix operators, implemented as methods of the type
 class \haddockid{MoC}. Since each MoC is determined by its tag system, we
 expose this 
 which are instances of this class. Thus an event's type will
 trigger an atom to behave in accordance to its associated MoC.\par

\item
a library of meaningful atom patterns as process constructors.
 (\emph{Check the \haddocktt{ForSyDe.Atom.MoC} module for extensive}
 \emph{documentation}).\par

\item
a set of data types defining tag systems through the structure
 of events (i.e. \emph{T} × \emph{V}). They are instances of the \haddockid{MoC}
 type class and define the rules of execution that will trigger an
 atom to behave in accordance to its associated MoC. For each
 supported MoC, \haddocktt{forsyde-atom} provides a module which defines the
 signal (event) type, but also a set of utilities and process
 constructors as specific instantiations of atom patterns.
 (\emph{Check the links in the \hyperref[section.i:MoC]{instances} section for}
 \emph{extensive documentation}).\par

\end{itemize}

\begin{haddockdesc}
\item[\begin{tabular}{@{}l}
class\ Applicative\ e\ =>\ MoC\ e\ where
\end{tabular}]\haddockbegindoc
This is a type class defining interfaces for the MoC layer
 atoms. Each model of computation exposes its tag system through a
 unique event constructor as an instance of this class, which
 defines \emph{T} × \emph{V}.\par
 To express all possible MoCs which can be described using
 a \emph{tagged} \emph{signal} \emph{model} we need to capture the most general
 form of their atoms. Recall that all atoms in the layered framework
 are represented as higher-order functions on structured types
 (instances of this class), taking functions of other (lower) layers
 as arguments. While this principle stands also for this layer, the
 functions taken as arguments need to be formatted for each MoC in
 particular in order to capture additional information, which we can
 call in general terms as the \emph{execution context}.\par
One typical example of additional information is the consumption
 and production rates of for data flow MoCs (e.g. SDF). In this case
 the passed functions are defined over "partitions" of events,
 i.e. groupings of events with the same partial order in relation
 to, for example, a process firing. The formal description of such a
 "formatted function" taken as argument by a MoC entity is:\par
\haddockeq{fig/eqs-moc-atom-formatted-func.pdf}\par
where \emph{a} and \emph{b} might be Cartesian products of different types,
 corresponding to how many signals the constructor is applied to or
 how many signals it yields, and each type is expressed as:\par
\haddockeq{fig/eqs-moc-atom-formatted-arg.pdf}\par
While, as you can see above, the execution context can be extracted
 from the type information, working with type-level parameters is
 not a trivial task in Haskell, especially if we want to describe a
 general and extensible type class. This is why we have chosen a
 pragmatic approach in implementing the \haddockid{MoC} class:\par
\begin{itemize}
\item
any (possible) Cartesian product of \emph{α} is represented using
 a recursive type, namely a list {\char 91}\emph{α}{\char 93}.\par

\item
as the execution context cannot (or can hardly) be extracted from
 the recursive type, in the most general case we pass both context
 \emph{and} argument as a pair (see each instance in particular). To aid
 in pairing contexts with each argument in a function, the \haddocktt{ctxt}
 utilities are provided (see \haddockid{ctxt22}).\par

\item
this artifice was masked using the generic type families \haddockid{Fun}
 and \haddocktt{Res}. \par

\end{itemize}

\haddockpremethods{}\textbf{Methods}
\begin{haddockdesc}
\item[\begin{tabular}{@{}l}\ \haddockid{Fun}\ e\ a\ b\ \end{tabular}]
\haddockbegindoc
This is a type family alias for a context-bound function passed as an argument to a MoC atom. In the most simple case it can be regarded as an enhanced -> type operator. While hiding the explicit definition of arguments, this implementation choice certainly has its advantages in avoiding unnecessary or redundant type constructors (see version 0.1.1 and prior). Aliases are replaced at compile time, thus not affecting run-time performance.\haddockeq{fig/eqs-moc-atom-function.pdf}\par

\item[\begin{tabular}{@{}l}\ \haddockid{Ret}\ e\ b\ \end{tabular}]
\haddockbegindoc
As with \haddockid{Fun}, this alias hides a context-bound value (e.g. function return). Although the definition seems to be redundant with \haddockid{Fun}, this alias is needed for utilities to recreate clean types again (see \haddockid{-*}).\haddockeq{fig/eqs-moc-atom-result.pdf}\par

\item[\begin{tabular}{@{}l}\haddockid{(-.-)}\ ::\ Fun\ e\ a\ b\ ->\ Stream\ (e\ a)\ ->\ Stream\ (e\ b)\ \end{tabular}]
\haddockbegindoc
This atom is mapping a function on values (in the presence of a context) to a signal, i.e. stream of tagged events. As ForSyDe deals with \emph{determinate}, \emph{functional} processes, this atom defines the (only) \emph{behavior} of a process in rapport to one input signal  \cite{Lee98}.\haddockeq{fig/eqs-moc-atom-dot.pdf}\par

\item[\begin{tabular}{@{}l}\haddockid{(-*-)}\ ::\ Stream\ (e\ (Fun\ e\ a\ b))\ ->\ Stream\ (e\ a)\ ->\ Stream\ (e\ b)\ \end{tabular}]
\haddockbegindoc
This atom synchronizes two signals, one carrying functions on values (in the presence of a context), and the other containing values, during which it applies the former on the latter. As concerning the process created, this atom defines a \emph{relation} between two signals  \cite{Lee98}.\haddockeq{fig/eqs-moc-atom-star.pdf}\par

\item[\begin{tabular}{@{}l}\haddockid{(-*)}\ ::\ Stream\ (e\ (Ret\ e\ b))\ ->\ Stream\ (e\ b)\ \end{tabular}]
\haddockbegindoc
Artificial \emph{utility} which drops the context and/or partitioning yielding a clean signal type.\haddockeq{fig/eqs-moc-atom-post.pdf}\par

\item[\begin{tabular}{@{}l}\haddockid{(-<-)}\ ::\ Stream\ (e\ a)\ ->\ Stream\ (e\ a)\ ->\ Stream\ (e\ a)\ \end{tabular}]
\haddockbegindoc
This atom appends a (partition of) events at the beginning of a signal. This atom is necessary to ensure \emph{complete partial order} of a signal and assures the \emph{least upper bound} necessary for example in the evaluation of feedback loops  \cite{Lee98}.\haddockeq{fig/eqs-moc-atom-pre.pdf}Notice the difference between the formal and the implemented type signatures. In the implementation the value/partition is wrapped inside an event type to enable smooth composition. You might also notice the type for the initial event(s) as being wrapped inside a signal constructor. This allows defining an DSL for this layer which is centered around signals exclusively, while also enabling to define atoms as homomorphisms to certain extent  \cite{Bird97}. Certain MoCs might have additional constraints on the first operand to be finite.\par

\item[\begin{tabular}{@{}l}\haddockid{(-\&-)}\ ::\ Stream\ (e\ a)\ ->\ Stream\ (e\ a)\ ->\ Stream\ (e\ a)\ \end{tabular}]
\haddockbegindoc
This atom allows the manipulation of tags in a signal in a restrictive way which preserves \emph{monotonicity} and \emph{continuity} in a process  \cite{Lee98}, namely by phase-shifting all tags in a signal with the appropriate metric corresponding to each MoC. Thus it preserves the characteristic function intact  \cite{Sander04}.\haddockeq{fig/eqs-moc-atom-phi.pdf}As with the \haddockid{-<-} atom, we can justify the type signature for smooth composition and the definition of atoms as homomorphisms to certain extent. This in turn allows the interpretation of the \haddockid{-\haddockpremethods{}\textbf{Methods}-} operator as aligning the phases of two signals: the second operand is aligned based on the first.\par

\end{haddockdesc}


\item[\begin{tabular}{@{}l}
instance\ MoC\ SY
\end{tabular}]\haddockbegindoc
Implenents the execution and synchronization semantics for the SY
 MoC through its atoms.\par


\item[\begin{tabular}{@{}l}
instance\ MoC\ SDF
\end{tabular}]\haddockbegindoc
Implenents the SDF semantics for the MoC atoms\par


\item[\begin{tabular}{@{}l}
instance\ MoC\ DE
\end{tabular}]\haddockbegindoc
Implenents the execution and synchronization semantics for the DE
 MoC through its atoms.\par


\item[\begin{tabular}{@{}l}
instance\ MoC\ CT
\end{tabular}]\haddockbegindoc
Implenents the execution and synchronization semantics for the CT
 MoC through its atoms.\par

\end{haddockdesc}
\subsection{The Skeleton Layer}
The skeleton layer describes recursive and regular composition
 of processes which expose inherent potential for parallelism. As
 such, it wraps lower layer entities (i.e. processes, signals),
 into regular structures called \emph{categorical types}. Most of the
 ground work for this layer is based on the categorical type
 theory \cite{Bird97}, which enable the description of
 algorithmic skeletons as high-level constructs encapsulating
 parallelism and communication with an associated cost complexity.\par
This layer provides:\par
\begin{itemize}
\item
3 atoms as infix operators which, as demonstrated in \cite{Bird97}
 and \cite{Skillicorn05}, are enough to describe \emph{all}
 algorithmic skeletons.\par

\item
a library of generic skeletons as specific atom patterns.
 (\emph{Check the \haddocktt{ForSyDe.Atom.Skeleton} module for extensive}
 \emph{documentation}).\par

\item
a set of different categorical types which implement these
 atoms, as instances of the \haddockid{Skeleton} type class. These types
 provide additional skeletons patterns of atoms which takes as
 arguments their own type constructors.
 (\emph{Check the links in the \hyperref[section.i:Skeleton]{instances} section for}
 \emph{extensive documentation}).\par

\end{itemize}

\begin{haddockdesc}
\item[\begin{tabular}{@{}l}
class\ Functor\ c\ =>\ Skeleton\ c\ where
\end{tabular}]\haddockbegindoc
Class containing all the Skeleton layer atoms.\par
This class is instantiated by a set of categorical types,
 i.e. types which describe an inherent potential for being evaluated
 in parallel. Skeletons are patterns from this layer. When skeletons
 take as arguments entities from the MoC layer (i.e. processes), the
 results themselves are parallel process networks which describe
 systems with an inherent potential to be implemented on parallel
 platforms. All skeletons can be described as composition of the
 three atoms below (\haddockid{=<<=} being just a specific instantiation of
 \haddockid{={\char '134}=}). This possible due to an existing theorem in the categorical
 type theory, also called the Bird-Merteens formalism
 \cite{Bird97}:\par
\par
\begin{description}
\item[factorization] A function on a categorical type is an algorithmic
 skeleton (i.e. catamorphism) \emph{iff} it can be represented in a
 factorized form, i.e. as a \emph{map} composed with a \emph{reduce}.
\end{description}Consequently, most of the skeletons for the implemented categorical
 types are described in their factorized form, taking as arguments
 either:\par
                 \begin{itemize}
                 \item
                 type constructors or functions derived from type constructors\par
                 
                 \item
                 processes, i.e. MoC layer entities\par
                 
                 \end{itemize}
                 Most of the ground-work on algorithmic skeletons on which this
 module is founded has been laid in \cite{Bird97},
 \cite{Skillicorn05} and it founds many
 of the frameworks collected in \cite{Gorlatch03}.\par
                 
\haddockpremethods{}\textbf{Methods}
\begin{haddockdesc}
\item[\begin{tabular}{@{}l}\haddockid{(=.=)}\ ::\ (a\ ->\ b)\ ->\ c\ a\ ->\ c\ b\ \end{tabular}]
\haddockbegindoc
Atom which maps a function on each element of a structure (i.e. categorical type), defined as:\haddockeq{fig/eqs-skel-atom-dot.pdf}\haddockid{=.=} together with \haddockid{=*=} form the map pattern.\par

\item[\begin{tabular}{@{}l}\haddockid{(=*=)}\ ::\ c\ (a\ ->\ b)\ ->\ c\ a\ ->\ c\ b\ \end{tabular}]
\haddockbegindoc
Atom which applies the functions contained by as structure (i.e. categorical type), on the elements of another structure, defined as:\haddockeq{fig/eqs-skel-atom-star.pdf}\haddockid{=.=} together with \haddockid{=*=} form the map pattern.\par

\item[\begin{tabular}{@{}l}\haddockid{(=\textbackslash=)}\ ::\ (a\ ->\ a\ ->\ a)\ ->\ c\ a\ ->\ a\ \end{tabular}]
\haddockbegindoc
Atom which reduces a structure to an element based on an \emph{associative} function, defined as:\haddockeq{fig/eqs-skel-atom-red.pdf}\par

\item[\begin{tabular}{@{}l}\haddockid{(=<<=)}\ \end{tabular}]
\haddockbegindoc
Skeleton which \emph{pipes} an element through all the functions contained by a structure. N.B.: this is not an atom. It has an implicit definition which might be augmented by instances of this class to include edge cases.\haddockeq{fig/eqs-skel-pattern-pipe.pdf}As the composition operation is not associative, we cannot treat pipe as a true reduction. Alas, it can still be exploited in parallel since it exposes another type of parallelism: time parallelism.\par

\item[\begin{tabular}{@{}l}\haddockid{first}\ ::\ c\ a\ ->\ a\ \end{tabular}]
\haddockbegindoc
Returns the first element in a structure.N.B.: this is not an atom. It has an implicit definition which might be replaced by instances of this class with a more efficient implementation.\haddockeq{fig/eqs-skel-pattern-first.pdf}\par

\item[\begin{tabular}{@{}l}\haddockid{last}\ ::\ c\ a\ ->\ a\ \end{tabular}]
\haddockbegindoc
Returns the last element in a structure.N.B.: this is not an atom. It has an implicit definition which might be replaced by instances of this class with a more efficient implementation.\haddockeq{fig/eqs-skel-pattern-last.pdf}\par

\end{haddockdesc}


\item[\begin{tabular}{@{}l}
instance\ Skeleton\ Vector
\end{tabular}]\haddockbegindoc
Ensures that \haddockid{Vector} is a structure associated with the Skeleton Layer.\par

\end{haddockdesc}

\subsection{Utilities}
The \haddockid{Atom} module exports a set of utility functions,
 mainly for aiding the designer to avoid working with zipped
 tuples which might pollute the design. Utilities are function
 without any semantical value (thus not considered atoms). They
 operate on and might alter the \emph{structure} of some datum, but it
 does not affect its state.\par
For a list of all the provided utilities, please consult the
 following module:\par

\begin{haddockdesc}
\item[\begin{tabular}{@{}l}
module\ ForSyDe.Atom.Utility
\end{tabular}]
\end{haddockdesc}
Among the most useful utilities we mentions the \haddocktt{unzip}
 function. Recall that in all our definitions for patterns, they
 were expressed in the most general form as functions from \emph{n}-ary
 Cartesian products to \emph{m}-ary Cartesian products. While partial
 application provides a versatile mechanism that can translate
 n-ary inputs into curried arguments (which is very powerful in
 combination with an applicative style), we cannot do so for
 return types. For the latter we must rely on tuples. But working
 with tuples of data wrapped in several layers of structures
 becomes extremely cumbersome. Take for example the case of a
 process constructed with \emph{pc} in equation (1) below. Using only
 the provided atoms to implement \emph{pc} would give us a process
 which returns only one signal of a tuple and not, as we would
 like, a tuple of signals of events.\par
\haddockeq{fig/misc-unzip.pdf}\par
Therefore, by implementing all data types associated with signals
 and events as instances of \haddockid{Functor}, we were able to provide a
 (set of) \emph{unzip} utility functions defined as in equation (2)
 above, in the \haddocktt{ForSyDe.Atom.Utility} module.  Mind that we call
 \emph{unzip} a utility and not an atom, since it has no
 synchronization nor behavior semantic. It just conveniently
 "lifts" the wrapped tuples in order to create "collections" of
 events and signals, and it is imposed by the mechanisms of the
 type system in the host language.\par

\begin{haddockdesc}
\item[\begin{tabular}{@{}l}
(||<)\ ::\ (Functor\ f1,\ Functor\ f2)\ =>\\\ \ \ \ \ \ \ \ \ f1\ (f2\ (a1,\ a2))\ ->\ (f1\ (f2\ a1),\ f1\ (f2\ a2))
\end{tabular}]\haddockbegindoc
This set of utility functions "unzip" nested n-tuples, provided
 as postfix operators. They are crucial for reconstructing data
 types from higher-order functions which input functions with
 multiple outputs. It relies on the nested types being instances of
 \haddockid{Functor}.\par
The operator convention is \haddocktt{(|+<+)}, where the number of \haddocktt{|}
 represent the number of layers the n-tuple is lifted, while the
 number of \haddocktt{<} + 1 is the order \emph{n} of the n-tuple.\par
\haddocktt{ForSyDe.Atom.Utility} exports the constructors below. Please
 follow the examples in the source code if they do not suffice:\par
\begin{code}
   |<,    |<<,    |<<<,    |<<<<,    |<<<<<,    |<<<<<<,    |<<<<<<<,    |<<<<<<<<,
  ||<,   ||<<,   ||<<<,   ||<<<<,   ||<<<<<,   ||<<<<<<,   ||<<<<<<<,   ||<<<<<<<<,
 |||<,  |||<<,  |||<<<,  |||<<<<,  |||<<<<<,  |||<<<<<<,  |||<<<<<<<,  |||<<<<<<<<,  
||||<, ||||<<, ||||<<<, ||||<<<<, ||||<<<<<, ||||<<<<<<, ||||<<<<<<<, ||||<<<<<<<<, \end{code}
Example:\par
\begin{interactive}
λ> :set -XPostfixOperators
λ> ([Just (1,2,3), Nothing, Just (4,5,6)] ||<<)
([Just 1,Nothing,Just 4],[Just 2,Nothing,Just 5],[Just 3,Nothing,Just 6])

\end{interactive}
\end{haddockdesc}

  \vspace{3ex}
\haddockmoduleheading{ForSyDe.Atom.ExB}
\label{module:ForSyDe.Atom.ExB}
\haddockbeginheader
{\haddockverb\begin{verbatim}
module ForSyDe.Atom.ExB (
    ExB(extend, (/.\), (/*\), (/&\), (/!\)),  res22,  filter,  filter', 
    degrade,  ignore22
  ) where\end{verbatim}}
\haddockendheader

This module exports the core entities of the extended behavior
 layer: interfaces for atoms and common patterns of atoms. It does
 \emph{NOT} export any implementation or instantiation of any specific
 behavior extension type.\par
\begin{mdframed}[style=reminder,frametitle=Reminder]Make sure to consult naming conventions in  \cref{sec:forsyde-atom:naming-convention} in order to interpret the names and type signatures correctly.\end{mdframed}\par

\subsection{Atoms}
\begin{haddockdesc}
\item[\begin{tabular}{@{}l}
class\ Functor\ b\ =>\ ExB\ b\ where
\end{tabular}]\haddockbegindoc
Class which defines the atoms for the extended behavior layer.\par
As its name suggests, this layer is extending the behavior of
 processes (or merely of functions if we ignore timing semantics
 completely), by expanding the domains of the wrapped layer
 (e.g. the set of values) with symbols having clearly-defined
 semantics (e.g. special events with known responses).\par
The types associated with this layer can simply be describes as:\par
\haddockeq{fig/eqs-exb-types.pdf}\par
where  \emph{α} is a base type and \emph{b} is the type extension,
 i.e. a set of symbols with clearly-defined semantics.\par
Extended behavior atoms are functions of these types, defined as
 interfaces in the \haddockid{ExB} type class.\par

\haddockpremethods{}\textbf{Methods}
\begin{haddockdesc}
\item[\begin{tabular}{@{}l}\haddockid{extend}\ ::\ a\ ->\ b\ a\ Source\ \end{tabular}]
\haddockbegindoc
Extends a value (from a layer below) with a set of symbols with known semantics, as described by a type instantiating this class.\par

\item[\begin{tabular}{@{}l}\haddockid{(/.\textbackslash)}\ ::\ (a\ ->\ a)\ ->\ b\ a\ ->\ b\ a\end{tabular}]
\haddockbegindoc
Basic functor operator. Lifts a function (from a layer below) into the domain of the extended behavior layer.\haddockeq{fig/eqs-exb-atom-func.pdf}\par

\item[\begin{tabular}{@{}l}\haddockid{(/*\textbackslash)}\ ::\ b\ (a\ ->\ a)\ ->\ b\ a\ ->\ b\ a\end{tabular}]
\haddockbegindoc
Applicative operator. Defines a res between two extended behavior symbols.\haddockeq{fig/eqs-exb-atom-app.pdf}\par

\item[\begin{tabular}{@{}l}\haddockid{(/\&\textbackslash)}\ ::\ b\ Bool\ ->\ b\ a\ ->\ b\ a\ Source\ \end{tabular}]
\haddockbegindoc
Predicate operator. Generates a defined behavior based on an extended Boolean predicate.\haddockeq{fig/eqs-exb-atom-phi.pdf}\par

\item[\begin{tabular}{@{}l}\haddockid{(/"!\textbackslash)}\ ::\ a\ ->\ b\ a\ ->\ a\ Source\ \end{tabular}]
\haddockbegindoc
Degrade operator. Degrades a behavior-extended value into a non-extended one (from a layer below), based on a kernel value. Used also to throw exceptions.\haddockeq{fig/eqs-exb-atom-deg.pdf}\par

\end{haddockdesc}


\item[\begin{tabular}{@{}l}
instance\ ExB\ AbstExt
\end{tabular}]\haddockbegindoc
Implements the absent semantics of the extended behavior atoms.\par

\end{haddockdesc}
\subsection{Patterns}
\begin{haddockdesc}
\item[\begin{tabular}{@{}l}
res22
\end{tabular}]\haddockbegindoc
\haddockbeginargs
\haddockdecltt{::} & ExB b \\
                     \haddockdecltt{=>} & \haddockdecltt{(a1
                                                          -> a2
                                                             -> (a1', a2'))} & function on values \\
                                                                               \haddockdecltt{->} & \haddockdecltt{b a1} & first input \\
                                                                                                                           \haddockdecltt{->} & \haddockdecltt{b a2} & second input \\
                                                                                                                                                                       \haddockdecltt{->} & \haddockdecltt{(b a1', b a2')} & tupled output \\
\end{tabulary}\par
\haddockeq{fig/eqs-exb-pattern-resolution.pdf}\par
The \haddocktt{res} behavior pattern lifts a function on values to the
 extended behavior domain, and applies a resolution between two
 extended behavior symbols.\par
Constructors: \haddocktt{res{\char 91}1-8{\char 93}{\char 91}1-4{\char 93}}.\par


\item[\begin{tabular}{@{}l}
filter\ ::\ ExB\ b\ =>\ b\ Bool\ ->\ b\ a\ ->\ b\ a
\end{tabular}]\haddockbegindoc
Prefix name for the prefix operator \haddockid{/{\char '46}{\char '134}}.\par


\item[\begin{tabular}{@{}l}
filter'\ ::\ ExB\ b\ =>\ Bool\ ->\ a\ ->\ b\ a
\end{tabular}]\haddockbegindoc
Same as \haddockid{filter} but takes base (non-extended) values as
 input arguments.\par


\item[\begin{tabular}{@{}l}
degrade\ ::\ ExB\ b\ =>\ a\ ->\ b\ a\ ->\ a
\end{tabular}]\haddockbegindoc
Prefix name for the degrade operator \haddockid{/"!{\char '134}}.\par


\item[\begin{tabular}{@{}l}
ignore22
\end{tabular}]\haddockbegindoc
\haddockbeginargs
\haddockdecltt{::} & ExB b \\
                     \haddockdecltt{=>} & \haddockdecltt{(a1
                                                          -> a2
                                                             -> a1'
                                                                -> a2'
                                                                   -> (a1, a2))} & function of \haddocktt{Y\ +\ X} arguments \\
                                                                                   \haddockdecltt{->} & \haddockdecltt{a1} & \\
                                                                                                                             \haddockdecltt{->} & \haddockdecltt{a2} & \\
                                                                                                                                                                       \haddockdecltt{->} & \haddockdecltt{b a1'} & \\
                                                                                                                                                                                                                    \haddockdecltt{->} & \haddockdecltt{b a2'} & \\
                                                                                                                                                                                                                                                                 \haddockdecltt{->} & \haddockdecltt{(a1, a2)} & \\
\end{tabulary}\par
\haddockeq{fig/eqs-exb-pattern-ignore.pdf}\par
The \haddocktt{ignoreXY} pattern takes a function of \haddocktt{Y\ +\ X} arguments, \haddocktt{Y}
 basic inputs followed by \haddocktt{X} behavior-extended inputs. The \haddocktt{X}
 behavior-extended arguments are subjugated to a res, and the
 result is then degraded using the first \haddocktt{Y} arguments as
 fallback. The effect is similar to "ignoring" a the result of a
 res function if ∈ \emph{b}.\par
The main application of this pattern is as extended behavior
 wrapper for state machine functions which do not "understand"
 extended behavior semantics, i.e. it simply propagates the current
 state (∈ \emph{α}) if the inputs (their res) belongs
 to the set of extended values (∈ \emph{b}).\par
Constructors: \haddocktt{ignore{\char 91}1-4{\char 93}{\char 91}1-4{\char 93}}.\par

\end{haddockdesc}
  \haddockmoduleheading{ForSyDe.Atom.ExB.Absent}
\label{module:ForSyDe.Atom.ExB.Absent}
\haddockbeginheader
{\haddockverb\begin{verbatim}
module ForSyDe.Atom.ExB.Absent (
    AbstExt(Abst, Prst) 
  ) where\end{verbatim}}
\haddockendheader

This module implements the constructors and assocuated utilities of
 a type which extends the behavior of a function to express "absent
 events" (see \cite{Halbwachs91}).\par
The \haddockid{AbstExt} type can be used directly with the atom patterns
 defined in \haddocktt{ForSyDe.Atom.ExB}, and no helpers or utilities are
 needed. Example usage:\par
\begin{interactive}
λ> res21 (+) (Prst 1) (Prst 2)
3
λ> res21 (+) Abst     Abst
⟂
λ> filter Abst         (Prst 1)
⟂
λ> filter (Prst False) (Prst 1)
⟂
λ> filter (Prst True)  (Prst 1)
1
λ> filter' False 1 :: AbstExt Int
⟂
λ> filter' True  1 :: AbstExt Int
1
λ> degrade 0 (Prst 1)
1
λ> degrade 0 Abst
0
λ> ignore11 (+) 1 (Prst 1)
2
λ> ignore11 (+) 1 Abst
1
λ> res21 (+) (Prst 1) Abst 
*** Exception: [ExB.Absent] Illegal occurrence of an absent and present event
\end{interactive}
           
\begin{haddockdesc}
\item[\begin{tabular}{@{}l}
data\ AbstExt\ a
\end{tabular}]\haddockbegindoc
\haddockbeginconstrs
\haddockdecltt{=} & \haddockdecltt{Abst} & ⊥ denotes the absence of a value \\
\haddockdecltt{|} & \haddockdecltt{Prst a} & ⊤ a present event with a value \\
\end{tabulary}\par

The \haddockid{AbstExt} type extends the base type with the '⊥'
 symbol denoting the absence of a value/event (see
 \cite{Halbwachs91}).\par


\item[\begin{tabular}{@{}l}
instance\ Functor\ AbstExt
\end{tabular}]\haddockbegindoc
\haddockid{Functor} instance. Bypasses the special values and maps a
 function to the wrapped value. Provides the \haddocktt{(<{\char '44}>)} operator.\par


\item[\begin{tabular}{@{}l}
instance\ Applicative\ AbstExt
\end{tabular}]\haddockbegindoc
\haddockid{Applicative} instance. Check source code for the lifting
 rules. Provides the \haddocktt{(<*>)} operator\par


\item[\begin{tabular}{@{}l}
instance\ ExB\ AbstExt
\end{tabular}]\haddockbegindoc
Implements the absent semantics of the extended behavior atoms.\par


\item[\begin{tabular}{@{}l}
instance\ Eq\ a\ =>\ Eq\ (AbstExt\ a)
\end{tabular}]

\item[\begin{tabular}{@{}l}
instance\ Read\ a\ =>\ Read\ (AbstExt\ a)
\end{tabular}]\haddockbegindoc
Reads the '{\char '137}' character to an \haddockid{Abst} and a normal value to
 \haddockid{Prst}-wrapped one.\par


\item[\begin{tabular}{@{}l}
instance\ Show\ a\ =>\ Show\ (AbstExt\ a)
\end{tabular}]\haddockbegindoc
Shows \haddockid{Abst} as '⊥', while a present event is represented
 with its value.\par


\item[\begin{tabular}{@{}l}
instance\ (Show\ a,\ Plottable\ a)\ =>\ Plottable\ (AbstExt\ a)
\end{tabular}]\haddockbegindoc
Absent-extended plottable types\par

\end{haddockdesc}
  \haddockmoduleheading{ForSyDe.Atom.MoC}
\label{module:ForSyDe.Atom.MoC}
\haddockbeginheader
{\haddockverb\begin{verbatim}
module ForSyDe.Atom.MoC (
    MoC(Fun, Ret, (-.-), (-*-), (-*), (-<-), (-&-)),  delay,  comb22, 
    reconfig22,  state22,  stated22,  moore22,  mealy22,  ctxt22,  warg,  wres, 
    (-*<)
  ) where\end{verbatim}}
\haddockendheader

This module exports the core entities of the MoC layer: interfaces
 for atoms and process constructors as patterns of atoms. It does
 \emph{NOT} export any implementation or instantiation of any specific
 MoC.\par
Current MoC implementations can be used by importing their
 respective modules:\par
\begin{itemize}
\item
\haddocktt{ForSyDe.Atom.MoC.CT}\par

\item
\haddocktt{ForSyDe.Atom.MoC.DE}\par

\item
\haddocktt{ForSyDe.Atom.MoC.SY}\par

\item
\haddocktt{ForSyDe.Atom.MoC.SDF}\par

\end{itemize}
\begin{mdframed}[style=reminder,frametitle=Reminder]Make sure to consult naming conventions in  \cref{sec:forsyde-atom:naming-convention} in order to interpret the names and type signatures correctly.\end{mdframed}\par

\subsection{Atoms}
\begin{haddockdesc}
\item[\begin{tabular}{@{}l}
class\ Applicative\ e\ =>\ MoC\ e\ where
\end{tabular}]\haddockbegindoc
This is a type class defining interfaces for the MoC layer
 atoms. Each model of computation exposes its tag system through a
 unique event constructor as an instance of this class, which
 defines \emph{T} × \emph{V}.\par
 To express all possible MoCs which can be described using
 a \emph{tagged} \emph{signal} \emph{model} we need to capture the most general
 form of their atoms. Recall that all atoms in the layered framework
 are represented as higher-order functions on structured types
 (instances of this class), taking functions of other (lower) layers
 as arguments. While this principle stands also for this layer, the
 functions taken as arguments need to be formatted for each MoC in
 particular in order to capture additional information, which we can
 call in general terms as the \emph{execution context}.\par
One typical example of additional information is the consumption
 and production rates of for data flow MoCs (e.g. SDF). In this case
 the passed functions are defined over "partitions" of events,
 i.e. groupings of events with the same partial order in relation
 to, for example, a process firing. The formal description of such a
 "formatted function" taken as argument by a MoC entity is:\par
\haddockeq{fig/eqs-moc-atom-formatted-func.pdf}\par
where \emph{a} and \emph{b} might be Cartesian products of different types,
 corresponding to how many signals the constructor is applied to or
 how many signals it yields, and each type is expressed as:\par
\haddockeq{fig/eqs-moc-atom-formatted-arg.pdf}\par
While, as you can see above, the execution context can be extracted
 from the type information, working with type-level parameters is
 not a trivial task in Haskell, especially if we want to describe a
 general and extensible type class. This is why we have chosen a
 pragmatic approach in implementing the \haddockid{MoC} class:\par
\begin{itemize}
\item
any (possible) Cartesian product of \emph{α} is represented using
 a recursive type, namely a list {\char 91}\emph{α}{\char 93}.\par

\item
as the execution context cannot (or can hardly) be extracted from
 the recursive type, in the most general case we pass both context
 \emph{and} argument as a pair (see each instance in particular). To aid
 in pairing contexts with each argument in a function, the \haddocktt{ctxt}
 utilities are provided (see \haddockid{ctxt22}).\par

\item
this artifice was masked using the generic type families \haddockid{Fun}
 and \haddocktt{Res}. \par

\end{itemize}

\haddockpremethods{}\textbf{Methods}
\begin{haddockdesc}
\item[\begin{tabular}{@{}l}\ \haddockid{Fun}\ e\ a\ b\ Source\ \end{tabular}]
\haddockbegindoc
This is a type family alias for a context-bound function passed as an argument to a MoC atom. In the most simple case it can be regarded as an enhanced -> type operator. While hiding the explicit definition of arguments, this implementation choice certainly has its advantages in avoiding unnecessary or redundant type constructors (see version 0.1.1 and prior). Aliases are replaced at compile time, thus not affecting run-time performance.\haddockeq{fig/eqs-moc-atom-function.pdf}\par

\item[\begin{tabular}{@{}l}\ \haddockid{Ret}\ e\ b\ Source\ \end{tabular}]
\haddockbegindoc
As with \haddockid{Fun}, this alias hides a context-bound value (e.g. function return). Although the definition seems to be redundant with \haddockid{Fun}, this alias is needed for utilities to recreate clean types again (see \haddockid{-*}).\haddockeq{fig/eqs-moc-atom-result.pdf}\par

\item[\begin{tabular}{@{}l}\haddockid{(-.-)}\ ::\ Fun\ e\ a\ b\ ->\ Stream\ (e\ a)\ ->\ Stream\ (e\ b)\ Source\ \end{tabular}]
\haddockbegindoc
This atom is mapping a function on values (in the presence of a context) to a signal, i.e. stream of tagged events. As ForSyDe deals with \emph{determinate}, \emph{functional} processes, this atom defines the (only) \emph{behavior} of a process in rapport to one input signal  \cite{Lee98}.\haddockeq{fig/eqs-moc-atom-dot.pdf}\par

\item[\begin{tabular}{@{}l}\haddockid{(-*-)}\ ::\ Stream\ (e\ (Fun\ e\ a\ b))\ ->\ Stream\ (e\ a)\ ->\ Stream\ (e\ b)\ Source\ \end{tabular}]
\haddockbegindoc
This atom synchronizes two signals, one carrying functions on values (in the presence of a context), and the other containing values, during which it applies the former on the latter. As concerning the process created, this atom defines a \emph{relation} between two signals  \cite{Lee98}.\haddockeq{fig/eqs-moc-atom-star.pdf}\par

\item[\begin{tabular}{@{}l}\haddockid{(-*)}\ ::\ Stream\ (e\ (Ret\ e\ b))\ ->\ Stream\ (e\ b)\ Source\ \end{tabular}]
\haddockbegindoc
Artificial \emph{utility} which drops the context and/or partitioning yielding a clean signal type.\haddockeq{fig/eqs-moc-atom-post.pdf}\par

\item[\begin{tabular}{@{}l}\haddockid{(-<-)}\ ::\ Stream\ (e\ a)\ ->\ Stream\ (e\ a)\ ->\ Stream\ (e\ a)\ Source\ \end{tabular}]
\haddockbegindoc
This atom appends a (partition of) events at the beginning of a signal. This atom is necessary to ensure \emph{complete partial order} of a signal and assures the \emph{least upper bound} necessary for example in the evaluation of feedback loops  \cite{Lee98}.\haddockeq{fig/eqs-moc-atom-pre.pdf}Notice the difference between the formal and the implemented type signatures. In the implementation the value/partition is wrapped inside an event type to enable smooth composition. You might also notice the type for the initial event(s) as being wrapped inside a signal constructor. This allows defining an DSL for this layer which is centered around signals exclusively, while also enabling to define atoms as homomorphisms to certain extent  \cite{Bird97}. Certain MoCs might have additional constraints on the first operand to be finite.\par

\item[\begin{tabular}{@{}l}\haddockid{(-\&-)}\ ::\ Stream\ (e\ a)\ ->\ Stream\ (e\ a)\ ->\ Stream\ (e\ a)\ Source\ \end{tabular}]
\haddockbegindoc
This atom allows the manipulation of tags in a signal in a restrictive way which preserves \emph{monotonicity} and \emph{continuity} in a process  \cite{Lee98}, namely by phase-shifting all tags in a signal with the appropriate metric corresponding to each MoC. Thus it preserves the characteristic function intact  \cite{Sander04}.\haddockeq{fig/eqs-moc-atom-phi.pdf}As with the \haddockid{-<-} atom, we can justify the type signature for smooth composition and the definition of atoms as homomorphisms to certain extent. This in turn allows the interpretation of the \haddockid{-\haddockpremethods{}\textbf{Methods}-} operator as aligning the phases of two signals: the second operand is aligned based on the first.\par

\end{haddockdesc}


\item[\begin{tabular}{@{}l}
instance\ MoC\ SY
\end{tabular}]\haddockbegindoc
Implenents the execution and synchronization semantics for the SY
 MoC through its atoms.\par


\item[\begin{tabular}{@{}l}
instance\ MoC\ SDF
\end{tabular}]\haddockbegindoc
Implenents the SDF semantics for the MoC atoms\par


\item[\begin{tabular}{@{}l}
instance\ MoC\ DE
\end{tabular}]\haddockbegindoc
Implenents the execution and synchronization semantics for the DE
 MoC through its atoms.\par


\item[\begin{tabular}{@{}l}
instance\ MoC\ CT
\end{tabular}]\haddockbegindoc
Implenents the execution and synchronization semantics for the CT
 MoC through its atoms.\par

\end{haddockdesc}
\subsection{Process constructors}
Process constructors are defined as patterns of MoC
 atoms. Check the \hyperref[naming_conv]{naming convention} of the API in
 the page description.\par

\begin{haddockdesc}
\item[\begin{tabular}{@{}l}
delay\ ::\ MoC\ e\ =>\ Stream\ (e\ a)\ ->\ Stream\ (e\ a)\ ->\ Stream\ (e\ a)
\end{tabular}]\haddockbegindoc
\haddockeq{fig/eqs-moc-pattern-delay.pdf}
   \haddockfig{fig/moc-pattern-delay.pdf}\par
The \haddockid{delay} process provides both initial token(s) and shifts the
 phase of the signal. In other words, it "delays" a signal with
 one or several events.\par
There is also an infix variant \haddockid{-{\char '46}>-} (\haddocktt{infixl\ 3}). To justify the
 first argument, see the documentation of the \haddockid{-<-} atom.\par


\item[\begin{tabular}{@{}l}
comb22
\end{tabular}]\haddockbegindoc
\haddockbeginargs
\haddockdecltt{::} & MoC e \\
                     \haddockdecltt{=>} & \haddockdecltt{Fun e a1 (Fun e a2 (Ret e b1, Ret e b2))} & combinational function (\hyperref[comb22f]{*}) \\
                                                                                                     \haddockdecltt{->} & \haddockdecltt{Stream (e a1)} & first input signal \\
                                                                                                                                                          \haddockdecltt{->} & \haddockdecltt{Stream (e a2)} & second input signal \\
                                                                                                                                                                                                               \haddockdecltt{->} & \haddockdecltt{(Stream (e b1), Stream (e b2))} & two output signals \\
\end{tabulary}\par
 \emph{(*) to be read } \haddocktt{a1\ ->\ a2\ ->\ (b1,\ b2)} \emph{where each}
 \emph{argument and result might be individually wrapped with a context}
 \emph{and might also express a partition.}\par
\haddockeq{fig/eqs-moc-pattern-comb.pdf}
 \haddockfig{fig/moc-pattern-comb.pdf}\par
The \haddocktt{comb} processes takes care of synchronization between signals
 and maps combinatorial functions on their event values. \par
This library exports constructors of type \haddocktt{comb{\char 91}1-8{\char 93}{\char 91}1-4{\char 93}}.\par


\item[\begin{tabular}{@{}l}
reconfig22
\end{tabular}]\haddockbegindoc
\haddockbeginargs
\haddockdecltt{::} & MoC e \\
                     \haddockdecltt{=>} & \haddockdecltt{Stream (e (Fun e a1 (Fun e a2 (Ret e b1, Ret e b2))))} & signal carrying functions (\hyperref[reconfig22f]{*}) \\
                                                                                                                  \haddockdecltt{->} & \haddockdecltt{Stream (e a1)} & first input signal \\
                                                                                                                                                                       \haddockdecltt{->} & \haddockdecltt{Stream (e a2)} & second input signal \\
                                                                                                                                                                                                                            \haddockdecltt{->} & \haddockdecltt{(Stream (e b1), Stream (e b2))} & two output signals \\
\end{tabulary}\par
 \emph{(*) to be read } \haddocktt{a1\ ->\ a2\ ->\ (b1,\ b2)} \emph{where each}
 \emph{argument and result might be individually wrapped with a context}
 \emph{and might also express a partition.}\par
\haddockeq{fig/eqs-moc-pattern-reconfig.pdf}
 \haddockfig{fig/moc-pattern-reconfig.pdf}\par
The \haddocktt{reconfig} processes constructs adaptive processes, where the
 first signal carries functions, and it is synchronized with all the
 other signals. \par
This library exports constructors of type \haddocktt{reconfig{\char 91}1-8{\char 93}{\char 91}1-4{\char 93}}.\par


\item[\begin{tabular}{@{}l}
state22
\end{tabular}]\haddockbegindoc
\haddockbeginargs
\haddockdecltt{::} & MoC e \\
                     \haddockdecltt{=>} & \haddockdecltt{Fun e st1 (Fun e st2 (Fun e a1 (Fun e a2 (Ret e st1, Ret e st2))))} & next state function (\hyperref[state22ns]{*}) \\
                                                                                                                               \haddockdecltt{->} & \haddockdecltt{(Stream (e st1), Stream (e st2))} & initial state(s) (\hyperref[state22i]{**}) \\
                                                                                                                                                                                                       \haddockdecltt{->} & \haddockdecltt{Stream (e a1)} & first input signal \\
                                                                                                                                                                                                                                                            \haddockdecltt{->} & \haddockdecltt{Stream (e a2)} & second input signal \\
                                                                                                                                                                                                                                                                                                                 \haddockdecltt{->} & \haddockdecltt{(Stream (e st1), Stream (e st2))} & output signals mirroring the next state(s). \\
\end{tabulary}\par
 \emph{(*) meaning } \haddocktt{st1\ ->\ st2\ ->\ a1\ ->\ a2\ ->\ (st1,st2)}
 \emph{where each argument and result might be individually wrapped}
 \emph{with a context and might also express a partition.}\par
 \emph{(**) see the documentation for \haddockid{-<-} for justification}
 \emph{of the type}\par
\haddockeq{fig/eqs-moc-pattern-state.pdf}
 \haddockfig{fig/moc-pattern-state.pdf}\par
The \haddocktt{state} processes generate process networks corresponding to a
 simple state machine like in the graph above. \par
This library exports constructors of type \haddocktt{state{\char 91}1-4{\char 93}{\char 91}1-4{\char 93}}.\par


\item[\begin{tabular}{@{}l}
stated22
\end{tabular}]\haddockbegindoc
\haddockbeginargs
\haddockdecltt{::} & MoC e \\
                     \haddockdecltt{=>} & \haddockdecltt{Fun e st1 (Fun e st2 (Fun e a1 (Fun e a2 (Ret e st1, Ret e st2))))} & next state function (\hyperref[stated22ns]{*}) \\
                                                                                                                               \haddockdecltt{->} & \haddockdecltt{(Stream (e st1), Stream (e st2))} & initial state(s) (\hyperref[stated22i]{**}) \\
                                                                                                                                                                                                       \haddockdecltt{->} & \haddockdecltt{Stream (e a1)} & first input signal \\
                                                                                                                                                                                                                                                            \haddockdecltt{->} & \haddockdecltt{Stream (e a2)} & second input signal \\
                                                                                                                                                                                                                                                                                                                 \haddockdecltt{->} & \haddockdecltt{(Stream (e st1), Stream (e st2))} & output signals mirroring the next state(s). \\
\end{tabulary}\par
 \emph{(*) meaning } \haddocktt{st1\ ->\ st2\ ->\ a1\ ->\ a2\ ->\ (st1,st2)}
 \emph{where each argument and result might be individually wrapped}
 \emph{with a context and might also express a partition.}\par
 \emph{(**) see the documentation for \haddockid{-<-} for justification}
 \emph{of the type}\par
\haddockeq{fig/eqs-moc-pattern-stated.pdf}
 \haddockfig{fig/moc-pattern-stated.pdf}\par
The \haddocktt{state} processes generate process networks corresponding to a
 simple state machine like in the graph above. The difference
 between \haddockid{state22} and \haddockid{stated22} is that the latter outputs the
 current state rather than the next one. There exists a variant with
 0 input signals, in which case the process is a signal
 generator.\par
This library exports constructors of type \haddocktt{stated{\char 91}0-4{\char 93}{\char 91}1-4{\char 93}}.\par


\item[\begin{tabular}{@{}l}
moore22
\end{tabular}]\haddockbegindoc
\haddockbeginargs
\haddockdecltt{::} & MoC e \\
                     \haddockdecltt{=>} & \haddockdecltt{Fun e st (Fun e a1 (Fun e a2 (Ret e st)))} & next state function (\hyperref[moore22ns]{*}) \\
                                                                                                      \haddockdecltt{->} & \haddockdecltt{Fun e st (Ret e b1, Ret e b2)} & output decoder (\hyperref[moore22od]{**}) \\
                                                                                                                                                                           \haddockdecltt{->} & \haddockdecltt{Stream (e st)} & initial state (\hyperref[moore22i]{***}) \\
                                                                                                                                                                                                                                \haddockdecltt{->} & \haddockdecltt{Stream (e a1)} & first input signal \\
                                                                                                                                                                                                                                                                                     \haddockdecltt{->} & \haddockdecltt{Stream (e a2)} & second input signal \\
                                                                                                                                                                                                                                                                                                                                          \haddockdecltt{->} & \haddockdecltt{(Stream (e b1), Stream (e b2))} & output signals \\
\end{tabulary}\par
 \emph{(*) meaning } \haddocktt{st\ ->\ a1\ ->\ a2\ ->\ st\ } \emph{where each}
 \emph{argument and result might be individually wrapped with a context}
 \emph{and might also express a partition.}\par
 \emph{(**) meaning } \haddocktt{st\ ->\ (b1,\ b2)\ } \emph{where each argument}
 \emph{and result might be individually wrapped with a context and might}
 \emph{also express a partition.}\par
 \emph{(***) see the documentation for \haddockid{-<-} for justification}
 \emph{of the type}\par
\haddockeq{fig/eqs-moc-pattern-moore.pdf}
 \haddockfig{fig/moc-pattern-moore.pdf}\par
The \haddocktt{moore} processes model Moore state machines.\par
This library exports constructors of type \haddocktt{moore{\char 91}1-4{\char 93}{\char 91}1-4{\char 93}}.\par


\item[\begin{tabular}{@{}l}
mealy22
\end{tabular}]\haddockbegindoc
\haddockbeginargs
\haddockdecltt{::} & MoC e \\
                     \haddockdecltt{=>} & \haddockdecltt{Fun e st (Fun e a1 (Fun e a2 (Ret e st)))} & next state function (\hyperref[mealy22ns]{*}) \\
                                                                                                      \haddockdecltt{->} & \haddockdecltt{Fun e st (Fun e a1 (Fun e a2 (Ret e b1, Ret e b2)))} & output decoder (\hyperref[mealy22od]{**}) \\
                                                                                                                                                                                                 \haddockdecltt{->} & \haddockdecltt{Stream (e st)} & initial state (\hyperref[mealy22i]{***}) \\
                                                                                                                                                                                                                                                      \haddockdecltt{->} & \haddockdecltt{Stream (e a1)} & first input signal \\
                                                                                                                                                                                                                                                                                                           \haddockdecltt{->} & \haddockdecltt{Stream (e a2)} & second input signal \\
                                                                                                                                                                                                                                                                                                                                                                \haddockdecltt{->} & \haddockdecltt{(Stream (e b1), Stream (e b2))} & output signals \\
\end{tabulary}\par
 \emph{(*) meaning } \haddocktt{st\ ->\ a1\ ->\ a2\ ->\ st\ } \emph{where each}
 \emph{argument and result might be individually wrapped with a context}
 \emph{and might also express a partition.}\par
 \emph{(**) meaning } \haddocktt{st\ ->\ a1\ ->\ a2\ ->\ (b1,\ b2)\ } \emph{where}
 \emph{each argument and result might be individually wrapped with a}
 \emph{context and might also express a partition.}\par
 \emph{(***) see the documentation for \haddockid{-<-} for justification}
 \emph{of the type}\par
\haddockeq{fig/eqs-moc-pattern-mealy.pdf}
 \haddockfig{fig/moc-pattern-mealy.pdf}\par
The \haddocktt{mealy} processes model Mealy state machines.\par
This library exports constructors of type \haddocktt{mealy{\char 91}1-4{\char 93}{\char 91}1-4{\char 93}}.\par

\end{haddockdesc}
\subsection{Utilities}
\begin{haddockdesc}
\item[\begin{tabular}{@{}l}
ctxt22
\end{tabular}]\haddockbegindoc
\haddockbeginargs
\haddockdecltt{::} & \haddockdecltt{(ctx, ctx)} & argument contexts (e.g. consumption rates in SDF) \\
                                                  \haddockdecltt{->} & \haddockdecltt{(ctx, ctx)} & result contexts (e.g. production rates in SDF) \\
                                                                                                    \haddockdecltt{->} & \haddockdecltt{(a1
                                                                                                                                         -> a2
                                                                                                                                            -> (b1, b2))} & function on values/partitions of values \\
                                                                                                                                                            \haddockdecltt{->} & \haddockdecltt{(ctx, a1
                                                                                                                                                                                                      -> (ctx, a2
                                                                                                                                                                                                               -> ((ctx, b1), (ctx, b2))))} & context-wrapped form of the previous function \\
\end{tabulary}\par
\haddockeq{fig/eqs-moc-atom-context.pdf}\par
Wraps a function with the context needed by some MoCs for their
 constructors (e.g. rates in SDF).\par
This library exports wrappers of type \haddocktt{ctxt{\char 91}1-8{\char 93}{\char 91}1-4{\char 93}}.\par


\item[\begin{tabular}{@{}l}
warg\ ::\ c\ ->\ (a\ ->\ b)\ ->\ (c,\ a\ ->\ b)
\end{tabular}]\haddockbegindoc
Attaches a context parameter to a function argument (e.g
 consumption rates in SDF). Used as kernel function in defining
 e.g. \haddockid{ctxt22}.\par


\item[\begin{tabular}{@{}l}
wres\ ::\ p\ ->\ b\ ->\ (p,\ b)
\end{tabular}]\haddockbegindoc
Attaches a context parameter to a function's result (e.g
 production rates in SDF). Used as kernel function in defining
 e.g. \haddockid{ctxt22}.\par


\item[\begin{tabular}{@{}l}
(-*<)\ ::\ MoC\ e\ =>\\\ \ \ \ \ \ \ \ \ Stream\ (e\ (Ret\ e\ b1,\ Ret\ e\ b2))\ ->\ (Stream\ (e\ b1),\ Stream\ (e\ b2))
\end{tabular}]\haddockbegindoc
Utilities for extending the \haddockid{-*} atom for dealing with tupled
 outputs. This library exports operators of form \haddocktt{-*<{\char '173}1,8{\char '175}}.\par

\end{haddockdesc}
  \haddockmoduleheading{ForSyDe.Atom.MoC.Stream}
\label{module:ForSyDe.Atom.MoC.Stream}
\haddockbeginheader
{\haddockverb\begin{verbatim}
module ForSyDe.Atom.MoC.Stream (
    Stream(NullS, (:-)),  ,  ,  ,  ,  ,  stream,  fromStream,  headS,  tailS, 
    lastS,  repeatS,  takeS,  dropS,  takeWhileS,  (+-+)
  ) where\end{verbatim}}
\haddockendheader

This module defines the shallow-embedded \haddockid{Stream} datatype and
 utility functions operating on it. In ForSyDe a signal is
 represented as a (partially or totally) \emph{ordered} sequence of
 events that enables processes to communicate and synchronize.  The
 \haddockid{Stream} type is but an ordered structure to encapsulate events as
 infinite streams.\par

\begin{haddockdesc}
\item[\begin{tabular}{@{}l}
data\ Stream\ e
\end{tabular}]\haddockbegindoc
\haddockbeginconstrs
\haddockdecltt{=} & \haddockdecltt{NullS} & terminates a signal \\
\haddockdecltt{|} & \haddockdecltt{e (:-) (Stream e)} & the default constructor appends an\\
 &&event to the head of the stream \\
\end{tabulary}\par

Defines a stream of events, encapsulating them in a structure
 isomorphic to an infinite list \cite{Bird87},
 thus all properties of lists may also be applied to
 \haddockid{Stream}s. While, in combination with lazy evaluation, it is
 possible to create and simulate infinite signals, we need to ensure
 that the first/previous event is always fully evaluated. This can
 be translated into the following rule:\par
\begin{description}
\item[zero-delay feedbacks] are forbidden, due to un-evaluated
 self-referential calls. In a feedback loop, there always has to be
 enough events to ensure the data flow.
\end{description}This rule imposes that the stream of data is uninterrupted in order
 to have an evaluatable kernel every time a new event is produced
 (i.e. to avoid deadlocks). Thus we can add the rule:\par
                 \begin{description}
                 \item[cleaning of output events] is also forbidden.  In other words, for
 each new input at any instant in time, a process must react with
 \emph{at least} one output event.
                 \end{description}

\item[\begin{tabular}{@{}l}
instance\ Functor\ Stream
\end{tabular}]\haddockbegindoc
allows for the mapping of an arbitrary function \haddocktt{(a\ ->\ b)} upon
 all the events of a \haddocktt{(Stream\ a)}.\par


\item[\begin{tabular}{@{}l}
instance\ Applicative\ Stream
\end{tabular}]\haddockbegindoc
enables the \haddockid{Stream} to behave like a \haddockid{ZipList}\par


\item[\begin{tabular}{@{}l}
instance\ Foldable\ Stream
\end{tabular}]\haddockbegindoc
provides folding functions useful for implementing utilities, such as \haddockid{length}.\par


\item[\begin{tabular}{@{}l}
instance\ Read\ a\ =>\ Read\ (Stream\ a)
\end{tabular}]\haddockbegindoc
signal \haddocktt{(1\ :-\ 2\ :-\ NullS)} is read using the string \haddocktt{"{\char '173}1,2{\char '175}"}.\par


\item[\begin{tabular}{@{}l}
instance\ Show\ a\ =>\ Show\ (Stream\ a)
\end{tabular}]\haddockbegindoc
signal \haddocktt{(1\ :-\ 2\ :-\ NullS)} is represented as \haddocktt{{\char '173}1,2{\char '175}}.\par


\item[\begin{tabular}{@{}l}
instance\ Plottable\ a\ =>\ Plot\ (Signal\ a)
\end{tabular}]\haddockbegindoc
\haddockid{SY} signals.\par


\item[\begin{tabular}{@{}l}
instance\ Plottable\ a\ =>\ Plot\ (Signal\ a)
\end{tabular}]\haddockbegindoc
\haddockid{SDF} signals.\par


\item[\begin{tabular}{@{}l}
instance\ Plottable\ a\ =>\ Plot\ (Signal\ a)
\end{tabular}]\haddockbegindoc
\haddockid{DE} signals.\par


\item[\begin{tabular}{@{}l}
instance\ Plottable\ a\ =>\ Plot\ (Signal\ a)
\end{tabular}]\haddockbegindoc
\haddockid{CT} signals.\par


\item[\begin{tabular}{@{}l}
stream\ ::\ {\char 91}a{\char 93}\ ->\ Stream\ a
\end{tabular}]\haddockbegindoc
The function \haddocktt{signal} converts a list into a signal.\par


\item[\begin{tabular}{@{}l}
fromStream\ ::\ Stream\ a\ ->\ {\char 91}a{\char 93}
\end{tabular}]\haddockbegindoc
The function \haddockid{fromStream} converts a signal into a list.\par


\item[\begin{tabular}{@{}l}
headS\ ::\ Stream\ a\ ->\ a
\end{tabular}]\haddockbegindoc
Returns the head of a signal.\par


\item[\begin{tabular}{@{}l}
tailS\ ::\ Stream\ e\ ->\ Stream\ e
\end{tabular}]\haddockbegindoc
Returns the tail of a signal\par


\item[\begin{tabular}{@{}l}
lastS\ ::\ Stream\ p\ ->\ p
\end{tabular}]\haddockbegindoc
Returns the last event in a signal.\par


\item[\begin{tabular}{@{}l}
repeatS\ ::\ a\ ->\ Stream\ a
\end{tabular}]\haddockbegindoc
Returns an infinite list containing the same repeated event.\par


\item[\begin{tabular}{@{}l}
takeS\ ::\ (Ord\ t,\ Num\ t)\ =>\ t\ ->\ Stream\ e\ ->\ Stream\ e
\end{tabular}]\haddockbegindoc
Returns the first \haddocktt{n} events in a signal.\par


\item[\begin{tabular}{@{}l}
dropS\ ::\ (Ord\ t,\ Num\ t)\ =>\ t\ ->\ Stream\ e\ ->\ Stream\ e
\end{tabular}]\haddockbegindoc
Drops the first \haddocktt{n} events in a signal.\par


\item[\begin{tabular}{@{}l}
takeWhileS\ ::\ (a\ ->\ Bool)\ ->\ Stream\ a\ ->\ Stream\ a
\end{tabular}]\haddockbegindoc
Returns the first events of a signal which comply to a condition.\par


\item[\begin{tabular}{@{}l}
(+-+)\ ::\ Stream\ e\ ->\ Stream\ e\ ->\ Stream\ e
\end{tabular}]\haddockbegindoc
Concatenates two signals.\par

\end{haddockdesc}
  \haddockmoduleheading{ForSyDe.Atom.MoC.Time}
\label{module:ForSyDe.Atom.MoC.Time}
\haddockbeginheader
{\haddockverb\begin{verbatim}
module ForSyDe.Atom.MoC.Time (
    Time,  time,  const,  e,  (*^*),  pi,  sin,  cos,  tan,  atan,  asin, 
    acos,  sqrt,  exp,  cosh,  sinh,  tanh,  atanh,  asinh,  acosh,  log
  ) where\end{verbatim}}
\haddockendheader

Collection of utility functions for working with \haddockid{Time}. While the
 \haddockid{CT} MoC describes time as being a non-disjoint
 continuum (represented in ForSyDe-Atom with \haddockid{Rational} numbers),
 most of the functions here are non-ideal approximations or
 conversions from floating point equivalents. The trigonometric
 functions are imported from the
 \href{https://hackage.haskell.org/package/numbers-3000.2.0.1}{numbers}
 package, with a fixed \haddocktt{eps} parameter.\par
These utilities are meant to get started with using the CT MoC, and
 should be used with caution if result fidelity is a requirement. In
 this case the user should find a native \haddockid{Rational} implementation
 for a particular function.\par

\begin{haddockdesc}
\item[\begin{tabular}{@{}l}
type\ Time\ =\ Rational
\end{tabular}]\haddockbegindoc
Type alias for the type to represent metric (continuous)
 time. Underneath we use \haddockid{Rational} that is able to represent any
 \emph{t} between \emph{t₁} < \emph{t₂} ∈ \emph{T}.\par


\item[\begin{tabular}{@{}l}
time\ ::\ TimeStamp\ ->\ Time
\end{tabular}]\haddockbegindoc
Converts \haddockid{TimeStamp} into \haddockid{Time} representation.\par


\item[\begin{tabular}{@{}l}
const\ ::\ a\ ->\ Time\ ->\ a
\end{tabular}]\haddockbegindoc
Returns a constant function.\par


\item[\begin{tabular}{@{}l}
e\ ::\ Time
\end{tabular}]\haddockbegindoc
Euler's number in \haddockid{Time} format. Converted from the \haddocktt{Prelude}
 equivalent, which is \haddockid{Floating}.\par


\item[\begin{tabular}{@{}l}
(*{\char '136}*)\ ::\ Time\ ->\ Time\ ->\ Time
\end{tabular}]\haddockbegindoc
"Power of" function taking \haddockid{Time}s as arguments. Converts back
 and forth to \haddockid{Floating}, as it uses the \haddockid{**} operator, so
 it is prone to conversion errors.\par


\item[\begin{tabular}{@{}l}
pi\ ::\ Time
\end{tabular}]\haddockbegindoc
\haddockid{Time} representation of the number π. Rational
 representation with a precision of \haddocktt{0.000001}.\par


\item[\begin{tabular}{@{}l}
sin\ ::\ Time\ ->\ Time
\end{tabular}]\haddockbegindoc
Sine of \haddockid{Time}. Rational representation with a precision of
 \haddocktt{0.000001}.\par


\item[\begin{tabular}{@{}l}
cos\ ::\ Time\ ->\ Time
\end{tabular}]\haddockbegindoc
Cosine of \haddockid{Time}. Rational representation with a precision of
 \haddocktt{0.000001}.\par


\item[\begin{tabular}{@{}l}
tan\ ::\ Time\ ->\ Time
\end{tabular}]\haddockbegindoc
Tangent of \haddockid{Time}. Rational representation with a precision of
 \haddocktt{0.000001}.\par


\item[\begin{tabular}{@{}l}
atan\ ::\ Time\ ->\ Time
\end{tabular}]\haddockbegindoc
Arctangent of \haddockid{Time}. Rational representation with a precision
 of \haddocktt{0.000001}.\par


\item[\begin{tabular}{@{}l}
asin\ ::\ Time\ ->\ Time
\end{tabular}]\haddockbegindoc
Arcsine of \haddockid{Time}. Rational representation with a precision of
 \haddocktt{0.000001}.\par


\item[\begin{tabular}{@{}l}
acos\ ::\ Time\ ->\ Time
\end{tabular}]\haddockbegindoc
Arccosine of \haddockid{Time}. Rational representation with a precision
 of \haddocktt{0.000001}.\par


\item[\begin{tabular}{@{}l}
sqrt\ ::\ Time\ ->\ Time
\end{tabular}]\haddockbegindoc
Square root of \haddockid{Time}. Rational representation with a precision
 of \haddocktt{0.000001}.\par


\item[\begin{tabular}{@{}l}
exp\ ::\ Time\ ->\ Time
\end{tabular}]\haddockbegindoc
Exponent of \haddockid{Time}. Rational representation with a precision of
 \haddocktt{0.000001}.\par


\item[\begin{tabular}{@{}l}
cosh\ ::\ Time\ ->\ Time
\end{tabular}]\haddockbegindoc
Hyperbolic cosine of \haddockid{Time}. Rational representation with a
 precision of \haddocktt{0.000001}.\par


\item[\begin{tabular}{@{}l}
sinh\ ::\ Time\ ->\ Time
\end{tabular}]\haddockbegindoc
Hyperbolic sine of \haddockid{Time}. Rational representation with a
 precision of \haddocktt{0.000001}.\par


\item[\begin{tabular}{@{}l}
tanh\ ::\ Time\ ->\ Time
\end{tabular}]\haddockbegindoc
Hyperbolic tangent of \haddockid{Time}. Rational representation with a
 precision of \haddocktt{0.000001}.\par


\item[\begin{tabular}{@{}l}
atanh\ ::\ Time\ ->\ Time
\end{tabular}]\haddockbegindoc
Hyperbolic arctangent of \haddockid{Time}. Rational representation with a
 precision of \haddocktt{0.000001}.\par


\item[\begin{tabular}{@{}l}
asinh\ ::\ Time\ ->\ Time
\end{tabular}]\haddockbegindoc
Hyperbolic arcsine of \haddockid{Time}. Rational representation with a
 precision of \haddocktt{0.000001}.\par


\item[\begin{tabular}{@{}l}
acosh\ ::\ Time\ ->\ Time
\end{tabular}]\haddockbegindoc
Hyperbolic arccosine of \haddockid{Time}. Rational representation with a
 precision of \haddocktt{0.000001}.\par


\item[\begin{tabular}{@{}l}
log\ ::\ Time\ ->\ Time
\end{tabular}]\haddockbegindoc
Natural logarithm of \haddockid{Time}. Rational representation with a
 precision of \haddocktt{0.000001}.\par

\end{haddockdesc}
  \haddockmoduleheading{ForSyDe.Atom.MoC.TimeStamp}
\label{module:ForSyDe.Atom.MoC.TimeStamp}
\haddockbeginheader
{\haddockverb\begin{verbatim}
module ForSyDe.Atom.MoC.TimeStamp (
    TimeStamp,  picosec,  nanosec,  microsec,  milisec,  sec,  minutes,  hours, 
    toTime,  pi, 
  ) where\end{verbatim}}
\haddockendheader

This module implements a timestamp data type, based on
 \haddocktt{Data.Time.Clock}. \par

\begin{haddockdesc}
\item[\begin{tabular}{@{}l}
type\ TimeStamp\ =\ DiffTime
\end{tabular}]\haddockbegindoc
Alias for the type representing discrete time. It is inherently
 quantizable, the quantum being a picosecond (10⁻¹²
 seconds), thus it can be considered order-isomorphic with a set of
 integers, i.e. between any two timestamps there is a finite number
 of timestamps. Moreover, a timestamp can be easily translated into
 a rational number representing fractions of a second, so the
 conversion between timestamps (discrete time) and rationals
 (analog/continuous time) is straightforward.\par
This type is used in the explicit tags of the
 \haddockid{DE} MoC (and subsequently the discrete event
 evaluation engine for simulating the \haddockid{CT} MoC).\par


\item[\begin{tabular}{@{}l}
picosec\ ::\ Integer\ ->\ TimeStamp
\end{tabular}]\haddockbegindoc
Specifies a timestamp in terms of picoseconds.\par


\item[\begin{tabular}{@{}l}
nanosec\ ::\ Integer\ ->\ TimeStamp
\end{tabular}]\haddockbegindoc
Specifies a timestamp in terms of nanoseconds.\par


\item[\begin{tabular}{@{}l}
microsec\ ::\ Integer\ ->\ TimeStamp
\end{tabular}]\haddockbegindoc
Specifies a timestamp in terms of microseconds.\par


\item[\begin{tabular}{@{}l}
milisec\ ::\ Integer\ ->\ TimeStamp
\end{tabular}]\haddockbegindoc
Specifies a timestamp in terms of miliseconds.\par


\item[\begin{tabular}{@{}l}
sec\ ::\ Integer\ ->\ TimeStamp
\end{tabular}]\haddockbegindoc
Specifies a timestamp in terms of seconds.\par


\item[\begin{tabular}{@{}l}
minutes\ ::\ Integer\ ->\ TimeStamp
\end{tabular}]\haddockbegindoc
Specifies a timestamp in terms of minutes.\par


\item[\begin{tabular}{@{}l}
hours\ ::\ Integer\ ->\ TimeStamp
\end{tabular}]\haddockbegindoc
Specifies a timestamp in terms of hours.\par


\item[\begin{tabular}{@{}l}
toTime\ ::\ TimeStamp\ ->\ Rational
\end{tabular}]\haddockbegindoc
Converts a timestamp to a rational number, used for describing
 continuous time.\par


\item[\begin{tabular}{@{}l}
pi\ ::\ TimeStamp
\end{tabular}]\haddockbegindoc
\haddockid{TimeStamp} representation of the number π. Converted from
 the \haddocktt{Prelude} equivalent, which is \haddockid{Floating}.\par

\end{haddockdesc}
  \haddockmoduleheading{ForSyDe.Atom.MoC.CT}
\label{module:ForSyDe.Atom.MoC.CT}
\haddockbeginheader
{\haddockverb\begin{verbatim}
module ForSyDe.Atom.MoC.CT (
    TimeStamp,  Time,  CT(CT, tag, phase, func),  Signal,  unit2,  infinite, 
    signal,  checkSignal,  delay,  delay',  comb22,  reconfig22,  constant2, 
    infinite2,  generate2,  stated22,  state22,  moore22,  mealy22,  toDE1, 
    sampDE2,  zipx,  unzipx,  unzipx'
  ) where\end{verbatim}}
\haddockendheader

The \haddocktt{CT} library implements the atoms holding the sematics for the
 continuous time computation model. It also provides a set of
 helpers for properly instantiating process network patterns as
 process constructors.\par
For working with time or timestamps please check the utilities
 provided by the \haddocktt{ForSyDe.Atom.MoC.Time} and
 \haddocktt{ForSyDe.Atom.MoC.TimeStamp} modules.\par
\begin{mdframed}[style=reminder,frametitle=Reminder]Make sure to consult naming conventions in  \cref{sec:forsyde-atom:naming-convention} in order to interpret the names and type signatures correctly.\end{mdframed}\par

\subsection{Continuous time (\haddocktt{CT}) event}
According to \cite{Lee98}, "{\char 91}regarding
 metric time{\char 93} at a minimum, \emph{T} is an Abelian group, in addition
 to being totally ordered. A consequence is that \emph{t₂} -
 \emph{t₁} is itself a tag ∀ \emph{t₁}, \emph{t₂} ∈
 \emph{T}. In a slightly more elaborate model of computation, \emph{T} has a
 metric. (...) A continuous-time system is a metric timed system
 \emph{Q} where \emph{T} is a continuum (a closed connected set)."\par
The continuous time (\haddocktt{CT}) MoC defines the closest behavior to
 what we could call "physical time", where signals cover the full
 span of a simulation as "functions of time" rather than
 "values". As such, we can state:\par
\begin{description}
\item[The CT MoC] is abstracting the execution semantics and describes
 a system where computation is performed continuously over a
 (possibly infinite) span of time.
\end{description}Below is an illustration of the behavior in time of the input and
 the output signals of a CT process:\par
                 \haddockfig{fig/moc-ct-example.pdf}\par
                 Our CT MoC is implemented as an enhanced version of
 \haddockid{DE} with respect to the \textbf{CT MoC} definition, in
 the sense that:\par
                 \begin{enumerate}
                 \item 
                 tags \emph{t} are also represented with
 \haddockid{TimeStamp}s, thus we can say that changes in a
 CT signal happen at discrete times (see below).\par
                 
                 \item 
                 values are represented as functions over a continuous span of
 time \emph{f(t)} rather than just a value or a series of values. The
 time domain is represented with rational numbers which, as
 compared to floating point numbers, do not suffer from inherent
 quantisation, being able to model true continuity, i.e. between
 any two arbitrary points in time there is an infinite amount of
 intermediate moments.\par
                 
                 \item 
                 The event constructor has also a \emph{phase} component \emph{φ},
 which is taken into consideration only when evaluating the event
 function, i.e. \emph{f (t + φ)}. This enables the modeling of
 "phase dispacements" of delay lines without altering the function
 itself (and thus increasing the complexity of the un-evaluated
 function graph). The phase needs to be reset during event
 synchronization.\par
                 
                 \end{enumerate}
                 These seemingly minor changes have deep implications in the
 expressiveness of a FoSyDe CT system and how we interpret
 it. Capturing the particularities of this MoC, we can formulate
 the following properties:\par
                 \haddockfig{fig/misc-ct-model.pdf}\par
                 \begin{enumerate}
                 \item 
                 \haddockid{CT} signals, due to their formation as streams of tagged
 events, represent \emph{discrete} changes in a continuous function
 over time (e.g. analog signal). While the functions carried by
 events are infinite (have always happened and will always
 happen), being carried by events in a tag system suggests that
 changes occur at discrete times. A CT signal can be represented
 by the analog circuit above, where the inputs are continuous
 signals, but the switch is discrete. Like in the
 \haddockid{DE} MoC, the absolute time 0 represent the time
 when the system started to be observed.\par
                 
                 \item 
                 the previous property is also proven by the fact that the
 evaluation engine of ForSyDe-Atom is inherently discrete,
 i.e. evaluation is performed whenever a new event occurs, in a
 dataflow manner. Allowing infinitely small distances between tags
 would hinder the advancement of simulation time.\par
                 
                 \item 
                 events carry \emph{functions} and not \emph{values}. In a lazy
 evaluation system like Haskell's, functions are kept symbolic
 until evaluation. This means that in a CT system computations are
 propagated as function graphs until a result is needed, e.g. a
 signal is plotted for arbitrary positions in time. This way
 intermediate quantization errors are eliminated, and the cost of
 higher plot resolution is the cost of evaluating the final
 results only.\par
                 
                 \item 
                 needless to say, for each \emph{t} ∈ \emph{T}, a signal is able to
 return (e.g. plot) the exact value \emph{v} for that particular \emph{t}.\par
                 
                 \item 
                 since itself the \haddockid{CT} MoC is just an enhanced
 \haddockid{DE} system, all atom evaluation properties are
 inherited from it: feedback loops need to advance time, atoms are
 forbidden to clean signals, and the conservative approach makes
 it ideal for parallel/distributed simulation.\par
                 
                 \item 
                 since \emph{T} is a total order, there is no need for an
 execution context (see \cref{module:ForSyDe.Atom.MoC}) and we can
 ignore the formatting of functions in \haddocktt{ForSyDe.Atom.MoC}, thus we
 can safely assume:\par
                 
                 \end{enumerate}
                 \haddockeq{fig/eqs-moc-timed-context.pdf}\par
                 
\begin{haddockdesc}
\item[\begin{tabular}{@{}l}
type\ TimeStamp\ =\ DiffTime
\end{tabular}]\haddockbegindoc
Alias for the type representing discrete time. It is inherently
 quantizable, the quantum being a picosecond (10⁻¹²
 seconds), thus it can be considered order-isomorphic with a set of
 integers, i.e. between any two timestamps there is a finite number
 of timestamps. Moreover, a timestamp can be easily translated into
 a rational number representing fractions of a second, so the
 conversion between timestamps (discrete time) and rationals
 (analog/continuous time) is straightforward.\par
This type is used in the explicit tags of the
 \haddockid{DE} MoC (and subsequently the discrete event
 evaluation engine for simulating the \haddockid{CT} MoC).\par


\item[\begin{tabular}{@{}l}
type\ Time\ =\ Rational
\end{tabular}]\haddockbegindoc
Type alias for the type to represent metric (continuous)
 time. Underneath we use \haddockid{Rational} that is able to represent any
 \emph{t} between \emph{t₁} < \emph{t₂} ∈ \emph{T}.\par


\item[\begin{tabular}{@{}l}
data\ CT\ a
\end{tabular}]\haddockbegindoc
\haddockbeginconstrs
\haddockdecltt{=} & \haddockdecltt{CT} & \\
                    \haddockdecltt{>}&\haddockdecltt{tag :: TimeStamp} & start time of event\\
                    \haddockdecltt{>}&\haddockdecltt{phase :: Time} & phase. Models function delays\\
                    \haddockdecltt{>}&\haddockdecltt{func :: Time -> a} & function of time\\
\end{tabulary}\par

The CT type, identifying a continuous time event and implementing an
 instance of the \haddockid{MoC} class.\par


\item[\begin{tabular}{@{}l}
instance\ Functor\ CT
\end{tabular}]\haddockbegindoc
Allows for mapping of functions on a CT event.\par


\item[\begin{tabular}{@{}l}
instance\ Applicative\ CT
\end{tabular}]\haddockbegindoc
Allows for lifting functions on a pair of CT events.\par


\item[\begin{tabular}{@{}l}
instance\ MoC\ CT
\end{tabular}]\haddockbegindoc
Implenents the execution and synchronization semantics for the CT
 MoC through its atoms.\par


\item[\begin{tabular}{@{}l}
instance\ Show\ a\ =>\ Show\ (CT\ a)
\end{tabular}]\haddockbegindoc
A non-ideal instance meant for debug purpose only. For each event
 it evaluates the function at the tag time \emph{only}!\par


\item[\begin{tabular}{@{}l}
instance\ Plottable\ a\ =>\ Plot\ (Signal\ a)
\end{tabular}]\haddockbegindoc
\haddockid{CT} signals.\par


\item[\begin{tabular}{@{}l}
instance\ type\ Ret\ CT\ b\ =\ b\\instance\ type\ Fun\ CT\ a\ b\ =\ a\ ->\ b
\end{tabular}]
\end{haddockdesc}
\subsection{Aliases {\char '46} utilities}
A set of type synonyms and utilities are provided for
 convenience. The API type signatures will feature these aliases
 to hide the cumbersome construction of atoms and patters as seen
 in \haddocktt{ForSyDe.Atom.MoC}.\par

\begin{haddockdesc}
\item[\begin{tabular}{@{}l}
type\ Signal\ a\ =\ Stream\ (CT\ a)
\end{tabular}]\haddockbegindoc
Type synonym for a CT signal, i.e. "a signal of CT events"\par


\item[\begin{tabular}{@{}l}
unit2\ ::\ ((TimeStamp,\ Time\ ->\ a1),\ (TimeStamp,\ Time\ ->\ a2))\\\ \ \ \ \ \ \ \ \ ->\ (Signal\ a1,\ Signal\ a2)
\end{tabular}]\haddockbegindoc
Wraps a (tuple of) pair(s) \haddocktt{(time,\ function)} into the equivalent
 unit signal(s), i.e. signal(s) with one event with the period
 \haddocktt{time} carrying \haddocktt{function}.\par
Helpers: \haddocktt{unit} and \haddocktt{unit{\char 91}2-4{\char 93}}.\par


\item[\begin{tabular}{@{}l}
infinite\ ::\ (Time\ ->\ a)\ ->\ Signal\ a
\end{tabular}]\haddockbegindoc
Creates an infinite signal.\par


\item[\begin{tabular}{@{}l}
signal\ ::\ {\char 91}(TimeStamp,\ Time\ ->\ a){\char 93}\ ->\ Signal\ a
\end{tabular}]\haddockbegindoc
Transforms a list of tuples such as the ones taken by \haddocktt{event}
 into a CT signal\par


\item[\begin{tabular}{@{}l}
checkSignal\ ::\ Stream\ (CT\ a)\ ->\ Stream\ (CT\ a)
\end{tabular}]\haddockbegindoc
Checks if a signal is well-formed or not, according to the CT MoC
 interpretation in \haddocktt{ForSyDe-Atom}.\par

\end{haddockdesc}
\subsection{\haddocktt{CT} process constuctors}
The CT process constructors are basically specific
 instantiations of patterns defined in \haddocktt{ForSyDe.Atom.MoC}. Some
 might also be wrapping functions in an extended behavioural
 model.\par
In the examples below we have imported and instantiated the
 functions such as \haddocktt{e'} \haddocktt{pi'}, \haddocktt{sin'} and \haddocktt{cos'} from the
 collection of utilities in \haddocktt{ForSyDe.Atom.MoC.Time} and
 \haddocktt{ForSyDe.Atom.MoC.TimeStamp}. Also, for the sake of documentation
 the interactive examples are only dumping the CT signals in data
 files using the \haddocktt{dumpDat} utility defined in
 \haddocktt{ForSyDe.Atom.Utility.Plot}, according to the custom \haddocktt{cfg}
 structure. These files can be further plotted by any tool of
 choice, or using the plotting utilities provided in the
 \haddocktt{ForSyDe.Atom.Utility.Plot} module.\par
\begin{interactive}
import ForSyDe.Atom.Utility.Plot
import ForSyDe.Atom.MoC.Time as Time
import ForSyDe.Atom.MoC.TimeStamp as TimeStamp
let pi'  = TimeStamp.pi
let exp' = Time.exp
let sin' = Time.sin
let cos' = Time.cos
let cfg  = defaultCfg {xmax=10, rate=0.1}\end{interactive}

\subsubsection{Simple}
These are mainly direct instantiations of patterns defined in
 \haddocktt{ForSyDe.Atom.MoC}, using DE-specific utilities.\par

\begin{haddockdesc}
\item[\begin{tabular}{@{}l}
delay
\end{tabular}]\haddockbegindoc
\haddockbeginargs
\haddockdecltt{::} & \haddockdecltt{TimeStamp} & time delay \\
                                                 \haddockdecltt{->} & \haddockdecltt{(Time
                                                                                      -> a)} & initial value \\
                                                                                               \haddockdecltt{->} & \haddockdecltt{Signal a} & input signal \\
                                                                                                                                               \haddockdecltt{->} & \haddockdecltt{Signal a} & output signal \\
\end{tabulary}\par
The \haddocktt{delay} process "delays" a signal with one
 event. Instantiates the \haddockid{delay} pattern. In the CT
 MoC, this process can be interpreted as an ideal "delay line".\par
\begin{interactive}
λ> let s  = infinite (sin')
λ> let s' = delay 2 (\_ -> 0) s
λ> dumpDat $ prepare cfg {labels=["delay"]} $ s'
Dumped delay in ./fig
["./fig/delay.dat"]

\end{interactive}\haddockfig{fig/moc-ct-pattern-delay.pdf}\par
           

\item[\begin{tabular}{@{}l}
delay'
\end{tabular}]\haddockbegindoc
\haddockbeginargs
\haddockdecltt{::} & \haddockdecltt{Signal a} & signal "borrowing" the initial event \\
                                                \haddockdecltt{->} & \haddockdecltt{Signal a} & input signal \\
                                                                                                \haddockdecltt{->} & \haddockdecltt{Signal a} & output signal \\
\end{tabulary}\par
Similar to the previous, but this is the raw instantiation of the
 \haddockid{delay} pattern. It "borrows" the first event from
 one signal and appends it at the head of another signal.\par
\begin{interactive}
λ> let s  = infinite (sin')
λ> let s' = signal [(0, \_ -> 0), (2, \_ -> 1)]
λ> dumpDat $ prepare cfg {labels=["delayp"]} $ delay' s' s
Dumped delayp in ./fig
["./fig/delayp.dat"]

\end{interactive}\haddockfig{fig/moc-ct-pattern-delayp.pdf}\par
           

\item[\begin{tabular}{@{}l}
comb22
\end{tabular}]\haddockbegindoc
\haddockbeginargs
\haddockdecltt{::} & \haddockdecltt{(a1
                                     -> a2 -> (b1, b2))} & function on values \\
                                                           \haddockdecltt{->} & \haddockdecltt{Signal a1} & first input signal \\
                                                                                                            \haddockdecltt{->} & \haddockdecltt{Signal a2} & second input signal \\
                                                                                                                                                             \haddockdecltt{->} & \haddockdecltt{(Signal b1, Signal b2)} & two output signals \\
\end{tabulary}\par
\haddocktt{comb} processes map combinatorial functions on signals and take
 care of synchronization between input signals. It instantiates the
 \haddocktt{comb} pattern (see \haddockid{comb22}).\par
Constructors: \haddocktt{comb{\char 91}1-4{\char 93}{\char 91}1-4{\char 93}}.\par
\begin{interactive}
λ> let s1 = infinite (sin')
λ> let s2 = signal [(0,\_->0),(pi',\_->1),(2*pi',\_->0),(3*pi',\_->1)]
λ> let o1 = comb11 (+1) s2
λ> let (o2_1, o2_2) = comb22 (\a b-> (a+b,a*b)) s1 s2
λ> dumpDat $ prepare cfg {labels=["comb11"]} o1
Dumped comb11 in ./fig
["./fig/comb11.dat"]
λ> dumpDat $ prepareL cfg {labels=["comb22-1","comb22-2"]} [o2_1, o2_2]
Dumped comb22-1, comb22-2 in ./fig
["./fig/comb22-1.dat","./fig/comb22-2.dat"]

\end{interactive}\haddockfig{fig/moc-ct-pattern-comb.pdf}\par
           

\item[\begin{tabular}{@{}l}
reconfig22
\end{tabular}]\haddockbegindoc
\haddockbeginargs
\haddockdecltt{::} & \haddockdecltt{Signal (a1
                                            -> a2
                                               -> (b1, b2))} & signal carrying functions \\
                                                               \haddockdecltt{->} & \haddockdecltt{Signal a1} & first input signal carrying arguments \\
                                                                                                                \haddockdecltt{->} & \haddockdecltt{Signal a2} & second input signal carrying arguments \\
                                                                                                                                                                 \haddockdecltt{->} & \haddockdecltt{(Signal b1, Signal b2)} & two output signals \\
\end{tabulary}\par
\haddocktt{reconfig} creates a CT adaptive process where the first signal
 carries functions and the other carry the arguments. It
 instantiates the \haddocktt{reconfig} atom pattern (see
 \haddockid{reconfig22}).\par
Constructors: \haddocktt{reconfig{\char 91}1-4{\char 93}{\char 91}1-4{\char 93}}.\par
\begin{interactive}
λ> let s1 = infinite (sin')
λ> let sf = signal [(0,\_->(*0)),(pi',\_->(+1)),(2*pi',\_->(*0)),(3*pi',\_->(+1))]
λ> dumpDat $ prepare cfg {labels=["reconfig"]} $ reconfig11 sf s1
Dumped reconfig in ./fig
["./fig/reconfig.dat"]

\end{interactive}

\item[\begin{tabular}{@{}l}
constant2
\end{tabular}]\haddockbegindoc
\haddockbeginargs
\haddockdecltt{::} & \haddockdecltt{(b1, b2)} & values to be repeated \\
                                                \haddockdecltt{->} & \haddockdecltt{(Signal b1, Signal b2)} & generated signals \\
\end{tabulary}\par
A generator for a constant signal. As compared with the
 \haddockid{constant2}, it just constructs an infinite
 signal with constant value (i.e. a signal with one event starting
 from time 0).\par
Constructors: \haddocktt{constant{\char 91}1-4{\char 93}}.\par
\begin{interactive}
λ> dumpDat $ prepare cfg {labels=["constant"]} $ constant1 2
Dumped constant in ./fig
["./fig/constant.dat"]

\end{interactive}\haddockfig{fig/moc-ct-pattern-constant.pdf}\par
           

\item[\begin{tabular}{@{}l}
infinite2
\end{tabular}]\haddockbegindoc
\haddockbeginargs
\haddockdecltt{::} & \haddockdecltt{(Time -> b1, Time
                                                 -> b2)} & values to be repeated \\
                                                           \haddockdecltt{->} & \haddockdecltt{(Signal b1, Signal b2)} & generated signals \\
\end{tabulary}\par
A generator for an infinite signal. Similar to \haddockid{constant2}.\par
Constructors: \haddocktt{infinite{\char 91}1-4{\char 93}}.\par
\begin{interactive}
λ> let (inf1, inf2) = infinite2 (sin', cos')
λ> dumpDat $ prepareL cfg {labels=["infinite2-1", "infinite2-2"]} [inf1, inf2]
Dumped infinite2-1, infinite2-2 in ./fig
["./fig/infinite2-1.dat","./fig/infinite2-2.dat"]

\end{interactive}\haddockfig{fig/moc-ct-pattern-infinite.pdf}\par
           

\item[\begin{tabular}{@{}l}
generate2
\end{tabular}]\haddockbegindoc
\haddockbeginargs
\haddockdecltt{::} & \haddockdecltt{(b1
                                     -> b2
                                        -> (b1, b2))} & function to generate next value \\
                                                        \haddockdecltt{->} & \haddockdecltt{((TimeStamp, Time
                                                                                                         -> b1), (TimeStamp, Time
                                                                                                                             -> b2))} & kernel values tupled with their generation rate. \\
                                                                                                                                        \haddockdecltt{->} & \haddockdecltt{(Signal b1, Signal b2)} & generated signals \\
\end{tabulary}\par
A signal generator based on a function and a kernel value. It
 is actually an instantiation of the \haddocktt{stated0X} constructor
 (check \haddockid{stated22}).\par
Constructors: \haddocktt{generate{\char 91}1-4{\char 93}}.\par
\begin{interactive}
λ> let { osc 0 = 1 ; osc 1 = 0 }
λ> dumpDat $ prepare cfg {labels=["generate"]} $ generate1 osc (pi', \_->0)
Dumped generate in ./fig
["./fig/generate.dat"]

\end{interactive}\haddockfig{fig/moc-ct-pattern-generate.pdf}\par
           Another example simulating an RC oscillator:\par
           \begin{quote}
           {\haddockverb\begin{verbatim}
           λ> let vs = 2                                -- Vs : supply voltage
λ> let r  = 100                              -- R : resistance
λ> let c  = 0.0005                           -- C : capacitance
λ> let vc t = vs * (1 - exp' (-t / (r * c))) -- Vc(t) : voltage charging through capacitor
λ> let ns v = vs + (-1 * v)                  -- next state : charging / discharging
λ> let rcOsc = generate1 ns (milisec 500, vc)
λ> dumpDat $ prepare cfg {labels=["rcosc"]} $ rcOsc
Dumped rcosc in ./fig
["./fig/rcosc.dat"]

\end{verbatim}}
           \end{quote}\haddockfig{fig/moc-ct-pattern-generate1.pdf}\par
                      

\item[\begin{tabular}{@{}l}
stated22
\end{tabular}]\haddockbegindoc
\haddockbeginargs
\haddockdecltt{::} & \haddockdecltt{(b1
                                     -> b2
                                        -> a1
                                           -> a2
                                              -> (b1, b2))} & next state function \\
                                                              \haddockdecltt{->} & \haddockdecltt{((TimeStamp, Time
                                                                                                               -> b1), (TimeStamp, Time
                                                                                                                                   -> b2))} & initial state values tupled with their initial delay \\
                                                                                                                                              \haddockdecltt{->} & \haddockdecltt{Signal a1} & first input signal \\
                                                                                                                                                                                               \haddockdecltt{->} & \haddockdecltt{Signal a2} & second input signal \\
                                                                                                                                                                                                                                                \haddockdecltt{->} & \haddockdecltt{(Signal b1, Signal b2)} & output signals \\
\end{tabulary}\par
\haddocktt{stated} is a state machine without an output decoder. It is an
 instantiation of the \haddocktt{state} MoC constructor (see
 \haddockid{stated22}).\par
Constructors: \haddocktt{stated{\char 91}1-4{\char 93}{\char 91}1-4{\char 93}}.\par
\begin{interactive}
λ> let { osc 0 a = a; osc _ a = 0 }
λ> let s1 = signal [(0,\_->1), (6,\_->0)]
λ> dumpDat $ prepare cfg {labels=["stated"]} $ stated11 osc (1,\_->0) s1
Dumped stated in ./fig
["./fig/stated.dat"]

\end{interactive}\haddockfig{fig/moc-ct-pattern-stated.pdf}\par
           

\item[\begin{tabular}{@{}l}
state22
\end{tabular}]\haddockbegindoc
\haddockbeginargs
\haddockdecltt{::} & \haddockdecltt{(b1
                                     -> b2
                                        -> a1
                                           -> a2
                                              -> (b1, b2))} & next state function \\
                                                              \haddockdecltt{->} & \haddockdecltt{((TimeStamp, Time
                                                                                                               -> b1), (TimeStamp, Time
                                                                                                                                   -> b2))} & initial state values tupled with their initial delay \\
                                                                                                                                              \haddockdecltt{->} & \haddockdecltt{Signal a1} & first input signal \\
                                                                                                                                                                                               \haddockdecltt{->} & \haddockdecltt{Signal a2} & second input signal \\
                                                                                                                                                                                                                                                \haddockdecltt{->} & \haddockdecltt{(Signal b1, Signal b2)} & output signals \\
\end{tabulary}\par
\haddocktt{state} is a state machine without an output decoder, and the
 state non-transparent. It is an instantiation of the \haddocktt{state} MoC
 constructor (see \haddockid{state22}).\par
Constructors: \haddocktt{state{\char 91}1-4{\char 93}{\char 91}1-4{\char 93}}.\par
\begin{interactive}
λ> let { osc 0 a = a; osc _ a = 0 }
λ> let s1 = signal [(0,\_->1), (6,\_->0)]
λ> dumpDat $ prepare cfg {labels=["state"]} $ state11 osc (1,\_->0) s1
Dumped state in ./fig
["./fig/state.dat"]

\end{interactive}\haddockfig{fig/moc-ct-pattern-state.pdf}                   \par
           

\item[\begin{tabular}{@{}l}
moore22
\end{tabular}]\haddockbegindoc
\haddockbeginargs
\haddockdecltt{::} & \haddockdecltt{(st
                                     -> a1
                                        -> a2 -> st)} & next state function \\
                                                        \haddockdecltt{->} & \haddockdecltt{(st
                                                                                             -> (b1, b2))} & output decoder \\
                                                                                                             \haddockdecltt{->} & \haddockdecltt{(TimeStamp, Time
                                                                                                                                                             -> st)} & initial state: tag and value \\
                                                                                                                                                                       \haddockdecltt{->} & \haddockdecltt{Signal a1} & \\
                                                                                                                                                                                                                        \haddockdecltt{->} & \haddockdecltt{Signal a2} & \\
                                                                                                                                                                                                                                                                         \haddockdecltt{->} & \haddockdecltt{(Signal b1, Signal b2)} & \\
\end{tabulary}\par
\haddocktt{moore} processes model Moore state machines. It is an
 instantiation of the \haddocktt{moore} MoC constructor (see
 \haddockid{moore22}).\par
Constructors: \haddocktt{moore{\char 91}1-4{\char 93}{\char 91}1-4{\char 93}}.\par
\begin{interactive}
λ> let { osc 0 a = a; osc _ a = 0 }
λ> let s1 = signal [(0,\_->1), (6,\_->0)]
λ> dumpDat $ prepare cfg {labels=["moore"]} $ moore11 osc (*3) (1,\_->0) s1
Dumped moore in ./fig
["./fig/moore.dat"]

\end{interactive}\haddockfig{fig/moc-ct-pattern-moore.pdf}          \par
           

\item[\begin{tabular}{@{}l}
mealy22
\end{tabular}]\haddockbegindoc
\haddockbeginargs
\haddockdecltt{::} & \haddockdecltt{(st
                                     -> a1
                                        -> a2 -> st)} & next state function \\
                                                        \haddockdecltt{->} & \haddockdecltt{(st
                                                                                             -> a1
                                                                                                -> a2
                                                                                                   -> (b1, b2))} & outpt decoder \\
                                                                                                                   \haddockdecltt{->} & \haddockdecltt{(TimeStamp, Time
                                                                                                                                                                   -> st)} & initial state: tag and value \\
                                                                                                                                                                             \haddockdecltt{->} & \haddockdecltt{Signal a1} & \\
                                                                                                                                                                                                                              \haddockdecltt{->} & \haddockdecltt{Signal a2} & \\
                                                                                                                                                                                                                                                                               \haddockdecltt{->} & \haddockdecltt{(Signal b1, Signal b2)} & \\
\end{tabulary}\par
\haddocktt{mealy} processes model Mealy state machines. It is an
 instantiation of the \haddocktt{mealy} MoC constructor
 (see \haddockid{mealy22}).\par
Constructors: \haddocktt{mealy{\char 91}1-4{\char 93}{\char 91}1-4{\char 93}}.\par
\begin{interactive}
λ> let { osc (-1) _ = 1; osc 1 _ = (-1) }
λ> let s1 = infinite sin'
λ> dumpDat $ prepare cfg {labels=["mealy"]} $ mealy11 osc (*) (pi',\_->1) s1
Dumped mealy in ./fig
["./fig/mealy.dat"]

\end{interactive}\haddockfig{fig/moc-ct-pattern-mealy.pdf}\par
           
\end{haddockdesc}
\subsubsection{Interfaces}
\begin{haddockdesc}
\item[\begin{tabular}{@{}l}
toDE1\ ::\ Signal\ a\ ->\ Signal\ (Time\ ->\ a)
\end{tabular}]\haddockbegindoc
Translates a (set of) \haddockid{CT} signal(s) into
 \haddockid{DE} semantics without loss of information. In
 \haddockid{DE}, the abstract function of time inferred by
 the \haddockid{CT} event loses its abstraction and it is
 "dropped" to explicit form, under a lower layer. In other words the
 implicit time semantics are lost, the carried value simply becoming
 an ordinary function.\par
Constructors: \haddocktt{toDE{\char 91}1-4{\char 93}}.\par
\haddockfig{fig/moc-ct-tode.pdf}\par


\item[\begin{tabular}{@{}l}
sampDE2
\end{tabular}]\haddockbegindoc
\haddockbeginargs
\haddockdecltt{::} & \haddockdecltt{Signal t} & \haddockid{DE} timestamp carrier  \\
                                                \haddockdecltt{->} & \haddockdecltt{Signal a} & \haddockid{CT} input \\
                                                                                                \haddockdecltt{->} & \haddockdecltt{Signal b} & \haddockid{CT} input \\
                                                                                                                                                \haddockdecltt{->} & \haddockdecltt{(Signal a, Signal b)} & \haddockid{DE} outputs \\
\end{tabulary}\par
Synchronizes a (set of) \haddockid{CT} signal(s) with a
 \haddockid{DE} carrier which holds the timestamps at
 which the CT signal must be sampled, and outputs the respective
 (set of) \haddockid{DE} signal(s).\par
Constructors: \haddocktt{sampDE{\char 91}1-4{\char 93}}.\par
\begin{interactive}
λ> let s = CT.infinite (fromRational . sin')
λ> let c = DE.generate1 id (pi'/2, 1)
λ> takeS 6 $ sampDE1 c s
{ 0.0 @0s, 1.0 @1.570796326794s, 1.793238520564752e-12 @3.141592653588s, -1.0 @4.712388980382s, 0.0 @6.283185307176s, 1.0 @7.85398163397s}

\end{interactive}\haddockfig{fig/moc-ct-sampde.pdf}\par
           

\item[\begin{tabular}{@{}l}
zipx\ ::\ Vector\ (Signal\ a)\ ->\ Signal\ (Vector\ a)
\end{tabular}]\haddockbegindoc
Synchronizes all the signals contained by a vector and zips them
 into one signal of vectors. It instantiates the
 \haddockid{zipx} skeleton.\par
\begin{interactive}
λ> let s1 = CT.signal [(0,const 1), (2,const 2), (6,const 3)]
λ> let s2 = CT.signal [(0,const 1), (2,const 2), (4,const 3)]
λ> let v1 = V.vector [s1,s1,s2,s2]
λ> zipx v1
{ <1,1,1,1> @0s, <2,2,2,2> @2s, <2,2,3,3> @4s, <3,3,3,3> @6s}

\end{interactive}See \haddockid{zipx} from the \haddocktt{ForSyDe.Atom.MoC.DE}
 library for a comprehensive visual example.\par
           

\item[\begin{tabular}{@{}l}
unzipx\ ::\ Integer\ ->\ Signal\ (Vector\ a)\ ->\ Vector\ (Signal\ a)
\end{tabular}]\haddockbegindoc
Unzips the vectors carried by a signal into a vector of
 signals. It instantiates the \haddockid{unzipx}
 skeleton. To avoid infinite recurrence, the user needs to provict
 the length of the output vector.\par
\begin{interactive}
λ> let v1 = V.vector [1,2,3,4]
λ> let s1 = CT.signal [(0,const v1),(2,const v1),(6,const v1)]
λ> unzipx 4 s1
<{ 4 @0s, 4 @2s, 4 @6s},{ 3 @0s, 3 @2s, 3 @6s},{ 2 @0s, 2 @2s, 2 @6s},{ 1 @0s, 1 @2s, 1 @6s}>

\end{interactive}See \haddockid{unzipx} from the \haddocktt{ForSyDe.Atom.MoC.DE}
 library for a comprehensive visual example.\par
           

\item[\begin{tabular}{@{}l}
unzipx'\ ::\ Signal\ (Vector\ a)\ ->\ Vector\ (Signal\ a)
\end{tabular}]\haddockbegindoc
Same as \haddockid{unzipx}, but "sniffs" the first event to determine the length of the output vector. Might have unsafe behavior!\par
\begin{interactive}
λ> let v1 = V.vector [1,2,3,4]
λ> let s1 = CT.signal [(0,const v1),(2,const v1),(6,const v1)]
λ> unzipx' s1
<{ 4 @0s, 4 @2s, 4 @6s},{ 3 @0s, 3 @2s, 3 @6s},{ 2 @0s, 2 @2s, 2 @6s},{ 1 @0s, 1 @2s, 1 @6s}>

\end{interactive}
\end{haddockdesc}
  \haddockmoduleheading{ForSyDe.Atom.MoC.DE}
\label{module:ForSyDe.Atom.MoC.DE}
\haddockbeginheader
{\haddockverb\begin{verbatim}
module ForSyDe.Atom.MoC.DE (
    TimeStamp,  DE(DE, tag, val),  Signal,  unit2,  infinite,  until,  signal, 
    checkSignal,  readSignal,  delay,  delay',  comb22,  reconfig22,  sync2, 
    constant2,  generate2,  stated22,  state22,  moore22,  mealy22,  toSY2, 
    toCT2,  zipx,  unzipx,  unzipx',  embedSY22
  ) where\end{verbatim}}
\haddockendheader

The \haddocktt{DE} library implements the atoms holding the sematics for the
 discrete event computation model. It also provides a set of helpers
 for properly instantiating process network patterns as process
 constructors.\par
\begin{mdframed}[style=reminder,frametitle=Reminder]Make sure to consult naming conventions in  \cref{sec:forsyde-atom:naming-convention} in order to interpret the names and type signatures correctly.\end{mdframed}\par

\subsection{Discrete event (\haddocktt{DE})}
According to \cite{Lee98}, "a
 discrete-event system is a timed system \emph{Q} where for all \emph{s}
 ∈ \emph{Q}, the tag sytem is order-isomorphic to a subset of the
 integers. Order-isomorphic means simply that there exists an
 order-preserving bijection between the events in \emph{T} and a subset
 of the integers (or the entire set of integers)."\par
The discrete event (\haddocktt{DE}) MoC does suggest the notion of physical
 time through its tags, also called timestamps. As the definition
 above implies, an important property of the DE tag system is that
 between any two timestamps \emph{tᵤ} and \emph{tᵥ} there is a
 \textbf{\emph{finite}} number of possible timestamps. Based on this we can
 formulate the folowing specialized definition:\par
\begin{description}
\item[The DE MoC] is abstracting the execution semantics of a system
 where synchronization is \emph{discretized} and \emph{time-aware}, and it
 is performed whenever a new event occurs.
\end{description}There are many variants of discrete event simulators, each of
 them implementing slight variations of the semantics stated in
 \cite{Lee98}. The execution model covered by
 the DE implementation of ForSyDe-atom may be described as a
 simplified "cycle simulator" with no delta-delay nor superdense
 time. The signals behave as "latched channels" (similar to an HDL
 simulator), and processes react instantaneously to any new
 event. While the simplicity of the execution engine is a desired
 one, more complex behaviors such as zero-time feedback, non-zero
 reaction time and communication protocols (e.g. lossy buffers)
 may be achieved by composing patterns from the \haddocktt{ForSyDe.Atom.MoC}
 and/or \haddocktt{ForSyDe.Atom.ExB} layers. Nevertheless, the DE behaviors
 possible within ForSyDe-Atom are included in the class of
 \emph{conservative simulators} as presented in
 \cite{Fujimoto00}, due to the dataflow
 nature of the evaluation mechanisms. Below you can see an example
 of a simple DE process, without behavior extensions:\par
                 \haddockfig{fig/moc-de-example.pdf}\par
                 Below are stated a few particularities of our DE MoC
 implementation:\par
                 \begin{enumerate}
                 \item 
                 According to \cite{Lee98}, our DE MoC is
 a one-sided system, i.e. time starts from an absolute 0. While
 negative time cannot be represented, signals can be phase-aligned
 with the help of the \haddockid{-{\char '46}-} atom. All signals
 need to start from timestamp 0, and events need to be positioned
 with their tags in strict ascending order. The \haddockid{checkSignal}
 utility enforces these rules.\par
                 
                 \item 
                 tags are explicit and a DE event will construct a type around
 both a tag and a value. Tags represent the start time of the
 event, the end time being implicit from the start time of the
 next event. By doing so, we ensure that the time domain is
 non-disjoint, i.e. a sub-case of continous time. \par
                 
                 \item 
                 according to the previous point, events are assumed to persist
 from their time of arrival until the next event arrives or, if
 there is no incoming event, until infinity. This default behavior
 can be be interpret signals as being either persistent channels
 (e.g. latched wires), or non-blocking buffers of size 1.\par
                 
                 \item 
                 as a consequence to the previous is that feedback loops will
 generate an infinite number of events (strictly preceding each
 other), since a loop updates the value after a certain delay, and
 any input is assumed to go to infinity. Thus we can now fully
 jutify the definition of the \haddockid{delay} pattern as
 consisting in a \emph{prepend} (i.e. generating the new value) and a
 \emph{phase shift} (i.e. advancing time with a positive integer). This
 is done in order to both preserve causality \emph{and} avoid deadlock.\par
                 
                 \item 
                 due to the reactive and dataflow natures of the execution
 system, DE processes \emph{are forbidden} to clean up events. Doing so
 might lead to deadlock wherever any feedback is involved. This
 means that a new event is created every time a new event arrives,
 regardless of what value it carries. This means that \emph{all} values
 are propagated, justifying our system's conservative approach
 \cite{Fujimoto00}. Atoms themselves do not
 clean signals, but using interfaces that do should be treated
 with extreme special care, as it is considered unsafe and
 deadlock-prone.\par
                 
                 \item 
                 due to the conservative approach, and to the fact that MoC
 atoms are independent synchronization entities, ForSyDe
 simulators are completely parallelizable, since processes are
 self-sufficient and do not depend on a global event queue (as
 compared to other cycle simulators).\par
                 
                 \item 
                 any signal from outside needs to be sane (\emph{T} must be a total
 order) before being injected into a ForSyDe process
 network. Helper functions are equipped with sanity
 checkers. Inside a ForSyDe process network, transformations are
 rate-monotonic, thus output signals are guaranteed to be sane.\par
                 
                 \item 
                 since \emph{T} is a total order, there is no need for an
 execution context (see \cref{module:ForSyDe.Atom.MoC}) and we can
 ignore the formatting of functions in \haddocktt{ForSyDe.Atom.MoC}, thus we
 can safely assume:\par
                 
                 \end{enumerate}
                 \haddockeq{fig/eqs-moc-timed-context.pdf}\par
                 
\begin{haddockdesc}
\item[\begin{tabular}{@{}l}
type\ TimeStamp\ =\ DiffTime
\end{tabular}]\haddockbegindoc
Alias for the type representing discrete time. It is inherently
 quantizable, the quantum being a picosecond (10⁻¹²
 seconds), thus it can be considered order-isomorphic with a set of
 integers, i.e. between any two timestamps there is a finite number
 of timestamps. Moreover, a timestamp can be easily translated into
 a rational number representing fractions of a second, so the
 conversion between timestamps (discrete time) and rationals
 (analog/continuous time) is straightforward.\par
This type is used in the explicit tags of the
 \haddockid{DE} MoC (and subsequently the discrete event
 evaluation engine for simulating the \haddockid{CT} MoC).\par


\item[\begin{tabular}{@{}l}
data\ DE\ a
\end{tabular}]\haddockbegindoc
\haddockbeginconstrs
\haddockdecltt{=} & \haddockdecltt{DE} & \\
                    \haddockdecltt{>}&\haddockdecltt{tag :: TimeStamp} & timestamp\\
                    \haddockdecltt{>}&\haddockdecltt{val :: a} & the value\\
\end{tabulary}\par

The DE event. It identifies a discrete event signal.\par


\item[\begin{tabular}{@{}l}
instance\ Functor\ DE
\end{tabular}]\haddockbegindoc
Allows for mapping of functions on a DE event.\par


\item[\begin{tabular}{@{}l}
instance\ Applicative\ DE
\end{tabular}]\haddockbegindoc
Allows for lifting functions on a pair of DE events.\par


\item[\begin{tabular}{@{}l}
instance\ MoC\ DE
\end{tabular}]\haddockbegindoc
Implenents the execution and synchronization semantics for the DE
 MoC through its atoms.\par


\item[\begin{tabular}{@{}l}
instance\ Eq\ a\ =>\ Eq\ (DE\ a)
\end{tabular}]

\item[\begin{tabular}{@{}l}
instance\ Read\ a\ =>\ Read\ (DE\ a)
\end{tabular}]\haddockbegindoc
Reads the string of type \haddocktt{v@t} as an event \haddocktt{DE\ t\ v}.\par


\item[\begin{tabular}{@{}l}
instance\ Show\ a\ =>\ Show\ (DE\ a)
\end{tabular}]\haddockbegindoc
Shows the event with tag \haddocktt{t} and value \haddocktt{v} as \haddocktt{\ v\ @t}.\par


\item[\begin{tabular}{@{}l}
instance\ Plottable\ a\ =>\ Plot\ (Signal\ a)
\end{tabular}]\haddockbegindoc
\haddockid{DE} signals.\par


\item[\begin{tabular}{@{}l}
instance\ type\ Ret\ DE\ b\ =\ b\\instance\ type\ Fun\ DE\ a\ b\ =\ a\ ->\ b
\end{tabular}]
\end{haddockdesc}
\subsection{Aliases {\char '46} utilities}
A set of type synonyms and utilities are provided for
 convenience. The API type signatures will feature these aliases
 to hide the cumbersome construction of atoms and patters as seen
 in \haddocktt{ForSyDe.Atom.MoC}.\par

\begin{haddockdesc}
\item[\begin{tabular}{@{}l}
type\ Signal\ a\ =\ Stream\ (DE\ a)
\end{tabular}]\haddockbegindoc
Type synonym for a SY signal, i.e. "a signal of SY events"\par


\item[\begin{tabular}{@{}l}
unit2\ ::\ ((TimeStamp,\ a1),\ (TimeStamp,\ a2))\\\ \ \ \ \ \ \ \ \ ->\ (Signal\ a1,\ Signal\ a2)
\end{tabular}]\haddockbegindoc
Wraps a (tuple of) pair(s) \haddocktt{(tag,\ value)} into the equivalent
 unit signal(s), in this case a signal with one event with the
 period \haddocktt{tag} carrying \haddocktt{value}.\par
Helpers: \haddocktt{unit} and \haddocktt{unit{\char 91}2-4{\char 93}}.\par


\item[\begin{tabular}{@{}l}
infinite\ ::\ a\ ->\ Signal\ a
\end{tabular}]\haddockbegindoc
Creates an infinite signal.\par


\item[\begin{tabular}{@{}l}
until\ ::\ TimeStamp\ ->\ Signal\ a\ ->\ Signal\ a
\end{tabular}]\haddockbegindoc
Takes the first part of the signal util a given timestamp. The
 last event of the resulting signal is at the given timestamp and
 carries the previous value. This utility is useful when plotting
 a signal, to specify the interval of plotting.\par


\item[\begin{tabular}{@{}l}
signal\ ::\ {\char 91}(TimeStamp,\ a){\char 93}\ ->\ Signal\ a
\end{tabular}]\haddockbegindoc
Transforms a list of tuples \haddocktt{(tag,\ value)} into a DE
 signal. Checks if it is well-formed.\par


\item[\begin{tabular}{@{}l}
checkSignal\ ::\ Stream\ (DE\ a)\ ->\ Stream\ (DE\ a)
\end{tabular}]\haddockbegindoc
Checks if a signal is well-formed or not, according to the DE MoC
 interpretation in \haddocktt{ForSyDe-Atom}.\par


\item[\begin{tabular}{@{}l}
readSignal\ ::\ Read\ a\ =>\ String\ ->\ Signal\ a
\end{tabular}]\haddockbegindoc
Reads a signal from a string and checks if it is well-formed.
 Like with the \haddocktt{read} function from \haddocktt{Prelude}, you must specify the
 type of the signal.\par
\begin{interactive}
λ> readSignal "{ 1@0, 2@2, 3@5, 4@7, 5@10 }" :: Signal Int
{ 1 @0s, 2 @2s, 3 @5s, 4 @7s, 5 @10s}
λ> readSignal "{ 1@0, 2@2, 3@5, 4@10, 5@7 }" :: Signal Int
{ 1 @0s, 2 @2s, 3 @5s*** Exception: [MoC.DE] malformed signal
λ> readSignal "{ 1@1, 2@2, 3@5, 4@7, 5@10 }" :: Signal Int
*** Exception: [MoC.DE] signal does not start from global 0
\end{interactive}
           
\end{haddockdesc}
\subsection{\haddocktt{DE} process constuctors}
The DE process constructors are basically specific
 instantiations of patterns defined in \haddocktt{ForSyDe.Atom.MoC}. Some
 might also be wrapping functions in an extended behavioural
 model.\par

\subsubsection{Simple}
These are mainly direct instantiations of patterns defined in
 \haddocktt{ForSyDe.Atom.MoC}, using DE-specific utilities.\par

\begin{haddockdesc}
\item[\begin{tabular}{@{}l}
delay
\end{tabular}]\haddockbegindoc
\haddockbeginargs
\haddockdecltt{::} & \haddockdecltt{TimeStamp} & time delay \\
                                                 \haddockdecltt{->} & \haddockdecltt{a} & initial value \\
                                                                                          \haddockdecltt{->} & \haddockdecltt{Signal a} & input signal \\
                                                                                                                                          \haddockdecltt{->} & \haddockdecltt{Signal a} & output signal \\
\end{tabulary}\par
The \haddocktt{delay} process "delays" a signal with one
 event. Instantiates the \haddockid{delay} pattern.\par
\begin{interactive}
λ> let s = readSignal "{1@0, 2@2, 3@6, 4@8, 5@9}" :: Signal Int
λ> delay 3 0 s
{ 0 @0s, 1 @3s, 2 @5s, 3 @9s, 4 @11s, 5 @12s}

\end{interactive}\haddockfig{fig/moc-de-pattern-delay.pdf}\par
           

\item[\begin{tabular}{@{}l}
delay'
\end{tabular}]\haddockbegindoc
\haddockbeginargs
\haddockdecltt{::} & \haddockdecltt{Signal a} & signal "borrowing" the initial event \\
                                                \haddockdecltt{->} & \haddockdecltt{Signal a} & input signal \\
                                                                                                \haddockdecltt{->} & \haddockdecltt{Signal a} & output signal \\
\end{tabulary}\par
Similar to the previous, but this is the raw instantiation of the
 \haddockid{delay} pattern. It "borrows" the first event from
 one signal and appends it at the head of another signal.\par
\begin{interactive}
λ> let s1 = readSignal "{1@0, 2@2, 3@6, 4@8, 5@9}" :: Signal Int
λ> let s2 = readSignal "{3@0, 4@4, 5@5, 6@8, 7@9}" :: Signal Int
λ> delay' s1 s2
{ 1 @0s, 3 @2s, 4 @6s, 5 @7s, 6 @10s, 7 @11s}

\end{interactive}\haddockfig{fig/moc-de-pattern-delayp.pdf}\par
           

\item[\begin{tabular}{@{}l}
comb22
\end{tabular}]\haddockbegindoc
\haddockbeginargs
\haddockdecltt{::} & \haddockdecltt{(a1
                                     -> a2 -> (b1, b2))} & function on values \\
                                                           \haddockdecltt{->} & \haddockdecltt{Signal a1} & first input signal \\
                                                                                                            \haddockdecltt{->} & \haddockdecltt{Signal a2} & second input signal \\
                                                                                                                                                             \haddockdecltt{->} & \haddockdecltt{(Signal b1, Signal b2)} & two output signals \\
\end{tabulary}\par
\haddocktt{comb} processes map combinatorial functions on signals and take
 care of synchronization between input signals. It instantiates the
 \haddocktt{comb} pattern (see \haddockid{comb22}).\par
Constructors: \haddocktt{comb{\char 91}1-4{\char 93}{\char 91}1-4{\char 93}}.\par
\begin{interactive}
λ> let s1 = infinite 1
λ> let s2 = readSignal "{1@0, 2@2, 3@6, 4@8, 5@9}" :: Signal Int
λ> comb11 (+1) s2
{ 2 @0s, 3 @2s, 4 @6s, 5 @8s, 6 @9s}
λ> comb22 (\a b-> (a+b,a-b)) s1 s2
({ 2 @0s, 3 @2s, 4 @6s, 5 @8s, 6 @9s},{ 0 @0s, -1 @2s, -2 @6s, -3 @8s, -4 @9s})

\end{interactive}\haddockfig{fig/moc-de-pattern-comb.pdf}\par
           

\item[\begin{tabular}{@{}l}
reconfig22
\end{tabular}]\haddockbegindoc
\haddockbeginargs
\haddockdecltt{::} & \haddockdecltt{Signal (a1
                                            -> a2
                                               -> (b1, b2))} & signal carrying functions \\
                                                               \haddockdecltt{->} & \haddockdecltt{Signal a1} & first input signal carrying arguments \\
                                                                                                                \haddockdecltt{->} & \haddockdecltt{Signal a2} & second input signal carrying arguments \\
                                                                                                                                                                 \haddockdecltt{->} & \haddockdecltt{(Signal b1, Signal b2)} & two output signals \\
\end{tabulary}\par
\haddocktt{reconfig} creates a DE adaptive process where the first signal
 carries functions and the other carry the arguments. It
 instantiates the \haddocktt{reconfig} atom pattern (see
 \haddockid{reconfig22}).\par
Constructors: \haddocktt{reconfig{\char 91}1-4{\char 93}{\char 91}1-4{\char 93}}.\par
\begin{interactive}
λ> let sf = signal [(0,(+1)),(2,(*2)),(5,(+1)),(7,(*2))]
λ> let s1 = signal [(0,1),(3,2),(5,3),(9,4)]
λ> reconfig11 sf s1
{ 2 @0s, 2 @2s, 4 @3s, 4 @5s, 6 @7s, 8 @9s}

\end{interactive}\haddockfig{fig/moc-de-pattern-reconfig.pdf}\par
           

\item[\begin{tabular}{@{}l}
sync2
\end{tabular}]\haddockbegindoc
\haddockbeginargs
\haddockdecltt{::} & \haddockdecltt{Signal a1} & first input signal \\
                                                 \haddockdecltt{->} & \haddockdecltt{Signal a2} & second input signal \\
                                                                                                  \haddockdecltt{->} & \haddockdecltt{(Signal a1, Signal a2)} & two output signals \\
\end{tabulary}\par
\haddocktt{sync} synchronizes multiple signals, so that they have the same
 set of tags, and consequently, the same number of events. It
 instantiates the \haddocktt{comb} atom pattern (see
 \haddockid{comb22}).\par
Constructors: \haddocktt{sync{\char 91}1-4{\char 93}}\par
\begin{interactive}
λ> let s1 = readSignal "{1@0, 2@2, 3@6, 4@8,  5@9}"  :: Signal Int
λ> let s2 = readSignal "{1@0, 2@5, 3@6, 4@10, 5@12}" :: Signal Int
λ> sync2 s1 s2
({ 1 @0s, 2 @2s, 2 @5s, 3 @6s, 4 @8s, 5 @9s, 5 @10s, 5 @12s},{ 1 @0s, 1 @2s, 2 @5s, 3 @6s, 3 @8s, 3 @9s, 4 @10s, 5 @12s})

\end{interactive}\haddockfig{fig/moc-de-pattern-sync.pdf}\par
           

\item[\begin{tabular}{@{}l}
constant2
\end{tabular}]\haddockbegindoc
\haddockbeginargs
\haddockdecltt{::} & \haddockdecltt{(b1, b2)} & values to be repeated \\
                                                \haddockdecltt{->} & \haddockdecltt{(Signal b1, Signal b2)} & generated signals \\
\end{tabulary}\par
A signal generator which keeps a value constant. As compared with
 the \haddockid{SY}, it just constructs an infinite signal
 with constant value (i.e. a signal with one event starting from
 time 0).\par
Constructors: \haddocktt{constant{\char 91}1-4{\char 93}}.\par
\begin{interactive}
λ> constant1 2
{ 2 @0s}

\end{interactive}\haddockfig{fig/moc-de-pattern-constant.pdf}\par
           

\item[\begin{tabular}{@{}l}
generate2
\end{tabular}]\haddockbegindoc
\haddockbeginargs
\haddockdecltt{::} & \haddockdecltt{(b1
                                     -> b2
                                        -> (b1, b2))} & function to generate next value \\
                                                        \haddockdecltt{->} & \haddockdecltt{((TimeStamp, b1), (TimeStamp, b2))} & kernel values tupled with their generation rate. \\
                                                                                                                                  \haddockdecltt{->} & \haddockdecltt{(Signal b1, Signal b2)} & generated signals \\
\end{tabulary}\par
A signal generator based on a function and a kernel value. It
 is actually an instantiation of the \haddocktt{stated0X} constructor
 (check \haddockid{stated22}).\par
Constructors: \haddocktt{generate{\char 91}1-4{\char 93}}.\par
\begin{interactive}
λ> let (s1,s2) = generate2 (\a b -> (a+1,b+2)) ((3,1),(1,2))
λ> takeS 5 s1
{ 1 @0s, 2 @3s, 2 @4s, 2 @5s, 3 @6s}
λ> takeS 7 s2
{ 2 @0s, 4 @1s, 6 @2s, 8 @3s, 10 @4s, 12 @5s, 14 @6s}

\end{interactive}\haddockfig{fig/moc-de-pattern-generate.pdf}\par
           

\item[\begin{tabular}{@{}l}
stated22
\end{tabular}]\haddockbegindoc
\haddockbeginargs
\haddockdecltt{::} & \haddockdecltt{(b1
                                     -> b2
                                        -> a1
                                           -> a2
                                              -> (b1, b2))} & next state function \\
                                                              \haddockdecltt{->} & \haddockdecltt{((TimeStamp, b1), (TimeStamp, b2))} & initial state values tupled with their initial delay \\
                                                                                                                                        \haddockdecltt{->} & \haddockdecltt{Signal a1} & first input signal \\
                                                                                                                                                                                         \haddockdecltt{->} & \haddockdecltt{Signal a2} & second input signal \\
                                                                                                                                                                                                                                          \haddockdecltt{->} & \haddockdecltt{(Signal b1, Signal b2)} & output signals \\
\end{tabulary}\par
\haddocktt{stated} is a state machine without an output decoder. It is an
 instantiation of the \haddocktt{state} MoC constructor (see
 \haddockid{stated22}).\par
Constructors: \haddocktt{stated{\char 91}1-4{\char 93}{\char 91}1-4{\char 93}}.\par
\begin{interactive}
λ> let s = readSignal "{1@0, 2@2, 3@6, 4@8, 5@9}" :: Signal Int
λ> takeS 7 $ stated11 (+) (6,1) s
{ 1 @0s, 2 @6s, 3 @8s, 5 @12s, 7 @14s, 8 @15s, 10 @18s}

\end{interactive}\haddockfig{fig/moc-de-pattern-stated.pdf}\par
           

\item[\begin{tabular}{@{}l}
state22
\end{tabular}]\haddockbegindoc
\haddockbeginargs
\haddockdecltt{::} & \haddockdecltt{(b1
                                     -> b2
                                        -> a1
                                           -> a2
                                              -> (b1, b2))} & next state function \\
                                                              \haddockdecltt{->} & \haddockdecltt{((TimeStamp, b1), (TimeStamp, b2))} & initial state values tupled with their initial delay \\
                                                                                                                                        \haddockdecltt{->} & \haddockdecltt{Signal a1} & first input signal \\
                                                                                                                                                                                         \haddockdecltt{->} & \haddockdecltt{Signal a2} & second input signal \\
                                                                                                                                                                                                                                          \haddockdecltt{->} & \haddockdecltt{(Signal b1, Signal b2)} & output signals \\
\end{tabulary}\par
\haddocktt{state} is a state machine without an output decoder, and the
 state non-transparent. It is an instantiation of the \haddocktt{state} MoC
 constructor (see \haddockid{state22}).\par
Constructors: \haddocktt{state{\char 91}1-4{\char 93}{\char 91}1-4{\char 93}}.\par
\begin{interactive}
λ> let s = readSignal "{1@0, 2@2, 3@6, 4@8, 5@9}" :: Signal Int
λ> takeS 7 $ state11 (+) (6,1) s
{ 2 @0s, 3 @2s, 5 @6s, 7 @8s, 8 @9s, 10 @12s, 12 @14s}

\end{interactive}\haddockfig{fig/moc-de-pattern-state.pdf}                   \par
           

\item[\begin{tabular}{@{}l}
moore22
\end{tabular}]\haddockbegindoc
\haddockbeginargs
\haddockdecltt{::} & \haddockdecltt{(st
                                     -> a1
                                        -> a2 -> st)} & next state function \\
                                                        \haddockdecltt{->} & \haddockdecltt{(st
                                                                                             -> (b1, b2))} & output decoder \\
                                                                                                             \haddockdecltt{->} & \haddockdecltt{(TimeStamp, st)} & initial state: tag and value \\
                                                                                                                                                                    \haddockdecltt{->} & \haddockdecltt{Signal a1} & \\
                                                                                                                                                                                                                     \haddockdecltt{->} & \haddockdecltt{Signal a2} & \\
                                                                                                                                                                                                                                                                      \haddockdecltt{->} & \haddockdecltt{(Signal b1, Signal b2)} & \\
\end{tabulary}\par
\haddocktt{moore} processes model Moore state machines. It is an
 instantiation of the \haddocktt{moore} MoC constructor (see
 \haddockid{moore22}).\par
Constructors: \haddocktt{moore{\char 91}1-4{\char 93}{\char 91}1-4{\char 93}}\par
\begin{interactive}
λ> let s = readSignal "{1@0, 2@2, 3@6, 4@8, 5@9}" :: Signal Int
λ> takeS 7 $ moore11 (+) (+1) (6,1) s
{ 2 @0s, 3 @6s, 4 @8s, 6 @12s, 8 @14s, 9 @15s, 11 @18s}

\end{interactive}\haddockfig{fig/moc-de-pattern-moore.pdf}          \par
           

\item[\begin{tabular}{@{}l}
mealy22
\end{tabular}]\haddockbegindoc
\haddockbeginargs
\haddockdecltt{::} & \haddockdecltt{(st
                                     -> a1
                                        -> a2 -> st)} & next state function \\
                                                        \haddockdecltt{->} & \haddockdecltt{(st
                                                                                             -> a1
                                                                                                -> a2
                                                                                                   -> (b1, b2))} & outpt decoder \\
                                                                                                                   \haddockdecltt{->} & \haddockdecltt{(TimeStamp, st)} & initial state: tag and value \\
                                                                                                                                                                          \haddockdecltt{->} & \haddockdecltt{Signal a1} & \\
                                                                                                                                                                                                                           \haddockdecltt{->} & \haddockdecltt{Signal a2} & \\
                                                                                                                                                                                                                                                                            \haddockdecltt{->} & \haddockdecltt{(Signal b1, Signal b2)} & \\
\end{tabulary}\par
\haddocktt{mealy} processes model Mealy state machines. It is an
 instantiation of the \haddocktt{mealy} MoC constructor
 (see \haddockid{mealy22}).\par
Constructors: \haddocktt{mealy{\char 91}1-4{\char 93}{\char 91}1-4{\char 93}}\par
\begin{interactive}
λ> let s = readSignal "{1@0, 2@2, 3@6, 4@8, 5@9}" :: Signal Int
λ> takeS 7 $ mealy11 (+) (-) (6,1) s
{ 0 @0s, -1 @2s, -1 @6s, -1 @8s, -2 @9s, 0 @12s, 2 @14s}

\end{interactive}\haddockfig{fig/moc-de-pattern-mealy.pdf}\par
           
\end{haddockdesc}
\subsubsection{Interface processes}
\begin{haddockdesc}
\item[\begin{tabular}{@{}l}
toSY2
\end{tabular}]\haddockbegindoc
\haddockbeginargs
\haddockdecltt{::} & \haddockdecltt{Signal a} & first input DE signal \\
                                                \haddockdecltt{->} & \haddockdecltt{Signal b} & second input DE signal \\
                                                                                                \haddockdecltt{->} & \haddockdecltt{(Signal TimeStamp, Signal a, Signal b)} & signal carrying timestamps tupled with the two output
 \haddockid{SY} signals \\
\end{tabulary}\par
Synchronizes a (set of) \haddockid{DE} signal(s) an
 strips off their explicit tags, outputting the equivalent
 \haddockid{SY} signal(s), tupled with an SY signal
 carrying the timestamps for the synchronization points.\par
Constructors : \haddocktt{toSY{\char 91}1-4{\char 93}}. \par
\begin{interactive}
λ> let s1 = DE.infinite 1
λ> let s2 = DE.readSignal "{1@0, 2@2, 3@6, 4@8, 5@9}" :: DE.Signal Int
λ> toSY2 s1 s2
({0s,2s,6s,8s,9s},{1,1,1,1,1},{1,2,3,4,5})

\end{interactive}\haddockfig{fig/moc-de-tosy.pdf}\par
           

\item[\begin{tabular}{@{}l}
toCT2
\end{tabular}]\haddockbegindoc
\haddockbeginargs
\haddockdecltt{::} & \haddockdecltt{Signal (Time
                                            -> a)} & first input DE signal \\
                                                     \haddockdecltt{->} & \haddockdecltt{Signal (Time
                                                                                                 -> b)} & second input DE signal \\
                                                                                                          \haddockdecltt{->} & \haddockdecltt{(Signal a, Signal b)} & two output \haddockid{CT} signals \\
\end{tabulary}\par
Semantic preserving transformation between a (set of) DE
 signal(s) and the equivalent CT signals. The
 \haddockid{DE} events must carry a function of \haddockid{Time}
 which will be lifted by providing it with \haddockid{CT}
 implicit time semantics.\par
Constructors: \haddocktt{toCT{\char 91}1-4{\char 93}}.\par
\haddockfig{fig/moc-de-toct.pdf}\par


\item[\begin{tabular}{@{}l}
zipx\ ::\ Vector\ (Signal\ a)\ ->\ Signal\ (Vector\ a)
\end{tabular}]\haddockbegindoc
Synchronizes all the signals contained by a vector and zips them
 into one signal of vectors. It instantiates the
 \haddockid{zipx} skeleton.\par
\begin{interactive}
λ> let s1 = DE.readSignal "{1@0, 2@2, 3@6, 4@8, 5@9}" :: DE.Signal Int
λ> let s2 = DE.readSignal "{1@0, 2@2, 3@4, 4@8, 5@9}" :: DE.Signal Int
λ> let v1 = V.vector [s1,s1,s2,s2]
λ> v1
<{ 1 @0s, 2 @2s, 3 @6s, 4 @8s, 5 @9s},{ 1 @0s, 2 @2s, 3 @6s, 4 @8s, 5 @9s},{ 1 @0s, 2 @2s, 3 @4s, 4 @8s, 5 @9s},{ 1 @0s, 2 @2s, 3 @4s, 4 @8s, 5 @9s}>
λ> zipx v1
{ <1,1,1,1> @0s, <2,2,2,2> @2s, <2,2,3,3> @4s, <3,3,3,3> @6s, <4,4,4,4> @8s, <5,5,5,5> @9s}

\end{interactive}\haddockfig{fig/moc-de-zipx.pdf}\par
           

\item[\begin{tabular}{@{}l}
unzipx\ ::\ Integer\ ->\ Signal\ (Vector\ a)\ ->\ Vector\ (Signal\ a)
\end{tabular}]\haddockbegindoc
Unzips the vectors carried by a signal into a vector of
 signals. It instantiates the \haddockid{unzipx}
 skeleton. To avoid infinite recurrence, the user needs to provide
 the length of the output vector.\par
\begin{interactive}
λ> let v1 = V.vector [1,2,3,4]
λ> let s1 = DE.signal [(0,v1),(2,v1),(6,v1),(8,v1),(9,v1)]
λ> s1
{ <1,2,3,4> @0s, <1,2,3,4> @2s, <1,2,3,4> @6s, <1,2,3,4> @8s, <1,2,3,4> @9s}
λ> unzipx 4 s1
<{ 1 @0s, 1 @2s, 1 @6s, 1 @8s, 1 @9s},{ 2 @0s, 2 @2s, 2 @6s, 2 @8s, 2 @9s},{ 3 @0s, 3 @2s, 3 @6s, 3 @8s, 3 @9s},{ 4 @0s, 4 @2s, 4 @6s, 4 @8s, 4 @9s}>

\end{interactive}\haddockfig{fig/moc-de-unzipx.pdf}\par
           

\item[\begin{tabular}{@{}l}
unzipx'\ ::\ Signal\ (Vector\ a)\ ->\ Vector\ (Signal\ a)
\end{tabular}]\haddockbegindoc
Same as \haddockid{unzipx}, but "sniffs" the first event to determine the length of the output vector. Might have unsafe behavior!\par
\begin{interactive}
λ> let v1 = V.vector [1,2,3,4]
λ> let s1 = DE.signal [(0,v1),(2,v1),(6,v1),(8,v1),(9,v1)]
λ> s1
{ <1,2,3,4> @0s, <1,2,3,4> @2s, <1,2,3,4> @6s, <1,2,3,4> @8s, <1,2,3,4> @9s}
λ> unzipx' s1
<{ 1 @0s, 1 @2s, 1 @6s, 1 @8s, 1 @9s},{ 2 @0s, 2 @2s, 2 @6s, 2 @8s, 2 @9s},{ 3 @0s, 3 @2s, 3 @6s, 3 @8s, 3 @9s},{ 4 @0s, 4 @2s, 4 @6s, 4 @8s, 4 @9s}>

\end{interactive}
\end{haddockdesc}
\subsubsection{Hybrid processes}
\begin{haddockdesc}
\item[\begin{tabular}{@{}l}
embedSY22
\end{tabular}]\haddockbegindoc
\haddockbeginargs
\haddockdecltt{::} & \haddockdecltt{(Signal a1
                                     -> Signal a2
                                        -> (Signal b1, Signal b2))} & \haddockid{SY} process \\
                                                                      \haddockdecltt{->} & \haddockdecltt{Signal a1} & first input DE signal \\
                                                                                                                       \haddockdecltt{->} & \haddockdecltt{Signal a2} & second input DE signal  \\
                                                                                                                                                                        \haddockdecltt{->} & \haddockdecltt{(Signal b1, Signal b2)} & two output \haddockid{DE} signals \\
\end{tabulary}\par
Embeds a \haddockid{SY} process inside a
 \haddockid{DE} environment. Internally, it synchronizes
 the input signals, translates them to SY, feeds them to a SY
 process and translates the result back to DE using the same input
 tags. Seen from outside, this process behaves like a DE process
 with "instantaneous response", even for feedback loops.\par
Constructors: \haddocktt{embedSY{\char 91}1-4{\char 93}{\char 91}1-4{\char 93}}.\par
For the following example, see the difference between its output
 and the one of \haddockid{stated22}\par
\begin{interactive}
λ> let s = readSignal "{1@0, 2@2, 3@6, 4@8, 5@9}" :: DE.Signal Int
λ> embedSY11 (SY.stated11 (+) 1) s
{ 1 @0s, 2 @2s, 4 @6s, 7 @8s, 11 @9s}

\end{interactive}\haddockfig{fig/moc-de-pattern-embedsy.pdf}\par
           
\end{haddockdesc}
  \haddockmoduleheading{ForSyDe.Atom.MoC.SDF}
\label{module:ForSyDe.Atom.MoC.SDF}
\haddockbeginheader
{\haddockverb\begin{verbatim}
module ForSyDe.Atom.MoC.SDF (
    SDF(SDF, val),  Signal,  Prod,  Cons,  signal,  readSignal,  delay, 
    delay',  comb22,  reconfig22,  constant2,  generate2,  stated22,  state22, 
    moore22,  mealy22,  toSY2,  zipx,  unzipx
  ) where\end{verbatim}}
\haddockendheader

The \haddocktt{SDF} library implements the atoms holding the sematics for the
 synchronous data flow computation model. It also provides a set of
 helpers for properly instantiating process network patterns as
 process constructors.\par
\begin{mdframed}[style=reminder,frametitle=Reminder]Make sure to consult naming conventions in  \cref{sec:forsyde-atom:naming-convention} in order to interpret the names and type signatures correctly.\end{mdframed}\par

\subsection{Synchronous data flow (\haddocktt{SDF}) event}
The synchronous data flow (\haddocktt{SDF}) MoC is the first untimed MoC
 implemented by the \haddocktt{forsyde-atom} framework. On untimed MoCs,
 \cite{Lee98} states that: "when tags are
 partially ordered rather than totally ordered, we say that the
 system is untimed. Untimed systems cannot have the same notion of
 causality as timed systems {\char 91}see \haddockid{SY}{\char 93}. (...)
 Processes defined in terms of constraints on the tags in the
 signals (...) have a \emph{consistent cut} rather than
 \emph{simultaneity}."  Regarding SDF, it states that "is a special
 case of Kahn process networks
 \cite{Kahn76}. A dataflow process is a Kahn
 process that is also sequential, where the events on the
 self-loop signal denote the firings of the dataflow actor. The
 firing rules of a dataflow actor are partial ordering constraints
 between these events and events on the inputs. (...)
 Produced/consumed events are defined in terms of relations with
 the events in the firing signal. It results that for the same
 firing \emph{i}, eᵢ < eₒ, as an intuitive sort of
 causality constraint."\par
Based on the above insights, we can formulate a simplified
 definition of the \haddocktt{forsyde-atom} interpretation of SDF:\par
\begin{description}
\item[The SDF MoC] is abstracting the execution semantics of a system
 where computation is performed according to firing rules where
 the production and the consumption rates are fixed.
\end{description}Below is a \emph{possible} behavior in time of the input and the
 output signals of a SDF process. Events sharing the same partial
 ordering in relation to one firing are overlined:\par
                 \haddockfig{fig/moc-sdf-example.pdf}\par
                 Implementing the SDF tag system implied a series of engineering
 decisions which lead to the following particularities:\par
                 \begin{enumerate}
                 \item 
                 signals represent FIFO channels, and tags are implicit from
 their position in the \haddockid{Stream}
 structure. Internally, \haddockid{SDF} signals have
 exactly the same structure as \haddockid{SY} signals,
 whereas the partial ordering is imposed by the processes alone.\par
                 
                 \item 
                 the \haddockid{SDF} event constructor wraps only a
 value.\par
                 
                 \item 
                 being an \emph{untimed MoC}, the order between events is partial to
 the firings of processes. An SDF atom will fire only when there
 are enough events to trigger its inputs. Once a firing occurs, it
 will take care of partitioning the input or output signals.\par
                 
                 \item 
                 SDF atoms \emph{do} require a context: the consumption \emph{c} and
 production \emph{p} rates. Also, the functions passed as arguments
 reflect the fact that multiple events are handled during a
 firing.\par
                 
                 \item 
                 the previous statement can be synthesized into the following
 execution context (see \cref{module:ForSyDe.Atom.MoC}), which also
 justifies the SDF implementation of \haddockid{Fun} and
 for \haddockid{Ret}:\par
                 
                 \end{enumerate}
                 \haddockeq{fig/eqs-moc-sdf-context.pdf}\par
                 
\begin{haddockdesc}
\item[\begin{tabular}{@{}l}
newtype\ SDF\ a
\end{tabular}]\haddockbegindoc
\haddockbeginconstrs
\haddockdecltt{=} & \haddockdecltt{SDF} & \\
                    \haddockdecltt{>}&\haddockdecltt{val :: a} &\\
\end{tabulary}\par

The CT type, identifying a discrete time event and implementing an
 instance of the \haddockid{MoC} class. A discrete event explicitates its tag
 which is represented as an integer.\par


\item[\begin{tabular}{@{}l}
instance\ Functor\ SDF
\end{tabular}]\haddockbegindoc
Allows for mapping of functions on a SDF event.\par


\item[\begin{tabular}{@{}l}
instance\ Applicative\ SDF
\end{tabular}]\haddockbegindoc
Allows for lifting functions on a pair of SDF events.\par


\item[\begin{tabular}{@{}l}
instance\ Foldable\ SDF\\instance\ Traversable\ SDF
\end{tabular}]

\item[\begin{tabular}{@{}l}
instance\ MoC\ SDF
\end{tabular}]\haddockbegindoc
Implenents the SDF semantics for the MoC atoms\par


\item[\begin{tabular}{@{}l}
instance\ Read\ a\ =>\ Read\ (SDF\ a)
\end{tabular}]\haddockbegindoc
Reads the value wrapped\par


\item[\begin{tabular}{@{}l}
instance\ Show\ a\ =>\ Show\ (SDF\ a)
\end{tabular}]\haddockbegindoc
Shows the value wrapped\par


\item[\begin{tabular}{@{}l}
instance\ Plottable\ a\ =>\ Plot\ (Signal\ a)
\end{tabular}]\haddockbegindoc
\haddockid{SDF} signals.\par


\item[\begin{tabular}{@{}l}
instance\ type\ Ret\ SDF\ a\ =\ (Prod,\ {\char 91}a{\char 93})\\instance\ type\ Fun\ SDF\ a\ b\ =\ (Cons,\ {\char 91}a{\char 93}\ ->\ b)
\end{tabular}]
\end{haddockdesc}
\subsection{Aliases {\char '46} utilities}
A set of type synonyms and utilities are provided for
 convenience. The API type signatures will feature these aliases
 to hide the cumbersome construction of atoms and patters as seen
 in \haddocktt{ForSyDe.Atom.MoC}.\par

\begin{haddockdesc}
\item[\begin{tabular}{@{}l}
type\ Signal\ a\ =\ Stream\ (SDF\ a)
\end{tabular}]\haddockbegindoc
Type synonym for a SY signal, i.e. "a signal of SY events"\par


\item[\begin{tabular}{@{}l}
type\ Prod\ =\ Int
\end{tabular}]\haddockbegindoc
Type synonym for consumption rate\par


\item[\begin{tabular}{@{}l}
type\ Cons\ =\ Int
\end{tabular}]\haddockbegindoc
Type synonym for production rate\par


\item[\begin{tabular}{@{}l}
signal\ ::\ {\char 91}a{\char 93}\ ->\ Signal\ a
\end{tabular}]\haddockbegindoc
Transforms a list of values into a SDF signal with only one
 partition, i.e. all events share the same (initial) tag.\par


\item[\begin{tabular}{@{}l}
readSignal\ ::\ Read\ a\ =>\ String\ ->\ Signal\ a
\end{tabular}]\haddockbegindoc
Reads a signal from a string. Like with the \haddocktt{read} function from
 \haddocktt{Prelude}, you must specify the tipe of the signal.\par
\begin{interactive}
λ> readSignal "{1,2,3,4,5}" :: Signal Int
{1,2,3,4,5}

\end{interactive}
\end{haddockdesc}
These SY process constructors are basically specific
 instantiations of the patterns of atoms defined in
 \haddocktt{ForSyDe.Atom.MoC}. Some are also wrapping functions in an
 extended behavioural model.\par

\subsubsection{Simple}
\begin{haddockdesc}
\item[\begin{tabular}{@{}l}
delay
\end{tabular}]\haddockbegindoc
\haddockbeginargs
\haddockdecltt{::} & \haddockdecltt{[a]} & list of initial values \\
                                           \haddockdecltt{->} & \haddockdecltt{Signal a} & input signal \\
                                                                                           \haddockdecltt{->} & \haddockdecltt{Signal a} & output signal \\
\end{tabulary}\par
The \haddocktt{delay} process "delays" a signal with initial events built
 from a list. It is an instantiation of the \haddockid{delay}
 constructor.\par
\begin{interactive}
λ> let s = signal [1,2,3,4,5]
λ> delay [0,0,0] s
{0,0,0,1,2,3,4,5}

\end{interactive}\haddockfig{fig/moc-sdf-pattern-delay.pdf}\par
           

\item[\begin{tabular}{@{}l}
delay'
\end{tabular}]\haddockbegindoc
\haddockbeginargs
\haddockdecltt{::} & \haddockdecltt{Signal a} & signal containing the initial tokens \\
                                                \haddockdecltt{->} & \haddockdecltt{Signal a} & input signal \\
                                                                                                \haddockdecltt{->} & \haddockdecltt{Signal a} & output signal \\
\end{tabulary}\par
Similar to the previous, but this is the raw instantiation of the
 \haddockid{delay} pattern. It appends the contents of one
 signal at the head of another signal.\par
\begin{interactive}
λ> let s1 = signal [0,0,0]
λ> let s2 = signal [1,2,3,4,5]
λ> delay' s1 s2
{0,0,0,1,2,3,4,5}

\end{interactive}\haddockfig{fig/moc-sdf-pattern-delayp.pdf}\par
           

\item[\begin{tabular}{@{}l}
comb22
\end{tabular}]\haddockbegindoc
\haddockbeginargs
\haddockdecltt{::} & \haddockdecltt{((Cons, Cons), (Prod, Prod), [a1]
                                                                 -> [a2]
                                                                    -> ([b1], [b2]))} & function on lists of values, tupled with consumption /
 production rates \\
                                                                                        \haddockdecltt{->} & \haddockdecltt{Signal a1} & first input signal \\
                                                                                                                                         \haddockdecltt{->} & \haddockdecltt{Signal a2} & second input signal \\
                                                                                                                                                                                          \haddockdecltt{->} & \haddockdecltt{(Signal b1, Signal b2)} & two output signals \\
\end{tabulary}\par
\haddocktt{comb} processes map combinatorial functions on signals and take
 care of synchronization between input signals. It instantiates the
 \haddocktt{comb} atom pattern (see \haddockid{comb22}).\par
Constructors: \haddocktt{comb{\char 91}1-4{\char 93}{\char 91}1-4{\char 93}}.\par
\begin{interactive}
λ> let s1 = signal [1..]
λ> let s2 = signal [1,1,1,1,1,1,1]
λ> let f [a,b,c] [d,e] = [a+d, c+e]
λ> comb21 ((3,2),2,f) s1 s2
{2,4,5,7,8,10}

\end{interactive}Incorrect usage (not covered by \haddocktt{doctest}):\par
           \begin{quote}
           {\haddockverb\begin{verbatim}
           λ> comb21 ((3,2),3,f) s1 s2
*** Exception: [MoC.SDF] Wrong production\end{verbatim}}
           \end{quote}
           \haddockfig{fig/moc-sdf-pattern-comb.pdf}\par
           

\item[\begin{tabular}{@{}l}
reconfig22
\end{tabular}]\haddockbegindoc
\haddockbeginargs
\haddockdecltt{::} & \haddockdecltt{((Cons, Cons), (Prod, Prod))} & \\
                                                                    \haddockdecltt{->} & \haddockdecltt{Signal ([a1]
                                                                                                                -> [a2]
                                                                                                                   -> ([b1], [b2]))} & function on lists of values, tupled with consumption /
 production rates \\
                                                                                                                                       \haddockdecltt{->} & \haddockdecltt{Signal a1} & first input signal \\
                                                                                                                                                                                        \haddockdecltt{->} & \haddockdecltt{Signal a2} & second input signal \\
                                                                                                                                                                                                                                         \haddockdecltt{->} & \haddockdecltt{(Signal b1, Signal b2)} & two output signals \\
\end{tabulary}\par
\haddocktt{reconfig} creates an SDF adaptive process where the first signal
 carries functions and the other carry the arguments. It
 instantiates the \haddocktt{reconfig} atom pattern (see
 \haddockid{reconfig22}). According to our SDF definition,
 the production and consumption rates need to be fixed, so they are
 passed as parameters to the constructor, whereas the first signal
 carries adaptive functions only. For the adaptive signal it only
 makes sense that the consumption rate is always 1.\par
Constructors: \haddocktt{reconfig{\char 91}1-4{\char 93}{\char 91}1-4{\char 93}}.\par
\begin{interactive}
λ> let f1 a = [sum a]
λ> let f2 a = [maximum a]
λ> let sf = signal [f1,f2,f1,f2,f1,f2,f1]
λ> let s1 = signal [1..]
λ> reconfig11 (4,1) sf s1
{10,8,42,16,74,24,106}

\end{interactive}\haddockfig{fig/moc-sdf-pattern-reconfig.pdf}\par
           

\item[\begin{tabular}{@{}l}
constant2
\end{tabular}]\haddockbegindoc
\haddockbeginargs
\haddockdecltt{::} & \haddockdecltt{([b1], [b2])} & values to be repeated \\
                                                    \haddockdecltt{->} & \haddockdecltt{(Signal b1, Signal b2)} & generated signals \\
\end{tabulary}\par
A signal generator which repeats the initial tokens
 indefinitely. It is actually an instantiation of the \haddocktt{stated0X}
 constructor (check \haddockid{stated22}).\par
Constructors: \haddocktt{constant{\char 91}1-4{\char 93}}.\par
\begin{interactive}
λ> let (s1, s2) = constant2 ([1,2,3],[2,1])
λ> takeS 7 s1
{1,2,3,1,2,3,1}
λ> takeS 5 s2
{2,1,2,1,2}

\end{interactive}\haddockfig{fig/moc-sdf-pattern-constant.pdf}\par
           

\item[\begin{tabular}{@{}l}
generate2
\end{tabular}]\haddockbegindoc
\haddockbeginargs
\haddockdecltt{::} & \haddockdecltt{((Cons, Cons), (Prod, Prod), [b1]
                                                                 -> [b2]
                                                                    -> ([b1], [b2]))} & function to generate next value, tupled with
 consumption / production rates \\
                                                                                        \haddockdecltt{->} & \haddockdecltt{([b1], [b2])} & values of initial tokens \\
                                                                                                                                            \haddockdecltt{->} & \haddockdecltt{(Signal b1, Signal b2)} & generated signals \\
\end{tabulary}\par
A signal generator based on a function and a kernel value. It
 is actually an instantiation of the \haddocktt{stated0X} constructor
 (check \haddockid{stated22}).\par
Constructors: \haddocktt{generate{\char 91}1-4{\char 93}}.\par
\begin{interactive}
λ> let f a b = ([sum a, sum a],[sum b, sum b, sum b])
λ> let (s1,s2) = generate2 ((2,3),(2,3),f) ([1,1],[2,2,2])
λ> takeS 7 s1
{1,1,2,2,4,4,8}
λ> takeS 8 s2
{2,2,2,6,6,6,18,18}

\end{interactive}\haddockfig{fig/moc-sdf-pattern-generate.pdf}\par
           

\item[\begin{tabular}{@{}l}
stated22
\end{tabular}]\haddockbegindoc
\haddockbeginargs
\haddockdecltt{::} & \haddockdecltt{((Cons, Cons, Cons, Cons), (Prod, Prod), [b1]
                                                                             -> [b2]
                                                                                -> [a1]
                                                                                   -> [a2]
                                                                                      -> ([b1], [b2]))} & next state function, tupled with
 consumption / production rates \\
                                                                                                          \haddockdecltt{->} & \haddockdecltt{([b1], [b2])} & initial state partitions of values \\
                                                                                                                                                              \haddockdecltt{->} & \haddockdecltt{Signal a1} & first input signal \\
                                                                                                                                                                                                               \haddockdecltt{->} & \haddockdecltt{Signal a2} & second input signal \\
                                                                                                                                                                                                                                                                \haddockdecltt{->} & \haddockdecltt{(Signal b1, Signal b2)} & output signals \\
\end{tabulary}\par
\haddocktt{stated} is a state machine without an output decoder. It is an
 instantiation of the \haddocktt{state} MoC constructor (see
 \haddockid{stated22}).\par
Constructors: \haddocktt{stated{\char 91}1-4{\char 93}{\char 91}1-4{\char 93}}.\par
\begin{interactive}
λ> let f [a] [b,c] = [a+b+c]
λ> let s = signal [1,2,3,4,5,6,7]
λ> stated11 ((1,2),1,f) [1] s
{1,4,11,22}

\end{interactive}\haddockfig{fig/moc-sdf-pattern-stated.pdf}\par
           

\item[\begin{tabular}{@{}l}
state22
\end{tabular}]\haddockbegindoc
\haddockbeginargs
\haddockdecltt{::} & \haddockdecltt{((Cons, Cons, Cons, Cons), (Prod, Prod), [b1]
                                                                             -> [b2]
                                                                                -> [a1]
                                                                                   -> [a2]
                                                                                      -> ([b1], [b2]))} & next state function, tupled with consumption /
 production rates \\
                                                                                                          \haddockdecltt{->} & \haddockdecltt{([b1], [b2])} & initial partitions of values \\
                                                                                                                                                              \haddockdecltt{->} & \haddockdecltt{Signal a1} & first input signal \\
                                                                                                                                                                                                               \haddockdecltt{->} & \haddockdecltt{Signal a2} & second input signal \\
                                                                                                                                                                                                                                                                \haddockdecltt{->} & \haddockdecltt{(Signal b1, Signal b2)} & output signals \\
\end{tabulary}\par
\haddocktt{state} is a state machine without an output decoder. It is an
 instantiation of the \haddocktt{stated} MoC constructor (see
 \haddockid{state22}).\par
Constructors: \haddocktt{state{\char 91}1-4{\char 93}{\char 91}1-4{\char 93}}.\par
\begin{interactive}
λ> let f [a] [b,c] = [a+b+c]
λ> let s = signal [1,2,3,4,5,6,7]
λ> state11 ((1,2),1,f) [1] s
{4,11,22}

\end{interactive}\haddockfig{fig/moc-sdf-pattern-state.pdf}\par
           

\item[\begin{tabular}{@{}l}
moore22
\end{tabular}]\haddockbegindoc
\haddockbeginargs
\haddockdecltt{::} & \haddockdecltt{((Cons, Cons, Cons), Prod, [st]
                                                               -> [a1]
                                                                  -> [a2]
                                                                     -> [st])} & next state function, tupled with consumption / production
 rates \\
                                                                                 \haddockdecltt{->} & \haddockdecltt{(Cons, (Prod, Prod), [st]
                                                                                                                                          -> ([b1], [b2]))} & output decoder, tupled with consumption / production
 rates \\
                                                                                                                                                              \haddockdecltt{->} & \haddockdecltt{[st]} & initial state values \\
                                                                                                                                                                                                          \haddockdecltt{->} & \haddockdecltt{Signal a1} & \\
                                                                                                                                                                                                                                                           \haddockdecltt{->} & \haddockdecltt{Signal a2} & \\
                                                                                                                                                                                                                                                                                                            \haddockdecltt{->} & \haddockdecltt{(Signal b1, Signal b2)} & \\
\end{tabulary}\par
\haddocktt{moore} processes model Moore state machines. It is an
 instantiation of the \haddocktt{moore} MoC constructor (see
 \haddockid{moore22}).\par
Constructors: \haddocktt{moore{\char 91}1-4{\char 93}{\char 91}1-4{\char 93}}.\par
\begin{interactive}
λ> let ns [a] [b,c] = [a+b+c]
λ> let od [a]       = [a+1,a*2]
λ> let s = signal [1,2,3,4,5,6,7]
λ> moore11 ((1,2),1,ns) (1,2,od) [1] s
{2,2,5,8,12,22,23,44}

\end{interactive}\haddockfig{fig/moc-sdf-pattern-moore.pdf}\par
           

\item[\begin{tabular}{@{}l}
mealy22
\end{tabular}]\haddockbegindoc
\haddockbeginargs
\haddockdecltt{::} & \haddockdecltt{((Cons, Cons, Cons), Prod, [st]
                                                               -> [a1]
                                                                  -> [a2]
                                                                     -> [st])} & next state function, tupled with consumption / production
 rates \\
                                                                                 \haddockdecltt{->} & \haddockdecltt{((Cons, Cons, Cons), (Prod, Prod), [st]
                                                                                                                                                        -> [a1]
                                                                                                                                                           -> [a2]
                                                                                                                                                              -> ([b1], [b2]))} & outpt decoder, tupled with consumption / production rates \\
                                                                                                                                                                                  \haddockdecltt{->} & \haddockdecltt{[st]} & initial state values \\
                                                                                                                                                                                                                              \haddockdecltt{->} & \haddockdecltt{Signal a1} & \\
                                                                                                                                                                                                                                                                               \haddockdecltt{->} & \haddockdecltt{Signal a2} & \\
                                                                                                                                                                                                                                                                                                                                \haddockdecltt{->} & \haddockdecltt{(Signal b1, Signal b2)} & \\
\end{tabulary}\par
\haddocktt{mealy} processes model Mealy state machines. It is an
 instantiation of the \haddocktt{mealy} MoC constructor
 (see \haddockid{mealy22}).\par
Constructors: \haddocktt{mealy{\char 91}1-4{\char 93}{\char 91}1-4{\char 93}}.\par
\begin{interactive}
λ> let ns [a] [b,c] = [a+b+c]
λ> let od [a] [b]   = [a+b,a*b]
λ> let s = signal [1,2,3,4,5,6,7]
λ> mealy11 ((1,2),1,ns) ((1,1),2,od) [1] s
{2,1,6,8,14,33,26,88}

\end{interactive}\haddockfig{fig/moc-sdf-pattern-mealy.pdf}\par
           
\end{haddockdesc}
\subsubsection{Interfaces}
\begin{haddockdesc}
\item[\begin{tabular}{@{}l}
toSY2\ ::\ Signal\ a\ ->\ Signal\ b\ ->\ (Signal\ a,\ Signal\ b)
\end{tabular}]\haddockbegindoc
Transforms a (set of) \haddockid{SDF} signal(s) into
 the equivalent \haddockid{SY} signal(s). The only change
 is the event consructor. The partial order of DE is interpreted as
 SY's total order, based on the positioning of events in the signals
 (e.g. FIFO buffers) at that moment.\par
Constructors: \haddocktt{toSY{\char 91}1-4{\char 93}}.\par
\begin{interactive}
λ> let s = SDF.signal [1,2,3,4,5]
λ> toSY s
{1,2,3,4,5}

\end{interactive}\haddockfig{fig/moc-sdf-tosy.pdf}\par
           

\item[\begin{tabular}{@{}l}
zipx
\end{tabular}]\haddockbegindoc
\haddockbeginargs
\haddockdecltt{::} & \haddockdecltt{Vector Cons} & consumption rates \\
                                                   \haddockdecltt{->} & \haddockdecltt{Vector (Signal a)} & vector of signals \\
                                                                                                            \haddockdecltt{->} & \haddockdecltt{Signal (Vector a)} & signal of vectors \\
\end{tabulary}\par
Consumes tokens from a vector of signals and merges them into a
 signal of vectors, with a production rate of 1. It instantiates the
 \haddockid{zipx} skeleton.\par
\begin{interactive}
λ> let s1 = SDF.signal [1,2,3,4,5]
λ> let s2 = SDF.signal [11,12,13,14,15]
λ> let v1 = V.vector [s1,s1,s2,s2]
λ> let r  = V.vector [2,1,2,1]
λ> v1
<{1,2,3,4,5},{1,2,3,4,5},{11,12,13,14,15},{11,12,13,14,15}>
λ> zipx r v1
{<1,2,1,11,12,11>,<3,4,2,13,14,12>}

\end{interactive}\haddockfig{fig/moc-sdf-zipx.pdf}\par
           

\item[\begin{tabular}{@{}l}
unzipx
\end{tabular}]\haddockbegindoc
\haddockbeginargs
\haddockdecltt{::} & \haddockdecltt{Vector Prod} & production rates (in reverse order) \\
                                                   \haddockdecltt{->} & \haddockdecltt{Signal (Vector a)} & signal of vectors \\
                                                                                                            \haddockdecltt{->} & \haddockdecltt{Vector (Signal a)} & vector of signals \\
\end{tabulary}\par
Consumes the vectors carried by a signal with a rate of 1, and
 unzips them into a vector of signals based on the user provided
 rates. It instantiates the \haddockid{unzipx}
 skeleton.\par
\textbf{OBS:} due to the \haddockid{recur} pattern
 contained by \haddockid{unzipx}, the vector of
 production rates needs to be provided in reverse order (see
 \haddocktt{ForSyDe.Atom.Skeleton.Vector}).\par
\begin{interactive}
λ> let s1 = SDF.signal [1,2,3,4,5]
λ> let s2 = SDF.signal [11,12,13,14,15]
λ> let v1 = V.vector [s1,s1,s2,s2]
λ> let r  = V.vector [2,1,2,1]
λ> let sz = zipx r v1
λ> v1
<{1,2,3,4,5},{1,2,3,4,5},{11,12,13,14,15},{11,12,13,14,15}>
λ> sz
{<1,2,1,11,12,11>,<3,4,2,13,14,12>}
λ> unzipx (V.reverse r) sz
<{1,2,3,4},{1,2},{11,12,13,14},{11,12}>

\end{interactive}\haddockfig{fig/moc-sdf-unzipx.pdf}\par
           
\end{haddockdesc}
  \haddockmoduleheading{ForSyDe.Atom.MoC.SY}
\label{module:ForSyDe.Atom.MoC.SY}
\haddockbeginheader
{\haddockverb\begin{verbatim}
module ForSyDe.Atom.MoC.SY (
    SY(SY, val),  Signal,  unit2,  signal,  readSignal,  delay,  comb22, 
    reconfig22,  constant2,  generate2,  stated22,  state22,  moore22, 
    mealy22,  when,  when',  is,  whenPresent,  filter,  filter',  fill,  hold, 
    reactAbst2,  toDE2,  toSDF2,  zipx,  unzipx,  unzipx'
  ) where\end{verbatim}}
\haddockendheader

The \haddocktt{SY} library implements the atoms holding the sematics for the
 synchronous computation model. It also provides a set of helpers
 for properly instantiating process network patterns as process
 constructors.\par
\begin{mdframed}[style=reminder,frametitle=Reminder]Make sure to consult naming conventions in  \cref{sec:forsyde-atom:naming-convention} in order to interpret the names and type signatures correctly.\end{mdframed}\par

\subsection{Synchronous (\haddocktt{SY}) event}
According to \cite{Lee98}, "two events
 are synchronous if they have the same tag, and two signals are
 synchronous if all events in one signal are synchronous to an
 event in the second signal and vice-versa. A system is
 synchronous if every signals in the system is synchronous to all
 the other signals in the system."\par
The synchronous (\haddocktt{SY}) MoC defines no notion of physical time,
 its tag system suggesting in fact the precedence among events. To
 further demystify its mechanisms, we can formulate the following
 proposition:\par
\begin{description}
\item[The SY MoC] is abstracting the execution semantics of a system
 where computation is assumed to perform instantaneously (with
 zero delay), at certain synchronization points, when data is
 assumed to be available.
\end{description}Below is a \emph{possible} behavior in time of the input and the
 output signals of a SY process, to illustrate these semantics:\par
                 \haddockfig{fig/moc-sy-example.pdf}\par
                 Implementing the SY tag system is straightforward if we consider
 the synchronous \haddockid{Signal} as an infinite list. In this case the
 tags are implicitly defined by the position of events in a
 signal: \emph{t₀} would correspond with the event at the head of
 a signal \emph{t₁} with the following event, etc... The only
 explicit parameter passed to a SY event constructor is the value
 it carries ∈ \emph{Vₑ}. As such, we can state the
 following particularities:\par
                 \begin{enumerate}
                 \item 
                 tags are implicit from the position in the
 \haddockid{Stream}, thus they are completely
 ignored in the type constructor.\par
                 
                 \item 
                 the type constructor wraps only a value\par
                 
                 \item 
                 being a \emph{timed MoC}, the order between events is total
 \cite{Lee98}.\par
                 
                 \item 
                 from the previous statement we can conclude that there is no
 need for an execution context (see \cref{module:ForSyDe.Atom.MoC})
 and we can ignore the formatting of functions in
 \haddocktt{ForSyDe.Atom.MoC}, thus we can safely assume:
 \haddockeq{fig/eqs-moc-timed-context.pdf}\par
                 
                 \end{enumerate}
                 
\begin{haddockdesc}
\item[\begin{tabular}{@{}l}
newtype\ SY\ a
\end{tabular}]\haddockbegindoc
\haddockbeginconstrs
\haddockdecltt{=} & \haddockdecltt{SY} & \\
                    \haddockdecltt{>}&\haddockdecltt{val :: a} &\\
\end{tabulary}\par

The SY event. It identifies a synchronous signal.\par


\item[\begin{tabular}{@{}l}
instance\ Functor\ SY
\end{tabular}]\haddockbegindoc
Allows for mapping of functions on a SY event.\par


\item[\begin{tabular}{@{}l}
instance\ Applicative\ SY
\end{tabular}]\haddockbegindoc
Allows for lifting functions on a pair of SY events.\par


\item[\begin{tabular}{@{}l}
instance\ MoC\ SY
\end{tabular}]\haddockbegindoc
Implenents the execution and synchronization semantics for the SY
 MoC through its atoms.\par


\item[\begin{tabular}{@{}l}
instance\ Read\ a\ =>\ Read\ (SY\ a)
\end{tabular}]\haddockbegindoc
Reads the value wrapped\par


\item[\begin{tabular}{@{}l}
instance\ Show\ a\ =>\ Show\ (SY\ a)
\end{tabular}]\haddockbegindoc
Shows the value wrapped\par


\item[\begin{tabular}{@{}l}
instance\ Plottable\ a\ =>\ Plot\ (Signal\ a)
\end{tabular}]\haddockbegindoc
\haddockid{SY} signals.\par


\item[\begin{tabular}{@{}l}
instance\ type\ Ret\ SY\ b\ =\ b\\instance\ type\ Fun\ SY\ a\ b\ =\ a\ ->\ b
\end{tabular}]
\end{haddockdesc}
\subsection{Aliases {\char '46} utilities}
A set of type synonyms and utilities are provided for
 convenience. The API type signatures will feature these aliases
 to hide the cumbersome construction of atoms and patters as seen
 in \haddocktt{ForSyDe.Atom.MoC}.\par

\begin{haddockdesc}
\item[\begin{tabular}{@{}l}
type\ Signal\ a\ =\ Stream\ (SY\ a)
\end{tabular}]\haddockbegindoc
Type synonym for a SY signal, i.e. "an ordered stream of SY
 events"\par


\item[\begin{tabular}{@{}l}
unit2\ ::\ (a1,\ a2)\ ->\ (Signal\ a1,\ Signal\ a2)
\end{tabular}]\haddockbegindoc
Wraps a (tuple of) value(s) into the equivalent unit signal(s).\par
Helpers: \haddocktt{unit}, \haddocktt{unit2}, \haddocktt{unit3}, \haddocktt{unit4}.\par


\item[\begin{tabular}{@{}l}
signal\ ::\ {\char 91}a{\char 93}\ ->\ Signal\ a
\end{tabular}]\haddockbegindoc
Transforms a list of values into a SY signal.\par


\item[\begin{tabular}{@{}l}
readSignal\ ::\ Read\ a\ =>\ String\ ->\ Signal\ a
\end{tabular}]\haddockbegindoc
Reads a signal from a string. Like with the \haddocktt{read} function from
 \haddocktt{Prelude}, you must specify the tipe of the signal.\par
\begin{interactive}
λ> readSignal "{1,2,3,4,5}" :: Signal Int
{1,2,3,4,5}

\end{interactive}
\end{haddockdesc}
\subsection{\haddocktt{SY} process constuctors}
These SY process constructors are basically specific
 instantiations of the patterns of atoms defined in
 \haddocktt{ForSyDe.Atom.MoC}. Some are also wrapping functions in an
 extended behavioural model.\par

\subsubsection{Simple}
\begin{haddockdesc}
\item[\begin{tabular}{@{}l}
delay
\end{tabular}]\haddockbegindoc
\haddockbeginargs
\haddockdecltt{::} & \haddockdecltt{a} & initial value \\
                                         \haddockdecltt{->} & \haddockdecltt{Signal a} & input signal \\
                                                                                         \haddockdecltt{->} & \haddockdecltt{Signal a} & output signal \\
\end{tabulary}\par
The \haddocktt{delay} process "delays" a signal with one
 event. Instantiates the \haddockid{delay} pattern.\par
\begin{interactive}
λ> let s = signal [1,2,3,4,5]
λ> delay 0 s
{0,1,2,3,4,5}

\end{interactive}\haddockfig{fig/moc-sy-pattern-delay.pdf}\par
           

\item[\begin{tabular}{@{}l}
comb22
\end{tabular}]\haddockbegindoc
\haddockbeginargs
\haddockdecltt{::} & \haddockdecltt{(a1
                                     -> a2 -> (b1, b2))} & function on values \\
                                                           \haddockdecltt{->} & \haddockdecltt{Signal a1} & first input signal \\
                                                                                                            \haddockdecltt{->} & \haddockdecltt{Signal a2} & second input signal \\
                                                                                                                                                             \haddockdecltt{->} & \haddockdecltt{(Signal b1, Signal b2)} & two output signals \\
\end{tabulary}\par
\haddocktt{comb} processes map combinatorial functions on signals and take
 care of synchronization between input signals. It instantiates the
 \haddocktt{comb} pattern (see \haddockid{comb22}).\par
Constructors: \haddocktt{comb{\char 91}1-4{\char 93}{\char 91}1-4{\char 93}}.\par
\begin{interactive}
λ> let s1 = signal [1..]
λ> let s2 = signal [1,1,1,1,1]
λ> comb11 (+1) s2
{2,2,2,2,2}
λ> comb22 (\a b-> (a+b,a-b)) s1 s2
({2,3,4,5,6},{0,1,2,3,4})

\end{interactive}\haddockfig{fig/moc-sy-pattern-comb.pdf}\par
           

\item[\begin{tabular}{@{}l}
reconfig22
\end{tabular}]\haddockbegindoc
\haddockbeginargs
\haddockdecltt{::} & \haddockdecltt{Signal (a1
                                            -> a2
                                               -> (b1, b2))} & signal carrying functions \\
                                                               \haddockdecltt{->} & \haddockdecltt{Signal a1} & first input signal carrying arguments \\
                                                                                                                \haddockdecltt{->} & \haddockdecltt{Signal a2} & second input signal carrying arguments \\
                                                                                                                                                                 \haddockdecltt{->} & \haddockdecltt{(Signal b1, Signal b2)} & two output signals \\
\end{tabulary}\par
\haddocktt{reconfig} creates an synchronous adaptive process where the
 first signal carries functions and the other carry the
 arguments. It instantiates the \haddocktt{reconfig} atom pattern (see
 \haddockid{reconfig22}).\par
Constructors: \haddocktt{reconfig{\char 91}1-4{\char 93}{\char 91}1-4{\char 93}}.\par
\begin{interactive}
λ> let sf = signal [(+1),(*2),(+1),(*2),(+1),(*2),(+1)]
λ> let s1 = signal [1..]
λ> reconfig11 sf s1
{2,4,4,8,6,12,8}

\end{interactive}\haddockfig{fig/moc-sy-pattern-reconfig.pdf}\par
           

\item[\begin{tabular}{@{}l}
constant2
\end{tabular}]\haddockbegindoc
\haddockbeginargs
\haddockdecltt{::} & \haddockdecltt{(b1, b2)} & values to be repeated \\
                                                \haddockdecltt{->} & \haddockdecltt{(Signal b1, Signal b2)} & generated signals \\
\end{tabulary}\par
A signal generator which keeps a value constant. It
 is actually an instantiation of the \haddocktt{stated0X} constructor
 (check \haddockid{stated22}).\par
Constructors: \haddocktt{constant{\char 91}1-4{\char 93}}.\par
\begin{interactive}
λ> let (s1, s2) = constant2 (1,2)
λ> takeS 3 s1
{1,1,1}
λ> takeS 5 s2
{2,2,2,2,2}

\end{interactive}\haddockfig{fig/moc-sy-pattern-constant.pdf}\par
           

\item[\begin{tabular}{@{}l}
generate2
\end{tabular}]\haddockbegindoc
\haddockbeginargs
\haddockdecltt{::} & \haddockdecltt{(b1
                                     -> b2
                                        -> (b1, b2))} & function to generate next value \\
                                                        \haddockdecltt{->} & \haddockdecltt{(b1, b2)} & kernel values \\
                                                                                                        \haddockdecltt{->} & \haddockdecltt{(Signal b1, Signal b2)} & generated signals \\
\end{tabulary}\par
A signal generator based on a function and a kernel value. It
 is actually an instantiation of the \haddocktt{stated0X} constructor
 (check \haddockid{stated22}).\par
Constructors: \haddocktt{generate{\char 91}1-4{\char 93}}.\par
\begin{interactive}
λ> let (s1,s2) = generate2 (\a b -> (a+1,b+2)) (1,2)
λ> takeS 5 s1
{1,2,3,4,5}
λ> takeS 7 s2
{2,4,6,8,10,12,14}

\end{interactive}\haddockfig{fig/moc-sy-pattern-generate.pdf}\par
           

\item[\begin{tabular}{@{}l}
stated22
\end{tabular}]\haddockbegindoc
\haddockbeginargs
\haddockdecltt{::} & \haddockdecltt{(b1
                                     -> b2
                                        -> a1
                                           -> a2
                                              -> (b1, b2))} & next state function \\
                                                              \haddockdecltt{->} & \haddockdecltt{(b1, b2)} & initial state values \\
                                                                                                              \haddockdecltt{->} & \haddockdecltt{Signal a1} & first input signal \\
                                                                                                                                                               \haddockdecltt{->} & \haddockdecltt{Signal a2} & second input signal \\
                                                                                                                                                                                                                \haddockdecltt{->} & \haddockdecltt{(Signal b1, Signal b2)} & output signals \\
\end{tabulary}\par
\haddocktt{stated} is a state machine without an output decoder. It is an
 instantiation of the \haddocktt{state} MoC constructor
 (see \haddockid{stated22}).\par
Constructors: \haddocktt{stated{\char 91}1-4{\char 93}{\char 91}1-4{\char 93}}.\par
\begin{interactive}
λ> let s1 = signal [1,2,3,4,5]
λ> stated11 (+) 1 s1
{1,2,4,7,11,16}

\end{interactive}\haddockfig{fig/moc-sy-pattern-stated.pdf}\par
           

\item[\begin{tabular}{@{}l}
state22
\end{tabular}]\haddockbegindoc
\haddockbeginargs
\haddockdecltt{::} & \haddockdecltt{(b1
                                     -> b2
                                        -> a1
                                           -> a2
                                              -> (b1, b2))} & next state function \\
                                                              \haddockdecltt{->} & \haddockdecltt{(b1, b2)} & initial state values \\
                                                                                                              \haddockdecltt{->} & \haddockdecltt{Signal a1} & first input signal \\
                                                                                                                                                               \haddockdecltt{->} & \haddockdecltt{Signal a2} & second input signal \\
                                                                                                                                                                                                                \haddockdecltt{->} & \haddockdecltt{(Signal b1, Signal b2)} & output signals \\
\end{tabulary}\par
\haddocktt{state} is a state machine without an output decoder. It is an
 instantiation of the \haddocktt{stated} MoC constructor
 (see \haddockid{state22}).\par
Constructors: \haddocktt{state{\char 91}1-4{\char 93}{\char 91}1-4{\char 93}}.\par
\begin{interactive}
λ> let s1 = signal [1,2,3,4,5]
λ> state11 (+) 1 s1
{2,4,7,11,16}

\end{interactive}\haddockfig{fig/moc-sy-pattern-state.pdf}\par
           

\item[\begin{tabular}{@{}l}
moore22
\end{tabular}]\haddockbegindoc
\haddockbeginargs
\haddockdecltt{::} & \haddockdecltt{(st
                                     -> a1
                                        -> a2 -> st)} & next state function \\
                                                        \haddockdecltt{->} & \haddockdecltt{(st
                                                                                             -> (b1, b2))} & output decoder \\
                                                                                                             \haddockdecltt{->} & \haddockdecltt{st} & initial state \\
                                                                                                                                                       \haddockdecltt{->} & \haddockdecltt{Signal a1} & \\
                                                                                                                                                                                                        \haddockdecltt{->} & \haddockdecltt{Signal a2} & \\
                                                                                                                                                                                                                                                         \haddockdecltt{->} & \haddockdecltt{(Signal b1, Signal b2)} & \\
\end{tabulary}\par
\haddocktt{moore} processes model Moore state machines. It is an
 instantiation of the \haddocktt{moore} MoC constructor
 (see \haddockid{moore22}).\par
Constructors: \haddocktt{moore{\char 91}1-4{\char 93}{\char 91}1-4{\char 93}}.\par
\begin{interactive}
λ> let s1 = signal [1,2,3,4,5]
λ> moore11 (+) (+1) 1 s1
{2,3,5,8,12,17}

\end{interactive}\haddockfig{fig/moc-sy-pattern-moore.pdf}\par
           

\item[\begin{tabular}{@{}l}
mealy22
\end{tabular}]\haddockbegindoc
\haddockbeginargs
\haddockdecltt{::} & \haddockdecltt{(st
                                     -> a1
                                        -> a2 -> st)} & next state function \\
                                                        \haddockdecltt{->} & \haddockdecltt{(st
                                                                                             -> a1
                                                                                                -> a2
                                                                                                   -> (b1, b2))} & outpt decoder \\
                                                                                                                   \haddockdecltt{->} & \haddockdecltt{st} & initial state \\
                                                                                                                                                             \haddockdecltt{->} & \haddockdecltt{Signal a1} & \\
                                                                                                                                                                                                              \haddockdecltt{->} & \haddockdecltt{Signal a2} & \\
                                                                                                                                                                                                                                                               \haddockdecltt{->} & \haddockdecltt{(Signal b1, Signal b2)} & \\
\end{tabulary}\par
\haddocktt{mealy} processes model Mealy state machines. It is an
 instantiation of the \haddocktt{mealy} MoC constructor
 (see \haddockid{mealy22}).\par
Constructors: \haddocktt{mealy{\char 91}1-4{\char 93}{\char 91}1-4{\char 93}}.\par
\begin{interactive}
λ> let s1 = signal [1,2,3,4,5]
λ> mealy11 (+) (-) 1 s1
{0,0,1,3,6}

\end{interactive}\haddockfig{fig/moc-sy-pattern-mealy.pdf}\par
           
\end{haddockdesc}
\subsubsection{Predicate behavior}
These processes manipulate the behavior of a signal based on
 predicates on their status.\par

\begin{haddockdesc}
\item[\begin{tabular}{@{}l}
when
\end{tabular}]\haddockbegindoc
\haddockbeginargs
\haddockdecltt{::} & \haddockdecltt{Signal (AbstExt Bool)} & Signal of predicates \\
                                                             \haddockdecltt{->} & \haddockdecltt{Signal (AbstExt a)} & Input signal \\
                                                                                                                       \haddockdecltt{->} & \haddockdecltt{Signal (AbstExt a)} & Output signal \\
\end{tabulary}\par
This process predicates the existence of values in a signal based
 on a signal of boolean values (conditions). It is similar to the
 \haddocktt{when} construct in the synchronous language Lustre
 \cite{Halbwachs91}, based on which clock
 calculus can be performed.\par
\textbf{OBS:} this process assumes that all signals carry
 absent-extended values, which is appropriate in describing
 multi-clock systems. For a version which inputs signals of
 non-extended values, check \haddockid{when'}.\par
\begin{interactive}
λ> let s1   = (signal . map Prst) [1,2,3,4,5]
λ> let pred = (signal . map Prst) [False,False,False,True,True]
λ> when pred s1
{⟂,⟂,⟂,4,5}

\end{interactive}\haddockfig{fig/moc-sy-pattern-when.pdf}\par
           

\item[\begin{tabular}{@{}l}
when'
\end{tabular}]\haddockbegindoc
\haddockbeginargs
\haddockdecltt{::} & \haddockdecltt{Signal Bool} & Signal of predicates \\
                                                   \haddockdecltt{->} & \haddockdecltt{Signal a} & Input signal \\
                                                                                                   \haddockdecltt{->} & \haddockdecltt{Signal (AbstExt a)} & Output signal \\
\end{tabulary}\par
Same as \haddockid{when} but inputs signals of non-extended values.\par
\begin{interactive}
λ> let s1   = signal [1,2,3,4,5]
λ> let pred = signal [False,False,False,True,True]
λ> when' pred s1
{⟂,⟂,⟂,4,5}

\end{interactive}

\item[\begin{tabular}{@{}l}
is\ ::\ Signal\ (AbstExt\ a)\ ->\ (a\ ->\ Bool)\ ->\ Signal\ (AbstExt\ Bool)
\end{tabular}]\haddockbegindoc
Simple wrapper for applying a predicate function on a signal of
 absent-extended events.\par
\begin{interactive}
λ> let s1   = signal $ map Prst [1,2,3,4,5]
λ> s1 `is` (>3)
{False,False,False,True,True}

\end{interactive}

\item[\begin{tabular}{@{}l}
whenPresent\ ::\ Signal\ (AbstExt\ a1)\\\ \ \ \ \ \ \ \ \ \ \ \ \ \ \ ->\ Signal\ (AbstExt\ a2)\ ->\ Signal\ (AbstExt\ a2)
\end{tabular}]\haddockbegindoc
Same as \haddockid{when} but triggering the output events merely based on
 the presence of the first input rather than a predicate function.\par
\begin{interactive}
λ> let s1   = signal $ map Prst [1,2,3,4,5]
λ> let sp   = signal [Prst 1, Prst 1, Abst, Prst 1, Abst]
λ> whenPresent sp s1
{1,2,⟂,4,⟂}

\end{interactive}

\item[\begin{tabular}{@{}l}
filter
\end{tabular}]\haddockbegindoc
\haddockbeginargs
\haddockdecltt{::} & \haddockdecltt{(a -> Bool)} & Predicate function \\
                                                   \haddockdecltt{->} & \haddockdecltt{Signal (AbstExt a)} & Input signal \\
                                                                                                             \haddockdecltt{->} & \haddockdecltt{Signal (AbstExt a)} & Output signal \\
\end{tabulary}\par
Filters out values to \haddockid{Abst} if they do not fulfill a predicate
 function.\par
\textbf{OBS:} this process assumes that all signals carry
 absent-extended values, which is appropriate in describing
 multi-clock systems. For a version which inputs signals of
 non-extended values, check \haddockid{filter'}.\par
\begin{interactive}
λ> let s1   = (signal . map Prst) [1,2,3,4,5]
λ> filter (>3) s1
{⟂,⟂,⟂,4,5}

\end{interactive}\haddockfig{fig/moc-sy-pattern-filter.pdf}\par
           

\item[\begin{tabular}{@{}l}
filter'
\end{tabular}]\haddockbegindoc
\haddockbeginargs
\haddockdecltt{::} & \haddockdecltt{(a -> Bool)} & Predicate function \\
                                                   \haddockdecltt{->} & \haddockdecltt{Signal a} & Input signal \\
                                                                                                   \haddockdecltt{->} & \haddockdecltt{Signal (AbstExt a)} & Output signal \\
\end{tabulary}\par
Same as \haddockid{filter} but inputs signals of non-extended values.\par
\begin{interactive}
λ> let s1   = signal [1,2,3,4,5]
λ> filter' (>3) s1
{⟂,⟂,⟂,4,5}

\end{interactive}

\item[\begin{tabular}{@{}l}
fill
\end{tabular}]\haddockbegindoc
\haddockbeginargs
\haddockdecltt{::} & \haddockdecltt{a} & Value to fill with \\
                                         \haddockdecltt{->} & \haddockdecltt{Signal (AbstExt a)} & Input \\
                                                                                                   \haddockdecltt{->} & \haddockdecltt{Signal a} & Output \\
\end{tabulary}\par
Fills absent events with a pre-defined value.\par
\begin{interactive}
λ> let s1   = signal [Abst, Abst, Prst 1, Prst 2, Abst, Prst 3]
λ> fill 0 s1
{0,0,1,2,0,3}

\end{interactive}\haddockfig{fig/moc-sy-pattern-fill.pdf}\par
           

\item[\begin{tabular}{@{}l}
hold
\end{tabular}]\haddockbegindoc
\haddockbeginargs
\haddockdecltt{::} & \haddockdecltt{a} & Value to fill with in case there was no previous value \\
                                         \haddockdecltt{->} & \haddockdecltt{Signal (AbstExt a)} & Input \\
                                                                                                   \haddockdecltt{->} & \haddockdecltt{Signal a} & Output \\
\end{tabulary}\par
Similar to \haddockid{fill}, but holds the last non-absent value if there
 was one. It implements a \haddocktt{state} pattern (see \haddockid{state22}).\par
\begin{interactive}
λ> let s1   = signal [Abst, Abst, Prst 1, Prst 2, Abst, Prst 3]
λ> hold 0 s1
{0,0,1,2,2,3}

\end{interactive}\haddockfig{fig/moc-sy-pattern-hold.pdf}\par
           

\item[\begin{tabular}{@{}l}
reactAbst2
\end{tabular}]\haddockbegindoc
\haddockbeginargs
\haddockdecltt{::} & \haddockdecltt{(Signal (AbstExt a1)
                                     -> Signal (AbstExt a2)
                                        -> Signal b)} & process which degrades the absent extension,
 e.g. holds present values \\
                                                        \haddockdecltt{->} & \haddockdecltt{Signal (AbstExt a1)} & \\
                                                                                                                   \haddockdecltt{->} & \haddockdecltt{Signal (AbstExt a2)} & \\
                                                                                                                                                                              \haddockdecltt{->} & \haddockdecltt{Signal (AbstExt b)} & absent-extended signal, properly reacting to the inputs \\
\end{tabulary}\par
Creates a wrapper enabling a reactive behvior to absent-extended
 signals for processes which would otherwise degrade the
 absent-extension (e.g. state machines with \haddocktt{ignore22} behavior).\par
Constructors: \haddocktt{reactAbst{\char 91}1-4{\char 93}}.\par
\begin{interactive}
λ> let s1 = readSignal "{1,1,1,_,1,_,1}" :: Signal (AbstExt Int)
λ> let proc = stated11 (B.ignore11 (+)) 0
λ> proc s1
{0,1,2,3,3,4,4,5}
λ> reactAbst1 proc s1
{0,1,2,⟂,3,⟂,4} 

\end{interactive}\haddockfig{fig/moc-sy-pattern-reactAbst.pdf}\par
           
\end{haddockdesc}
\subsection{Interfaces}
\begin{haddockdesc}
\item[\begin{tabular}{@{}l}
toDE2
\end{tabular}]\haddockbegindoc
\haddockbeginargs
\haddockdecltt{::} & \haddockdecltt{Signal TimeStamp} & SY signal carrying \haddockid{DE} timestamps \\
                                                        \haddockdecltt{->} & \haddockdecltt{Signal a} & first input SY signal \\
                                                                                                        \haddockdecltt{->} & \haddockdecltt{Signal b} & second input SY signal \\
                                                                                                                                                        \haddockdecltt{->} & \haddockdecltt{(Signal a, Signal b)} & two output \haddockid{DE} signals \\
\end{tabulary}\par
Wraps explicit timestamps to a (set of) \haddockid{SY}
 signal(s), rendering the equivalent synchronized
 \haddockid{DE} signal(s).\par
Constructors: \haddocktt{toDE}, \haddocktt{toDE2}, \haddocktt{toDE3}, \haddocktt{toDE4}.\par
\begin{interactive}
λ> let s1 = SY.signal [0,3,4,6,9]
λ> let s2 = SY.signal [1,2,3,4,5]
λ> toDE s1 s2
{ 1 @0s, 2 @3s, 3 @4s, 4 @6s, 5 @9s}

\end{interactive}\haddockfig{fig/moc-sy-tode.pdf}\par
           

\item[\begin{tabular}{@{}l}
toSDF2\ ::\ Signal\ a\ ->\ Signal\ b\ ->\ (Signal\ a,\ Signal\ b)
\end{tabular}]\haddockbegindoc
Transforms a (set of) \haddockid{SY} signal(s) into the
 equivalent \haddockid{SDF} signal(s). The only change is
 the event consructor. The total order of SY is interpreted as
 partial order by the next SDF process downstream.\par
Constructors: \haddocktt{toSDF}, \haddocktt{toSDF2}, \haddocktt{toSDF3}, \haddocktt{toSDF4}.\par
\begin{interactive}
λ> let s = SY.signal [1,2,3,4,5]
λ> toSDF s
{1,2,3,4,5}

\end{interactive}\haddockfig{fig/moc-sy-tosdf.pdf}\par
           

\item[\begin{tabular}{@{}l}
zipx\ ::\ Vector\ (Signal\ a)\ ->\ Signal\ (Vector\ a)
\end{tabular}]\haddockbegindoc
Synchronizes all the signals contained by a vector and zips them
 into one signal of vectors. It instantiates the
 \haddockid{zipx} skeleton.\par
\begin{interactive}
λ> let s1 = SY.signal [1,2,3,4,5]
λ> let s2 = SY.signal [11,12,13,14,15]
λ> let v1 = V.vector [s1,s1,s2,s2]
λ> v1
<{1,2,3,4,5},{1,2,3,4,5},{11,12,13,14,15},{11,12,13,14,15}>
λ> zipx v1
{<1,1,11,11>,<2,2,12,12>,<3,3,13,13>,<4,4,14,14>,<5,5,15,15>}

\end{interactive}\haddockfig{fig/moc-sy-zipx.pdf}\par
           

\item[\begin{tabular}{@{}l}
unzipx\ ::\ Integer\ ->\ Signal\ (Vector\ a)\ ->\ Vector\ (Signal\ a)
\end{tabular}]\haddockbegindoc
Unzips the vectors carried by a signal into a vector of
 signals. It instantiates the \haddockid{unzipx}
 skeleton. To avoid infinite recurrence, the user needs to provide
 the length of the output vector.\par
\begin{interactive}
λ> let v1 = V.vector [1,2,3,4]
λ> let s1 = SY.signal [v1,v1,v1,v1,v1]
λ> s1
{<1,2,3,4>,<1,2,3,4>,<1,2,3,4>,<1,2,3,4>,<1,2,3,4>}
λ> unzipx 4 s1
<{1,1,1,1,1},{2,2,2,2,2},{3,3,3,3,3},{4,4,4,4,4}>

\end{interactive}\haddockfig{fig/moc-sy-zipx.pdf}\par
           

\item[\begin{tabular}{@{}l}
unzipx'\ ::\ Signal\ (Vector\ a)\ ->\ Vector\ (Signal\ a)
\end{tabular}]\haddockbegindoc
Same as \haddockid{unzipx}, but "sniffs" the first event to determine the length of the output vector. Might have unsafe behavior!\par
\begin{interactive}
λ> let v1 = V.vector [1,2,3,4]
λ> let s1 = SY.signal [v1,v1,v1,v1,v1]
λ> s1
{<1,2,3,4>,<1,2,3,4>,<1,2,3,4>,<1,2,3,4>,<1,2,3,4>}
λ> unzipx' s1
<{1,1,1,1,1},{2,2,2,2,2},{3,3,3,3,3},{4,4,4,4,4}>

\end{interactive}
\end{haddockdesc}
  \haddockmoduleheading{ForSyDe.Atom.Skeleton}
\label{module:ForSyDe.Atom.Skeleton}
\haddockbeginheader
{\haddockverb\begin{verbatim}
module ForSyDe.Atom.Skeleton (
    Skeleton((=.=), (=*=), (=\=), (=<<=), first, last),  farm22,  reduce, 
    reducei,  pipe,  pipe2
  ) where\end{verbatim}}
\haddockendheader

This module exports a type class with the interfaces for the
 Skeleton layer atoms. It does \emph{NOT} export any implementation of
 atoms not any constructor as composition of atoms.\par
\begin{mdframed}[style=reminder,frametitle=Reminder]Make sure to consult naming conventions in  \cref{sec:forsyde-atom:naming-convention} in order to interpret the names and type signatures correctly.\end{mdframed}\par

\subsection{Atoms}
\begin{haddockdesc}
\item[\begin{tabular}{@{}l}
class\ Functor\ c\ =>\ Skeleton\ c\ where
\end{tabular}]\haddockbegindoc
Class containing all the Skeleton layer atoms.\par
This class is instantiated by a set of categorical types,
 i.e. types which describe an inherent potential for being evaluated
 in parallel. Skeletons are patterns from this layer. When skeletons
 take as arguments entities from the MoC layer (i.e. processes), the
 results themselves are parallel process networks which describe
 systems with an inherent potential to be implemented on parallel
 platforms. All skeletons can be described as composition of the
 three atoms below (\haddockid{=<<=} being just a specific instantiation of
 \haddockid{={\char '134}=}). This possible due to an existing theorem in the categorical
 type theory, also called the Bird-Merteens formalism
 \cite{Bird97}:\par
\par
\begin{description}
\item[factorization] A function on a categorical type is an algorithmic
 skeleton (i.e. catamorphism) \emph{iff} it can be represented in a
 factorized form, i.e. as a \emph{map} composed with a \emph{reduce}.
\end{description}Consequently, most of the skeletons for the implemented categorical
 types are described in their factorized form, taking as arguments
 either:\par
                 \begin{itemize}
                 \item
                 type constructors or functions derived from type constructors\par
                 
                 \item
                 processes, i.e. MoC layer entities\par
                 
                 \end{itemize}
                 Most of the ground-work on algorithmic skeletons on which this
 module is founded has been laid in \cite{Bird97},
 \cite{Skillicorn05} and it founds many
 of the frameworks collected in \cite{Gorlatch03}.\par
                 
\haddockpremethods{}\textbf{Methods}

\item[\begin{tabular}{@{}l}
instance\ Skeleton\ Vector
\end{tabular}]\haddockbegindoc
Ensures that \haddockid{Vector} is a structure associated with the Skeleton Layer.\par

\end{haddockdesc}
\subsection{Skeleton constructors}
Patterns of in the skeleton layer are provided, like all other
 patterns in ForSyDe-Atom, as constructors. If the layer below
 this one is the \haddockid{MoC} layer, i.e. the functions
 taken as arguments are processes, then these skeletons can be
 regarded as process network constructors, as the structures
 created are process networks with inherent potential for parallel
 implementation.\par

\begin{haddockdesc}
\item[\begin{tabular}{@{}l}
farm22\ ::\ Skeleton\ c\ =>\\\ \ \ \ \ \ \ \ \ \ (a1\ ->\ a2\ ->\ (b1,\ b2))\ ->\ c\ a1\ ->\ c\ a2\ ->\ (c\ b1,\ c\ b2)
\end{tabular}]\haddockbegindoc
\haddocktt{farm} maps a function on a vector. It is the embodiment of the
 \haddocktt{map} homomorphism, and its naming is inspired from the pattern
 predominant in HPC. Indeed, if we consider the layer below as being
 the \haddockid{MoC} layer (i.e. the passed functions are
 processes), the resulting structure could be regarded as a "farm of
 data-parallel processes".\par
Constructors: \haddocktt{farm{\char 91}1-8{\char 93}{\char 91}1-4{\char 93}}.\par
\haddockeq{fig/eqs-skel-pattern-farm.pdf}
 \haddockfig{fig/skel-pattern-farm.pdf}\par


\item[\begin{tabular}{@{}l}
reduce
\end{tabular}]\haddockbegindoc
\haddockbeginargs
\haddockdecltt{::} & Skeleton c \\
                     \haddockdecltt{=>} & \haddockdecltt{(a
                                                          -> a
                                                             -> a)} & associative function (*) \\
                                                                      \haddockdecltt{->} & \haddockdecltt{c a} & structure \\
                                                                                                                 \haddockdecltt{->} & \haddockdecltt{a} & reduced element \\
\end{tabulary}\par
Infix name for the \haddockid{={\char '134}=} atom operator.\par
(*) if the operation is not associative then the network can be
 treated like a pipeline.\par


\item[\begin{tabular}{@{}l}
reducei
\end{tabular}]\haddockbegindoc
\haddockbeginargs
\haddockdecltt{::} & Skeleton c \\
                     \haddockdecltt{=>} & \haddockdecltt{(a
                                                          -> a
                                                             -> a)} & associative function (*) \\
                                                                      \haddockdecltt{->} & \haddockdecltt{a} & initial element of structure \\
                                                                                                               \haddockdecltt{->} & \haddockdecltt{c a} & structure \\
                                                                                                                                                          \haddockdecltt{->} & \haddockdecltt{a} & reduced element \\
\end{tabulary}\par
\haddockid{reducei} is special case of \haddockid{reduce} where an initial element is
 specified outside the reduced vector. It is implemented as a
 \haddockid{pipe} with switched arguments, and the reduction function is
 constrained to be associative. It is semantically equivalent to the
 pattern depicted below.\par
(*) if the operation is not associative then the network is
 semantically equivalent to \haddocktt{pipe1} (see \haddockid{pipe2}).\par
\haddockeq{fig/eqs-skel-pattern-reducei.pdf}
 \haddockfig{fig/skel-pattern-reducei.pdf}\par


\item[\begin{tabular}{@{}l}
pipe
\end{tabular}]\haddockbegindoc
\haddockbeginargs
\haddockdecltt{::} & Skeleton c \\
                     \haddockdecltt{=>} & \haddockdecltt{c (a
                                                            -> a)} & vector of functions \\
                                                                     \haddockdecltt{->} & \haddockdecltt{a} & kernel element \\
                                                                                                              \haddockdecltt{->} & \haddockdecltt{a} & result  \\
\end{tabulary}\par
Infix name for the \haddockid{=<<=} skeleton operator.\par


\item[\begin{tabular}{@{}l}
pipe2\ ::\ Skeleton\ c\ =>\\\ \ \ \ \ \ \ \ \ (a1\ ->\ a2\ ->\ a\ ->\ a)\ ->\ c\ a1\ ->\ c\ a2\ ->\ a\ ->\ a
\end{tabular}]\haddockbegindoc
The \haddocktt{pipe} constructors are a more generic form of the \haddockid{=<<=}
 (\haddockid{pipe}) skeleton apt for successive partial application and create
 more robust parameterizable pipeline networks.\par
Constructors: \haddocktt{comb{\char 91}1-8{\char 93}}.\par
\haddockeq{fig/eqs-skel-pattern-pipe1.pdf}
 \haddockfig{fig/skel-pattern-pipe1.pdf}\par

\end{haddockdesc}
  \haddockmoduleheading{ForSyDe.Atom.Skeleton.Vector}
\label{module:ForSyDe.Atom.Skeleton.Vector}
\haddockbeginheader
{\haddockverb\begin{verbatim}
module ForSyDe.Atom.Skeleton.Vector (
    Vector(Null, (:>)),  null,  unit,  (<++>),  vector,  fromVector,  indexes, 
    isNull,  (<:),  farm22,  reduce,  prefix,  suffix,  pipe,  (=/=),  recur, 
    cascade2,  mesh2,  length,  index,  fanout,  fanoutn,  generate,  iterate, 
    first,  last,  inits,  tails,  init,  tail,  concat,  reverse,  group, 
    shiftr,  shiftl,  rotr,  rotl,  take,  drop,  takeWhile,  filterIdx,  odds, 
    evens,  stride,  get,  (<@),  (<@!),  gather1,  (<@>),  replace,  scatter, 
    bitrev,  duals,  unduals,  zipx,  unzipx
  ) where\end{verbatim}}
\haddockendheader

This module defines the data type \haddockid{Vector} as a categorical type,
 and implements the atoms for the \haddockid{Skeleton}
 class. Algorithmic skeletons for \haddockid{Vector} are mostly described in
 their factorized form, which ensures that they are catamorphisms
 (see the \href{ForSyDe-Atom-Skeleton.html#factorization}{factorization}
 theorem). Where efficiency or practicality is a concern, some
 skeletons are implemented as recurrences. One can still prove that
 they are catamorphisms through alternative theorems (see
 \cite{Skillicorn05}).\par
\begin{mdframed}[style=reminder,frametitle=Reminder]Make sure to consult naming conventions in  \cref{sec:forsyde-atom:naming-convention} in order to interpret the names and type signatures correctly.\end{mdframed}\par

\subsection{Vector data type}
\begin{haddockdesc}
\item[\begin{tabular}{@{}l}
data\ Vector\ a
\end{tabular}]\haddockbegindoc
\haddockbeginconstrs
\haddockdecltt{=} & \haddockdecltt{Null} & Null element. Terminates a vector. \\
\haddockdecltt{|} & \haddockdecltt{a (:>) (Vector a)} & appends an element at the head of a vector. \\
\end{tabulary}\par

The  \haddockid{Vector}, or at least its interpretation, is the
 exact equivalent of an infinite list, as defined in
 \cite{Bird97}. Its name though is borrowed
 from \cite{Reekie95}, since it is more
 suggestive in the context of process networks.\par
According to \cite{Bird97}, \haddockid{Vector}
 should be implemented as following:\par
\begin{code}
data Vector a = Null                   -- null element
              | Unit a                 -- singleton vector
              | Vector a <++> Vector a -- concatenate two vectors\end{code}
This construction suggests the possibility of splitting a \haddockid{Vector}
 into multiple parts and evaluating it in parallel. Due to reasons
 of efficiency, and to ensure that the structure is flat and
 homogeneous, \haddockid{Vector} is implemented using the same constructors as
 an infinite list like in \cite{Bird87} (see
 below). When defining skeletons of vectors we will not use the real
 constructors though, but the theoretical ones defined above and
 provided as \hyperref[g:2]{functions} . This way we align ForSyDe-Atom's
 \haddockid{Vector} type with the categorical type theory and its theorems.\par
Another particularity of \haddockid{Vector} is that it instantiates the
 reduction atom \haddockid{={\char '134}=} as a \emph{right fold}, as it is the most efficient
 implementation in the context of lazy evaluation. As a consequence
 reduction is performed \textbf{\emph{from right to left}}. This is noticeable
 especially in the case of pipeline-based skeletons (where \haddockid{pipe}
 itself is a reduction with the right-associative composition
 operator \haddockid{.}) is performed from right to left, which
 comes in natural when considering the order of function
 composition. Thus for \haddockid{reduce}-based skeletons (e.g. \haddocktt{prefix},
 \haddocktt{suffix}, \haddocktt{recur}, \haddocktt{cascade}, \haddocktt{mesh}) the result vectors shall be
 read from end to beginning.\par


\item[\begin{tabular}{@{}l}
instance\ Functor\ Vector
\end{tabular}]\haddockbegindoc
Provides an implementation for \haddockid{=.=}.\par


\item[\begin{tabular}{@{}l}
instance\ Applicative\ Vector
\end{tabular}]\haddockbegindoc
Provides an implementation for \haddockid{=*=}.\par


\item[\begin{tabular}{@{}l}
instance\ Foldable\ Vector
\end{tabular}]\haddockbegindoc
Provides an implementation for \haddockid{={\char '134}=}.\par


\item[\begin{tabular}{@{}l}
instance\ Skeleton\ Vector
\end{tabular}]\haddockbegindoc
Ensures that \haddockid{Vector} is a structure associated with the Skeleton Layer.\par


\item[\begin{tabular}{@{}l}
instance\ Eq\ a\ =>\ Eq\ (Vector\ a)
\end{tabular}]

\item[\begin{tabular}{@{}l}
instance\ Read\ a\ =>\ Read\ (Vector\ a)
\end{tabular}]\haddockbegindoc
The vector 1 :> 2 :> Null is read using the string "<1,2>".\par


\item[\begin{tabular}{@{}l}
instance\ Show\ a\ =>\ Show\ (Vector\ a)
\end{tabular}]\haddockbegindoc
The vector 1 :> 2 :> Null is represented as <1,2>.\par


\item[\begin{tabular}{@{}l}
instance\ Plottable\ a\ =>\ Plottable\ (Vector\ a)
\end{tabular}]\haddockbegindoc
Vectors of plottable types\par


\item[\begin{tabular}{@{}l}
instance\ Plottable\ a\ =>\ Plot\ (Vector\ a)
\end{tabular}]\haddockbegindoc
vectors of coordinates\par

\end{haddockdesc}
\subsection{"Constructors"}
Theoretical constructors for the \haddockid{Vector} type, used in the
 definition of skeletons as catamorphisms.\par

\begin{haddockdesc}
\item[\begin{tabular}{@{}l}
null\ ::\ Vector\ a
\end{tabular}]\haddockbegindoc
Constructs a null vector.\par
\begin{interactive}
λ> null
<>

\end{interactive}

\item[\begin{tabular}{@{}l}
unit\ ::\ a\ ->\ Vector\ a
\end{tabular}]\haddockbegindoc
Constructs a singleton vector.\par
\begin{interactive}
λ> unit 1
<1>

\end{interactive}

\item[\begin{tabular}{@{}l}
(<++>)\ ::\ Vector\ a\ ->\ Vector\ a\ ->\ Vector\ a
\end{tabular}]\haddockbegindoc
Constructs a vector by appending two existing vectors.\par
\begin{interactive}
λ> unit 1 <++> unit 2
<1,2>

\end{interactive}
\end{haddockdesc}
\subsection{Utilities}
\begin{haddockdesc}
\item[\begin{tabular}{@{}l}
vector\ ::\ {\char 91}a{\char 93}\ ->\ Vector\ a
\end{tabular}]\haddockbegindoc
Converts a list to a vector.\par


\item[\begin{tabular}{@{}l}
fromVector\ ::\ Vector\ a\ ->\ {\char 91}a{\char 93}
\end{tabular}]\haddockbegindoc
Converts a vector to a list.\par


\item[\begin{tabular}{@{}l}
indexes\ ::\ Vector\ Integer
\end{tabular}]\haddockbegindoc
Creates the infinite vector:\par
\begin{interactive}
<1,2,3,4,...>\end{interactive}
Used mainly for operation on indexes.\par


\item[\begin{tabular}{@{}l}
isNull\ ::\ Vector\ a\ ->\ Bool
\end{tabular}]\haddockbegindoc
Returns \haddocktt{True} if the argument is a null vector.\par


\item[\begin{tabular}{@{}l}
(<:)\ ::\ Vector\ a\ ->\ a\ ->\ Vector\ a
\end{tabular}]\haddockbegindoc
Appends an element at the end of a vector.\par

\end{haddockdesc}
\subsection{Skeletons}
Algorithmic skeletons on vectors are mainly presented in terms
 of compositions of the atoms associated with the
 \haddocktt{Skeleton} Layer. When defining them,
 we use the following operators:\par
\haddockeq{fig/eqs-skel-vector-operators.pdf}\par
where:\par
\begin{itemize}
\item
(1) is the \haddockid{unit} constructor, constructing a singleton vector.\par

\item
(2) is the \haddockid{<++>} constructor, concatenating two vectors.\par

\item
(3) is the \haddocktt{<@!>} selector. The subscript notation is used to
 denote element at position \emph{n} in a vector.\par

\item
(4) suggests an arbitrary selector which returns a vector with
 another one's elements, based on some indices. The shown example
 is an alternative notation for the \haddockid{tail} skeleton.\par

\end{itemize}

\subsubsection{Functional networks}
This sub-category denotes skeletons (patterns) which are take
 functions as arguments. If the functions are
 \haddockid{MoC} layer entities, i.e. processes, then these
 patterns are capable of constructing parallel process
 networks. Using the applicative mechanism, the designer has a
 high degree of freedom when customizing process networks through
 systematic partial application, rendering numerous possible
 usages for the same pattern. To avoid over-encumbering the
 figures, they depict small test cases, which might not expose the
 full potential of the constructors.\par
see the \href{ForSyDe-Atom.html#naming_conv}{naming convention} rules
 on how to interpret, use and develop your own constructors.\par

\begin{haddockdesc}
\item[\begin{tabular}{@{}l}
farm22
\end{tabular}]\haddockbegindoc
\haddockbeginargs
\haddockdecltt{::} & \haddockdecltt{(a1
                                     -> a2
                                        -> (b1, b2))} & function (e.g. process) \\
                                                        \haddockdecltt{->} & \haddockdecltt{Vector a1} & first input vector \\
                                                                                                         \haddockdecltt{->} & \haddockdecltt{Vector a2} & second input vector \\
                                                                                                                                                          \haddockdecltt{->} & \haddockdecltt{(Vector b1, Vector b2)} & two output vectors \\
\end{tabulary}\par
\haddocktt{farm} is simply the \haddockid{Vector} instance of the skeletom \haddocktt{farm}
 pattern (see \haddockid{farm22}). If the function taken
 as argument is a process, then it creates a farm network of data
 parallel processes.\par
Constructors: \haddocktt{farm{\char 91}1-4{\char 93}{\char 91}1-4{\char 93}}.\par
\begin{interactive}
λ> let v1 = vector [1,2,3,4,5]
λ> S.farm21 (+) v1 v1
<2,4,6,8,10>
λ> let s1 = SY.signal [1,2,3,4,5]
λ> let v2 = vector [s1,s1,s1]
λ> S.farm11 (comb11 (+1)) v2
<{2,3,4,5,6},{2,3,4,5,6},{2,3,4,5,6}>
λ> S.farm21 (\x -> comb11 (+x)) v1 v2
<{2,3,4,5,6},{3,4,5,6,7},{4,5,6,7,8}>

\end{interactive}\haddockeq{fig/eqs-skel-vector-farm.pdf}\par
           \haddockdoublefig{fig/skel-vector-func-farm.pdf}{fig/skel-vector-func-farm-net.pdf}\par
           

\item[\begin{tabular}{@{}l}
reduce\ ::\ (a\ ->\ a\ ->\ a)\ ->\ Vector\ a\ ->\ a
\end{tabular}]\haddockbegindoc
As the name suggests, it reduces a vector to an element based on
 an associative function. If the function is not associative, it can be treated like a pipeline.\par
\haddockid{Vector} instantiates the skeletons for both
 \haddockid{reduce} and \haddockid{reducei}.\par
\begin{interactive}
λ> let v1 = vector [1,2,3,4,5]
λ> S.reduce (+) v1
15
λ> let s1 = SY.signal [1,2,3,4,5]
λ> let s2 = SY.signal [10,10,10,10,10]
λ> let v2 = vector [s1,s1,s1]
λ> S.reduce (comb21 (+)) v2
{3,6,9,12,15}
λ> S.reducei (comb21 (+)) s2 v2
{13,16,19,22,25}

\end{interactive}\haddockdoublefig{fig/skel-vector-func-reducei.pdf}{fig/skel-vector-func-reducei-net.pdf}\par
           

\item[\begin{tabular}{@{}l}
prefix\ ::\ (b\ ->\ b\ ->\ b)\ ->\ Vector\ b\ ->\ Vector\ b
\end{tabular}]\haddockbegindoc
\haddocktt{prefix} peforms the \emph{parallel prefix} operation on a vector.
 Equivalent process networks are constructed if processes are passed
 as arguments.\par
Similar to \haddockid{reduce} and \haddockid{reducei}, two versions \haddockid{prefix} and
 \haddocktt{prefixi} are provided.\par
\begin{interactive}
λ> let v1 = vector [1,2,3,4,5]
λ> prefix (+) v1
<15,14,12,9,5>
λ> let s1 = SY.signal [1,2,3,4,5]
λ> let s2 = SY.signal [10,10,10,10,10]
λ> let v2 = vector [s1,s1,s1]
λ> prefix (comb21 (+)) v2
<{3,6,9,12,15},{2,4,6,8,10},{1,2,3,4,5}>
λ> prefixi (comb21 (+)) s2 v2
<{13,16,19,22,25},{12,14,16,18,20},{11,12,13,14,15}>

\end{interactive}\haddockeq{fig/eqs-skel-vector-prefix.pdf}\par
           \haddockdoublefig{fig/skel-vector-func-prefix.pdf}{fig/skel-vector-func-prefix-net.pdf}\par
           \haddockdoublefig{fig/skel-vector-func-prefixi.pdf}{fig/skel-vector-func-prefixi-net.pdf}\par
           

\item[\begin{tabular}{@{}l}
suffix\ ::\ (b\ ->\ b\ ->\ b)\ ->\ Vector\ b\ ->\ Vector\ b
\end{tabular}]\haddockbegindoc
\haddocktt{suffix} peforms the \emph{parallel suffix} operation on a vector.
 Equivalent process networks are constructed if processes are passed
 as arguments.\par
Similar to \haddockid{reduce} and \haddockid{reducei}, two versions \haddockid{suffix} and
 \haddocktt{suffixi} are provided.\par
\begin{interactive}
λ> let v1 = vector [1,2,3,4,5]
λ> suffix (+) v1
<1,3,6,10,15>
λ> let s1 = SY.signal [1,2,3,4,5]
λ> let s2 = SY.signal [10,10,10,10,10]
λ> let v2 = vector [s1,s1,s1]
λ> suffix (comb21 (+)) v2
<{1,2,3,4,5},{2,4,6,8,10},{3,6,9,12,15}>
λ> suffixi (comb21 (+)) s2 v2
<{11,12,13,14,15},{12,14,16,18,20},{13,16,19,22,25}>

\end{interactive}\haddockeq{fig/eqs-skel-vector-suffix.pdf}\par
           \haddockdoublefig{fig/skel-vector-func-suffix.pdf}{fig/skel-vector-func-suffix-net.pdf}\par
           \haddockdoublefig{fig/skel-vector-func-suffixi.pdf}{fig/skel-vector-func-suffixi-net.pdf}\par
           

\item[\begin{tabular}{@{}l}
pipe
\end{tabular}]\haddockbegindoc
\haddockbeginargs
\haddockdecltt{::} & \haddockdecltt{Vector (a -> a)} & vector of functions \\
                                                       \haddockdecltt{->} & \haddockdecltt{a} & input \\
                                                                                                \haddockdecltt{->} & \haddockdecltt{a} & output \\
\end{tabulary}\par
\haddocktt{pipe} creates a pipeline of functions from a vector. \haddockid{pipe}
  simply instantiates the \haddockid{=<<=} atom whereas \haddocktt{pipeX} instantiate
  their omologi from the \haddocktt{ForSyDe.Atom.Skeleton} module (see
  \haddockid{pipe2}).\par
\textbf{OBS:} the pipelining is done in the order dictated by the
 function composition operator: from right to left.\par
Constructors: \haddocktt{pipe{\char 91}1-4{\char 93}}.\par
\begin{interactive}
λ> let v1 = vector [(+1),(+1),(+1)]
λ> S.pipe v1 1
4
λ> let s1 = SY.signal [1,2,3,4]
λ> let v2 = vector [1,2,3,4]
λ> S.pipe1 (\x -> comb11 (+x)) v2 s1
{11,12,13,14}

\end{interactive}\haddockdoublefig{fig/skel-vector-func-pipe.pdf}{fig/skel-vector-func-pipe-net.pdf}\par
           

\item[\begin{tabular}{@{}l}
(=/=)\ ::\ Vector\ (a\ ->\ a)\ ->\ a\ ->\ Vector\ a
\end{tabular}]\haddockbegindoc
Infix operator for \haddockid{recur}.\par


\item[\begin{tabular}{@{}l}
recur
\end{tabular}]\haddockbegindoc
\haddockbeginargs
\haddockdecltt{::} & \haddockdecltt{Vector (a -> a)} & vector of functions \\
                                                       \haddockdecltt{->} & \haddockdecltt{a} & input \\
                                                                                                \haddockdecltt{->} & \haddockdecltt{Vector a} & output  \\
\end{tabulary}\par
\haddocktt{recur} creates a systolic array from a vector of
 functions. Just like \haddockid{pipe} and \haddocktt{pipeX}, there exists a raw
 \haddockid{recur} version with an infix operator \haddockid{=/=}, and the enhanced
 \haddocktt{recurX} which is meant for systematic partial application of a
 function on an arbitrary number of vectors until the desired vector
 of functions is obtained.\par
Constructors: \haddocktt{(=/=)}, \haddocktt{recur}, \haddocktt{recuri}, \haddocktt{recur{\char 91}1-4{\char 93}{\char 91}1-4{\char 93}}.\par
\begin{interactive}
λ> let v1 = vector [(+1),(+1),(+1)]
λ> recur v1 1
<4,3,2>
λ> recuri v1 1
<4,3,2,1>
λ> let s1 = SY.signal [1,2,3,4]
λ> let v2 = vector [1,2,3,4]
λ> recur1 (\x -> comb11 (+x)) v2 s1
<{11,12,13,14},{10,11,12,13},{8,9,10,11},{5,6,7,8}>

\end{interactive}\haddockeq{fig/eqs-skel-vector-recur.pdf}\par
           \haddockdoublefig{fig/skel-vector-func-recur.pdf}{fig/skel-vector-func-recur-net.pdf}\par
           

\item[\begin{tabular}{@{}l}
cascade2
\end{tabular}]\haddockbegindoc
\haddockbeginargs
\haddockdecltt{::} & \haddockdecltt{(a2
                                     -> a1
                                        -> a
                                           -> a
                                              -> a)} & \haddocktt{function41} which needs to be applied to \haddocktt{function21} \\
                                                       \haddockdecltt{->} & \haddockdecltt{Vector (Vector a2)} & fills in the first argument in the function above \\
                                                                                                                 \haddockdecltt{->} & \haddockdecltt{Vector (Vector a1)} & fills in the second argument in the function above \\
                                                                                                                                                                           \haddockdecltt{->} & \haddockdecltt{Vector a} & first input vector (e.g. of signals) \\
                                                                                                                                                                                                                           \haddockdecltt{->} & \haddockdecltt{Vector a} & second input vector (e.g. of signals) \\
                                                                                                                                                                                                                                                                           \haddockdecltt{->} & \haddockdecltt{Vector a} & output \\
\end{tabulary}\par
\haddocktt{cascade} creates a "cascading mesh" as a result of piping a
 vector into a vector of recur arrays. \par
Constructors: \haddocktt{cascade}, \haddocktt{cascade{\char 91}1-4{\char 93}}.\par
\begin{interactive}
λ> let v1 = vector [1,2,3,4]
λ> cascade (+) v1 v1
<238,119,49,14>
λ> let s1 = SY.signal [1,2,3,4]
λ> let vs = vector [s1, s1, s1]
λ> cascade (comb21 (+)) vs vs
<{20,40,60,80},{10,20,30,40},{4,8,12,16}>
λ> let vv = vector [vector [1,-1,1], vector [-1,1,-1], vector [1,-1,1] ]
λ> cascade1 (\x -> comb21 (\y z-> x*(y+z))) vv vs vs
<{16,32,48,64},{8,16,24,32},{-2,-4,-6,-8}>

\end{interactive}\haddockeq{fig/eqs-skel-vector-cascade.pdf}\par
           \haddockdoublefig{fig/skel-vector-func-cascade.pdf}{fig/skel-vector-func-cascade-net.pdf}\par
           

\item[\begin{tabular}{@{}l}
mesh2
\end{tabular}]\haddockbegindoc
\haddockbeginargs
\haddockdecltt{::} & \haddockdecltt{(a2
                                     -> a1
                                        -> a
                                           -> a
                                              -> a)} & \haddocktt{function41} which needs to be applied to \haddocktt{function21} \\
                                                       \haddockdecltt{->} & \haddockdecltt{Vector (Vector a2)} & fills in the first argument in the function above \\
                                                                                                                 \haddockdecltt{->} & \haddockdecltt{Vector (Vector a1)} & fills in the second argument in the function above \\
                                                                                                                                                                           \haddockdecltt{->} & \haddockdecltt{Vector a} & first input vector (e.g. of signals) \\
                                                                                                                                                                                                                           \haddockdecltt{->} & \haddockdecltt{Vector a} & second input vector (e.g. of signals) \\
                                                                                                                                                                                                                                                                           \haddockdecltt{->} & \haddockdecltt{Vector (Vector a)} & output, a 2D vector \\
\end{tabulary}\par
\haddocktt{mesh} creates a 2D systolic array as a result of piping a vector
 into a vector of 1D systolic arrays.\par
Constructors: \haddocktt{mesh}, \haddocktt{mesh{\char 91}1-4{\char 93}}.\par
\begin{interactive}
λ> let v1 = vector [1,2,3,4]
λ> mesh (+) v1 v1
<<238,119,49,14>,<119,70,35,13>,<49,35,22,11>,<14,13,11,8>>
λ> let s1 = SY.signal [1,2,3,4]
λ> let vs = vector [s1, s1, s1]
λ> mesh (comb21 (+)) vs vs
<<{20,40,60,80},{10,20,30,40},{4,8,12,16}>,<{10,20,30,40},{6,12,18,24},{3,6,9,12}>,<{4,8,12,16}, {3,6,9,12},{2,4,6,8}>>
λ> let vv = vector [vector [1,-1,1], vector [-1,1,-1], vector [1,-1,1]]
λ> mesh1 (\x -> comb21 (\y z-> x*(y+z))) vv vs vs
<<{16,32,48,64},{8,16,24,32},{-2,-4,-6,-8}>,<{8,16,24,32},{-6,-12,-18,-24},{-3,-6,-9,-12}>, <{-2,-4,-6,-8},{-3,-6,-9,-12},{2,4,6,8}>>

\end{interactive}\haddockeq{fig/eqs-skel-vector-mesh.pdf}\par
           \haddockdoublefig{fig/skel-vector-func-mesh.pdf}{fig/skel-vector-func-mesh-net.pdf}\par
           
\end{haddockdesc}
\subsubsection{Queries}
Queries return various information about a vector. They are
 also built as skeletons.\par

\begin{haddockdesc}
\item[\begin{tabular}{@{}l}
length\ ::\ Num\ p\ =>\ Vector\ a\ ->\ p
\end{tabular}]\haddockbegindoc
returns the number of elements in a value.\par
\begin{interactive}
λ> length $ vector [1,2,3,4,5]
5

\end{interactive}\haddockeq{fig/eqs-skel-vector-length.pdf}\par
           

\item[\begin{tabular}{@{}l}
index\ ::\ Vector\ a2\ ->\ Vector\ Integer
\end{tabular}]\haddockbegindoc
returns a vector with the indexes from another vector.\par
\begin{interactive}
λ> index $ vector [1,1,1,1,1,1,1]
<1,2,3,4,5,6,7>

\end{interactive}
\end{haddockdesc}
\subsubsection{Generators}
Generators are specific applications of the \haddocktt{prefix} or
 \haddocktt{suffix} skeletons.\par

\begin{haddockdesc}
\item[\begin{tabular}{@{}l}
fanout\ ::\ t\ ->\ Vector\ t
\end{tabular}]\haddockbegindoc
\haddockid{fanout} repeats an element. As a process network it distributes
 the same value or signal to all the connected processes down the
 line. Depending on the target platform and the refinement decisions
 involved, it may be interpreted in the following implementations:\par
\begin{itemize}
\item
global or shared memory in case of a massively parallel platform
  (e.g. GPU)\par

\item
a (static) memory or cache location in memory-driven
  architectures (e.g. CPU)\par

\item
a fanout in case of a HDL system\par

\item
a broadcast in case of a distributed system\par

\end{itemize}


\item[\begin{tabular}{@{}l}
fanoutn\ ::\ (Num\ t,\ Ord\ t)\ =>\ t\ ->\ a\ ->\ Vector\ a
\end{tabular}]\haddockbegindoc
\haddockid{fanoutn} is the same as \haddockid{fanout}, but the length of the result
 is also provided.\par
\begin{interactive}
λ> fanoutn 5 1
<1,1,1,1,1>

\end{interactive}

\item[\begin{tabular}{@{}l}
generate\ ::\ (Num\ t,\ Ord\ t)\ =>\ t\ ->\ (a\ ->\ a)\ ->\ a\ ->\ Vector\ a
\end{tabular}]\haddockbegindoc
\haddockid{generate} creates a vector based on a kernel function. It is
 just a restricted version of \haddockid{recur}.\par
\begin{interactive}
λ> generate 5 (+1) 1
<6,5,4,3,2>

\end{interactive}\haddockeq{fig/eqs-skel-vector-generate.pdf}\par
           

\item[\begin{tabular}{@{}l}
iterate\ ::\ (Num\ t,\ Ord\ t)\ =>\ t\ ->\ (a\ ->\ a)\ ->\ a\ ->\ Vector\ a
\end{tabular}]\haddockbegindoc
\haddockid{iterate} is a version of \haddockid{generate} which keeps the initial
 element as well. It is a restricted version of \haddocktt{recuri}.\par
\begin{interactive}
λ> iterate 5 (+1) 1
<5,4,3,2,1>

\end{interactive}
\end{haddockdesc}
\subsubsection{Permutators}
Permutators perform operations on the very structure of
 vectors, and make heavy use of the vector constructors.\par

\begin{haddockdesc}
\item[\begin{tabular}{@{}l}
first\ ::\ Vector\ a\ ->\ a
\end{tabular}]\haddockbegindoc
Instance of \haddockid{first}\par
\begin{interactive}
λ> S.first $ vector [1,2,3,4,5]
1

\end{interactive}

\item[\begin{tabular}{@{}l}
last\ ::\ Vector\ a\ ->\ a
\end{tabular}]\haddockbegindoc
Instance of \haddockid{last}\par
\begin{interactive}
λ> S.last $ vector [1,2,3,4,5]
5

\end{interactive}

\item[\begin{tabular}{@{}l}
inits\ ::\ Vector\ a\ ->\ Vector\ (Vector\ a)
\end{tabular}]\haddockbegindoc
creates a vector of all the initial segments in a vector.\par
\begin{interactive}
λ> inits $ vector [1,2,3,4,5]
<<1>,<1,2>,<1,2,3>,<1,2,3,4>,<1,2,3,4,5>>

\end{interactive}\haddockeq{fig/eqs-skel-vector-inits.pdf}\par
           \haddockdoublefig{fig/skel-vector-comm-inits.pdf}{fig/skel-vector-comm-inits-net.pdf}\par
           

\item[\begin{tabular}{@{}l}
tails\ ::\ Vector\ a\ ->\ Vector\ (Vector\ a)
\end{tabular}]\haddockbegindoc
creates a vector of all the final segments in a vector.\par
\begin{interactive}
λ> tails $ vector [1,2,3,4,5]
<<1,2,3,4,5>,<2,3,4,5>,<3,4,5>,<4,5>,<5>>

\end{interactive}\haddockeq{fig/eqs-skel-vector-tails.pdf}\par
           \haddockdoublefig{fig/skel-vector-comm-tails.pdf}{fig/skel-vector-comm-tails-net.pdf}\par
           

\item[\begin{tabular}{@{}l}
init\ ::\ Vector\ a\ ->\ Vector\ a
\end{tabular}]\haddockbegindoc
Returns the initial segment of a vector.\par
\begin{interactive}
λ> init $ vector [1,2,3,4,5]
<1,2,3,4>

\end{interactive}\haddockeq{fig/eqs-skel-vector-init.pdf}\par
           

\item[\begin{tabular}{@{}l}
tail\ ::\ Vector\ a\ ->\ Vector\ a
\end{tabular}]\haddockbegindoc
Returns the tail of a vector.\par
\begin{interactive}
λ> tail $ vector [1,2,3,4,5]
<2,3,4,5>

\end{interactive}\haddockeq{fig/eqs-skel-vector-tail.pdf}\par
           

\item[\begin{tabular}{@{}l}
concat\ ::\ Vector\ (Vector\ a)\ ->\ Vector\ a
\end{tabular}]\haddockbegindoc
concatenates a vector of vectors.\par
\begin{interactive}
λ> concat $ vector [vector[1,2,3,4], vector[5,6,7]]
<1,2,3,4,5,6,7>

\end{interactive}\haddockeq{fig/eqs-skel-vector-concat.pdf}\par
           

\item[\begin{tabular}{@{}l}
reverse\ ::\ Vector\ a\ ->\ Vector\ a
\end{tabular}]\haddockbegindoc
reverses the elements in a vector.\par
\begin{interactive}
λ> reverse $ vector [1,2,3,4,5]
<5,4,3,2,1>

\end{interactive}\haddockeq{fig/eqs-skel-vector-reverse.pdf}\par
           \haddockdoublefig{fig/skel-vector-comm-reverse.pdf}{fig/skel-vector-comm-reverse-net.pdf}\par
           

\item[\begin{tabular}{@{}l}
group\ ::\ Integer\ ->\ Vector\ a\ ->\ Vector\ (Vector\ a)
\end{tabular}]\haddockbegindoc
groups a vector into sub-vectors of \emph{n} elements.\par
\begin{interactive}
λ> group 3 $ vector [1,2,3,4,5,6,7,8]
<<1,2,3>,<4,5,6>,<7,8>>

\end{interactive}\haddockeq{fig/eqs-skel-vector-group.pdf}\par
           \haddockdoublefig{fig/skel-vector-comm-group.pdf}{fig/skel-vector-comm-group-net.pdf}\par
           

\item[\begin{tabular}{@{}l}
shiftr\ ::\ Vector\ a\ ->\ a\ ->\ Vector\ a
\end{tabular}]\haddockbegindoc
right-shifts a vector with an element.\par
\begin{interactive}
λ> vector [1,2,3,4] `shiftr` 8
<8,1,2,3>

\end{interactive}\haddockdoublefig{fig/skel-vector-comm-shiftr.pdf}{fig/skel-vector-comm-shiftr-net.pdf}\par
           

\item[\begin{tabular}{@{}l}
shiftl\ ::\ Vector\ a\ ->\ a\ ->\ Vector\ a
\end{tabular}]\haddockbegindoc
left-shifts a vector with an element.\par
\begin{interactive}
λ> vector [1,2,3,4] `shiftl` 8
<2,3,4,8>

\end{interactive}\haddockdoublefig{fig/skel-vector-comm-shiftl.pdf}{fig/skel-vector-comm-shiftl-net.pdf}\par
           

\item[\begin{tabular}{@{}l}
rotr\ ::\ Vector\ a\ ->\ Vector\ a
\end{tabular}]\haddockbegindoc
rotates a vector to the right.\par
\begin{interactive}
λ> rotr $ vector [1,2,3,4]
<4,1,2,3>

\end{interactive}\haddockdoublefig{fig/skel-vector-comm-rotr.pdf}{fig/skel-vector-comm-rotr-net.pdf}\par
           

\item[\begin{tabular}{@{}l}
rotl\ ::\ Vector\ a\ ->\ Vector\ a
\end{tabular}]\haddockbegindoc
rotates a vector to the left.\par
\begin{interactive}
λ> rotl $ vector [1,2,3,4]
<2,3,4,1>

\end{interactive}\haddockdoublefig{fig/skel-vector-comm-rotl.pdf}{fig/skel-vector-comm-rotl-net.pdf}\par
           

\item[\begin{tabular}{@{}l}
take\ ::\ Integer\ ->\ Vector\ a\ ->\ Vector\ a
\end{tabular}]\haddockbegindoc
takes the first \emph{n} elements of a vector.\par
\begin{interactive}
λ> take 5 $ vector [1,2,3,4,5,6,7,8,9]
<1,2,3,4,5>

\end{interactive}\haddockeq{fig/eqs-skel-vector-take.pdf}\par
           

\item[\begin{tabular}{@{}l}
drop\ ::\ Integer\ ->\ Vector\ a\ ->\ Vector\ a
\end{tabular}]\haddockbegindoc
drops the first \emph{n} elements of a vector.\par
\begin{interactive}
λ> drop 5 $ vector [1,2,3,4,5,6,7,8,9]
<6,7,8,9>

\end{interactive}\haddockeq{fig/eqs-skel-vector-drop.pdf}\par
           

\item[\begin{tabular}{@{}l}
takeWhile\ ::\ (a\ ->\ Bool)\ ->\ Vector\ a\ ->\ Vector\ a
\end{tabular}]\haddockbegindoc
takes the first elements in a vector until the first element that
 does not fulfill a predicate.\par
\begin{interactive}
λ> takeWhile (<5) $ vector [1,2,3,4,5,6,7,8,9]
<1,2,3,4>

\end{interactive}\haddockeq{fig/eqs-skel-vector-takewhile.pdf}\par
           

\item[\begin{tabular}{@{}l}
filterIdx\ ::\ (Integer\ ->\ Bool)\ ->\ Vector\ a\ ->\ Vector\ a
\end{tabular}]\haddockbegindoc
returns a vector containing only the elements of another vector
 whose index satisfies a predicate.\par
\begin{interactive}
λ> filterIdx (\x -> x `mod` 3 == 0) $ vector [0,1,2,3,4,5,6,7,8,9]
<2,5,8>

\end{interactive}\haddockeq{fig/eqs-skel-vector-filteridx.pdf}\par
           \haddockdoublefig{fig/skel-vector-comm-filteridx.pdf}{fig/skel-vector-comm-filteridx-net.pdf}\par
           

\item[\begin{tabular}{@{}l}
odds\ ::\ Vector\ a\ ->\ Vector\ a
\end{tabular}]\haddockbegindoc
\haddockeq{fig/eqs-skel-vector-odds.pdf}\par


\item[\begin{tabular}{@{}l}
evens\ ::\ Vector\ a\ ->\ Vector\ a
\end{tabular}]\haddockbegindoc
\haddockeq{fig/eqs-skel-vector-evens.pdf}\par


\item[\begin{tabular}{@{}l}
stride
\end{tabular}]\haddockbegindoc
\haddockbeginargs
\haddockdecltt{::} & \haddockdecltt{Integer} & first index \\
                                               \haddockdecltt{->} & \haddockdecltt{Integer} & stride length \\
                                                                                              \haddockdecltt{->} & \haddockdecltt{Vector a} & \\
                                                                                                                                              \haddockdecltt{->} & \haddockdecltt{Vector a} & \\
\end{tabulary}\par
does a stride-selection on a vector.\par
\begin{interactive}
λ> stride 1 3 $ vector [1,2,3,4,5,6,7,8,9]
<1,4,7>

\end{interactive}\haddockeq{fig/eqs-skel-vector-stride.pdf}\par
           \haddockdoublefig{fig/skel-vector-comm-inits.pdf}{fig/skel-vector-comm-inits-net.pdf}\par
           

\item[\begin{tabular}{@{}l}
get\ ::\ Integer\ ->\ Vector\ a\ ->\ Maybe\ a
\end{tabular}]\haddockbegindoc
returns the \emph{n}-th element in a vector, or \haddocktt{Nothing} if \emph{n > l}.\par
\begin{interactive}
λ> get 3 $ vector [1,2,3,4,5]
Just 3

\end{interactive}\haddockeq{fig/eqs-skel-vector-get.pdf}\par
           

\item[\begin{tabular}{@{}l}
(<@)\ ::\ Vector\ a\ ->\ Integer\ ->\ Maybe\ a
\end{tabular}]\haddockbegindoc
the same as \haddockid{get} but with flipped arguments.\par


\item[\begin{tabular}{@{}l}
(<@!)\ ::\ Vector\ p\ ->\ Integer\ ->\ p
\end{tabular}]\haddockbegindoc
unsafe version of \haddockid{<@>}. Throws an exception if \emph{n > l}.\par


\item[\begin{tabular}{@{}l}
gather1
\end{tabular}]\haddockbegindoc
\haddockbeginargs
\haddockdecltt{::} & \haddockdecltt{Vector Integer} & vector of indexes \\
                                                      \haddockdecltt{->} & \haddockdecltt{Vector a} & input vector \\
                                                                                                      \haddockdecltt{->} & \haddockdecltt{Vector (Maybe a)} & \\
\end{tabulary}\par
selects the elements in a vector at the incexes contained by another vector.\par
The following versions of this skeleton are available, the number
 suggesting how many nested vectors it is operating upon: \haddocktt{gather{\char 91}1-5{\char 93}}\par
\begin{interactive}
λ> let ix = vector [vector [1,3,4], vector [3,5,1], vector [5,8,9]]
λ> let v = vector [11,12,13,14,15]
λ> gather2 ix v
<<Just 11,Just 13,Just 14>,<Just 13,Just 15,Just 11>,<Just 15,Nothing,Nothing>>

\end{interactive}\haddockeq{fig/eqs-skel-vector-gather.pdf}\par
           \haddockdoublefig{fig/skel-vector-comm-gather.pdf}{fig/skel-vector-comm-gather-net.pdf}\par
           

\item[\begin{tabular}{@{}l}
(<@>)
\end{tabular}]\haddockbegindoc
\haddockbeginargs
\haddockdecltt{::} & \haddockdecltt{Vector a} & input vector \\
                                                \haddockdecltt{->} & \haddockdecltt{Vector Integer} & vector of indexes \\
                                                                                                      \haddockdecltt{->} & \haddockdecltt{Vector (Maybe a)} & \\
\end{tabulary}\par
the same as \haddockid{gather1} but with flipped arguments\par
The following versions of this skeleton are available, the number
 suggesting how many nested vectors it is operating upon.\par
\begin{code}
(<@>), (<<@>>), (<<<@>>>), (<<<<@>>>>), (<<<<<@>>>>>),\end{code}


\item[\begin{tabular}{@{}l}
replace\ ::\ Integer\ ->\ p\ ->\ Vector\ p\ ->\ Vector\ p
\end{tabular}]\haddockbegindoc
replaces the \emph{n}-th element in a vector with another.\par
\begin{interactive}
λ> replace 5 15 $ vector [1,2,3,4,5,6,7,8,9]
<1,2,3,4,15,6,7,8,9>

\end{interactive}\haddockeq{fig/eqs-skel-vector-replace.pdf}\par
           \haddockdoublefig{fig/skel-vector-comm-replace.pdf}{fig/skel-vector-comm-replace-net.pdf}\par
           

\item[\begin{tabular}{@{}l}
scatter\ ::\ Vector\ Integer\ ->\ Vector\ p\ ->\ Vector\ p\ ->\ Vector\ p
\end{tabular}]\haddockbegindoc
scatters the elements in a vector based on the indexes contained by another vector.\par
\begin{interactive}
λ> scatter (vector [2,4,5]) (vector [0,0,0,0,0,0,0,0]) (vector [1,1,1])
<0,1,0,1,1,0,0,0>

\end{interactive}\haddockeq{fig/eqs-skel-vector-scatter.pdf}\par
           \haddockdoublefig{fig/skel-vector-comm-scatter.pdf}{fig/skel-vector-comm-scatter-net.pdf}\par
           

\item[\begin{tabular}{@{}l}
bitrev\ ::\ Vector\ a\ ->\ Vector\ a
\end{tabular}]\haddockbegindoc
performs a bit-reverse permutation.\par
\haddockdoublefig{fig/skel-vector-comm-bitrev.pdf}{fig/skel-vector-comm-bitrev-net.pdf}\par
\begin{interactive}
λ> bitrev $ vector ["000","001","010","011","100","101","110","111"]
<"111","011","101","001","110","010","100","000">

\end{interactive}

\item[\begin{tabular}{@{}l}
duals\ ::\ Vector\ b2\ ->\ (Vector\ b2,\ Vector\ b2)
\end{tabular}]\haddockbegindoc
splits a vector in two equal parts.\par
\begin{interactive}
λ> duals $ vector [1,2,3,4,5,6,7]
(<1,2,3>,<4,5,6>)

\end{interactive}

\item[\begin{tabular}{@{}l}
unduals\ ::\ Vector\ a\ ->\ Vector\ a\ ->\ Vector\ a
\end{tabular}]\haddockbegindoc
concatenates a previously split vector. See also \haddockid{duals}\par

\end{haddockdesc}
\subsubsection{Interfaces}
\begin{haddockdesc}
\item[\begin{tabular}{@{}l}
zipx
\end{tabular}]\haddockbegindoc
\haddockbeginargs
\haddockdecltt{::} & MoC e \\
                     \haddockdecltt{=>} & \haddockdecltt{Vector ((Vector a
                                                                  -> Vector a
                                                                     -> Vector a)
                                                                 -> Fun e (Vector a) (Fun e (Vector a) (Ret e (Vector a))))} & vector of MoC-specific context wrappers for the function
 \haddockid{<++>} \\
                                                                                                                               \haddockdecltt{->} & \haddockdecltt{Vector (Stream (e a))} & input vector of signals \\
                                                                                                                                                                                            \haddockdecltt{->} & \haddockdecltt{Stream (e (Vector a))} & output signal of vectors \\
\end{tabulary}\par
\haddockid{zipx} is a template skeleton for "zipping" a vector of
 signals. It synchronizes all signals (of the same MoC) in a vector
 and outputs one signal with vectors of the synced values. For each
 signal in the input vector it requires a function which
 \emph{translates} a partition of events (see \haddocktt{ForSyDe.Atom.MoC}) into
 sub-vectors.\par
There exist helper intances of the \haddockid{zipx} skeleton interface for
 all supported MoCs.\par
\haddockeq{fig/eqs-skel-vector-zipx.pdf}
 \haddockfig{fig/skel-vector-comm-zipx.pdf}\par


\item[\begin{tabular}{@{}l}
unzipx\ ::\ MoC\ e\ =>\\\ \ \ \ \ \ \ \ \ \ (Vector\ a\ ->\ Vector\ (Ret\ e\ a))\\\ \ \ \ \ \ \ \ \ \ ->\ Integer\ ->\ Stream\ (e\ (Vector\ a))\ ->\ Vector\ (Stream\ (e\ a))
\end{tabular}]\haddockbegindoc
\haddockid{unzipx} is a template skeleton to unzip a signal carrying
 vectors into a vector of multiple signals. It required a function
 that \emph{splits} a vector of values into a vector of event partitions
 belonging to output signals. Unlike \haddockid{zipx}, it also requires the
 number of output signals. The reason for this is that it is
 impossible to determine the length of the output vector without
 "sniffing" the content of the input events, which is out of the
 scope of skeletons and may lead to unsafe behavior. The length of
 the output vector is needed in order to avoid infinite recurrence.\par
There exist helper intances of the \haddockid{unzipx} skeleton interface for
 all supported MoCs.\par
\haddockeq{fig/eqs-skel-vector-unzipx.pdf}{fig/skel-vector-comm-unzipx.pdf}\par

\end{haddockdesc}
  % \haddockmoduleheading{ForSyDe.Atom.Utility}
\label{module:ForSyDe.Atom.Utility}
\haddockbeginheader
{\haddockverb\begin{verbatim}
module ForSyDe.Atom.Utility (
    module ForSyDe.Atom.Utility.Tuple
  ) where\end{verbatim}}
\haddockendheader

This parent module contains sub-modules that concern utility
 functions. Due to legacy reasons, only the
 \haddocktt{ForSyDe.Atom.Utility.Tuple} module is re-exported. All other
 sub-modules need to be imported explicitly.\par

This module contains utility functions dealing with tuples.\par

\begin{haddockdesc}
\item[\begin{tabular}{@{}l}
module\ ForSyDe.Atom.Utility.Tuple
\end{tabular}]
\end{haddockdesc}
  \haddockmoduleheading{ForSyDe.Atom.Utility.Plot}
\label{module:ForSyDe.Atom.Utility.Plot}
\haddockbeginheader
{\haddockverb\begin{verbatim}
module ForSyDe.Atom.Utility.Plot (
    Config(Cfg, verbose, path, title, rate, xmax, labels, fire, other), 
    defaultCfg,  silentCfg,  noJunkCfg,  prepare,  prepareL,  prepareV, 
    showDat,  dumpDat,  plotGnu,  heatmapGnu,  showLatex,  dumpLatex, 
    plotLatex,  Plottable(toCoord),  Plot(sample, sample', takeUntil, getInfo), 
    PInfo(Info, typeid, command, measure, style, stacking, sparse),  Samples, 
    PlotData
  ) where\end{verbatim}}
\haddockendheader

This module imports plotting and data dumping functions working
 with "plottable" data types, i.e. instances of the \haddockid{Plot} and
 \haddockid{Plottable} type classes.\par

\subsection{User API}
The following commands are frequently used as part of the
 normal modeling routine.\par

\subsubsection{Configuration settings}
\begin{haddockdesc}
\item[\begin{tabular}{@{}l}
data\ Config
\end{tabular}]\haddockbegindoc
\haddockbeginconstrs
\haddockdecltt{=} & \haddockdecltt{Cfg} & \\
                    \haddockdecltt{>}&\haddockdecltt{verbose :: Bool} & verbose printouts on terminal\\
                    \haddockdecltt{>}&\haddockdecltt{path :: String} & directory where all dumped files will be found\\
                    \haddockdecltt{>}&\haddockdecltt{title :: String} & base name for dumped files\\
                    \haddockdecltt{>}&\haddockdecltt{rate :: Float} & sample rate if relevant. Useful for explicit-tagged signals, ignored otherwise.\\
                    \haddockdecltt{>}&\haddockdecltt{xmax :: Float} & Maximum X coordinate. Mandatory for infinite structures, optional otherwise.\\
                    \haddockdecltt{>}&\haddockdecltt{labels :: [String]} & list of labels with the names of the structures plotted\\
                    \haddockdecltt{>}&\haddockdecltt{fire :: Bool} & if relevant, fires a plotting or compiling program.\\
                    \haddockdecltt{>}&\haddockdecltt{other :: Bool} & if relevant, dumps additional scripts and plots.\\
\end{tabulary}\par

Record structure containing configuration settings for the
 plotting commands.\par


\item[\begin{tabular}{@{}l}
instance\ Show\ Config
\end{tabular}]

\item[\begin{tabular}{@{}l}
defaultCfg\ ::\ Config
\end{tabular}]\haddockbegindoc
Default configuration: verbose, dump everything possible, fire
 whatever program needed. Check source for settings.\par
Example usage:\par
\begin{interactive}
λ> defaultCfg {xmax = 15, verbose = False, labels = ["john","doe"]}
Cfg {verbose = False, path = "./fig", title = "plot", rate = 1.0e-2, xmax = 15.0, labels = ["john","doe"], fire = True, other = True}

\end{interactive}

\item[\begin{tabular}{@{}l}
silentCfg\ ::\ Config
\end{tabular}]\haddockbegindoc
Silent configuration: does not fire any program or print our
 unnecessary info. Check source for settings.\par


\item[\begin{tabular}{@{}l}
noJunkCfg\ ::\ Config
\end{tabular}]\haddockbegindoc
Clean configuration: verbose, does not dump more than necessary,
 fire whatever program needed. Check source for settings.\par

\end{haddockdesc}
\subsubsection{Data preparation}
\begin{haddockdesc}
\item[\begin{tabular}{@{}l}
prepare
\end{tabular}]\haddockbegindoc
\haddockbeginargs
\haddockdecltt{::} & Plot a \\
                     \haddockdecltt{=>} & \haddockdecltt{Config} & configuration settings \\
                                                                   \haddockdecltt{->} & \haddockdecltt{a} & plottable data type \\
                                                                                                            \haddockdecltt{->} & \haddockdecltt{PlotData} & structure ready for dumping \\
\end{tabulary}\par
Prepares a single plottable data structure to be dumped and/or
 plotted.\par


\item[\begin{tabular}{@{}l}
prepareL\ ::\ Plot\ a\ =>\ Config\ ->\ {\char 91}a{\char 93}\ ->\ PlotData
\end{tabular}]\haddockbegindoc
Prepares a list of plottable data structures to be dumped and/or
 plotted. See \haddockid{prepare}.\par


\item[\begin{tabular}{@{}l}
prepareV\ ::\ Plot\ a\ =>\ Config\ ->\ Vector\ a\ ->\ PlotData
\end{tabular}]\haddockbegindoc
Prepares a vector of plottable data structures to be dumped
 and/or plotted. See \haddockid{prepare}.\par

\end{haddockdesc}
\subsubsection{Dumping and plotting data}
\begin{haddockdesc}
\item[\begin{tabular}{@{}l}
showDat\ ::\ PlotData\ ->\ IO\ ()
\end{tabular}]\haddockbegindoc
Prints out the sampled contents of a \haddockid{prepare}d data set.\par


\item[\begin{tabular}{@{}l}
dumpDat\ ::\ PlotData\ ->\ IO\ {\char 91}String{\char 93}
\end{tabular}]\haddockbegindoc
Dumps the sampled contents of a \haddockid{prepare}d data set into separate
 \haddocktt{.dat} files.\par


\item[\begin{tabular}{@{}l}
plotGnu\ ::\ PlotData\ ->\ IO\ ()
\end{tabular}]\haddockbegindoc
Generates a GNUplot script and \haddocktt{.dat} files for plotting the
 sampled contents of a \haddockid{prepare}d data set. Depending on the
 configuration settings, it also dumps LaTeX and PDF plots, and
 fires the script.\par
\textbf{OBS:} needless to say that \href{http://www.gnuplot.info/}{GNUplot}
 needs to be installed in order to use this command. Also, in order
 to fire GNUplot from a ghci session you might need to install
 \haddocktt{gnuplot-x11}.\par


\item[\begin{tabular}{@{}l}
heatmapGnu\ ::\ PlotData\ ->\ IO\ ()
\end{tabular}]\haddockbegindoc
Similar to \haddockid{plotGnu} but creates a heatmap plot using the GNUplot
 engine. For this, the input needs to contain at least two columns
 of data, otherwise the plot does not show anything, i.e. the
 samples need to be lists or vectors of two or more elements.\par
\textbf{OBS:} same dependencies are needed as for \haddockid{plotGnu}.\par


\item[\begin{tabular}{@{}l}
showLatex\ ::\ PlotData\ ->\ IO\ ()
\end{tabular}]\haddockbegindoc
Prints out a LaTeX environment from a \haddockid{prepare}d data set. This
 environment should be paste inside a \haddocktt{tikzpicture} in a document
 title which imports the ForSyDe-LaTeX package.\par


\item[\begin{tabular}{@{}l}
dumpLatex\ ::\ PlotData\ ->\ IO\ {\char 91}String{\char 93}
\end{tabular}]\haddockbegindoc
Dumps a set of formatted data files with the extension \haddocktt{.flx}
 that can be imported by a LaTeX document which uses the
 ForSyDe-LaTeX package.\par


\item[\begin{tabular}{@{}l}
plotLatex\ ::\ PlotData\ ->\ IO\ ()
\end{tabular}]\haddockbegindoc
Creates a standalone LaTeX document which uses the ForSyDe-LaTeX
 package, plotting a \haddockid{prepare}d data set. Depending on the
 configuration settings, the command \haddocktt{pdflatex} may also be invoked
 to compile a pdf image.\par
\textbf{OBS:} A LaTeX compiler is required to run the \haddocktt{pdflatex}
 command. The \href{https://github.com/forsyde/forsyde-latex}{ForSyDe-LaTeX}
 package also needs to be installed according to the instructions on
 the project web page.\par

\end{haddockdesc}
\subsection{The data types}
Below the data types involved are shown and the plottable
 structures are documented.\par

\begin{haddockdesc}
\item[\begin{tabular}{@{}l}
class\ Plottable\ a\ where
\end{tabular}]\haddockbegindoc
This class gathers types which can be sampled and converted to a
 numerical string which can be read and interpreted by a plotter
 engine.\par

\haddockpremethods{}\textbf{Methods}
\begin{haddockdesc}
\item[\begin{tabular}{@{}l}\haddockid{toCoord}\ ::\ a\ ->\ String\ Source\ \end{tabular}]
\haddockbegindoc
Transforms the input type into a coordinate string.\par

\item[\begin{tabular}{@{}l}\haddockid{sample}\ ::\ Float\ ->\ a\ ->\ Samples\ Source\ \end{tabular}]
\haddockbegindoc
Samples the data according to a given step size.\par

\item[\begin{tabular}{@{}l}\haddockid{sample}\ ::\ a\ ->\ Samples\ Source\ ,\ \end{tabular}]
\haddockbegindoc
Samples the data according to the internal structure.\par

\item[\begin{tabular}{@{}l}\haddockid{takeUntil}\ ::\ Float\ ->\ a\ ->\ a\ Source\ \end{tabular}]
\haddockbegindoc
Takes the first samples until a given tag.\par

\item[\begin{tabular}{@{}l}\haddockid{getInfo}\ ::\ a\ ->\ PInfo\ Source\ \end{tabular}]
\haddockbegindoc
Returns static information about the data type.\par

\end{haddockdesc}


\item[\begin{tabular}{@{}l}
instance\ (Show\ a,\ Real\ a)\ =>\ Plottable\ a
\end{tabular}]\haddockbegindoc
Real numbers that can be converted to a floating point representation\par


\item[\begin{tabular}{@{}l}
instance\ Plottable\ TimeStamp
\end{tabular}]\haddockbegindoc
Time stamps\par


\item[\begin{tabular}{@{}l}
instance\ Plottable\ a\ =>\ Plottable\ (Vector\ a)
\end{tabular}]\haddockbegindoc
Vectors of plottable types\par


\item[\begin{tabular}{@{}l}
instance\ (Show\ a,\ Plottable\ a)\ =>\ Plottable\ (AbstExt\ a)
\end{tabular}]\haddockbegindoc
Absent-extended plottable types\par


\item[\begin{tabular}{@{}l}
class\ Plot\ a\ where
\end{tabular}]\haddockbegindoc
This class gathers all ForSyDe-Atom structures that can be
 plotted.\par

\haddockpremethods{}\textbf{Methods}
\begin{haddockdesc}
\item[\begin{tabular}{@{}l}\haddockid{toCoord}\ ::\ a\ ->\ String\ Source\ \end{tabular}]
\haddockbegindoc
Transforms the input type into a coordinate string.\par

\item[\begin{tabular}{@{}l}\haddockid{sample}\ ::\ Float\ ->\ a\ ->\ Samples\ Source\ \end{tabular}]
\haddockbegindoc
Samples the data according to a given step size.\par

\item[\begin{tabular}{@{}l}\haddockid{sample}\ ::\ a\ ->\ Samples\ Source\ ,\ \end{tabular}]
\haddockbegindoc
Samples the data according to the internal structure.\par

\item[\begin{tabular}{@{}l}\haddockid{takeUntil}\ ::\ Float\ ->\ a\ ->\ a\ Source\ \end{tabular}]
\haddockbegindoc
Takes the first samples until a given tag.\par

\item[\begin{tabular}{@{}l}\haddockid{getInfo}\ ::\ a\ ->\ PInfo\ Source\ \end{tabular}]
\haddockbegindoc
Returns static information about the data type.\par

\end{haddockdesc}


\item[\begin{tabular}{@{}l}
instance\ Plottable\ a\ =>\ Plot\ (Vector\ a)
\end{tabular}]\haddockbegindoc
vectors of coordinates\par


\item[\begin{tabular}{@{}l}
instance\ Plottable\ a\ =>\ Plot\ (Signal\ a)
\end{tabular}]\haddockbegindoc
\haddockid{SY} signals.\par


\item[\begin{tabular}{@{}l}
instance\ Plottable\ a\ =>\ Plot\ (Signal\ a)
\end{tabular}]\haddockbegindoc
\haddockid{SDF} signals.\par


\item[\begin{tabular}{@{}l}
instance\ Plottable\ a\ =>\ Plot\ (Signal\ a)
\end{tabular}]\haddockbegindoc
\haddockid{DE} signals.\par


\item[\begin{tabular}{@{}l}
instance\ Plottable\ a\ =>\ Plot\ (Signal\ a)
\end{tabular}]\haddockbegindoc
\haddockid{CT} signals.\par


\item[\begin{tabular}{@{}l}
data\ PInfo
\end{tabular}]\haddockbegindoc
\haddockbeginconstrs
\haddockdecltt{=} & \haddockdecltt{Info} & \\
                    \haddockdecltt{>}&\haddockdecltt{typeid :: String} & id used usually in implicit tags\\
                    \haddockdecltt{>}&\haddockdecltt{command :: String} & LaTeX identifier\\
                    \haddockdecltt{>}&\haddockdecltt{measure :: String} & unit of measure\\
                    \haddockdecltt{>}&\haddockdecltt{style :: String} & style tweaking in the GNUplot script\\
                    \haddockdecltt{>}&\haddockdecltt{stacking :: Bool} & if the plot is stacking\\
                    \haddockdecltt{>}&\haddockdecltt{sparse :: Bool} & if the sampled data is sparse instead of dense\\
\end{tabulary}\par

Static information of each plottable data type.\par


\item[\begin{tabular}{@{}l}
instance\ Show\ PInfo
\end{tabular}]

\item[\begin{tabular}{@{}l}
type\ Samples\ =\ {\char 91}(String,\ String){\char 93}
\end{tabular}]\haddockbegindoc
Alias for sampled data \par


\item[\begin{tabular}{@{}l}
type\ PlotData\ =\ (Config,\ PInfo,\ {\char 91}(String,\ Samples){\char 93})
\end{tabular}]\haddockbegindoc
Alias for a data set \haddockid{prepare}d to be plotted.\par

\end{haddockdesc}
  \haddockmoduleheading{ForSyDe.Atom.Utility.Tuple}
\label{module:ForSyDe.Atom.Utility.Tuple}
\haddockbeginheader
{\haddockverb\begin{verbatim}
module ForSyDe.Atom.Utility.Tuple (
    at22,  (||<),  (<>),  ($$)
  ) where\end{verbatim}}
\haddockendheader

This module implements general purpose utility functions. It mainly
 hosts functions dealing with tuples. Utility are provided for up
 until 9-tuples. Follow the examples in the source code in case it
 does not suffice.\par
\begin{mdframed}[style=reminder,frametitle=Reminder]Make sure to consult naming conventions in  \cref{sec:forsyde-atom:naming-convention} in order to interpret the names and type signatures correctly.\end{mdframed}\par

\begin{haddockdesc}
\item[\begin{tabular}{@{}l}
at22\ ::\ (a1,\ a2)\ ->\ a2
\end{tabular}]\haddockbegindoc
The \haddocktt{at}\emph{xy} functions return the \emph{y}-th element of an \emph{x}-tuple.\par
\haddocktt{ForSyDe.Atom.Utility} exports the constructors below. Please
 follow the examples in the source code if they do not suffice:\par
\begin{code}
at21, at22,
at31, at32, at33,
at41, at42, at43, at44,
at51, at52, at53, at54, at55,
at61, at62, at63, at64, at65, at66, 
at71, at72, at73, at74, at75, at76, at77,
at81, at82, at83, at84, at85, at86, at87, at88,
at91, at92, at93, at94, at95, at96, at97, at98, at99,\end{code}
Example:\par
\begin{interactive}
λ> at53 (1,2,3,4,5)
3

\end{interactive}

\item[\begin{tabular}{@{}l}
(||<)\ ::\ (Functor\ f1,\ Functor\ f2)\ =>\\\ \ \ \ \ \ \ \ \ f1\ (f2\ (a1,\ a2))\ ->\ (f1\ (f2\ a1),\ f1\ (f2\ a2))
\end{tabular}]\haddockbegindoc
This set of utility functions "unzip" nested n-tuples, provided
 as postfix operators. They are crucial for reconstructing data
 types from higher-order functions which input functions with
 multiple outputs. It relies on the nested types being instances of
 \haddockid{Functor}.\par
The operator convention is \haddocktt{(|+<+)}, where the number of \haddocktt{|}
 represent the number of layers the n-tuple is lifted, while the
 number of \haddocktt{<} + 1 is the order \emph{n} of the n-tuple.\par
\haddocktt{ForSyDe.Atom.Utility} exports the constructors below. Please
 follow the examples in the source code if they do not suffice:\par
\begin{code}
   |<,    |<<,    |<<<,    |<<<<,    |<<<<<,    |<<<<<<,    |<<<<<<<,    |<<<<<<<<,
  ||<,   ||<<,   ||<<<,   ||<<<<,   ||<<<<<,   ||<<<<<<,   ||<<<<<<<,   ||<<<<<<<<,
 |||<,  |||<<,  |||<<<,  |||<<<<,  |||<<<<<,  |||<<<<<<,  |||<<<<<<<,  |||<<<<<<<<,  
||||<, ||||<<, ||||<<<, ||||<<<<, ||||<<<<<, ||||<<<<<<, ||||<<<<<<<, ||||<<<<<<<<, \end{code}
Example:\par
\begin{interactive}
λ> :set -XPostfixOperators
λ> ([Just (1,2,3), Nothing, Just (4,5,6)] ||<<)
([Just 1,Nothing,Just 4],[Just 2,Nothing,Just 5],[Just 3,Nothing,Just 6])

\end{interactive}

\item[\begin{tabular}{@{}l}
(<>)\ ::\ (a1\ ->\ a2\ ->\ b1)\ ->\ (a1,\ a2)\ ->\ b1
\end{tabular}]\haddockbegindoc
Infix currying operators used for convenience. \par
The operator convention is \haddocktt{(<>+)}, where the number of \haddocktt{>} + 1 is
 the order \emph{n} of the n-tuple.\par
\haddocktt{ForSyDe.Atom.Utility} exports the constructors below. Please
 follow the examples in the source code if they do not suffice:\par
\begin{code}
<>, <>>, <>>>, <>>>>, <>>>>>, <>>>>>>, <>>>>>>>, <>>>>>>>>\end{code}
Example:\par
\begin{interactive}
λ> (+) <> (1,2)
3

\end{interactive}

\item[\begin{tabular}{@{}l}
({\char '44}{\char '44})\ ::\ (a1\ ->\ b1,\ a2\ ->\ b2)\ ->\ (a1,\ a2)\ ->\ (b1,\ b2)
\end{tabular}]\haddockbegindoc
Infix function application operator for tuples. \par
The operator convention is \haddocktt{({\char '44}+)}, where the number of \haddocktt{{\char '44}} is the
 order \emph{n} of the n-tuple. For Applying a function on nontuples we
 rely on \haddockid{{\char '44}} provided by \haddocktt{Prelude}.\par
\haddocktt{ForSyDe.Atom.Utility} exports the constructors below. Please
 follow the examples in the source code if they do not suffice:\par
\begin{code}
$$, $$$, $$$$, $$$$$, $$$$$$, $$$$$$$, $$$$$$$$, $$$$$$$$$\end{code}
Example:\par
\begin{interactive}
λ> ((+),(-)) $$ (1,1) $$ (2,2)
(3,-1)

\end{interactive}
\end{haddockdesc}
  
  \printbibliography[heading=subbibliography]
  \printindex
\end{refsection}

\end{document}

%%% Local Variables:
%%% TeX-command-default: "Make"
%%% mode: latex
%%% TeX-master: t
%%% End:
