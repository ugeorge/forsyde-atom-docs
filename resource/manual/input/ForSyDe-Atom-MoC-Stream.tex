\haddockmoduleheading{ForSyDe.Atom.MoC.Stream}
\label{module:ForSyDe.Atom.MoC.Stream}
\haddockbeginheader
{\haddockverb\begin{verbatim}
module ForSyDe.Atom.MoC.Stream (
    Stream(NullS, (:-)),  ,  ,  ,  ,  ,  stream,  fromStream,  headS,  tailS, 
    lastS,  repeatS,  takeS,  dropS,  takeWhileS,  (+-+)
  ) where\end{verbatim}}
\haddockendheader

This module defines the shallow-embedded \haddockid{Stream} datatype and
 utility functions operating on it. In ForSyDe a signal is
 represented as a (partially or totally) \emph{ordered} sequence of
 events that enables processes to communicate and synchronize.  The
 \haddockid{Stream} type is but an ordered structure to encapsulate events as
 infinite streams.\par

\begin{haddockdesc}
\item[\begin{tabular}{@{}l}
data\ Stream\ e
\end{tabular}]\haddockbegindoc
\haddockbeginconstrs
\haddockdecltt{=} & \haddockdecltt{NullS} & terminates a signal \\
\haddockdecltt{|} & \haddockdecltt{e (:-) (Stream e)} & the default constructor appends an\\
 &&event to the head of the stream \\
\end{tabulary}\par

Defines a stream of events, encapsulating them in a structure
 isomorphic to an infinite list \cite{Bird87},
 thus all properties of lists may also be applied to
 \haddockid{Stream}s. While, in combination with lazy evaluation, it is
 possible to create and simulate infinite signals, we need to ensure
 that the first/previous event is always fully evaluated. This can
 be translated into the following rule:\par
\begin{description}
\item[zero-delay feedbacks] are forbidden, due to un-evaluated
 self-referential calls. In a feedback loop, there always has to be
 enough events to ensure the data flow.
\end{description}This rule imposes that the stream of data is uninterrupted in order
 to have an evaluatable kernel every time a new event is produced
 (i.e. to avoid deadlocks). Thus we can add the rule:\par
                 \begin{description}
                 \item[cleaning of output events] is also forbidden.  In other words, for
 each new input at any instant in time, a process must react with
 \emph{at least} one output event.
                 \end{description}

\item[\begin{tabular}{@{}l}
instance\ Functor\ Stream
\end{tabular}]\haddockbegindoc
allows for the mapping of an arbitrary function \haddocktt{(a\ ->\ b)} upon
 all the events of a \haddocktt{(Stream\ a)}.\par


\item[\begin{tabular}{@{}l}
instance\ Applicative\ Stream
\end{tabular}]\haddockbegindoc
enables the \haddockid{Stream} to behave like a \haddockid{ZipList}\par


\item[\begin{tabular}{@{}l}
instance\ Foldable\ Stream
\end{tabular}]\haddockbegindoc
provides folding functions useful for implementing utilities, such as \haddockid{length}.\par


\item[\begin{tabular}{@{}l}
instance\ Read\ a\ =>\ Read\ (Stream\ a)
\end{tabular}]\haddockbegindoc
signal \haddocktt{(1\ :-\ 2\ :-\ NullS)} is read using the string \haddocktt{"{\char '173}1,2{\char '175}"}.\par


\item[\begin{tabular}{@{}l}
instance\ Show\ a\ =>\ Show\ (Stream\ a)
\end{tabular}]\haddockbegindoc
signal \haddocktt{(1\ :-\ 2\ :-\ NullS)} is represented as \haddocktt{{\char '173}1,2{\char '175}}.\par


\item[\begin{tabular}{@{}l}
instance\ Plottable\ a\ =>\ Plot\ (Signal\ a)
\end{tabular}]\haddockbegindoc
\haddockid{SY} signals.\par


\item[\begin{tabular}{@{}l}
instance\ Plottable\ a\ =>\ Plot\ (Signal\ a)
\end{tabular}]\haddockbegindoc
\haddockid{SDF} signals.\par


\item[\begin{tabular}{@{}l}
instance\ Plottable\ a\ =>\ Plot\ (Signal\ a)
\end{tabular}]\haddockbegindoc
\haddockid{DE} signals.\par


\item[\begin{tabular}{@{}l}
instance\ Plottable\ a\ =>\ Plot\ (Signal\ a)
\end{tabular}]\haddockbegindoc
\haddockid{CT} signals.\par


\item[\begin{tabular}{@{}l}
stream\ ::\ {\char 91}a{\char 93}\ ->\ Stream\ a
\end{tabular}]\haddockbegindoc
The function \haddocktt{signal} converts a list into a signal.\par


\item[\begin{tabular}{@{}l}
fromStream\ ::\ Stream\ a\ ->\ {\char 91}a{\char 93}
\end{tabular}]\haddockbegindoc
The function \haddockid{fromStream} converts a signal into a list.\par


\item[\begin{tabular}{@{}l}
headS\ ::\ Stream\ a\ ->\ a
\end{tabular}]\haddockbegindoc
Returns the head of a signal.\par


\item[\begin{tabular}{@{}l}
tailS\ ::\ Stream\ e\ ->\ Stream\ e
\end{tabular}]\haddockbegindoc
Returns the tail of a signal\par


\item[\begin{tabular}{@{}l}
lastS\ ::\ Stream\ p\ ->\ p
\end{tabular}]\haddockbegindoc
Returns the last event in a signal.\par


\item[\begin{tabular}{@{}l}
repeatS\ ::\ a\ ->\ Stream\ a
\end{tabular}]\haddockbegindoc
Returns an infinite list containing the same repeated event.\par


\item[\begin{tabular}{@{}l}
takeS\ ::\ (Ord\ t,\ Num\ t)\ =>\ t\ ->\ Stream\ e\ ->\ Stream\ e
\end{tabular}]\haddockbegindoc
Returns the first \haddocktt{n} events in a signal.\par


\item[\begin{tabular}{@{}l}
dropS\ ::\ (Ord\ t,\ Num\ t)\ =>\ t\ ->\ Stream\ e\ ->\ Stream\ e
\end{tabular}]\haddockbegindoc
Drops the first \haddocktt{n} events in a signal.\par


\item[\begin{tabular}{@{}l}
takeWhileS\ ::\ (a\ ->\ Bool)\ ->\ Stream\ a\ ->\ Stream\ a
\end{tabular}]\haddockbegindoc
Returns the first events of a signal which comply to a condition.\par


\item[\begin{tabular}{@{}l}
(+-+)\ ::\ Stream\ e\ ->\ Stream\ e\ ->\ Stream\ e
\end{tabular}]\haddockbegindoc
Concatenates two signals.\par

\end{haddockdesc}