\section{Purpose \& organization}
\label{sec:purp-organ}

This book is a living document which gathers material related to \textsc{ForSyDe-Atom} and binds it in form of a user manual. Most of the text contained by this book originates from actual inline or literate source code documentation, in form of examples, tutorials, reports and even library API documentation. This means that this document evolves with the \textsc{ForSyDe-Atom} project itself and is periodically updated.

\textsc{ForSyDe-Atom} is a shallow-embedded DSL in the functional programming language Haskell for modeling cyber-physical and parallel systems. It enforces a disciplined way of modeling by separating the manifold concerns of systems into orthogonal \emph{layers}. The \textsc{ForSyDe-Atom} formal framework aims to providing (where possible) a minimum set of primary common building blocks for each layer called \emph{atoms}, capturing elementary semantics. Even so, the modeling framework provides library blocks and modules commonly used in CPS defined in terms of \emph{patterns} of atoms. For more information on \textsc{ForSyDe-Atom} itself, please consult the associated scientific publications or the API extended documentation\footnote{currently available online at \url{https://forsyde.github.io/forsyde-atom/}}.

This document is structured in two parts. The first part gathers examples, tutorials and experiments in a learning progression. Each chapter associated with (and actually generated from) a \href{https://www.haskell.org/cabal/}{Haskell Cabal project} included in the \texttt{forsyde-atom-examples}\footnote{available online at \url{https://github.com/forsyde/forsyde-atom-examples}} repository. This means that the first part of the book is meant to be read in parallel with running the associated example project which conveniently exports functions to test the listed code ``on-the-fly''. The second part of the book is the actual inline API documentation of \textsc{ForSyDe-Atom} generated with \href{https://www.haskell.org/haddock/}{Haddock}, which serves also as an extended library report and provides information both on theoretical and implementation issues.

\begin{summary}
  If for any reason you have difficulties following the document or you encounter bugs or discrepancies in the code and text do not hesitate to contact the author(s) or maintainer(s) on GitHub or by email. 
\end{summary}

%%% Local Variables:
%%% mode: latex
%%% TeX-master: "manual"
%%% End:
