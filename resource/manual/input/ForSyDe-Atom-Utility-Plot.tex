\haddockmoduleheading{ForSyDe.Atom.Utility.Plot}
\label{module:ForSyDe.Atom.Utility.Plot}
\haddockbeginheader
{\haddockverb\begin{verbatim}
module ForSyDe.Atom.Utility.Plot (
    Config(Cfg, verbose, path, title, rate, xmax, labels, fire, other), 
    defaultCfg,  silentCfg,  noJunkCfg,  prepare,  prepareL,  prepareV, 
    showDat,  dumpDat,  plotGnu,  heatmapGnu,  showLatex,  dumpLatex, 
    plotLatex,  Plottable(toCoord),  Plot(sample, sample', takeUntil, getInfo), 
    PInfo(Info, typeid, command, measure, style, stacking, sparse),  Samples, 
    PlotData
  ) where\end{verbatim}}
\haddockendheader

This module imports plotting and data dumping functions working
 with "plottable" data types, i.e. instances of the \haddockid{Plot} and
 \haddockid{Plottable} type classes.\par

\subsection{User API}
The following commands are frequently used as part of the
 normal modeling routine.\par

\subsubsection{Configuration settings}
\begin{haddockdesc}
\item[\begin{tabular}{@{}l}
data\ Config
\end{tabular}]\haddockbegindoc
\haddockbeginconstrs
\haddockdecltt{=} & \haddockdecltt{Cfg} & \\
                    \haddockdecltt{>}&\haddockdecltt{verbose :: Bool} & verbose printouts on terminal\\
                    \haddockdecltt{>}&\haddockdecltt{path :: String} & directory where all dumped files will be found\\
                    \haddockdecltt{>}&\haddockdecltt{title :: String} & base name for dumped files\\
                    \haddockdecltt{>}&\haddockdecltt{rate :: Float} & sample rate if relevant. Useful for explicit-tagged signals, ignored otherwise.\\
                    \haddockdecltt{>}&\haddockdecltt{xmax :: Float} & Maximum X coordinate. Mandatory for infinite structures, optional otherwise.\\
                    \haddockdecltt{>}&\haddockdecltt{labels :: [String]} & list of labels with the names of the structures plotted\\
                    \haddockdecltt{>}&\haddockdecltt{fire :: Bool} & if relevant, fires a plotting or compiling program.\\
                    \haddockdecltt{>}&\haddockdecltt{other :: Bool} & if relevant, dumps additional scripts and plots.\\
\end{tabulary}\par

Record structure containing configuration settings for the
 plotting commands.\par


\item[\begin{tabular}{@{}l}
instance\ Show\ Config
\end{tabular}]

\item[\begin{tabular}{@{}l}
defaultCfg\ ::\ Config
\end{tabular}]\haddockbegindoc
Default configuration: verbose, dump everything possible, fire
 whatever program needed. Check source for settings.\par
Example usage:\par
\begin{interactive}
λ> defaultCfg {xmax = 15, verbose = False, labels = ["john","doe"]}
Cfg {verbose = False, path = "./fig", title = "plot", rate = 1.0e-2, xmax = 15.0, labels = ["john","doe"], fire = True, other = True}

\end{interactive}

\item[\begin{tabular}{@{}l}
silentCfg\ ::\ Config
\end{tabular}]\haddockbegindoc
Silent configuration: does not fire any program or print our
 unnecessary info. Check source for settings.\par


\item[\begin{tabular}{@{}l}
noJunkCfg\ ::\ Config
\end{tabular}]\haddockbegindoc
Clean configuration: verbose, does not dump more than necessary,
 fire whatever program needed. Check source for settings.\par

\end{haddockdesc}
\subsubsection{Data preparation}
\begin{haddockdesc}
\item[\begin{tabular}{@{}l}
prepare
\end{tabular}]\haddockbegindoc
\haddockbeginargs
\haddockdecltt{::} & Plot a \\
                     \haddockdecltt{=>} & \haddockdecltt{Config} & configuration settings \\
                                                                   \haddockdecltt{->} & \haddockdecltt{a} & plottable data type \\
                                                                                                            \haddockdecltt{->} & \haddockdecltt{PlotData} & structure ready for dumping \\
\end{tabulary}\par
Prepares a single plottable data structure to be dumped and/or
 plotted.\par


\item[\begin{tabular}{@{}l}
prepareL\ ::\ Plot\ a\ =>\ Config\ ->\ {\char 91}a{\char 93}\ ->\ PlotData
\end{tabular}]\haddockbegindoc
Prepares a list of plottable data structures to be dumped and/or
 plotted. See \haddockid{prepare}.\par


\item[\begin{tabular}{@{}l}
prepareV\ ::\ Plot\ a\ =>\ Config\ ->\ Vector\ a\ ->\ PlotData
\end{tabular}]\haddockbegindoc
Prepares a vector of plottable data structures to be dumped
 and/or plotted. See \haddockid{prepare}.\par

\end{haddockdesc}
\subsubsection{Dumping and plotting data}
\begin{haddockdesc}
\item[\begin{tabular}{@{}l}
showDat\ ::\ PlotData\ ->\ IO\ ()
\end{tabular}]\haddockbegindoc
Prints out the sampled contents of a \haddockid{prepare}d data set.\par


\item[\begin{tabular}{@{}l}
dumpDat\ ::\ PlotData\ ->\ IO\ {\char 91}String{\char 93}
\end{tabular}]\haddockbegindoc
Dumps the sampled contents of a \haddockid{prepare}d data set into separate
 \haddocktt{.dat} files.\par


\item[\begin{tabular}{@{}l}
plotGnu\ ::\ PlotData\ ->\ IO\ ()
\end{tabular}]\haddockbegindoc
Generates a GNUplot script and \haddocktt{.dat} files for plotting the
 sampled contents of a \haddockid{prepare}d data set. Depending on the
 configuration settings, it also dumps LaTeX and PDF plots, and
 fires the script.\par
\textbf{OBS:} needless to say that \href{http://www.gnuplot.info/}{GNUplot}
 needs to be installed in order to use this command. Also, in order
 to fire GNUplot from a ghci session you might need to install
 \haddocktt{gnuplot-x11}.\par


\item[\begin{tabular}{@{}l}
heatmapGnu\ ::\ PlotData\ ->\ IO\ ()
\end{tabular}]\haddockbegindoc
Similar to \haddockid{plotGnu} but creates a heatmap plot using the GNUplot
 engine. For this, the input needs to contain at least two columns
 of data, otherwise the plot does not show anything, i.e. the
 samples need to be lists or vectors of two or more elements.\par
\textbf{OBS:} same dependencies are needed as for \haddockid{plotGnu}.\par


\item[\begin{tabular}{@{}l}
showLatex\ ::\ PlotData\ ->\ IO\ ()
\end{tabular}]\haddockbegindoc
Prints out a LaTeX environment from a \haddockid{prepare}d data set. This
 environment should be paste inside a \haddocktt{tikzpicture} in a document
 title which imports the ForSyDe-LaTeX package.\par


\item[\begin{tabular}{@{}l}
dumpLatex\ ::\ PlotData\ ->\ IO\ {\char 91}String{\char 93}
\end{tabular}]\haddockbegindoc
Dumps a set of formatted data files with the extension \haddocktt{.flx}
 that can be imported by a LaTeX document which uses the
 ForSyDe-LaTeX package.\par


\item[\begin{tabular}{@{}l}
plotLatex\ ::\ PlotData\ ->\ IO\ ()
\end{tabular}]\haddockbegindoc
Creates a standalone LaTeX document which uses the ForSyDe-LaTeX
 package, plotting a \haddockid{prepare}d data set. Depending on the
 configuration settings, the command \haddocktt{pdflatex} may also be invoked
 to compile a pdf image.\par
\textbf{OBS:} A LaTeX compiler is required to run the \haddocktt{pdflatex}
 command. The \href{https://github.com/forsyde/forsyde-latex}{ForSyDe-LaTeX}
 package also needs to be installed according to the instructions on
 the project web page.\par

\end{haddockdesc}
\subsection{The data types}
Below the data types involved are shown and the plottable
 structures are documented.\par

\begin{haddockdesc}
\item[\begin{tabular}{@{}l}
class\ Plottable\ a\ where
\end{tabular}]\haddockbegindoc
This class gathers types which can be sampled and converted to a
 numerical string which can be read and interpreted by a plotter
 engine.\par

\haddockpremethods{}\textbf{Methods}
\begin{haddockdesc}
\item[\begin{tabular}{@{}l}\haddockid{toCoord}\ ::\ a\ ->\ String\ Source\ \end{tabular}]
\haddockbegindoc
Transforms the input type into a coordinate string.\par

\item[\begin{tabular}{@{}l}\haddockid{sample}\ ::\ Float\ ->\ a\ ->\ Samples\ Source\ \end{tabular}]
\haddockbegindoc
Samples the data according to a given step size.\par

\item[\begin{tabular}{@{}l}\haddockid{sample}\ ::\ a\ ->\ Samples\ Source\ ,\ \end{tabular}]
\haddockbegindoc
Samples the data according to the internal structure.\par

\item[\begin{tabular}{@{}l}\haddockid{takeUntil}\ ::\ Float\ ->\ a\ ->\ a\ Source\ \end{tabular}]
\haddockbegindoc
Takes the first samples until a given tag.\par

\item[\begin{tabular}{@{}l}\haddockid{getInfo}\ ::\ a\ ->\ PInfo\ Source\ \end{tabular}]
\haddockbegindoc
Returns static information about the data type.\par

\end{haddockdesc}


\item[\begin{tabular}{@{}l}
instance\ (Show\ a,\ Real\ a)\ =>\ Plottable\ a
\end{tabular}]\haddockbegindoc
Real numbers that can be converted to a floating point representation\par


\item[\begin{tabular}{@{}l}
instance\ Plottable\ TimeStamp
\end{tabular}]\haddockbegindoc
Time stamps\par


\item[\begin{tabular}{@{}l}
instance\ Plottable\ a\ =>\ Plottable\ (Vector\ a)
\end{tabular}]\haddockbegindoc
Vectors of plottable types\par


\item[\begin{tabular}{@{}l}
instance\ (Show\ a,\ Plottable\ a)\ =>\ Plottable\ (AbstExt\ a)
\end{tabular}]\haddockbegindoc
Absent-extended plottable types\par


\item[\begin{tabular}{@{}l}
class\ Plot\ a\ where
\end{tabular}]\haddockbegindoc
This class gathers all ForSyDe-Atom structures that can be
 plotted.\par

\haddockpremethods{}\textbf{Methods}
\begin{haddockdesc}
\item[\begin{tabular}{@{}l}\haddockid{toCoord}\ ::\ a\ ->\ String\ Source\ \end{tabular}]
\haddockbegindoc
Transforms the input type into a coordinate string.\par

\item[\begin{tabular}{@{}l}\haddockid{sample}\ ::\ Float\ ->\ a\ ->\ Samples\ Source\ \end{tabular}]
\haddockbegindoc
Samples the data according to a given step size.\par

\item[\begin{tabular}{@{}l}\haddockid{sample}\ ::\ a\ ->\ Samples\ Source\ ,\ \end{tabular}]
\haddockbegindoc
Samples the data according to the internal structure.\par

\item[\begin{tabular}{@{}l}\haddockid{takeUntil}\ ::\ Float\ ->\ a\ ->\ a\ Source\ \end{tabular}]
\haddockbegindoc
Takes the first samples until a given tag.\par

\item[\begin{tabular}{@{}l}\haddockid{getInfo}\ ::\ a\ ->\ PInfo\ Source\ \end{tabular}]
\haddockbegindoc
Returns static information about the data type.\par

\end{haddockdesc}


\item[\begin{tabular}{@{}l}
instance\ Plottable\ a\ =>\ Plot\ (Vector\ a)
\end{tabular}]\haddockbegindoc
vectors of coordinates\par


\item[\begin{tabular}{@{}l}
instance\ Plottable\ a\ =>\ Plot\ (Signal\ a)
\end{tabular}]\haddockbegindoc
\haddockid{SY} signals.\par


\item[\begin{tabular}{@{}l}
instance\ Plottable\ a\ =>\ Plot\ (Signal\ a)
\end{tabular}]\haddockbegindoc
\haddockid{SDF} signals.\par


\item[\begin{tabular}{@{}l}
instance\ Plottable\ a\ =>\ Plot\ (Signal\ a)
\end{tabular}]\haddockbegindoc
\haddockid{DE} signals.\par


\item[\begin{tabular}{@{}l}
instance\ Plottable\ a\ =>\ Plot\ (Signal\ a)
\end{tabular}]\haddockbegindoc
\haddockid{CT} signals.\par


\item[\begin{tabular}{@{}l}
data\ PInfo
\end{tabular}]\haddockbegindoc
\haddockbeginconstrs
\haddockdecltt{=} & \haddockdecltt{Info} & \\
                    \haddockdecltt{>}&\haddockdecltt{typeid :: String} & id used usually in implicit tags\\
                    \haddockdecltt{>}&\haddockdecltt{command :: String} & LaTeX identifier\\
                    \haddockdecltt{>}&\haddockdecltt{measure :: String} & unit of measure\\
                    \haddockdecltt{>}&\haddockdecltt{style :: String} & style tweaking in the GNUplot script\\
                    \haddockdecltt{>}&\haddockdecltt{stacking :: Bool} & if the plot is stacking\\
                    \haddockdecltt{>}&\haddockdecltt{sparse :: Bool} & if the sampled data is sparse instead of dense\\
\end{tabulary}\par

Static information of each plottable data type.\par


\item[\begin{tabular}{@{}l}
instance\ Show\ PInfo
\end{tabular}]

\item[\begin{tabular}{@{}l}
type\ Samples\ =\ {\char 91}(String,\ String){\char 93}
\end{tabular}]\haddockbegindoc
Alias for sampled data \par


\item[\begin{tabular}{@{}l}
type\ PlotData\ =\ (Config,\ PInfo,\ {\char 91}(String,\ Samples){\char 93})
\end{tabular}]\haddockbegindoc
Alias for a data set \haddockid{prepare}d to be plotted.\par

\end{haddockdesc}