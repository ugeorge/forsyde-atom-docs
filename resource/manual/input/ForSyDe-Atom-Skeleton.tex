\haddockmoduleheading{ForSyDe.Atom.Skeleton}
\label{module:ForSyDe.Atom.Skeleton}
\haddockbeginheader
{\haddockverb\begin{verbatim}
module ForSyDe.Atom.Skeleton (
    Skeleton((=.=), (=*=), (=\=), (=<<=), first, last),  farm22,  reduce, 
    reducei,  pipe,  pipe2
  ) where\end{verbatim}}
\haddockendheader

This module exports a type class with the interfaces for the
 Skeleton layer atoms. It does \emph{NOT} export any implementation of
 atoms not any constructor as composition of atoms.\par
\begin{mdframed}[style=reminder,frametitle=Reminder]Make sure to consult naming conventions in  \cref{sec:forsyde-atom:naming-convention} in order to interpret the names and type signatures correctly.\end{mdframed}\par

\subsection{Atoms}
\begin{haddockdesc}
\item[\begin{tabular}{@{}l}
class\ Functor\ c\ =>\ Skeleton\ c\ where
\end{tabular}]\haddockbegindoc
Class containing all the Skeleton layer atoms.\par
This class is instantiated by a set of categorical types,
 i.e. types which describe an inherent potential for being evaluated
 in parallel. Skeletons are patterns from this layer. When skeletons
 take as arguments entities from the MoC layer (i.e. processes), the
 results themselves are parallel process networks which describe
 systems with an inherent potential to be implemented on parallel
 platforms. All skeletons can be described as composition of the
 three atoms below (\haddockid{=<<=} being just a specific instantiation of
 \haddockid{={\char '134}=}). This possible due to an existing theorem in the categorical
 type theory, also called the Bird-Merteens formalism
 \cite{Bird97}:\par
\par
\begin{description}
\item[factorization] A function on a categorical type is an algorithmic
 skeleton (i.e. catamorphism) \emph{iff} it can be represented in a
 factorized form, i.e. as a \emph{map} composed with a \emph{reduce}.
\end{description}Consequently, most of the skeletons for the implemented categorical
 types are described in their factorized form, taking as arguments
 either:\par
                 \begin{itemize}
                 \item
                 type constructors or functions derived from type constructors\par
                 
                 \item
                 processes, i.e. MoC layer entities\par
                 
                 \end{itemize}
                 Most of the ground-work on algorithmic skeletons on which this
 module is founded has been laid in \cite{Bird97},
 \cite{Skillicorn05} and it founds many
 of the frameworks collected in \cite{Gorlatch03}.\par
                 
\haddockpremethods{}\textbf{Methods}
\begin{haddockdesc}
\item[\begin{tabular}{@{}l}\haddockid{(=.=)}\ ::\ (a\ ->\ b)\ ->\ c\ a\ ->\ c\ b\ \end{tabular}]
\haddockbegindoc
Atom which maps a function on each element of a structure (i.e. categorical type), defined as:\haddockeq{fig/eqs-skel-atom-dot.pdf}\haddockid{=.=} together with \haddockid{=*=} form the map pattern.\par

\item[\begin{tabular}{@{}l}\haddockid{(=*=)}\ ::\ c\ (a\ ->\ b)\ ->\ c\ a\ ->\ c\ b\ \end{tabular}]
\haddockbegindoc
Atom which applies the functions contained by as structure (i.e. categorical type), on the elements of another structure, defined as:\haddockeq{fig/eqs-skel-atom-star.pdf}\haddockid{=.=} together with \haddockid{=*=} form the map pattern.\par

\item[\begin{tabular}{@{}l}\haddockid{(=\textbackslash=)}\ ::\ (a\ ->\ a\ ->\ a)\ ->\ c\ a\ ->\ a\ \end{tabular}]
\haddockbegindoc
Atom which reduces a structure to an element based on an \emph{associative} function, defined as:\haddockeq{fig/eqs-skel-atom-red.pdf}\par

\item[\begin{tabular}{@{}l}\haddockid{(=<<=)}\ \end{tabular}]
\haddockbegindoc
Skeleton which \emph{pipes} an element through all the functions contained by a structure. N.B.: this is not an atom. It has an implicit definition which might be augmented by instances of this class to include edge cases.\haddockeq{fig/eqs-skel-pattern-pipe.pdf}As the composition operation is not associative, we cannot treat pipe as a true reduction. Alas, it can still be exploited in parallel since it exposes another type of parallelism: time parallelism.\par

\item[\begin{tabular}{@{}l}\haddockid{first}\ ::\ c\ a\ ->\ a\ \end{tabular}]
\haddockbegindoc
Returns the first element in a structure.N.B.: this is not an atom. It has an implicit definition which might be replaced by instances of this class with a more efficient implementation.\haddockeq{fig/eqs-skel-pattern-first.pdf}\par

\item[\begin{tabular}{@{}l}\haddockid{last}\ ::\ c\ a\ ->\ a\ \end{tabular}]
\haddockbegindoc
Returns the last element in a structure.N.B.: this is not an atom. It has an implicit definition which might be replaced by instances of this class with a more efficient implementation.\haddockeq{fig/eqs-skel-pattern-last.pdf}\par

\end{haddockdesc}


\item[\begin{tabular}{@{}l}
instance\ Skeleton\ Vector
\end{tabular}]\haddockbegindoc
Ensures that \haddockid{Vector} is a structure associated with the Skeleton Layer.\par

\end{haddockdesc}

\subsection{Skeleton constructors}
Patterns of in the skeleton layer are provided, like all other
 patterns in ForSyDe-Atom, as constructors. If the layer below
 this one is the \haddockid{MoC} layer, i.e. the functions
 taken as arguments are processes, then these skeletons can be
 regarded as process network constructors, as the structures
 created are process networks with inherent potential for parallel
 implementation.\par

\begin{haddockdesc}
\item[\begin{tabular}{@{}l}
farm22\ ::\ Skeleton\ c\ =>\\\ \ \ \ \ \ \ \ \ \ (a1\ ->\ a2\ ->\ (b1,\ b2))\ ->\ c\ a1\ ->\ c\ a2\ ->\ (c\ b1,\ c\ b2)
\end{tabular}]\haddockbegindoc
\haddocktt{farm} maps a function on a vector. It is the embodiment of the
 \haddocktt{map} homomorphism, and its naming is inspired from the pattern
 predominant in HPC. Indeed, if we consider the layer below as being
 the \haddockid{MoC} layer (i.e. the passed functions are
 processes), the resulting structure could be regarded as a "farm of
 data-parallel processes".\par
Constructors: \haddocktt{farm{\char 91}1-8{\char 93}{\char 91}1-4{\char 93}}.\par
\haddockeq{fig/eqs-skel-pattern-farm.pdf}
 \haddockfig{fig/skel-pattern-farm.pdf}\par


\item[\begin{tabular}{@{}l}
reduce
\end{tabular}]\haddockbegindoc
\haddockbeginargs
\haddockdecltt{::} & Skeleton c \\
                     \haddockdecltt{=>} & \haddockdecltt{(a
                                                          -> a
                                                             -> a)} & associative function (*) \\
                                                                      \haddockdecltt{->} & \haddockdecltt{c a} & structure \\
                                                                                                                 \haddockdecltt{->} & \haddockdecltt{a} & reduced element \\
\end{tabulary}\par
Infix name for the \haddockid{={\char '134}=} atom operator.\par
(*) if the operation is not associative then the network can be
 treated like a pipeline.\par


\item[\begin{tabular}{@{}l}
reducei
\end{tabular}]\haddockbegindoc
\haddockbeginargs
\haddockdecltt{::} & Skeleton c \\
                     \haddockdecltt{=>} & \haddockdecltt{(a
                                                          -> a
                                                             -> a)} & associative function (*) \\
                                                                      \haddockdecltt{->} & \haddockdecltt{a} & initial element of structure \\
                                                                                                               \haddockdecltt{->} & \haddockdecltt{c a} & structure \\
                                                                                                                                                          \haddockdecltt{->} & \haddockdecltt{a} & reduced element \\
\end{tabulary}\par
\haddockid{reducei} is special case of \haddockid{reduce} where an initial element is
 specified outside the reduced vector. It is implemented as a
 \haddockid{pipe} with switched arguments, and the reduction function is
 constrained to be associative. It is semantically equivalent to the
 pattern depicted below.\par
(*) if the operation is not associative then the network is
 semantically equivalent to \haddocktt{pipe1} (see \haddockid{pipe2}).\par
\haddockeq{fig/eqs-skel-pattern-reducei.pdf}
 \haddockfig{fig/skel-pattern-reducei.pdf}\par


\item[\begin{tabular}{@{}l}
pipe
\end{tabular}]\haddockbegindoc
\haddockbeginargs
\haddockdecltt{::} & Skeleton c \\
                     \haddockdecltt{=>} & \haddockdecltt{c (a
                                                            -> a)} & vector of functions \\
                                                                     \haddockdecltt{->} & \haddockdecltt{a} & kernel element \\
                                                                                                              \haddockdecltt{->} & \haddockdecltt{a} & result  \\
\end{tabulary}\par
Infix name for the \haddockid{=<<=} skeleton operator.\par


\item[\begin{tabular}{@{}l}
pipe2\ ::\ Skeleton\ c\ =>\\\ \ \ \ \ \ \ \ \ (a1\ ->\ a2\ ->\ a\ ->\ a)\ ->\ c\ a1\ ->\ c\ a2\ ->\ a\ ->\ a
\end{tabular}]\haddockbegindoc
The \haddocktt{pipe} constructors are a more generic form of the \haddockid{=<<=}
 (\haddockid{pipe}) skeleton apt for successive partial application and create
 more robust parameterizable pipeline networks.\par
Constructors: \haddocktt{comb{\char 91}1-8{\char 93}}.\par
\haddockeq{fig/eqs-skel-pattern-pipe1.pdf}
 \haddockfig{fig/skel-pattern-pipe1.pdf}\par

\end{haddockdesc}