\haddockmoduleheading{ForSyDe.Atom.MoC.Time}
\label{module:ForSyDe.Atom.MoC.Time}
\haddockbeginheader
{\haddockverb\begin{verbatim}
module ForSyDe.Atom.MoC.Time (
    Time,  time,  const,  e,  (*^*),  pi,  sin,  cos,  tan,  atan,  asin, 
    acos,  sqrt,  exp,  cosh,  sinh,  tanh,  atanh,  asinh,  acosh,  log
  ) where\end{verbatim}}
\haddockendheader

Collection of utility functions for working with \haddockid{Time}. While the
 \haddockid{CT} MoC describes time as being a non-disjoint
 continuum (represented in ForSyDe-Atom with \haddockid{Rational} numbers),
 most of the functions here are non-ideal approximations or
 conversions from floating point equivalents. The trigonometric
 functions are imported from the
 \href{https://hackage.haskell.org/package/numbers-3000.2.0.1}{numbers}
 package, with a fixed \haddocktt{eps} parameter.\par
These utilities are meant to get started with using the CT MoC, and
 should be used with caution if result fidelity is a requirement. In
 this case the user should find a native \haddockid{Rational} implementation
 for a particular function.\par

\begin{haddockdesc}
\item[\begin{tabular}{@{}l}
type\ Time\ =\ Rational
\end{tabular}]\haddockbegindoc
Type alias for the type to represent metric (continuous)
 time. Underneath we use \haddockid{Rational} that is able to represent any
 \emph{t} between \emph{t₁} < \emph{t₂} ∈ \emph{T}.\par


\item[\begin{tabular}{@{}l}
time\ ::\ TimeStamp\ ->\ Time
\end{tabular}]\haddockbegindoc
Converts \haddockid{TimeStamp} into \haddockid{Time} representation.\par


\item[\begin{tabular}{@{}l}
const\ ::\ a\ ->\ Time\ ->\ a
\end{tabular}]\haddockbegindoc
Returns a constant function.\par


\item[\begin{tabular}{@{}l}
e\ ::\ Time
\end{tabular}]\haddockbegindoc
Euler's number in \haddockid{Time} format. Converted from the \haddocktt{Prelude}
 equivalent, which is \haddockid{Floating}.\par


\item[\begin{tabular}{@{}l}
(*{\char '136}*)\ ::\ Time\ ->\ Time\ ->\ Time
\end{tabular}]\haddockbegindoc
"Power of" function taking \haddockid{Time}s as arguments. Converts back
 and forth to \haddockid{Floating}, as it uses the \haddockid{**} operator, so
 it is prone to conversion errors.\par


\item[\begin{tabular}{@{}l}
pi\ ::\ Time
\end{tabular}]\haddockbegindoc
\haddockid{Time} representation of the number π. Rational
 representation with a precision of \haddocktt{0.000001}.\par


\item[\begin{tabular}{@{}l}
sin\ ::\ Time\ ->\ Time
\end{tabular}]\haddockbegindoc
Sine of \haddockid{Time}. Rational representation with a precision of
 \haddocktt{0.000001}.\par


\item[\begin{tabular}{@{}l}
cos\ ::\ Time\ ->\ Time
\end{tabular}]\haddockbegindoc
Cosine of \haddockid{Time}. Rational representation with a precision of
 \haddocktt{0.000001}.\par


\item[\begin{tabular}{@{}l}
tan\ ::\ Time\ ->\ Time
\end{tabular}]\haddockbegindoc
Tangent of \haddockid{Time}. Rational representation with a precision of
 \haddocktt{0.000001}.\par


\item[\begin{tabular}{@{}l}
atan\ ::\ Time\ ->\ Time
\end{tabular}]\haddockbegindoc
Arctangent of \haddockid{Time}. Rational representation with a precision
 of \haddocktt{0.000001}.\par


\item[\begin{tabular}{@{}l}
asin\ ::\ Time\ ->\ Time
\end{tabular}]\haddockbegindoc
Arcsine of \haddockid{Time}. Rational representation with a precision of
 \haddocktt{0.000001}.\par


\item[\begin{tabular}{@{}l}
acos\ ::\ Time\ ->\ Time
\end{tabular}]\haddockbegindoc
Arccosine of \haddockid{Time}. Rational representation with a precision
 of \haddocktt{0.000001}.\par


\item[\begin{tabular}{@{}l}
sqrt\ ::\ Time\ ->\ Time
\end{tabular}]\haddockbegindoc
Square root of \haddockid{Time}. Rational representation with a precision
 of \haddocktt{0.000001}.\par


\item[\begin{tabular}{@{}l}
exp\ ::\ Time\ ->\ Time
\end{tabular}]\haddockbegindoc
Exponent of \haddockid{Time}. Rational representation with a precision of
 \haddocktt{0.000001}.\par


\item[\begin{tabular}{@{}l}
cosh\ ::\ Time\ ->\ Time
\end{tabular}]\haddockbegindoc
Hyperbolic cosine of \haddockid{Time}. Rational representation with a
 precision of \haddocktt{0.000001}.\par


\item[\begin{tabular}{@{}l}
sinh\ ::\ Time\ ->\ Time
\end{tabular}]\haddockbegindoc
Hyperbolic sine of \haddockid{Time}. Rational representation with a
 precision of \haddocktt{0.000001}.\par


\item[\begin{tabular}{@{}l}
tanh\ ::\ Time\ ->\ Time
\end{tabular}]\haddockbegindoc
Hyperbolic tangent of \haddockid{Time}. Rational representation with a
 precision of \haddocktt{0.000001}.\par


\item[\begin{tabular}{@{}l}
atanh\ ::\ Time\ ->\ Time
\end{tabular}]\haddockbegindoc
Hyperbolic arctangent of \haddockid{Time}. Rational representation with a
 precision of \haddocktt{0.000001}.\par


\item[\begin{tabular}{@{}l}
asinh\ ::\ Time\ ->\ Time
\end{tabular}]\haddockbegindoc
Hyperbolic arcsine of \haddockid{Time}. Rational representation with a
 precision of \haddocktt{0.000001}.\par


\item[\begin{tabular}{@{}l}
acosh\ ::\ Time\ ->\ Time
\end{tabular}]\haddockbegindoc
Hyperbolic arccosine of \haddockid{Time}. Rational representation with a
 precision of \haddocktt{0.000001}.\par


\item[\begin{tabular}{@{}l}
log\ ::\ Time\ ->\ Time
\end{tabular}]\haddockbegindoc
Natural logarithm of \haddockid{Time}. Rational representation with a
 precision of \haddocktt{0.000001}.\par

\end{haddockdesc}