\section{Introduction}
\label{sec:forsyde-atom:introduction}

The ForSyDe (Formal System Design) methodology has been developed
with the objective to move system design to a higher level of
abstraction and to bridge the abstraction gap by transformational
design refinement. It targets the modelling and characterization of
cyber-physical systems inheriting the theory of
\href{http://ieeexplore.ieee.org/stamp/stamp.jsp?arnumber=736561}{Models of Computation (MoCs)},
providing both a correct-by-construction execution model and
analyzable entry point for further synthesis flows. For more information
about ForSyDe and its associated projects please consult the
\href{https://github.com/forsyde}{ForSyDe webpage}.\par
The \haddocktt{forsyde-atom} library is a shallow-embedded DSL implementing
the execution semantics of an \emph{atom-based approach} to ForSyDe. Its
purpose is to provide a modeling framework for cyber-physical
systems and to serve as a proof-of-concept for a future (non-strict
typed) DSL. Adhering to the formalism set by the
\href{http://www.diva-portal.org/smash/get/diva2:9340/FULLTEXT01.pdf}{PhD Thesis of Ingo Sander},
the current framework views systems as networks of processes
communicating through signals. The \emph{atom-based approach} to ForSyDe
adds some new important concepts:\par
\begin{itemize}
\item
the separation of concerns through semantically-independent
\textbf{\emph{layers}} of computation, behavior, synchronization and
structure. Each layer offers a different analyzable view of a
given system and is described using the concept of
\emph{higher-order functions}.\par

\item
the description of each layer as a network of primitive building
blocks called \textbf{\emph{atoms}}. Each atom embeds an undividable operation,
and wraps functions of lower layers with the semantics dictated by
a higher layer. They are described using the powerful concept of
\emph{applicative functors}.\par

\item
the complete \textbf{\emph{autonomy}} of atoms in relation to the patterns
they build. As such, process networks or constructors are nothing
but \emph{structural} ad-hoc compositions that aid in achieving complex
behaviors whereas the actual behavior is dictated by the atoms
alone. This is proved (for the time being) for the
\emph{Synchronization Layer} by implementing all MoCs as instances of
\emph{only one type class}.\par

\end{itemize}

While the host language limits the possibility of providing general
(e.g. proces) constructors due to type-strictness, the
\haddocktt{forsyde-atom} framework is by all means complete. In this sense the
user can create her own custom \emph{correct-by-design} constructors and
networks as compositions of the provided atoms and utilities. Also,
this haddock-generated page is organized both as an API
documentation and as a technical report to facilitate te use and
understanding of the formal principles behing the design process.\par

\subsection{Naming convention}
\label{sec:forsyde-atom:naming-convention}

All multi-parameter patterns and
utilities provided by the library API as higher-order functions are
named along the lines of \haddocktt{functionMN} where \haddocktt{M} represents the
number of \textbf{\emph{curried}} inputs (i.e. \haddocktt{a1\ ->\ a2\ ->\ ...\ ->\ aM}),
while \haddocktt{N} represents the number of \textbf{\emph{tupled}} outputs
(i.e. \haddocktt{(b1,b2,...,bN)}). To avoid repetition we only provide
documentation for functions with 2 inputs and 2 outputs
(i.e. \haddocktt{function22}), while the available ones are mentioned as a 
regex (i.e. \haddocktt{function[1-4][1-4]}). In case the provided functions do not suffice,
feel free to implement your own patterns following the example in \cref{sec:getting-started:making-your-own}.\par
