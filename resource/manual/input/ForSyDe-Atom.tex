\haddockmoduleheading{ForSyDe.Atom}
\label{module:ForSyDe.Atom}
\haddockbeginheader
{\haddockverb\begin{verbatim}
module ForSyDe.Atom (
    ExB(extend, (/.\), (/*\), (/&\), (/!\)),  Stream(NullS, (:-)), 
    module ForSyDe.Atom.MoC.Stream,  Time,  TimeStamp, 
    MoC(Fun, Ret, (-.-), (-*-), (-*), (-<-), (-&-)), 
    Skeleton((=.=), (=*=), (=\=), (=<<=), first, last), 
    module ForSyDe.Atom.Utility,  (||<)
  ) where\end{verbatim}}
\haddockendheader

The formal foundation upon which ForSyDe \cite{Sander04}
 defines its semantics is the \emph{tagged signal model}
 \cite{Lee98}.  This is a denotational
 framework introduced by Lee and Sangiovanni-Vincentelli as a common
 meta model for describing properties of concurrent systems in
 general terms as sets of possible behaviors. Systems are regarded
 as \emph{compositions} of \emph{processes} acting on \emph{signals} which are sets
 of \emph{tagged events}. Signals are characterized by a \emph{tag system}
 which determines causality between events, and could model time,
 precedence relationships, synchronization points, and other key
 properties. Based on how tag systems are defined, one can identify
 several \emph{Models of Computations (MoCs)} as classes of behaviors
 dictating the semantics of execution and concurrency in a network
 of processes.\par
These concepts are the supporting pillars of ForSyDe's philosophy,
 and state the purpose of the \haddocktt{forsyde-atom} library: it is supposed
 to be a modelling framework used as a proof-of-concept for the
 atom-based approach to cyber-physical systems
 \cite{Ungureanu17}. This approach
 extends the tagged signal model by systematically deconstructing
 processes to their basic semantics and recreating them using a
 minimal language of primitive building blocks called \emph{atoms}. It
 also tries to expand the scope of this model by exploiting more
 aspects than just timing, by adding primitives for parallelism,
 behavior, etc.\par
The API documentation is structured as follows: this page provides
 an overview of the general notions and concepts, gently introducing
 the separate modules and the motivation behind them. Each major
 module corresponds to a separate \emph{layer}
 \cite{Ungureanu17} implemented as a type
 class. The documentation pages for each layer and for each of their
 instances contains in-depth knowledge and examples, and can be
 accessed from the contents page or by following the links
 suggested. For more complex examples and tutorials follow the links
 in the \href{https://github.com/forsyde/forsyde-atom}{project web page}.\par
 \begin{mdframed}[style=reminder,frametitle=Reminder]Make sure to consult naming conventions in  \cref{sec:forsyde-atom:naming-convention} in order to interpret the names and type signatures correctly.\end{mdframed}\par

\subsection{The layered process model}
The \haddocktt{forsyde-atom} project is led by three main policies:\par
\begin{enumerate}
\item 
in order to cope with the complexity of cyber-physical systems
 (CPS) it tries to separate the concerns such as computation,
 timing, synchronization, parallelism, structure, behavior, etc.\par

\item 
in order to have a small, ideally minimal grammar to reason
 about systems correctness, it aims to provide primitive
 (indivisible) operators called \emph{atoms} as building blocks for
 independently developing complex aspects of a system's execution
 through means of composition or generalization.\par

\item 
in order to express complex behaviors with a minimal grammar,
 it decouples structure (composition) from meaning (semantics),
 the only semantics carriers being atoms. Thus complex behaviors
 can be described in terms of \emph{patterns of atoms}. Using ad-hoc
 polymorphism, atoms can be overloaded with different semantics
 triggered by the data type they input, whereas their composition
 is always the same.\par

\end{enumerate}
\begin{description}
\item[\haddocktt{atom}] the elementary (primitive, indivisible) constructor
 which embeds a set of semantics relevant for their respective
 layer (e.g. timing, behavioural, structural, etc.)
\item[\haddocktt{atom\ patterns}] meaningful compositions of atoms. They are
 provided as constructors which need to be properly instantiated
 in order to be used. We also use the term "pattern" to
 differentiate atom compositions as constructors from atoms as
 constructors.
\end{description}The first policy, i.e. the separation of concerns led to the
 so-called \emph{layered process model} which is reflected in the
 library by providing separate major modules associated with each
 layer. Layers as such are independent collections of entities for
 modeling different aspects of CPS. These aspects interact through
 means of higher-order functions, wrapping each other in as
 structured fashion in a way which can be visualized as below.\par
                 
 \haddockfig{fig/misc-layered-model.pdf}\par
                 Layers are implemented as type classes which imply:\par
                 \begin{itemize}
                 \item
                 \textbf{atoms} as function signatures belonging to the type class;\par
                 
                 \item
                 \textbf{patterns} which are compositions atoms, provided as constructors;\par
                 
                 \item
                 \textbf{data types} for all the classes of behaviors concerning the
 aspect described by the layer in question. These types
 instantiate the above type class and overload the atoms with
 semantics in accordance to the behavior described. For example,
 the \haddockid{MoC} layer is currently instantiated by
 types describing the \haddockid{CT},
 \haddockid{DE}, \haddockid{SY} and
 \haddockid{SDF} MoCs.\par
                 
                 \end{itemize}
                 In order to model interleaving aspects of CPS, layers interact
 with each other through means of higher order functions. As such,
 each layer describes some atoms as higher-order functions which
 take entities belonging to another layer as arguments.
 Intrinsically, the data types belonging to a layer may be wrapping
 types of other layers, as depicted in the \hyperref[layered-model]{figure}
 above. For a short comprehensive overview on layers, please refer
 to \cite{Ungureanu17}.\par
                 By convention, the first (innermost) layer is always the
 \emph{function layer} which describes arbitrary functions on data and
 expresses the system's functional aspects. In the following
 paragraphs we will give an overview of the "outer" layers
 currently implemented in \haddocktt{forsyde-atom}, which in comparison,
 express the extra-functional aspects of a system (timing,
 behavior, synchronization, and so on).\par
                 
\subsection{The Extended Behavior (ExB) Layer}
As seen in \hyperref[layered-model]{layered process model}, the extended
 behavior layer expands the set of possible behaviors implied by a
 layer (typically the function layer), by defining a set of
 symbols with \emph{known} semantics, and adding it to (i.e. wrapping)
 the pool of possible values or states.\par
While semantically the \haddockid{ExB} layer extends the value pool in
 order to express special events (e.g. error messages or even the
 complete absence of events), it practically provides an
 independent environment to model events with a default/known
 response, independently of the data path. These responses are
 particularly captured by atoms, thus enforcing the high-level
 separation of concerns between e.g. control and data paths.\par
This layer provides:\par
\begin{itemize}
\item
a set of extended behavior atoms defining the interfaces for
 the resolution and response functions, as part of the \haddockid{ExB} type
 class \emph{(see below)}.\par

\item
a library of function wrappers as specific atom patterns
 (\emph{Check the \haddocktt{ForSyDe.Atom.ExB} module for extensive}
 \emph{documentation}).\par

\item
a set of data types defining classes of behaviors and
 instantiating the \haddockid{ExB} type class (\emph{check the links in the}
 \emph{\hyperref[section.i:ExB]{instances} section for extensive documentation}).\par

\end{itemize}

\begin{haddockdesc}
\item[\begin{tabular}{@{}l}
class\ Functor\ b\ =>\ ExB\ b\ where
\end{tabular}]\haddockbegindoc
Class which defines the atoms for the extended behavior layer.\par
As its name suggests, this layer is extending the behavior of
 processes (or merely of functions if we ignore timing semantics
 completely), by expanding the domains of the wrapped layer
 (e.g. the set of values) with symbols having clearly-defined
 semantics (e.g. special events with known responses).\par
The types associated with this layer can simply be describes as:\par
\haddockeq{fig/eqs-exb-types.pdf}\par
where  \emph{α} is a base type and \emph{b} is the type extension,
 i.e. a set of symbols with clearly-defined semantics.\par
Extended behavior atoms are functions of these types, defined as
 interfaces in the \haddockid{ExB} type class.\par

\haddockpremethods{}\textbf{Methods}
\begin{haddockdesc}
\item[\begin{tabular}{@{}l}\haddockid{extend}\ ::\ a\ ->\ b\ a\ \end{tabular}]
\haddockbegindoc
Extends a value (from a layer below) with a set of symbols with known semantics, as described by a type instantiating this class.\par

\item[\begin{tabular}{@{}l}\haddockid{(/.\textbackslash)}\ ::\ (a\ ->\ a)\ ->\ b\ a\ ->\ b\ a\end{tabular}]
\haddockbegindoc
Basic functor operator. Lifts a function (from a layer below) into the domain of the extended behavior layer.\haddockeq{fig/eqs-exb-atom-func.pdf}\par

\item[\begin{tabular}{@{}l}\haddockid{(/*\textbackslash)}\ ::\ b\ (a\ ->\ a)\ ->\ b\ a\ ->\ b\ a\end{tabular}]
\haddockbegindoc
Applicative operator. Defines a res between two extended behavior symbols.\haddockeq{fig/eqs-exb-atom-app.pdf}\par

\item[\begin{tabular}{@{}l}\haddockid{(/\&\textbackslash)}\ ::\ b\ Bool\ ->\ b\ a\ ->\ b\ a\ \end{tabular}]
\haddockbegindoc
Predicate operator. Generates a defined behavior based on an extended Boolean predicate.\haddockeq{fig/eqs-exb-atom-phi.pdf}\par

\item[\begin{tabular}{@{}l}\haddockid{(/"!\textbackslash)}\ ::\ a\ ->\ b\ a\ ->\ a\ \end{tabular}]
\haddockbegindoc
Degrade operator. Degrades a behavior-extended value into a non-extended one (from a layer below), based on a kernel value. Used also to throw exceptions.\haddockeq{fig/eqs-exb-atom-deg.pdf}\par

\end{haddockdesc}


\item[\begin{tabular}{@{}l}
instance\ ExB\ AbstExt
\end{tabular}]\haddockbegindoc
Implements the absent semantics of the extended behavior atoms.\par

\end{haddockdesc}
\subsection{The Model of Computation (MoC) Layer}
This layer represents a major part of the \haddocktt{forsyde-atom}
 library and is concerned in modeling the timing aspects of
 CPS. While its foundations have been layered in the classical
 ForSyDe \cite{Sander04}, it is mainly inspired from
 \cite{Lee98} an it tries to follow the tagged signal model as
 closely as it is permitted by the host language, and with the
 adaptations require by the atom approach.\par
Although a short introduction of the tagged signal model has been
 written in the introduction of this documentation, we feel
 obliged to provide a primer in the classical ForSyDe theory in
 order to understand how everything fits together.\par

\subsubsection{Signals}
\cite{Lee98} defines signals as (ordered) sets of events
 where each event is composed of a tag \emph{T} and a value
 \emph{V}. Similarly, in ForSyDe a signal is defined as a (partially or
 totally) \emph{ordered sequence} of events that enables processes to
 communicate and synchronize. Sequencing might infer an implicit
 order of events, but more importantly it determines an order of
 evaluation, which is a key piece of a simulation engine.\par
\haddockeq{fig/misc-tagged-signal.pdf}\par
In ForSyDe-Atom, sequencing is achieved using the \haddockid{Stream} data
 type, inspired from \cite{Reekie95}. In ForSyDe-Atom,
 signals are streams that carry \emph{events}, where each type of event
 is identified by a type constructor which defines its tag
 system. In other words, we can state that through its tag system,
 a signal is \emph{bound} to a MoC.\par

\begin{haddockdesc}
\item[\begin{tabular}{@{}l}
data\ Stream\ e
\end{tabular}]\haddockbegindoc
\haddockbeginconstrs
\haddockdecltt{=} & \haddockdecltt{NullS} & terminates a signal \\
\haddockdecltt{|} & \haddockdecltt{e (:-) (Stream e)} & the default constructor appends an\\
 &&event to the head of the stream \\
\end{tabulary}\par

Defines a stream of events, encapsulating them in a structure
 isomorphic to an infinite list \cite{Bird87},
 thus all properties of lists may also be applied to
 \haddockid{Stream}s. While, in combination with lazy evaluation, it is
 possible to create and simulate infinite signals, we need to ensure
 that the first/previous event is always fully evaluated. This can
 be translated into the following rule:\par
\begin{description}
\item[zero-delay feedbacks] are forbidden, due to un-evaluated
 self-referential calls. In a feedback loop, there always has to be
 enough events to ensure the data flow.
\end{description}This rule imposes that the stream of data is uninterrupted in order
 to have an evaluatable kernel every time a new event is produced
 (i.e. to avoid deadlocks). Thus we can add the rule:\par
                 \begin{description}
                 \item[cleaning of output events] is also forbidden.  In other words, for
 each new input at any instant in time, a process must react with
 \emph{at least} one output event.
                 \end{description}

\item[\begin{tabular}{@{}l}
instance\ Functor\ Stream
\end{tabular}]\haddockbegindoc
allows for the mapping of an arbitrary function \haddocktt{(a\ ->\ b)} upon
 all the events of a \haddocktt{(Stream\ a)}.\par


\item[\begin{tabular}{@{}l}
instance\ Applicative\ Stream
\end{tabular}]\haddockbegindoc
enables the \haddockid{Stream} to behave like a \haddockid{ZipList}\par


\item[\begin{tabular}{@{}l}
instance\ Foldable\ Stream
\end{tabular}]\haddockbegindoc
provides folding functions useful for implementing utilities, such as \haddockid{length}.\par


\item[\begin{tabular}{@{}l}
instance\ Read\ a\ =>\ Read\ (Stream\ a)
\end{tabular}]\haddockbegindoc
signal \haddocktt{(1\ :-\ 2\ :-\ NullS)} is read using the string \haddocktt{"{\char '173}1,2{\char '175}"}.\par


\item[\begin{tabular}{@{}l}
instance\ Show\ a\ =>\ Show\ (Stream\ a)
\end{tabular}]\haddockbegindoc
signal \haddocktt{(1\ :-\ 2\ :-\ NullS)} is represented as \haddocktt{{\char '173}1,2{\char '175}}.\par


\item[\begin{tabular}{@{}l}
instance\ Plottable\ a\ =>\ Plot\ (Signal\ a)
\end{tabular}]\haddockbegindoc
\haddockid{SY} signals.\par


\item[\begin{tabular}{@{}l}
instance\ Plottable\ a\ =>\ Plot\ (Signal\ a)
\end{tabular}]\haddockbegindoc
\haddockid{SDF} signals.\par


\item[\begin{tabular}{@{}l}
instance\ Plottable\ a\ =>\ Plot\ (Signal\ a)
\end{tabular}]\haddockbegindoc
\haddockid{DE} signals.\par


\item[\begin{tabular}{@{}l}
instance\ Plottable\ a\ =>\ Plot\ (Signal\ a)
\end{tabular}]\haddockbegindoc
\haddockid{CT} signals.\par

\end{haddockdesc}
For extended documentation and a list of all utilities associated
 with the \haddockid{Stream} type you can consult:\par

\begin{haddockdesc}
\item[\begin{tabular}{@{}l}
module\ ForSyDe.Atom.MoC.Stream
\end{tabular}]
\end{haddockdesc}
\subsubsection{Processes}
As described in \cite{Lee98}, processes are either "set of
 possible behaviors" of signals or "relations" between multiple
 signals. One can describe complex systems by composing processes,
 which in this case is interpreted as the "intersection of the
 behaviors of each of the processes being involved".\par
\begin{description}
\item[monotonicity] In order to ensure causal order and determinancy,
 processes need to be \emph{monotonic} \cite{Lee98}. A signal's
 tags (if explicit) \emph{must be} a partial or total order and all tag
 alterations must be monotonic.
\end{description}ForSyDe inherits this definition with respect to a functional
 view, thus a \textbf{process} \emph{p} is a functional mapping over (the
 history of) signals. A process can \emph{only} be instantiated using a
 \textbf{process constructor} \emph{pc}, which is a higher order function
 embedding MoC semantics and/or a specific composition, but
 lacking functionality.\par
                 
 \haddockeq{fig/misc-process.pdf}\par
                 Since processes are functions, process composition is equivalent
 to function composition. This means that composing two processes
 \haddocktt{p1} and \haddocktt{p2} grants the process \haddocktt{p2\ .\ p1}\par
\begin{quote}
  {\haddockverb\begin{verbatim}
p1      :: Signal α -> Signal β
p2      :: Signal β -> Signal γ
p2 . p1 :: Signal α -> Signal γ\end{verbatim}}
                 \end{quote}
                 This implies that there is a signal \haddocktt{Signal\ $\beta$ } that
  "streams" the result from \haddocktt{p1} to \haddocktt{p2}, as suggested in the
  drawing:\par
                 \haddockeq{fig/misc-ser-composition.pdf}\par
                 \textbf{Process networks} describe ForSyDe systems in terms of
 compositions of processes and originate from Reekie's process
 nets \cite{Reekie95}. A process network is a process itself,
 i.e. function from signal(s) to signal(s). The composition above
 \haddocktt{p2\ .\ p1\ } can also be regarded as a process network.\par
                 In ForSyDe-Atom atoms can be regarded as process constructors as
 their instantiations are functions on signals of events.
 Instantiations of atom patterns are the exact equivalent of
 process networks, which themselves are also processes, depending
 on the level of abstraction you are working with (hierarchical
 blocks vs. flat structures).\par
                 To understand the versatility of composition and partial
 application in building process constructors, consider the
 example above where composition of two processes infers a signal
 between them. This mechanism also works when composing
 constructors (un-instantiated atoms), which yields another
 constructor. By instantiating (fully applying) the new
 constructor we obtain a process network equivalent to the
 composition of the respective primitive processes obtained by
 instantiating (fully applying) the component atoms, like in the
 example below:\par
                 \haddockeq{fig/misc-process-constructor.pdf}\par
                 Now if we visualize process networks as graphs, where processes
 are nodes and signals are edges, a meaningful process composition
 could be regarded as graph patterns. Therefore it is safe to
 associate process constructors as patterns in process networks.\par
                 
\subsubsection{Models of Computation}
As mentioned in the introduction, \emph{MoCs} are classes of behaviors
 dictating the semantics of execution and concurrency in a network
 of processes. Based on the definitions of their tag systems
 ForSyDe identifies MoCs as:\par
\begin{enumerate}
\item 
\emph{timed} where \emph{T} is a totally ordered set and \emph{t} express the
 notion of physical time (e.g. continuous time
 \haddockid{CT}, discrete event
 \haddockid{DE}) or precedence (e.g. synchronous
 \haddockid{SY});\par

\item 
\emph{untimed}, where \emph{T} is a partially ordered set and \emph{t} is
 expressed in terms of constraints on the tags in signals
 (e.g. dataflow, synchronous data flow
 \haddockid{SDF}).\par

\end{enumerate}
As concerning MoCs, ForSyDe implements the execution semantics
 \emph{through process constructors}, abstracting the timing model and
 inferring a schedule of the process network. In ForSyDe-Atom all
 atoms embed operating semantics dictated by a certain MoC and are
 side-effect-free. This ensures the functional correctness of a
 system even from early design stages.\par

\subsubsection{Representing Time}
For explicit time representation, ForSyDe-atom provides two
 distinct data types.\par

\begin{haddockdesc}
\item[\begin{tabular}{@{}l}
type\ Time\ =\ Rational
\end{tabular}]\haddockbegindoc
Type alias for the type to represent metric (continuous)
 time. Underneath we use \haddockid{Rational} that is able to represent any
 \emph{t} between \emph{t₁} < \emph{t₂} ∈ \emph{T}.\par


\item[\begin{tabular}{@{}l}
type\ TimeStamp\ =\ DiffTime
\end{tabular}]\haddockbegindoc
Alias for the type representing discrete time. It is inherently
 quantizable, the quantum being a picosecond (10⁻¹²
 seconds), thus it can be considered order-isomorphic with a set of
 integers, i.e. between any two timestamps there is a finite number
 of timestamps. Moreover, a timestamp can be easily translated into
 a rational number representing fractions of a second, so the
 conversion between timestamps (discrete time) and rationals
 (analog/continuous time) is straightforward.\par
This type is used in the explicit tags of the
 \haddockid{DE} MoC (and subsequently the discrete event
 evaluation engine for simulating the \haddockid{CT} MoC).\par

\end{haddockdesc}
\subsubsection{MoC Layer Overview}
This layer consists of:\par
\begin{itemize}
\item
4 atoms as infix operators, implemented as methods of the type
 class \haddockid{MoC}. Since each MoC is determined by its tag system, we
 expose this 
 which are instances of this class. Thus an event's type will
 trigger an atom to behave in accordance to its associated MoC.\par

\item
a library of meaningful atom patterns as process constructors.
 (\emph{Check the \haddocktt{ForSyDe.Atom.MoC} module for extensive}
 \emph{documentation}).\par

\item
a set of data types defining tag systems through the structure
 of events (i.e. \emph{T} × \emph{V}). They are instances of the \haddockid{MoC}
 type class and define the rules of execution that will trigger an
 atom to behave in accordance to its associated MoC. For each
 supported MoC, \haddocktt{forsyde-atom} provides a module which defines the
 signal (event) type, but also a set of utilities and process
 constructors as specific instantiations of atom patterns.
 (\emph{Check the links in the \hyperref[section.i:MoC]{instances} section for}
 \emph{extensive documentation}).\par

\end{itemize}

\begin{haddockdesc}
\item[\begin{tabular}{@{}l}
class\ Applicative\ e\ =>\ MoC\ e\ where
\end{tabular}]\haddockbegindoc
This is a type class defining interfaces for the MoC layer
 atoms. Each model of computation exposes its tag system through a
 unique event constructor as an instance of this class, which
 defines \emph{T} × \emph{V}.\par
 To express all possible MoCs which can be described using
 a \emph{tagged} \emph{signal} \emph{model} we need to capture the most general
 form of their atoms. Recall that all atoms in the layered framework
 are represented as higher-order functions on structured types
 (instances of this class), taking functions of other (lower) layers
 as arguments. While this principle stands also for this layer, the
 functions taken as arguments need to be formatted for each MoC in
 particular in order to capture additional information, which we can
 call in general terms as the \emph{execution context}.\par
One typical example of additional information is the consumption
 and production rates of for data flow MoCs (e.g. SDF). In this case
 the passed functions are defined over "partitions" of events,
 i.e. groupings of events with the same partial order in relation
 to, for example, a process firing. The formal description of such a
 "formatted function" taken as argument by a MoC entity is:\par
\haddockeq{fig/eqs-moc-atom-formatted-func.pdf}\par
where \emph{a} and \emph{b} might be Cartesian products of different types,
 corresponding to how many signals the constructor is applied to or
 how many signals it yields, and each type is expressed as:\par
\haddockeq{fig/eqs-moc-atom-formatted-arg.pdf}\par
While, as you can see above, the execution context can be extracted
 from the type information, working with type-level parameters is
 not a trivial task in Haskell, especially if we want to describe a
 general and extensible type class. This is why we have chosen a
 pragmatic approach in implementing the \haddockid{MoC} class:\par
\begin{itemize}
\item
any (possible) Cartesian product of \emph{α} is represented using
 a recursive type, namely a list {\char 91}\emph{α}{\char 93}.\par

\item
as the execution context cannot (or can hardly) be extracted from
 the recursive type, in the most general case we pass both context
 \emph{and} argument as a pair (see each instance in particular). To aid
 in pairing contexts with each argument in a function, the \haddocktt{ctxt}
 utilities are provided (see \haddockid{ctxt22}).\par

\item
this artifice was masked using the generic type families \haddockid{Fun}
 and \haddocktt{Res}. \par

\end{itemize}

\haddockpremethods{}\textbf{Methods}
\begin{haddockdesc}
\item[\begin{tabular}{@{}l}\ \haddockid{Fun}\ e\ a\ b\ \end{tabular}]
\haddockbegindoc
This is a type family alias for a context-bound function passed as an argument to a MoC atom. In the most simple case it can be regarded as an enhanced -> type operator. While hiding the explicit definition of arguments, this implementation choice certainly has its advantages in avoiding unnecessary or redundant type constructors (see version 0.1.1 and prior). Aliases are replaced at compile time, thus not affecting run-time performance.\haddockeq{fig/eqs-moc-atom-function.pdf}\par

\item[\begin{tabular}{@{}l}\ \haddockid{Ret}\ e\ b\ \end{tabular}]
\haddockbegindoc
As with \haddockid{Fun}, this alias hides a context-bound value (e.g. function return). Although the definition seems to be redundant with \haddockid{Fun}, this alias is needed for utilities to recreate clean types again (see \haddockid{-*}).\haddockeq{fig/eqs-moc-atom-result.pdf}\par

\item[\begin{tabular}{@{}l}\haddockid{(-.-)}\ ::\ Fun\ e\ a\ b\ ->\ Stream\ (e\ a)\ ->\ Stream\ (e\ b)\ \end{tabular}]
\haddockbegindoc
This atom is mapping a function on values (in the presence of a context) to a signal, i.e. stream of tagged events. As ForSyDe deals with \emph{determinate}, \emph{functional} processes, this atom defines the (only) \emph{behavior} of a process in rapport to one input signal  \cite{Lee98}.\haddockeq{fig/eqs-moc-atom-dot.pdf}\par

\item[\begin{tabular}{@{}l}\haddockid{(-*-)}\ ::\ Stream\ (e\ (Fun\ e\ a\ b))\ ->\ Stream\ (e\ a)\ ->\ Stream\ (e\ b)\ \end{tabular}]
\haddockbegindoc
This atom synchronizes two signals, one carrying functions on values (in the presence of a context), and the other containing values, during which it applies the former on the latter. As concerning the process created, this atom defines a \emph{relation} between two signals  \cite{Lee98}.\haddockeq{fig/eqs-moc-atom-star.pdf}\par

\item[\begin{tabular}{@{}l}\haddockid{(-*)}\ ::\ Stream\ (e\ (Ret\ e\ b))\ ->\ Stream\ (e\ b)\ \end{tabular}]
\haddockbegindoc
Artificial \emph{utility} which drops the context and/or partitioning yielding a clean signal type.\haddockeq{fig/eqs-moc-atom-post.pdf}\par

\item[\begin{tabular}{@{}l}\haddockid{(-<-)}\ ::\ Stream\ (e\ a)\ ->\ Stream\ (e\ a)\ ->\ Stream\ (e\ a)\ \end{tabular}]
\haddockbegindoc
This atom appends a (partition of) events at the beginning of a signal. This atom is necessary to ensure \emph{complete partial order} of a signal and assures the \emph{least upper bound} necessary for example in the evaluation of feedback loops  \cite{Lee98}.\haddockeq{fig/eqs-moc-atom-pre.pdf}Notice the difference between the formal and the implemented type signatures. In the implementation the value/partition is wrapped inside an event type to enable smooth composition. You might also notice the type for the initial event(s) as being wrapped inside a signal constructor. This allows defining an DSL for this layer which is centered around signals exclusively, while also enabling to define atoms as homomorphisms to certain extent  \cite{Bird97}. Certain MoCs might have additional constraints on the first operand to be finite.\par

\item[\begin{tabular}{@{}l}\haddockid{(-\&-)}\ ::\ Stream\ (e\ a)\ ->\ Stream\ (e\ a)\ ->\ Stream\ (e\ a)\ \end{tabular}]
\haddockbegindoc
This atom allows the manipulation of tags in a signal in a restrictive way which preserves \emph{monotonicity} and \emph{continuity} in a process  \cite{Lee98}, namely by phase-shifting all tags in a signal with the appropriate metric corresponding to each MoC. Thus it preserves the characteristic function intact  \cite{Sander04}.\haddockeq{fig/eqs-moc-atom-phi.pdf}As with the \haddockid{-<-} atom, we can justify the type signature for smooth composition and the definition of atoms as homomorphisms to certain extent. This in turn allows the interpretation of the \haddockid{-\haddockpremethods{}\textbf{Methods}-} operator as aligning the phases of two signals: the second operand is aligned based on the first.\par

\end{haddockdesc}


\item[\begin{tabular}{@{}l}
instance\ MoC\ SY
\end{tabular}]\haddockbegindoc
Implenents the execution and synchronization semantics for the SY
 MoC through its atoms.\par


\item[\begin{tabular}{@{}l}
instance\ MoC\ SDF
\end{tabular}]\haddockbegindoc
Implenents the SDF semantics for the MoC atoms\par


\item[\begin{tabular}{@{}l}
instance\ MoC\ DE
\end{tabular}]\haddockbegindoc
Implenents the execution and synchronization semantics for the DE
 MoC through its atoms.\par


\item[\begin{tabular}{@{}l}
instance\ MoC\ CT
\end{tabular}]\haddockbegindoc
Implenents the execution and synchronization semantics for the CT
 MoC through its atoms.\par

\end{haddockdesc}
\subsection{The Skeleton Layer}
The skeleton layer describes recursive and regular composition
 of processes which expose inherent potential for parallelism. As
 such, it wraps lower layer entities (i.e. processes, signals),
 into regular structures called \emph{categorical types}. Most of the
 ground work for this layer is based on the categorical type
 theory \cite{Bird97}, which enable the description of
 algorithmic skeletons as high-level constructs encapsulating
 parallelism and communication with an associated cost complexity.\par
This layer provides:\par
\begin{itemize}
\item
3 atoms as infix operators which, as demonstrated in \cite{Bird97}
 and \cite{Skillicorn05}, are enough to describe \emph{all}
 algorithmic skeletons.\par

\item
a library of generic skeletons as specific atom patterns.
 (\emph{Check the \haddocktt{ForSyDe.Atom.Skeleton} module for extensive}
 \emph{documentation}).\par

\item
a set of different categorical types which implement these
 atoms, as instances of the \haddockid{Skeleton} type class. These types
 provide additional skeletons patterns of atoms which takes as
 arguments their own type constructors.
 (\emph{Check the links in the \hyperref[section.i:Skeleton]{instances} section for}
 \emph{extensive documentation}).\par

\end{itemize}

\begin{haddockdesc}
\item[\begin{tabular}{@{}l}
class\ Functor\ c\ =>\ Skeleton\ c\ where
\end{tabular}]\haddockbegindoc
Class containing all the Skeleton layer atoms.\par
This class is instantiated by a set of categorical types,
 i.e. types which describe an inherent potential for being evaluated
 in parallel. Skeletons are patterns from this layer. When skeletons
 take as arguments entities from the MoC layer (i.e. processes), the
 results themselves are parallel process networks which describe
 systems with an inherent potential to be implemented on parallel
 platforms. All skeletons can be described as composition of the
 three atoms below (\haddockid{=<<=} being just a specific instantiation of
 \haddockid{={\char '134}=}). This possible due to an existing theorem in the categorical
 type theory, also called the Bird-Merteens formalism
 \cite{Bird97}:\par
\par
\begin{description}
\item[factorization] A function on a categorical type is an algorithmic
 skeleton (i.e. catamorphism) \emph{iff} it can be represented in a
 factorized form, i.e. as a \emph{map} composed with a \emph{reduce}.
\end{description}Consequently, most of the skeletons for the implemented categorical
 types are described in their factorized form, taking as arguments
 either:\par
                 \begin{itemize}
                 \item
                 type constructors or functions derived from type constructors\par
                 
                 \item
                 processes, i.e. MoC layer entities\par
                 
                 \end{itemize}
                 Most of the ground-work on algorithmic skeletons on which this
 module is founded has been laid in \cite{Bird97},
 \cite{Skillicorn05} and it founds many
 of the frameworks collected in \cite{Gorlatch03}.\par
                 
\haddockpremethods{}\textbf{Methods}
\begin{haddockdesc}
\item[\begin{tabular}{@{}l}\haddockid{(=.=)}\ ::\ (a\ ->\ b)\ ->\ c\ a\ ->\ c\ b\ \end{tabular}]
\haddockbegindoc
Atom which maps a function on each element of a structure (i.e. categorical type), defined as:\haddockeq{fig/eqs-skel-atom-dot.pdf}\haddockid{=.=} together with \haddockid{=*=} form the map pattern.\par

\item[\begin{tabular}{@{}l}\haddockid{(=*=)}\ ::\ c\ (a\ ->\ b)\ ->\ c\ a\ ->\ c\ b\ \end{tabular}]
\haddockbegindoc
Atom which applies the functions contained by as structure (i.e. categorical type), on the elements of another structure, defined as:\haddockeq{fig/eqs-skel-atom-star.pdf}\haddockid{=.=} together with \haddockid{=*=} form the map pattern.\par

\item[\begin{tabular}{@{}l}\haddockid{(=\textbackslash=)}\ ::\ (a\ ->\ a\ ->\ a)\ ->\ c\ a\ ->\ a\ \end{tabular}]
\haddockbegindoc
Atom which reduces a structure to an element based on an \emph{associative} function, defined as:\haddockeq{fig/eqs-skel-atom-red.pdf}\par

\item[\begin{tabular}{@{}l}\haddockid{(=<<=)}\ \end{tabular}]
\haddockbegindoc
Skeleton which \emph{pipes} an element through all the functions contained by a structure. N.B.: this is not an atom. It has an implicit definition which might be augmented by instances of this class to include edge cases.\haddockeq{fig/eqs-skel-pattern-pipe.pdf}As the composition operation is not associative, we cannot treat pipe as a true reduction. Alas, it can still be exploited in parallel since it exposes another type of parallelism: time parallelism.\par

\item[\begin{tabular}{@{}l}\haddockid{first}\ ::\ c\ a\ ->\ a\ \end{tabular}]
\haddockbegindoc
Returns the first element in a structure.N.B.: this is not an atom. It has an implicit definition which might be replaced by instances of this class with a more efficient implementation.\haddockeq{fig/eqs-skel-pattern-first.pdf}\par

\item[\begin{tabular}{@{}l}\haddockid{last}\ ::\ c\ a\ ->\ a\ \end{tabular}]
\haddockbegindoc
Returns the last element in a structure.N.B.: this is not an atom. It has an implicit definition which might be replaced by instances of this class with a more efficient implementation.\haddockeq{fig/eqs-skel-pattern-last.pdf}\par

\end{haddockdesc}


\item[\begin{tabular}{@{}l}
instance\ Skeleton\ Vector
\end{tabular}]\haddockbegindoc
Ensures that \haddockid{Vector} is a structure associated with the Skeleton Layer.\par

\end{haddockdesc}

\subsection{Utilities}
The \haddockid{Atom} module exports a set of utility functions,
 mainly for aiding the designer to avoid working with zipped
 tuples which might pollute the design. Utilities are function
 without any semantical value (thus not considered atoms). They
 operate on and might alter the \emph{structure} of some datum, but it
 does not affect its state.\par
For a list of all the provided utilities, please consult the
 following module:\par

\begin{haddockdesc}
\item[\begin{tabular}{@{}l}
module\ ForSyDe.Atom.Utility
\end{tabular}]
\end{haddockdesc}
Among the most useful utilities we mentions the \haddocktt{unzip}
 function. Recall that in all our definitions for patterns, they
 were expressed in the most general form as functions from \emph{n}-ary
 Cartesian products to \emph{m}-ary Cartesian products. While partial
 application provides a versatile mechanism that can translate
 n-ary inputs into curried arguments (which is very powerful in
 combination with an applicative style), we cannot do so for
 return types. For the latter we must rely on tuples. But working
 with tuples of data wrapped in several layers of structures
 becomes extremely cumbersome. Take for example the case of a
 process constructed with \emph{pc} in equation (1) below. Using only
 the provided atoms to implement \emph{pc} would give us a process
 which returns only one signal of a tuple and not, as we would
 like, a tuple of signals of events.\par
\haddockeq{fig/misc-unzip.pdf}\par
Therefore, by implementing all data types associated with signals
 and events as instances of \haddockid{Functor}, we were able to provide a
 (set of) \emph{unzip} utility functions defined as in equation (2)
 above, in the \haddocktt{ForSyDe.Atom.Utility} module.  Mind that we call
 \emph{unzip} a utility and not an atom, since it has no
 synchronization nor behavior semantic. It just conveniently
 "lifts" the wrapped tuples in order to create "collections" of
 events and signals, and it is imposed by the mechanisms of the
 type system in the host language.\par

\begin{haddockdesc}
\item[\begin{tabular}{@{}l}
(||<)\ ::\ (Functor\ f1,\ Functor\ f2)\ =>\\\ \ \ \ \ \ \ \ \ f1\ (f2\ (a1,\ a2))\ ->\ (f1\ (f2\ a1),\ f1\ (f2\ a2))
\end{tabular}]\haddockbegindoc
This set of utility functions "unzip" nested n-tuples, provided
 as postfix operators. They are crucial for reconstructing data
 types from higher-order functions which input functions with
 multiple outputs. It relies on the nested types being instances of
 \haddockid{Functor}.\par
The operator convention is \haddocktt{(|+<+)}, where the number of \haddocktt{|}
 represent the number of layers the n-tuple is lifted, while the
 number of \haddocktt{<} + 1 is the order \emph{n} of the n-tuple.\par
\haddocktt{ForSyDe.Atom.Utility} exports the constructors below. Please
 follow the examples in the source code if they do not suffice:\par
\begin{code}
   |<,    |<<,    |<<<,    |<<<<,    |<<<<<,    |<<<<<<,    |<<<<<<<,    |<<<<<<<<,
  ||<,   ||<<,   ||<<<,   ||<<<<,   ||<<<<<,   ||<<<<<<,   ||<<<<<<<,   ||<<<<<<<<,
 |||<,  |||<<,  |||<<<,  |||<<<<,  |||<<<<<,  |||<<<<<<,  |||<<<<<<<,  |||<<<<<<<<,  
||||<, ||||<<, ||||<<<, ||||<<<<, ||||<<<<<, ||||<<<<<<, ||||<<<<<<<, ||||<<<<<<<<, \end{code}
Example:\par
\begin{interactive}
λ> :set -XPostfixOperators
λ> ([Just (1,2,3), Nothing, Just (4,5,6)] ||<<)
([Just 1,Nothing,Just 4],[Just 2,Nothing,Just 5],[Just 3,Nothing,Just 6])

\end{interactive}
\end{haddockdesc}
