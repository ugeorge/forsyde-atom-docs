\documentclass{standalone} 
\usepackage[tikz,plot,math]{forsyde}
\usepackage{forsyde-atom-docs}

\pgfkeys{/signal plot keys/.cd,
  initial/.style = {
    name=sigplot,
    label pos=mid, 
    step=5,
    xscale=.8,
    yscale=.5,
    xshift=0pt,
    yshift=0pt,
    signal sep=1,
    at={0,0},
    anchor=mid,
  }
}%

\begin{document}

\begin{docimage}{example}
  \begin{tikzpicture}
    \standard[process, ni=2, no=1, moc=ct, f={$+$}, type=comb](p1){};
    \begin{signalsCT}[timestamps=3.14, grid=3.14, inputs={p1.west}]{10}
      \signalCT*[outline,ordinate=0,ymin=-1,ymax=1]{%
        data/ct-examp-i1.flx}
      \signalCT*[outline,ymin=0,ymax=1]{%
        data/ct-examp-i2.flx}
    \end{signalsCT}
    \path (sigplot.e1) edge[s,-|-,->] (p1.w1);
    \path (sigplot.e2) edge[s,-|-,->] (p1.w2);
    \begin{signalsCT}[timestamps=3.14, grid=3.14, outputs={p1.east}]{10}
      \signalCT*[outline,ymin=0,ymax=1]{%
        data/ct-examp-o1.flx}
    \end{signalsCT}
    \signal[-|-] (p1.e1) -> (sigplot.w1);
  \end{tikzpicture}
\end{docimage} 


\begin{docimage}{pattern-delay}
  \begin{tikzpicture}
    \standard[process, moc=ct, f={$2$;$const\ 0$}, type=delay](p1){};
    \begin{signalsCT}[timestamps=3.14, grid=3.14, inputs={p1.west}]{10}
      \signalCT*[outline,ordinate=0,ymin=-1,ymax=1]{%
        data/ct-delay-i1.flx}
    \end{signalsCT}
    \path (sigplot.e1) edge[s,-|-,->] (p1.w1);
    \begin{signalsCT}[timestamps=3.14, grid=3.14, outputs={p1.east}]{10}
      \signalCT*[outline,ordinate=0,ymin=-1,ymax=1]{%
        data/ct-delay-o1.flx}
    \end{signalsCT}
    \signal[-|-] (p1.e1) -> (sigplot.w1);    
  \end{tikzpicture}
\end{docimage} 

\begin{docimage}{pattern-delayp}
  \begin{tikzpicture}
    \standard[process, ni=2, moc=ct, type=delay'](p1){};
    \begin{signalsCT}[timestamps=3.14, grid=3.14, inputs={p1.west}]{10}
      \signalCT*[outline,ordinate=0,ymin=-1,ymax=1]{%
        data/ct-delayp-i1.flx}
      \signalCT*[outline,ymin=-1,ymax=1]{%
        data/ct-delayp-i2.flx}
    \end{signalsCT}
    \path (sigplot.e1) edge[s,-|-,->] (p1.w1);
    \path (sigplot.e2) edge[s,-|-,->] (p1.w2);
    \begin{signalsCT}[timestamps=3.14, grid=3.14, outputs={p1.east}]{10}
      \signalCT*[outline,ordinate=0,ymin=-1,ymax=1]{%
        data/ct-delayp-o1.flx}
    \end{signalsCT}
    \signal[-|-] (p1.e1) -> (sigplot.w1);    
  \end{tikzpicture}
\end{docimage} 

\begin{docimage}{pattern-comb}
  \begin{tikzpicture}
    \standard[process, ni=2, no=2, moc=ct, f={$\lambda\ a\ b\rightarrow (a+b,a*b)$}, type=comb](p1){};
    \begin{signalsCT}[timestamps=3.14, grid=3.14, inputs={p1.west}]{10}
      \signalCT*[outline,ordinate=0,ymin=-1,ymax=1]{%
        data/ct-comb-i1.flx}
      \signalCT*[outline,ymin=0,ymax=1]{%
        data/ct-comb-i2.flx}
    \end{signalsCT}
    \path (sigplot.e1) edge[s,-|-,->] (p1.w1);
    \path (sigplot.e2) edge[s,-|-,->] (p1.w2);
    \begin{signalsCT}[timestamps=3.14, grid=3.14, outputs={p1.east}]{10}
      \signalCT*[outline,ordinate=0,ymin=-1,ymax=1]{%
        data/ct-comb-o1.flx}
      \signalCT*[outline,ordinate=0,ymin=-1,ymax=1]{%
        data/ct-comb-o2.flx}
    \end{signalsCT}
    \signal[-|-] (p1.e1) -> (sigplot.w1);    
    \signal[-|-] (p1.e2) -> (sigplot.w2);
  \end{tikzpicture}
\end{docimage} 

\begin{docimage}{pattern-constant}
  \begin{tikzpicture}
    \standard[process, moc=ct, f={$2$}, type=constant](p1){};
    \begin{signalsCT}[timestamps=3.14, grid=3.14, outputs={p1.east}]{10}
      \signalCT*[outline,ymin=0,ymax=4]{%
        data/ct-constant-o1.flx}
    \end{signalsCT}
    \signal[-|-] (p1.e1) -> (sigplot.w1);    
  \end{tikzpicture}
\end{docimage} 

\begin{docimage}{pattern-infinite}
  \begin{tikzpicture}
    \standard[process, no=2, moc=ct, f={$\noexpand\sin, \noexpand\cos$}, type=infinite](p1){};
    \begin{signalsCT}[timestamps=3.14, grid=3.14, outputs={p1.east}]{10}
      \signalCT*[outline,ordinate=0,ymin=-1,ymax=1]{%
        data/ct-infinite-o1.flx}
      \signalCT*[outline,ordinate=0,ymin=-1,ymax=1]{%
        data/ct-infinite-o2.flx}
    \end{signalsCT}
    \signal[-|-] (p1.e1) -> (sigplot.w1);    
    \signal[-|-] (p1.e2) -> (sigplot.w2);    
  \end{tikzpicture}
\end{docimage}

\begin{docimage}{pattern-generate}
  \begin{tikzpicture}
    \standard[process, moc=ct, f={$\noexpand\mathit{osc}$;$\noexpand\pi, \noexpand\mathit{const}(0)$}, type=generate](p1){};
    \begin{signalsCT}[timestamps=3.14, grid=3.14, outputs={p1.east}]{10}
      \signalCT*[outline,ymin=0,ymax=1]{%
        data/ct-generate-o1.flx}
    \end{signalsCT}
    \signal[-|-] (p1.e1) -> (sigplot.w1);    
  \end{tikzpicture}
\end{docimage} 

\begin{docimage}{pattern-generate1}  
  \begin{tikzpicture}
    \standard[process, moc=ct, f={$ns$;$500ms, V_C$}, type=generate](p1){};
    \begin{signalsCT}[timestamps=1, grid=1, outputs={p1.east}]{3}
      \signalCT*[outline,ordinate=0,ymin=0,ymax=2]{%
        data/ct-generate1-o1.flx}
    \end{signalsCT}
    \signal[-|-] (p1.e1) -> (sigplot.w1);    
  \end{tikzpicture}
\end{docimage}

\begin{docimage}{pattern-stated}
  \begin{tikzpicture}
    \standard[process, moc=ct, f={$\noexpand\mathit{osc}$;$1, \noexpand\mathit{const}(0)$}, type=stated](p1){};
    \begin{signalsCT}[timestamps=5, grid=5, inputs={p1.west}]{10}
      \signalCT*[outline,ymin=0,ymax=1]{%
        data/ct-stated-i1.flx}
    \end{signalsCT}
    \path (sigplot.e1) edge[s,-|-,->] (p1.w1);
    \begin{signalsCT}[timestamps=5, grid=5, outputs={p1.east}]{10}
      \signalCT*[outline,ymin=0,ymax=1]{%
        data/ct-stated-o1.flx}
    \end{signalsCT}
    \signal[-|-] (p1.e1) -> (sigplot.w1);    
  \end{tikzpicture}
\end{docimage} 

\begin{docimage}{pattern-state}
  \begin{tikzpicture}
    \standard[process, moc=ct, f={$\noexpand\mathit{osc}$;$1, \noexpand\mathit{const}(0)$}, type=state](p1){};
    \begin{signalsCT}[timestamps=5, grid=5, inputs={p1.west}]{10}
      \signalCT*[outline,ymin=0,ymax=1]{%
        data/ct-state-i1.flx}
    \end{signalsCT}
    \path (sigplot.e1) edge[s,-|-,->] (p1.w1);
    \begin{signalsCT}[timestamps=5, grid=5, outputs={p1.east}]{10}
      \signalCT*[outline,ymin=0,ymax=1]{%
        data/ct-state-o1.flx}
    \end{signalsCT}
    \signal[-|-] (p1.e1) -> (sigplot.w1);    
  \end{tikzpicture}
\end{docimage} 

\begin{docimage}{pattern-moore}
  \begin{tikzpicture}
    \standard[process, moc=ct, f={$\noexpand\mathit{osc}$;$\times 3$;$1, \noexpand\mathit{const}(0)$}, type=moore](p1){};
    \begin{signalsCT}[timestamps=5, grid=5, inputs={p1.west}]{10}
      \signalCT*[outline,ymin=0,ymax=1]{%
        data/ct-moore-i1.flx}
    \end{signalsCT}
    \path (sigplot.e1) edge[s,-|-,->] (p1.w1);
    \begin{signalsCT}[timestamps=5, grid=5, outputs={p1.east}]{10}
      \signalCT*[outline,ymin=0,ymax=3]{%
        data/ct-moore-o1.flx}
    \end{signalsCT}
    \signal[-|-] (p1.e1) -> (sigplot.w1);    
  \end{tikzpicture}
\end{docimage}

\begin{docimage}{pattern-mealy}
  \begin{tikzpicture}
    \standard[process, moc=ct, f={$\noexpand\mathit{osc}$;$\times$;$\noexpand\pi, \noexpand\mathit{const}(1)$}, type=mealy](p1){};
    \begin{signalsCT}[timestamps=5, grid=3.14, inputs={p1.west}]{10}
      \signalCT*[outline,ordinate=0,ymin=-1,ymax=1]{%
        data/ct-mealy-i1.flx}
    \end{signalsCT}
    \path (sigplot.e1) edge[s,-|-,->] (p1.w1);
    \begin{signalsCT}[timestamps=5, grid=3.14, outputs={p1.east}]{10}
      \signalCT*[outline,ordinate=0,ymin=-1,ymax=1]{%
        data/ct-mealy-o1.flx}
    \end{signalsCT}
    \signal[-|-] (p1.e1) -> (sigplot.w1);    
  \end{tikzpicture}
\end{docimage}

\begin{docimage}{tode}
  \begin{tikzpicture}
    \standard[process, ni=1, no=1, moc=ct, type=toDE](p1){};
    \begin{signalsCT}[grid and time=3, inputs={p1.west}, xscale=1]{9}
      \signalCT*[outline,ordinate=0,ymin=-1,ymax=1]{%
        data/ct-tode-i1.flx}
    \end{signalsCT}
    \signal[-|-] (sigplot.e1) -> (p1.w1);
    \begin{signalsDE}[timestamps=3, grid=3, outputs={p1.east}]{9}
      \signalDE{ $\sin$ :0, $\cos$ :3, $\sin$ :6, 0:200 }
    \end{signalsDE}
    \signal[-|-] (p1.e1) -> (sigplot.w1);
  \end{tikzpicture}
\end{docimage} 

\begin{docimage}{sampde}
  \begin{tikzpicture}
    \standard[process, ni=2, no=1, moc=ct, type=toDE](p1){};
    \begin{signalsCT}[timestamps=1.57, grid=1.57, at={p1.west}, anchor=north east, xshift=-.5cm, yshift=-.2cm, xscale=1]{7.86}
      \signalCT*[outline,ordinate=0,ymin=-1,ymax=1]{%
        data/ct-sampde-i1.flx}
    \end{signalsCT}
    \signal[-|-] (sigplot.e1) -> (p1.w2);
    \begin{signalsDE}[timestamps=3.14, grid=3.14, at={p1.west}, anchor=south east, xshift=-.4cm, yshift=.2cm]{8}
      \signalDE*[trunc]{%
        data/ct-sampde-i2.flx}
    \end{signalsDE}
    \signal[-|-] (sigplot.e1) -> (p1.w1);
    \begin{signalsDE}[timestamps=3.14, grid=3.14, outputs={p1.east}]{9}
      \signalDE*[trunc]{%
        data/ct-sampde-o1.flx}
    \end{signalsDE}
    \signal[-|-] (p1.e1) -> (sigplot.w1);
  \end{tikzpicture}
\end{docimage} 


\end{document}

%%% Local Variables:
%%% TeX-command-default: "Make"
%%% mode: latex
%%% TeX-master: t
%%% End:
